\documentclass[oribibl]{llncs}
\usepackage{url}                  % format URLs
\usepackage[colorlinks=true,allcolors=blue,breaklinks,draft=false]{hyperref}   % hyperlinks, including DOIs and URLs in bibliography
\usepackage{amsmath}
\usepackage{xspace}
\usepackage{algorithm}
\usepackage[noend]{algpseudocode}
\usepackage{subcaption}
\usepackage{xcolor}
\usepackage{tikz}
\usepackage{pgfplots}


\definecolor{blues1}{RGB}{198, 219, 239}
\definecolor{blues2}{RGB}{158, 202, 225}
\definecolor{blues3}{RGB}{107, 174, 214}
\definecolor{blues4}{RGB}{49, 130, 189}
\definecolor{blues5}{RGB}{8, 81, 156}
\definecolor{antiquefuchsia}{rgb}{0.57, 0.36, 0.51}
\definecolor{asparagus}{rgb}{0.53, 0.66, 0.42}
\definecolor{darkspringgreen}{rgb}{0.09, 0.45, 0.27}
\definecolor{darkslategray}{rgb}{0.18, 0.31, 0.31}
\definecolor{coralred}{rgb}{1.0, 0.25, 0.25}

\pgfplotsset{mystyle/.style={%
        %width=6cm,
        %ylabel={mystyle (kg)},
        %xlabel={Eggs (no.)},
        xmin=1,xmax=32,
        enlargelimits=true,
        %xmajorgrids=false,
		ymajorgrids=true,
        grid=major,
        grid style={dashed, gray!30},
        %symbolic x coords={1,2,4,8,16,32},
        ylabel style={align=center},
        xtick={1,4,8,16,32}, 
        scaled y ticks=base 10:-6,
        ytick scale label code/.code={},
    	yticklabel={\pgfmathprintnumber{\tick}}
        %tick align=outside,
}}

\pgfplotsset{unbalanced/.style={%
	   width=0.35\textwidth,
       legend columns=-1,
	   legend entries={\autoTree, \danaTree ,\lockfreeTree, \stmTree, \domTree, },
	   legend to name=unbalancedLegened,
}}


\pgfplotsset{balanced/.style={%
	   width=0.35\textwidth,
       legend columns=-1,
	   legend entries={\autoTreap, \danaAVL,\bronson, \friendly,  \stmTreap,
	   \domTreap, \globalTreap}, legend to name=balancedLegened,
}}

\pgfplotsset{skiplist/.style={%
	   width=\textwidth,
       ymin=0,ymax=2800000,
       xtick={2,4,8,16,32}, 
       legend columns=-1,
	   legend entries={\autoSkiplist, \kary , \skiplist, \stmSkiplist,
	   \domSkiplist}, legend to name=skiplistLegened,
}}

\pgfplotsset{skiplist1000/.style={%
	   width=\textwidth,
       ymin=0,ymax=150000,
       scaled y ticks=base 10:-3,
       xtick={2,4,8,16,32}, 
       legend columns=-1,
	   legend entries={\autoSkiplist, \kary, \skiplist, \stmSkiplist,
	   \domSkiplist}, legend to name=skiplistLegened1000,
}}

\pgfplotsset{skiplistUpdate/.style={%
	   width=0.35\textwidth,
       ymin=0,ymax=5500000,
       xtick={1,2,4,8,16,32}, 
       legend columns=-1,
	   legend entries={\autoSkiplist, \kary , \skiplist, \stmSkiplist,
	   \domSkiplist}, legend to name=skiplistLegenedUpdate,
}}

\newcommand{\Xomit}[1]{}

%-------------Theorem Definitions ---------------%
\newtheorem{theorem}{Theorem}[section]
\newtheorem{lemma}[theorem]{Lemma}
\newtheorem{claim}[theorem]{Claim}
\newtheorem{observation}[theorem]{Observation}
\newtheorem{proposition}[theorem]{Proposition}
\newtheorem{corollary}[theorem]{Corollary}

\newenvironment{proof}[1][Proof]{\begin{trivlist}
\item[\hskip \labelsep {\bfseries #1}]}{\qedsymb\end{trivlist}}
\newenvironment{definition}[1][Definition]{\begin{trivlist}
\item[\hskip \labelsep {\bfseries #1}]}{\end{trivlist}}
\newenvironment{example}[1][Example]{\begin{trivlist}
\item[\hskip \labelsep {\bfseries #1}]}{\end{trivlist}}
\newenvironment{remark}[1][Remark]{\begin{trivlist}
\item[\hskip \labelsep {\bfseries #1}]}{\end{trivlist}}


\newcommand{\qed}{\nobreak \ifvmode \relax \else
      \ifdim\lastskip<1.5em \hskip-\lastskip
      \hskip1.5em plus0em minus0.5em \fi \nobreak
      \vrule height0.75em width0.5em depth0.25em\fi}
\newcommand{\qedsymb}{\hfill{\rule{2mm}{2mm}}}


%--------------------------------------------------%

\newcommand{\code}[1]{\textsf{#1}}
\newcommand{\readV}{\code{read\_version}\xspace}
\newcommand{\readSet}{\code{read\_set}\xspace}
\newcommand{\writeV}{\code{write\_version}\xspace}

\newcommand{\reqI}{\textbf{LPR1}\xspace}
\newcommand{\reqII}{\textbf{LPR2}\xspace}

%---------Evaluation Macros-----------------------%
\newcommand{\autoTree}{LR-Tree\xspace}
\newcommand{\autoTreap}{LR-Treap\xspace}
\newcommand{\optAutoTree}{Opt-LR-Tree\xspace}
\newcommand{\optAutoTreap}{Opt-LR-Treap\xspace}
\newcommand{\autoSkiplist}{LR-Skiplist\xspace}
\newcommand{\danaTree}{LO-Tree\xspace}
\newcommand{\danaAVL}{LO-AVL\xspace}
\newcommand{\bronson}{Snap-Tree\xspace}
\newcommand{\friendly}{CF-Tree\xspace}
\newcommand{\skiplist}{Java-Skiplist\xspace}
\newcommand{\kary}{k-Tree\xspace}
\newcommand{\lockfreeTree}{LF-Tree\xspace}
\newcommand{\globalTree}{Global-Tree\xspace}
\newcommand{\globalTreap}{Global-Treap\xspace}
\newcommand{\domTree}{Lock-Tree\xspace}
\newcommand{\domTreap}{Lock-Treap\xspace}
\newcommand{\stmTree}{STM-Tree\xspace}
\newcommand{\stmTreap}{STM-Treap\xspace}
\newcommand{\stmSkiplist}{STM-Skiplist\xspace}
\newcommand{\domSkiplist}{Lock-Skiplist\xspace}
\newcommand{\getOP}{\textsc{get}\xspace}


%---------Comments -----------------------%
\newcommand{\Idit}[1]{{\color{red}{[\textbf{Idit:} #1 ]}}}
\newcommand{\guy}[1]{{\color{red}{[\textbf{guy:} #1 ]}}}
\newcommand{\eshcar}[1]{{\textcolor{violet}{\{{\bf eshcar:} \em #1\}}}}
\newcommand{\maya}[1]{{\textcolor{magenta}{\{{\bf maya:} \em #1\}}}}




\begin{document}



\title{Towards Automatic Lock Removal\\ for Scalable Synchronization}

\author{Ivar Ekeland\inst{1} \and Roger Temam\inst{2}}
\author{
Maya Arbel\inst{1}\fnmsep\inst{2}
\and Guy Golan Gueta\inst{1}  
\and Eshcar Hillel\inst{1}  
\and Idit Keidar\inst{1}\fnmsep\inst{2}} 
\institute{Yahoo Labs, Haifa, Israel
\and The Technion, Haifa, Israel}



\maketitle

\begin{abstract}
We present a \emph{code transformation} for concurrent data structures,
which increases their scalability without sacrificing correctness.
Our transformation takes lock-based code and replaces some of the
locking steps therein with optimistic synchronization in order to reduce contention. The main idea is to
have each operation perform an optimistic traversal of the data structure
as long as no shared memory locations are updated, and then proceed with
pessimistic code. The transformed code inherits essential
properties of the original one, including linearizability, serializability,
and deadlock freedom.
%It reduces contention, both on locks and on access to shared memory locations.
%When applying our transformation to
%hand-over-hand locking solutions, we obtain significantly superior scalability.

Our work complements existing pessimistic transformations that make
sequential code thread-safe by adding locks.
In essence, we provide a way to optimize such transformations by reducing
synchronization bottlenecks (for example, locking the root of a tree).
The resulting code scales well and significantly outperforms
pessimistic approaches. We further compare our synthesized code to state-of-the-art
data structures implemented by experts.
We find that its performance is comparable %, and sometimes even superior,
to that achieved by the custom-tailored implementations.
Our work thus shows the promise that automated approaches
bear for overcoming the difficulty involved in manually
hand-crafting concurrent data structures.

\end{abstract}



\thispagestyle{empty}





\section{Introduction} \label{sec:intro}

\subsection{Automatic Lock Removal}
% Parallelizing data strucutres is important for performance
The steady increase in the number of cores in today's computers is driving software developers to allow more and more parallelism.
An important focal point for such efforts is scaling the concurrency of shared data structures, which are often a principal friction point among threads.
%; recent work has illustrated that improved data structure concurrency can lead to
%benefits in overall system performance~\cite{clsm-poster}.
%It is therefore not surprising that many recent works have been dedicated to developing scalable concurrent data
Many recent works have been dedicated to developing scalable concurrent data
structures (e.g.,~\cite{ArbelA2014,DrachslerVY2014,NatarajanM2014,BrownER2014,CrainGR2013,BraginskyP2012,
AfekKKMT2012,EllenFRB2010,BronsonCCO2010,HerlihyLLS2007,fraser2004practical,Michael:1996}),
some of which are widely used in real-world systems~\cite{Ohad:OOPSLA11}.

% They are difficult to build and of resticted use
Each of these projects generally focuses on a single data
structure (for example, a binary search tree~\cite{BronsonCCO2010} or a queue~\cite{Michael:1996}) and manually optimizes its implementation. These data structures are developed by concurrency experts, typically PhDs or PhD candidates.
Proving the correctness of such custom-tailored data structures is painstaking;
for example, the proofs of \cite{BraginskyP2012,EllenFRB2010} are $31$ and $20$ pages long
respectively.
The rationale behind dedicating so much effort to one data structure is that it is
generic and can be used by many applications. Nevertheless,  systems often use data structures in unique ways
that necessitate changing or extending their code (e.g.,~\cite{levelDB,jmonkey,OhadThesis,zyulkyarov2009atomic}), which limits the usability of custom-tailored
implementations. Hence, the return-on-investment for such endeavors may be suboptimal.
Here, we propose an approach to facilitate this labor-intensive process by automatic means,
making scalable synchronization more readily available.

% We give a transformation
Specifically, we present in Section~\ref{sec:algorithm} a source-to-source
code transformation that takes a lock-based concurrent data structure implementation as its input
and generates more scalable code for the same data structure via judicious use of optimism.
%%%Section~\ref{sec:model} details our model and assumptions, and
%%%Section~\ref{sec:algorithm} specifies the transformation.
Our approach combines optimism and pessimism in a new, practical, way.
In striking the balance between the two, we exploit the common access pattern in data structure operations,
(for example, tree insertion or removal), which typically begin by traversing the data structure (to the insertion or removal point), and then perform (mostly) local updates at that location.
Our transformation replaces locking steps in the initial read-only traversal of each operation with
optimistic synchronization, whereas the update phase employs the original lock-based synchronization.
Our work may thus be seen as a form of software lock elision for read-only operation prefixes.

% Best of both worlds
Combining optimism and pessimism allows us to achieve ``the best of both worlds'' -- while the
optimistic traversal increases concurrency and eliminates bottlenecks,
the use of pessimistic updates saves the overhead associated with speculative or deferred shared
memory updates, (as in software transactional memory~\cite{HLR:SLCA2010}).
The partially-optimistic execution is compatible with the original code, which permits us to re-execute operations
pessimistically when too many conflicts occur, avoiding livelocks.
%Furthermore, it allows for code optimizations
%that make the optimistic execution fail in some conflict-free cases (for example, when too many items would have been locked
%by the original code), since we can always fallback upon lock-based execution.

% Properties of our transformation
We show in Section~\ref{sec:proof} that our transformation preserves the external behaviour of the original lock-based code.
%all essential properties of the original code: serializability, linearizability, and deadlock-freedom.
In other words, if the original code is correct (in the sense of serializability, linearizability, and deadlock-freedom), so is the
transformed version. Moreover, the transformation
%does not introduce any new accesses to shared data; in particular, it
refrains from introducing a shared global clock (as used in some transactional memory systems~\cite{DBLP:conf/eurosys/ShalevS06}) or other
shared data structures. Thus, if the original code is \emph{disjoint access parallel}~\cite{Israeli:1994:DIS:197917.198079}, i.e., threads
that access disjoint (abstract) data objects do not contend on (low level) shared memory locations, then this
property holds also for the transformed code.

\subsection{Fully Automatic Parallelization}
% Automatic parallelization
One important use case for our transformation is to apply it in conjunction with automatic lock-based
parallelization mechanisms~\cite{Gueta2011,MZGB:POPL06}.
The latter automatically instrument sequential code (at compile time)
and add fine-grained lock and unlock instructions that ensure its safety in concurrent executions.
%For example, \emph{domination locking}~\cite{Gueta2011} is applicable to trees or forests
%(including binary search trees, b-trees, treaps~\cite{AragonS1989}, and self adjusting heaps~\cite{Sleator:SAH1986:SAH}); it
%employs a variant of hand-over-hand locking~\cite{SilberschatzK1980},
%acquiring and releasing locks as it goes down the tree.
%%Other approaches are applicable to DAGS~\cite{dag-locking}.
Our evaluation shows that, by themselves, solutions of this sort may scale poorly.
%~\cite{Gueta2011}.
This is due to synchronization bottlenecks, e.g., the root of a tree,
which is locked by all operations.
By subsequently applying our transformation, one can optimize
the lock-based code they produce, yielding a \emph{fully automatic approach to
scalable parallelization of sequential code}.

%% Why not read-write locks
%It is worth noting that the aforementioned mechanisms synthesize code that uses conventional symmetric locks,
%which is the type of locks handled by our transformation. We are not aware of any automatic transformation
%inserting read-write locks. We further note that read-write locks
%are more costly than conventional ones, and, moreover, threads using
%read locks contend on the shared memory locations employed by the lock's implementation~\cite{xxx}.
%In contrast, with our transformation, threads executing the optimistic phase do not contend on locks,
%and are completely invisible to other threads.

\subsection{Evaluation}
In Section~\ref{sec:eval} we evaluate our transformation by generating three data structures-- an unbalanced search tree, a treap
(randomized balanced search tree),
and a skip list that supports range queries. We synthesize the first two from sequential implementations using the algorithm of~\cite{Gueta2011}, followed by our transformation.
For the skip list, we manually add fine-grained locks (in a straightforward manner), and then apply our transformation.
All examples are implemented in Java. We evaluate the scalability of the resulting code
in a range of workload scenarios on a $32$-core machine.
In all cases, the lock-based implementations do not scale --
their throughput remains flat as the number of running threads increases. In contrast, the code generated by our transformation
is scalable, and its throughput continues to grow with the number of threads.

We compare our synthesized code to state-of-the-art data structures
that were hand-crafted by experts in the field~\cite{DrachslerVY2014,BronsonCCO2010,ConcurrentSkipList,fraser2004practical}.
%and resulted in publications in leading venues~\cite{DrachslerVY2014,BronsonCCO2010}.
Our results show that the
fully-automatically generated code for the search tree and treap
achieves comparable performance to that of
custom-tailored solutions.

We also consider a data structure that supports range queries, which are required by
many applications (e.g.,~\cite{levelDB,FerroJKRY14}). To this end we implement a skip list -- a data structure that naturally supports range queries.
While range queries implemented using the iterators available in the Java concurrency library's skip list~\cite{ConcurrentSkipList} perform
somewhat better than ones in our synthesized code, it is important to note that these iterators are \emph{not}
linearizable (atomic), and only support so-called weak consistency, whereas range queries in our implementation are linearizable.
A notable previous linearizable implementation is due to Bronson et al.~\cite{BronsonCCO2010},
and it is significantly out-performed by our synthesized code.
%
This illustrates the benefit of the broad applicability
of our automatic approach compared to specific custom-tailored implementations.

Note that the alternative automatic approach we compare with is domination locking~\cite{Gueta2011} which scales poorly.
%\Idit{Say something about transactional memory.}
%The only other automatic parallelization approach we are familiar with is software transactional memory~\cite{xxx},
%which has been shown to perform even worse than a global lock~\cite{}, hence we do not empirically
%compare our solution with it.
Software transactional memory is another approach for automatic parallelization,
yet due to the significant overhead associated with this approach~\cite{Cascaval:2008,DuffyTM2010}, it is often complemented
by custom-tailored data structure implementations whose operations
can be called from within transactions~\cite{Herlihy:2008,Koskinen:2010,NathanBronson11}.
%
Further discussion of related work appears in Section~\ref{sec:related}.

To conclude, this paper demonstrates that automatic synchronization, based on a careful combination of optimistic and
pessimistic concurrency control, is a promising approach for bringing legacy code to emerging computer architectures.
While this paper illustrates the method for tree and skip list data structures, we believe that the general direction may be more broadly applicable, and maybe used with a variety of locking schemes, such as two phase locking.
Section~\ref{sec:discussion} concludes the paper and touches on some directions for future work. 
%\section{Model and Definitions}\label{sec:model}



\paragraph{Shared Memory Data Structures}

A \emph{data structure} defines a set of \emph{operations} that may be invoked by
clients of the data structure, potentially concurrently.
%
Operations have parameters and local variables, which are private to the invocation of the operations.
(Thus, these are thread-local variables.)
%
We assume that the code of each operation is represented by a separate control-flow graph (CFG).

The operations of a data structure interact via a shared memory which is composed of a set of \emph{shared objects}.
%
Each shared object supports atomic \emph{read} (load) and \emph{write} (store) instructions.
%
Below, we extend shared objects to also support locks.

Every thread executes a sequence of operations, each of which is invoked with certain parameters and returns a response.
An operation's execution consists of a sequence of primitive \emph{steps}, beginning with an \emph{invoke} step, followed by
atomic accesses to shared objects, and ending with a \emph{return} step. Steps also modify the executing thread's local variables.

A \emph{configuration} is an assignment of values to all shared objects and local variables. Thus, each step takes the system from one
configuration to another. Steps are deterministically defined by the data structure's code and the current configuration.
We assume that each data structure has a single \emph{initial configuration}.

An \emph{execution} is an alternating sequence of configurations and steps,
$C_0,s_1,C_1, \ldots,s_i,C_i,\ldots,$
where $C_0$ is the initial configuration,
and each configuration $C_i$ is the result of
executing step $s_i$ on configuration $C_{i-1}$.
We only consider finite executions in this paper.
An execution is \emph{sequential} if steps of different operations are not interleaved.
In other words, a sequential execution is a sequence of operation executions.

\paragraph{Locks}
Each shared object serves as a lock for itself.
It supports atomic \emph{lock}, \emph{tryLock}, \emph{unlock} and \emph{isLockedByAnotherThread} instructions.
Locks are exclusive (i.e., a lock can be held by at most one thread at a time).
The execution of a thread
trying to acquire a lock (by a \emph{lock} instruction) which is
held by another thread is blocked until a time when the
lock is available (i.e., is not held by any thread).
The other instructions never block the execution.
The \emph{tryLock} instruction returns \emph{false} if the lock is currently held by another thread, otherwise it acquires the lock and returns \emph{true}.
The \emph{isLockedByAnotherThread} instruction returns \emph{true}, if and only if, the lock is currently held by another thread.

We assume that in the given code every (read
or write) access by an operation to a shared object is performed when the
executing thread holds the lock on that object.
We also assume that the locking in the given code only uses \emph{lock} and \emph{unlock} instructions 
(i.e., the other locking instructions are not used).



\paragraph{Correctness}

The correctness of a data structure is defined in terms of its external behavior, as reflected in values returned by invoked operations.
Correctness of a code transformation is proven by showing that the synthesized code's executions are equivalent to ones of the original code,
where two executions are  \emph{equivalent} if every thread invokes the same
operations in the same order  in both executions, and gets the same result for each operation. More formally, we say in this paper that a code transformation is \emph{correct} if every execution of the transformed code
is equivalent to some execution of the original code.

The widely-used correctness criterion of serializability relies on equivalence to sequential executions in order to
link a data structure's behavior under concurrency to its sequentially specified behavior. Since equivalence is transitive,
we get that any code transformation satisfying our correctness notion, when applied to serializable code, yields code that is also serializable.
If the code transformation further ensures the real-time order of operations (i.e., operations that do not overlap appear in the same order in
executions of the transformed and original code), then linearizability (atomicity) is also invariant under the transformation.
Another important aspect of correctness is preserving the progress conditions of the original code, for example, deadlock-freedom.

In this paper, we are not concerned with internal consistency (as required e.g., by opacity~\cite{GuerraouiK2008} or the validity notion of~\cite{LevAriCK2014}),
which restricts the configurations an operation might see during its execution.
This is because our code transformation uses timeouts and exception handlers to overcome unexpected behavior that may arise when a thread sees an inconsistent view of global variables (similar to~\cite{Nakaike:2010}).




\renewcommand{\ttdefault}{pcr}
%\algrenewcommand\textkeyword{\texttt}
\algrenewcommand\algorithmicif{\texttt{if}}
\algrenewcommand\algorithmicthen{\texttt{then}}
\algrenewcommand\algorithmicfunction{\textsc{Function}}
\algrenewcommand\algorithmicforall{\texttt{for all}}
\algrenewcommand\algorithmicdo{\texttt{do}}
\algrenewcommand\textproc{\textit}
\newcommand{\codesize}{\footnotesize}



\section{Transformation}\label{sec:algorithm}

We present an algorithm for a source-to-source transformation, whose
goal is to optimize the code of a given data structure implemented using lock-based concurrency control.
In Section~\ref{ssec:locks}, we detail our assumptions about the given code and the locks it uses.
\Xomit{
%% Idit: omitted to save space
The transformation produces a combination of pessimistic and optimistic synchronization
by replacing part of the locking (in the given code) with optimistic concurrency control.
}
Section~\ref{ssec:overview} overviews our general approach to combining optimism and pessimism,
while
%The synthesized code consists of three phases - an optimistic read-only phase, a validation phase,
%and a pessimistic update phase. 
%The transformation first partitions the given code into a read-only phase and
%an update phase as described in Section~\ref{ssec:extendedTran}, and
%then instruments each of these phases separately and adds the validation phase between the two;
Section~\ref{ssec:transformation} details how the code is instrumented. 
%of each of the three phases is produced.

\subsection{Lock-based Data Structures}\label{ssec:locks}

A \emph{data structure} defines a set of \emph{operations} that may be invoked by
clients of the data structure, potentially concurrently.
%
Operations have parameters and local (private) variables. %, which are private to the invocation, and are thus thread-local.
%
The operations interact via \emph{shared memory variables}, which are also called \emph{shared objects}.
%
Each shared object supports atomic \emph{read} (load) and \emph{write} (store) instructions.
%
%Below, we extend shared objects to also support locks.
More formal definitions appear in Appendix~\ref{sec:formal-proof}. 

In addition, each shared object is associated with a lock, which can be unique to the object or common to several (or even all) objects. 
The object supports atomic \emph{lock} and \emph{unlock} instructions.
Locks are exclusive (i.e., a lock can be held by at most one thread at a time), and blocking.
We assume that in the given code every (read
or write) access by an operation to a shared object is performed when the
executing thread holds the lock associated with that object.


The given code only uses the \emph{lock} and \emph{unlock} instructions, while the transformed code can apply in addition atomic non-blocking
\emph{tryLock}  and \emph{isLockedByAnother} instructions: 
 \emph{tryLock}  returns \emph{false} if the lock is currently held by another thread, otherwise it acquires the lock and returns \emph{true};
 \emph{isLockedByAnother}  returns \emph{true} if and only if the lock is currently held by another thread.


\subsection{Combining Optimism and Pessimism}\label{ssec:overview}

% Optimism
%Generally speaking, 
Optimistic concurrency control is a form of synchronization, which accesses shared variables without using locks in the hope that they will not be modified by others before the end of the operation (or more generally, the transaction). To verify the latter, optimistic concurrency control relies on \emph{validation}, which is typically implemented using \emph{version numbers}. If validation fails, the operation restarts. Optimistic execution of update operations requires either performing roll-back (reverting variables to their old values) upon validation failure, or deferring writes to commit time; both approaches induce significant overhead~\cite{Cascaval:2008}. We therefore refrain from speculative shared memory updates.

%The main idea:
The main idea behind our approach is to judiciously use optimistic synchronization only as long as an operation does not update shared state;
we use a standard approach based on version numbers to allow validation of optimistic reads.
Once an operation
writes to shared memory, we revert to pessimistic (lock-based) synchronization. In
other words, we rely on validation %at the end of the read-only prefix of an operation 
in order to render redundant
locks that would have been acquired and freed before the first update.
This scheme is particularly suitable for data structures,
since the common behavior of their operations
is to first traverse the data structure, and then
perform modifications.
Since the read only prefix has no side effects we can sandbox it by catching exceptions and infinite loops, and defer validation to the end of the traversal.

Conceptually, our approach thus divides an operation into three phases: an optimistic \emph{read-only phase},
a pessimistic \emph{update phase} and a \emph{validation phase} that conjoins them.
The read-only phase traverses the data structure without taking any locks, while maintaining 
in thread-local variables
sufficient information to later ensure the correctness of the traversal.
The read phase is \emph{invisible} to other threads, as it updates no shared variables.
The update phase uses the original pessimistic (lock-based) synchronization, with the addition of updating version numbers.
The validation phase bridges between the optimistic and pessimistic ones.
It first locks the objects for which a lock would have been held at this point by
the original locking code, and then validates the correctness
of the read-only phase. This allows the
update phase to run as if an execution of the original pessimistic synchronization
took place. If the validation fails, the operation
restarts. In order to avoid livelock, we set a threshold on the number of restarts.
If the threshold is exceeded, the code falls back on pessimistic execution.
We show below that
it is safe to do so, since our semi-optimistic code is compatible
with the fully pessimistic one.


\paragraph{Phase Transition}
%\label{ssec:extendedTran}
In many cases, the transition from the read-only phase to the update phase occurs at a statically-defined code location. For example, many data structure operations begin with a read-only traversal to locate the key of interest, and when it is found, proceed to execute code that modifies the data structure. This is the case in all the examples we consider in Section~\ref{sec:eval} below.

More generally, it is possible to switch from the optimistic read-only execution (via the validation phase) to pessimistic execution at any point before the first update. Moreover, the phase transition point can be determined dynamically at run time.

One possible way to dynamically track the execution mode is using a flag \textbf{opt}, initialized to true, indicating the optimistic phase. 
Every shared memory update operation is then instrumented with code that checks \textbf{opt}, and if it is true, executes the validation phase followed by setting \textbf{opt} to false and continuing the execution from the same location. 
%Every lock operation is instrumented with a check of \textbf{opt} to determine whether it is to be executed optimistically or pessimistically. 

\subsection{Transforming the Code Phases}\label{ssec:transformation}

\newcommand{\spOne}{\hspace{-3mm}\ }
\newcommand{\spZero}{\hspace{-3mm}}
\begin{figure*}
\codesize
	\begin{center}
	\begin{subfigure}[b]{.44\textwidth}
		\begin{algorithmic}[1]{}
		{\ttfamily
			\Function{addThird}{List list, Node new} \label{code:begin}
%			\Statex -----------------------
			\Statex ----------
\Comment{\textrm{read-only phase}}
			\State                               \label{code:beginRead}
			\State                               
            \State{\spOne}\textbf{list.lock()}
			\State{\spOne}Node prev = list.head
			\State{\spOne}\textbf{prev.lock()}
            \State{\spOne}\textbf{list.unlock()}
			\State{\spOne}Node succ = prev.next
			\State{\spOne}\textbf{succ.lock()}
			\State{\spZero}\textbf{prev.unlock()}
			\State{\spZero}prev = succ
			\State{\spZero}succ = succ.next
			\State{\spZero}\textbf{succ.lock()}  \label{code:endRead}
%			\Statex -----------------------
			\Statex ----------
			\State                               \label{code:beginValidation}
			\State
			\State
			\State
			\State
			\State
			\State
            \State                               \label{code:endValidation}
%			\Statex -----------------------
			\Statex ----------
							\Comment{\textrm{update phase}}
			\State{\spZero}prev.next = new       \label{code:beginUpdate}
			\State{\spZero}\textbf{new.lock()}
			\State{\spZero}new.next = succ
            \State
			\State{\spZero}\textbf{prev.unlock()}
            \State
			\State{\spZero}\textbf{new.unlock()}
            \State
			\State{\spZero}\textbf{succ.unlock()}  \label{code:endUpdate}
			\EndFunction
			}
		\end{algorithmic}
		\caption{Code with original locking} \label{figure:transformation:before}
	\end{subfigure}
\hspace{0.01\textwidth}
	\begin{subfigure}[b]{.53\textwidth}
		\begin{algorithmic}[1]{}
		{\ttfamily
			\Function{addThird}{List list, Node new} \label{code:begin}
%			\Statex -----------------------
			\Statex ----------
			\Comment{\textrm{read-only phase}}
            \State{\spOne}\textbf{lockedSet.init()} \label{code:initSets1}
            \State{\spOne}\textbf{readSet.init()} \label{code:initSets2}
            \State{\spOne}\textbf{if !track(list)  then {goto} \ref{code:begin}} \label{code:readGhaseGoto0}
			\State{\spOne}Node prev = list.head
			\State{\spOne}\textbf{if !track(prev)  then {goto} \ref{code:begin}} \label{code:readGhaseGoto1}
            \State{\spOne}\textbf{lockedSet.remove(list)} \label{code:lockedSet:remove1}
			\State{\spOne}Node succ = prev.next
			\State{\spOne}\textbf{if !track(succ) then {goto} \ref{code:begin}}  \label{code:readGhaseGoto2}
			\State{\spZero}\textbf{lockedSet.remove(prev)} \label{code:lockedSet:remove2}
			\State{\spZero}prev = succ
			\State{\spZero}succ = succ.next
			\State{\spZero}\textbf{if !track(succ) then {goto} \ref{code:begin}} \label{code:readGhaseGoto3}
%			\Statex -----------------------
			\Statex ----------
			\Comment{\textrm{validation phase}}
			\State{\spZero}\textbf{read fence} \label{code:fence}
			\State{\spZero}\textbf{for all obj in lockedSet do} \label{code:validateLockedSet}	
            \State{\spZero}\ \ \textbf{if !obj.tryLock() then}
            \State{\spZero}\ \ \ \ \ \textbf{unlockAll()}
            \State{\spZero}\ \ \ \ \ \textbf{{goto} \ref{code:begin}} \label{code:validateGoto1}
			\State{\spZero}\textbf{if !validateReadSet() then} 		\label{code:validateReadSet}
				\State{\spZero}\ \ \textbf{unlockAll()}
				\State{\spZero}\ \ \textbf{{goto} \ref{code:begin}} \label{code:validateGoto2}
				%\Comment Restart Operation
%			\Statex -----------------------
			\Statex ----------
			\Comment{\textrm{update phase}}
			\State{\spZero}prev.next = new
			\State{\spZero}\textbf{new.lock()}
			\State{\spZero}new.next = succ			
			\State{\spZero}\textbf{prev.incVersion}
			\State{\spZero}\textbf{prev.unlock()}
			\State{\spZero}\textbf{new.incVersion}
			\State{\spZero}\textbf{new.unlock()}
			\State{\spZero}\textbf{succ.incVersion}
			\State{\spZero}\textbf{succ.unlock()}

			\EndFunction
			}
		\end{algorithmic}
		\caption{The code produced by our  transformation}\label{figure:transformation:after}
	\end{subfigure}
	%\bigskip
	%\hline
	\end{center}
\vspace{-4mm}
	\caption{Code transformation example.
	The synchronization code is in bold.
			\label{figure:transformation}}
\end{figure*}

\begin{figure}
\centering
%\begin{minipage}{0.49\textwidth}
%\centering
\codesize
\begin{algorithmic}[1]{}
		{\ttfamily
		\Function{track}{obj}
		\State{}lockedSet.add(obj) \label{code:lockedSet:add}
			\State long ver = obj.getVersion() \label{code:track:getVersion}
			\State readSet.add($\langle$obj,ver$\rangle$)
%\Statex	\hspace{-30mm} \Comment{\textrm{Eager validation}}
%			\State if {$\exists$<o,v>$\in$readSet such that o=obj and v!=ver} then
			% return false \label{code:track:verifyVersion}			
			\If{$\langle$obj,v$\rangle\in$readSet and v!=ver} \label{code:track:verifyVersion}
%				\State return false
				 return false
			\EndIf
			\If{obj.isLockedByAnother()} \label{code:track:verifyUnlocked}
%				\State return false \label{code:track:verifyUnlockedB}
				 return false \label{code:track:verifyUnlockedB}
			\EndIf
			\State return true
		\EndFunction
		}
\end{algorithmic}
\caption{In read-only phase, locking is replaced by
tracking locks and read objects' versions.
\label{figure::track}}
%\end{minipage}\hfill
\end{figure}
\begin{figure}
%\begin{minipage}{0.47\textwidth}
\centering
\codesize
\begin{algorithmic}[1]{}
		{\ttfamily
		\Function{validateReadSet}{}()
		\ForAll {$\langle$obj,ver$\rangle$ in readSet}
			\If{obj.isLockedByAnother()}
			\State return false \Comment{\textrm{validation failed (locked object)}} \label{code:validate:verifyUnlocked}
%			\State \Comment{\textrm{(locked object)}}
			\EndIf
			\If{obj.getVersion() != ver} 
				\State return false \Comment{\textrm{validation failed (different version)}} \label{code:validate:verifyVersion}
%				\State \Comment{\textrm{(different version)}}
			\EndIf
		\EndFor
		\State retrun true \Comment{\textrm{validation succeed}}
		\EndFunction
		}
\end{algorithmic}
\caption{Read set validation: verify that  objects are unlocked and their versions are unchanged.\label{figure::validate}}
%\end{minipage}
\end{figure}

We now describe how we synthesize the code for each of the phases. 
%The regular three-phase flow is described in
%Section~\ref{sssec:alg-normal}, and exceptions are described in Section~\ref{sssec:alg-abnormal}.
We first describe the regular three-phase flow, and then continue with describing the exceptional cases.

\subsubsection{Normal Flow}
\label{sssec:alg-normal}

We illustrate the transformation for a simple code snippet that adds a new element as the third node in a linked list. Each node is associated with a  lock.
The original and transformed code are provided in Figure \ref{figure:transformation}. The latter uses
the tracking and validation functions in Figures \ref{figure::track} and
\ref{figure::validate}, resp.
For clarity of exposition, we present a statically instrumented version, without tracking the phases using \textbf{opt}.

%\paragraph{Version Numbers}

Our transformation instruments each lock with an additional field \emph{version}. We assume each object supports \emph{getVersion} and \emph{incVersion} instruction to read and increment the version number of the lock associated with the object. We invoke \emph{incVersion} when holding the lock, and are therefore are not concerned about contention.
%Later we will show that
%If $o$ is not locked, then this field  represents the current version number of $o$.
Note that each lock has its own version, i.e., version numbers of different locks are independent of each other.

\paragraph{Read-only Phase}
% In this phase, we replace all the lock and unlock instructions with local tracking.
In this phase the executing thread is invisible to other threads, i.e., 
 avoids  contention on shared memory both in terms of writing and in terms of locking.
During this phase, our synchronization maintains two thread-local multi-sets: \emph{lockedSet} and \emph{readSet}.
The \emph{lockedSet} tracks the objects that were supposed to be locked by the original synchronization.
%
The \emph{readSet} tracks versions of all objects read by the
operation, in order to allow us to later validate that the operation has observed a consistent view of shared memory.

At the beginning of the read-only phase, we insert code that initializes \emph{lockedSet} and \emph{readSet} to be empty (see  lines~\ref{code:initSets1}-\ref{code:initSets2} of Figure~\ref{figure:transformation:after}).
Throughout the read-only phase, (i.e., when \textbf{opt} is true with dynamic phase transitions), 
we replace every lock and unlock instruction with the corresponding code in Table~\ref{Ta:readOnlyTransformation}.
A lock instruction on object $o$ is replaced with code that tracks the object and the version of its lock in
 \emph{lockedSet} and \emph{readSet} (see Figure~\ref{figure::track}).
An unlock instruction on object $o$ is replaced with code that removes $o$ from \emph{lockedSet} (see
lines \ref{code:beginRead}-\ref{code:endRead} of Figure~\ref{figure:transformation:after}).

\begin{table}
\codesize
\ttfamily
{\tt
\begin{center}
\begin{tabular}{|l|l|}
\hline
\textbf{Original code} & \textbf{Transformed Code}\\
\hline
\textit{x.lock()}&
\textit{if !track(x) then goto $S$}
\\
\hline
\textit{x.unlock()}&
\textit{lockedSet.remove(x)}
\\
\hline
\end{tabular}
\end{center}
}
\caption{Transformation for read-only phase:
each locking instruction (left column) is replaced with the corresponding code on the right;
 $S$  denotes the beginning of the operation.
}
\label{Ta:readOnlyTransformation}
\end{table}

In Figure~\ref{figure::track} (lines \ref{code:track:verifyVersion}-\ref{code:track:verifyUnlockedB}), we use an eager validation scheme\footnote{Eager validation is not required for correctness.}: 
If the object already exists in \emph{readSet}, we check that the current version of its lock is equal
to the version in \emph{readSet}; and if the versions are different  the operation restarts (line~\ref{code:track:verifyVersion}).
Similarly, it is  checked to be unlocked, and the operation restarts if it is locked (line~\ref{code:track:verifyUnlocked}).


% Optimization
Although it only accesses thread-local data structures, lock tracking induces a certain overhead due to the need to search a lock in the \emph{lockedSet} in order to unlock it. (In our experiments presented below, in large data structures, this overhead slows operations down by up to $40\%$).
We suggest some optimizations to mitigate this cost. 
First, we observe that 
the \emph{lockedSet} does not need to be tracked in read-only operations, which a compiler can easily detect. 
 We can further avoid this overhead in update operations in certain 
 cases by relying on the structure of the transformed code. For example, if the lock-based code is created from sequential code 
using domination locking~\cite{Gueta2011}, then at any given time in the read phase, it holds locks on a well-defined set of objects -- the ones currently pointed by the operation's local variables. When applying our transformation to code generated by this scheme, we can optimize it to remove lock-tracking, and instead populate the \emph{lockedSet} with the appropriate locks immediately before executing the validation phase.

\paragraph{Validation Phase}
The code of the validation phase is invoked between the read-only phase and the update phase (lines \ref{code:beginValidation}-\ref{code:endValidation} of Figure~\ref{figure:transformation:after}).
It locks the objects that are left in \emph{lockedSet} and validates the objects in \emph{readSet}.
To avoid deadlocks, the locks are acquired using a \emph{tryLock}
instruction.
If any \emph{tryLock} fails, the code unlocks  all
previously acquired locks and restarts from the beginning
(lines \ref{code:validateLockedSet}-\ref{code:validateGoto1}).

The function \emph{validateReadSet} in Figure~\ref{figure::validate} verifies that the objects in the read set have not been updated.
%
The function checks that each object in the read set is not locked by another thread,
and that the current version of the lock associated with the object matches the version saved in the
\emph{readSet}.
This check guarantees that the object was not locked from the time it was read until
the time it was validated.
Since operations write only to
locked objects, it follows that the object was not changed.
This \emph{readSet} validation can be viewed as a double collect~\cite{Afek:1993:ASS:153724.153741}
of all objects accessed by the read-only phase.
%
The operation is restarted if the validation fails (lines \ref{code:validateReadSet}-\ref{code:validateGoto2}).

% Fences
We assume that, following standard practice in lock implementations, 
the function \textit{isLockedByAnother} imposes a \emph{memory fence} (barrier). 
This ensures that the lock and version are read during \textit{track} before the object's value is read optimistically
during the read-only phase. 
To ensure that the second read of the lock and version, during the validation phase, succeeds the 
optimistic read of the object's value, we precede the validation phase with a memory fence as well (line~\ref{code:fence}).  
Note that it suffices to impose a \emph{read fence} (sometimes called acquire or load fence) 
prior to the validation as well as during \textit{isLockedByAnother}, because this part of the code does not include
writes to shared memory.

\paragraph{Update Phase}
In this phase our transformation preserves the original locking while maintaining the versions of the objects, i.e., the version of an object $o$ is incremented every time $o$ is unlocked.
Here, (i.e., in case  \textbf{opt} is false with dynamic phase transitions), 
before each unlock instruction \emph{\ttfamily x.unlock()} we insert the code \emph{\ttfamily x.incVersion()} .
An example is shown in lines \ref{code:beginUpdate}-\ref{code:endUpdate} of Figure~\ref{figure:transformation:after}.


\subsubsection{Exceptions from Regular Flow}
\label{sssec:alg-abnormal}


%\paragraph{Inconsistent Views}
%Other than when reading objects already in the read set, the read phase does not validate past reads during its executions ---
The read phase does not validate past reads during its executions (other than when re-reading the same variable). 
As a result, it may observe an inconsistent state of shared memory, which may lead to infinite loops or spurious exceptions (as explained, e.g., in~\cite{HLR:SLCA2010}).
%
We avoid such infinite loops using a timeout.
If the timeout expires before the read-only phase completes, read set
validation takes place (via the function \emph{validateReadSet}). If the validation fails, the operation is restarted.
This is realized by inserting code that examines the timeout in every loop iteration in the original code.
%
Similarly, we avoid spurious exceptions by catching all exceptions and
performing validation. Here too, if the validation fails, the operation is restarted. Otherwise, the exception is handled as in
the original code.

%rebuttal discussion
Our sandboxing relies on properties of managed languages like Java or C\#: 
\begin{enumerate}
\item We can identify all instructions that may update shared memory and end the read phase before they occur. 

\item The ability to capture all exits from a block via the try-finally mechanism ensures that we never exit the read phase without performing validation. 

\item The code is not self-modifying and hence the tracking and validation code is executed as intended. 

\item The speculative execution does not alter the references to the thread-local variables we introduce (readSet, lockedSet) since they are constant references to well-typed objects. 

Hence, our tracking and validation code executes correctly. 
\end{enumerate}

While recent work~\cite{DiceHKLM14} has shown that differing validation to the end of a transaction can be unsafe, this problem does not occur in our solution. The key problem shown there is that access to the object on which the conflict is checked (namely the lock) is deferred until after other unchecked shared accesses, which could potentially be inconsistent and cause the lock not to be accessed. In our case, on the other hand, all accesses to shared data are recorded for validation purposes. If an object that should be accessed (like the lock in the lock elision case) is not accessed because of earlier conflicts, these earlier conflicts will be detected and the transaction will abort. 

%\paragraph{Early Phase Branching}
Note that, using our transformation, the shared state at the end of the validation phase
is identical to the state that would have been reached had the code been executed pessimistically from
the outset. Hence, the three-phase version of the code is compatible with the instrumented
pessimistic version. This means that if the optimistic phase is unsuccessful for any reason, we can always
fall back on the pessimistic version. Moreover, we can switch from optimistic to pessimistic synchronization
\emph{at any point} during the read phase.
We use this property in two ways.
%, as we now describe.
First, we avoid livelocks by limiting the number of restarts due to conflicts:
The validation phase tracks the number of restarts in a thread-local variable.
If this number exceeds a certain threshold, we perform the entire operation optimistically.

Second, this property offers the optimistic implementation the liberty of
failing spuriously, even in the absence of conflicts, because it can always fall back on the safe pessimistic version
of the code.
One can take advantage of this liberty, and implement the \emph{readSet} using a constant size array.
%Our implementation takes advantage of this liberty, and uses constant size arrays for the \emph{lockedSet} and \emph{readSet}.
In case the array becomes full,  the optimistic version cannot proceed, but there is no 
need to start the operation anew.
Instead, one can immediately perform the validation phase, which, if successful, switches to a pessimistic modus operandi, after having acquired all the needed locks.

\newcommand{\op}{\emph{\textsc{op}}}
\newcommand{\opt}{\textsc{opt}}

\section{Analysis}
\label{sec:proof}

We show  three properties of our automatic transformation. First, we prove that the  transformation is correct, i.e., 
every execution of the  synthesized code is equivalent to some execution of the \emph{locking algorithm}, that is, the
sequential code instrumented with the locking protocol.
This implies that if the locking protocol ensures serializability, then so does our protocol.
Second, we show that the equivalent execution preserves the real-time order of the original one, 
which implies that linearizability is also invariant under the transformation.
Finally, we argue that our transformation preserves deadlock-freedom.
In this section, we provide informal correctness arguments. A formal proof is deferred to Appendix~\ref{sec:formal-proof}.

\paragraph{Transformation Correctness}

Let $\pi$ be a finite execution of the transformed algorithm. We will show an equivalent execution of the locking algorithm.
Let \op\ be the set of operations in $\pi$.
First, we project object versions out of $\pi$'s configurations, and remove all accesses (reads and writes) to object versions.
That is, we replace steps that access versions with local steps that modify the operation's local memory only.
(Note that we get an execution with exactly the same invocations, responses, local states, and shared object states, but without 
versions). 
Second, note that each operation $op \in$ \op\ executes at most one successful read-only
phase, namely a complete read-only phase followed by a successful validation phase.
For each such $op$, we remove the prefix of $op$ that precedes the successful read-only phase.
This includes completely removing operations that have no successful read-only phase.
We call the resulting execution $\pi'$.

We next iteratively perturb $\pi'$ to construct an equivalent execution of the locking protocol.
We order the operations in \op\ according to the order in which the first steps of their respective (successful) read set validation phases occur in $\pi'$.
Denote the sequence of these first validations steps $e_1, e_2, \ldots, e_k$, and the respective operations
$op_1, op_2, \ldots, op_k$ in this order. Let $\pi_0 = \pi'$.
In each iteration $i \geq 1$, we construct $\pi_i$ by replacing the execution of $op_i$
in $\pi_{i-1}$ with an execution of $op_i$ that follows the locking protocol. To do this, we first move all steps of $op_i$ that precede 
its first validation step, $e_i$, to occur immediately before $e_i$ (we know that the read steps return the same values, since validation 
succeeds). We then add steps that lock these objects before their read steps (and unlock
them) as dictated by the locking protocol.
%We then add steps that lock all of them immediately before these moved read
% steps. Finally, we add unlock steps immediately before $e_i$ for objects that are in the read set but not in the lock set during $e_i$, 
Finally we remove the tryLock and validation steps by $op_i$. 

Our ability to lock each object in $op_i$'s read set follows from the following observation:
\begin{observation}
Between the first time in which $op_i$ first reads an object $o$ in its read-only phase, and until the first step of $op_i$'s validation phase, no
 thread locks $o$. 
\end{observation}
The observation follows immediately from the fact that $op_i$'s read validation is successful and that every lock step increases the
respective object's version number. 

It is easy to see that $\pi_{i-1}$ and $\pi_i$ are equivalent. By repeating this for all operations in \op, we get an execution $\pi_k$ of the locking 
protocol.

\paragraph{Real-Time Order}
It is easy to see that $\pi_k$ preserves the real-time order of $\pi$, since it does not change the order of invoke or return steps. 

\paragraph{Progress}
The read and validation phases of our instrumented code do not use blocking locks -- the read-phase does not use locks at all, whereas the 
validation phase uses tryLocks. Therefore, both phases are non-blocking. In principle, the optimistic approach may lead to livelocks, but 
our algorithm fall-back on the pessimistic approach following a bounded number of restarts, and hence cannot livelock. We get that any lack
of progress must be due to blocking in the locking algorithm. Since the section of the code executing the locking protocol is unchanged, and
since we ensure that it begins when holding the same locks as in the original protocol, we get that our transformation does not introduce any
source of spurious blocking that is not present in the original locking protocol. 




\section{Evaluation}
\label{sec:eval}

We evaluate the performance of our approach on two types of data
structures. In Section~\ref{sec:readwrite} we consider search trees
supporting insert, delete, and get operations, whereas Section~\ref{sec:range}
focuses on data structures that, in addition, support range queries that retrieve all
keys within a given range. We compare the performance of our approach in terms of
throughput to fully pessimistic solutions applying fine-grain locking. These
algorithms also serve as the lock-based reference implementation at the base of
our semi-optimistic implementations.
We further
compare our approach to software transactional memory and hand-crafted state-of-the-art data structure
implementations supporting the same functionality.

We also measured the performance of global lock-based implementations.
In all workloads, the results were identical or inferior to those
achieved by pessimistic fine-grain locking. We hence
omitted these results to avoid obscuring the presentation.

We use the micro-benchmark suite \textit{Synchrobench}~\cite{Gramoli2015}, configured as described below. 
%We follow a standard evaluation methodology
%(\cite{DrachslerVY2014,NatarajanM2014,BrownER2014,ArbelA2014}). 
Each experiment
consists of $5$ trials. A trial is a five second run in which each thread continuously executes
randomly chosen operations drawn from the workload distribution, with keys
selected uniformly at random from the range $[0,2\cdot10^6]$.
Each trial begins by initiating a new data structure with
%and applying an untimed pre-filling phase, which continues until the size of the data structure is within 5\% of
$10^6$ records. The presented results are the average throughput over all trials.
%(with the pre-filling phase excluded).

%% Eshcar: discuss small trees? the results are not the same
%We also experimented with a smaller range ($[0,2\cdot10^4]$) to test different
%contention levels; since the results showed similar trends, they are omitted here.

All implementations are written in Java. We ran the experiments on a dedicated machine with
four Intel Xeon E5-4650 processors, each with $8$ cores, for a total of $32$ threads
(with hyper-threading disabled).
We used Ubuntu 12.04.4 LTS and Java Runtime Environment (build
1.7.0\_51-b13) using the 64-Bit Server VM (build 24.51-b03, mixed mode).

\subsection{Insert-Delete-Get Operations}
\label{sec:readwrite}

We start by benchmarking a search-tree supporting the basic insert,
delete, and get (lookup) operations. Our experiments evaluate unbalanced as well
as balanced trees.

We employ textbook sequential implementations of an unbalanced binary
tree, and a treap~\cite{AragonS1989}. To
generate pessimistic lock-based implementations, we synthesize
concurrent code by applying the domination locking technique to the sequential
data structures. The resulting algorithms are denoted \domTree and \domTreap.
Finally, we manually apply our lock-removal transformation to the reference
implementations to get our semi-optimistic versions of the code, which we call
\autoTree and \autoTreap, respectively.
We also applied Deuce~\cite{Deuce2010}, which is a Java implementation of the TL2 algorithm~\cite{DiceSS2006} to the sequential implementations. The resulting algorithms are denoted \stmTree and \stmTreap.

We further compare our implementations to their hand-crafted state-of-the-art counterparts\footnote{Unless described otherwise implementations are provided by Synchrobench}. We compare \autoTree to
\begin{description}
\setlength{\itemsep}{0pt}
\setlength{\parskip}{0pt}
\item[\danaTree] The locked-based unbalanced tree of Drachsler et al.~\cite{DrachslerVY2014}\footnote{Implementation provided by the authors.}.
\item [\lockfreeTree] The lock-free unbalanced tree of Ellen et al.~\cite{EllenFRB2010}.
\end{description}
\autoTreap is evaluated against three hand-crafted implementations
\begin{description}
\setlength{\itemsep}{0pt}
\setlength{\parskip}{0pt}
\item[\danaAVL] The locked-based relaxed balanced AVL tree of
				Drachsler et al.~\cite{DrachslerVY2014}.%\footnote{Implementation available at \\\texttt{https://github.com/logicalordering/trees}}.
\item[\bronson] The locked based relaxed balanced AVL tree
				of Bronson et al.~\cite{BronsonCCO2010}.%\footnote{Implementation available at \\texttt{https://github.com/nbronson/snaptree}}.
\item[\friendly] The Contention-Friendly Tree of Crain et al.~\cite{CrainGR2013}.
%\item[\skiplist] The non-blocking skip-list in the the Java standard library; based on the work of Fraser and Harris~\cite{fraser2004practical}.
\end{description}

We evaluate performance in three representative workloads distributions: a
\emph{read-only} workload comprised of $100\%$ lookup operations, a \emph{write-dominated}
workload consisting of insert and delete operations ($50\%$ each), and a
\emph{mixed workload} with $50\%$ lookups, $25\%$ inserts, and $25\%$
deletes.

\paragraph{Results}
Figure~\ref{evaluation:results:unbalanced}
shows the throughput of unbalanced data structures and Figure~\ref{evaluation:results:balanced} shows
the throughput of the balanced ones. We see that our semi-optimistic
solution is far superior to the previous, fully-pessimistic,
automated approach; it successfully overcomes the bottlenecks associated with lock contention
in the \domTree and \domTreap implementations. 

For both balanced and unbalanced trees, our approach outperforms STM with 2x to 3x throughput.
This might be due to the tradeoff between validation overhead vs not concerning with internal consistency.
To ensure opacity~\cite{GuerraouiK2008}, each time an object is read in STM, the transaction checks that the version of the lock protecting the object is valid. 
Our algorithms read the version instead of locking the object which happens less frequently, typically once during the operation.
This is where sandboxing pays off -- we allow operations to observe inconsistent views and hence have improved performance.


%In many scenarios, 
Our solution comes close to custom-tailored implementations.
The results for the read-only workload show the main overhead
of our automatic approach. By profiling the code, we learned
that the bulk of this overhead stems from the need to track all read objects,
which is inherent to our transformation.
This is in contrast with the hand-crafted implementations,
which have small overhead on reads in this scenario, thanks to either
wait-free reads (in \danaAVL, \danaTree, \friendly and \lockfreeTree ), or optimistic validation (in \bronson).

As the ratio of updates in the workload increases, our automatic implementation
closes this gap.
In other words, the transformed code deals well with update contention.
This might be due to the fact that once
an update phase begins, the operation is not delayed due to concurrent
read-only operations.


\subsection{Range Queries}
\label{sec:range}

Next we evaluate the performance of our approach when the data
structure supports a more intricate functionality like range queries.
\Xomit{
Hand
crafting an implementation of a data structure that supports atomic
(linearizable) range queries is challenging.
The implementations that do support iterating through records may impose an
additional overhead on the regular read and write operations to ensure
progress of range queries.
The results in this section demonstrate that our method
allows generating a correct and efficient code, which is otherwise difficult
to obtain.
}
We use a skip list, which readily supports range queries by
nature of its linked-structure. The core of the implementation is the key lookup
method; once reaching the key, a record can be added or be removed in place, and
an iteration of subsequent records can be executed by traversing
the bottom-level linked-list.

The domination locking scheme cannot be efficiently applied to the skip list
structure since it is a DAG rather than a tree. Instead, we
manually devise a pessimistic locking protocol. Our
algorithm, (inspired by the one in~\cite{HerlihyS2008}), applies
hand-over-hand locking at each level, so that at the end of the search, the
operation holds locks on two records in each level, which define the minimal interval within this level
containing the lookup key (or the first lookup key in the case of a range query). Upon
reaching the bottom level, unnecessary locks are released, as follows: update operations only keep
locks on nodes they intend to modify, whereas
%that will be modified; while holding these locks any
%modification can be executed in isolation from other update operations.
range
queries keep the locks in the level with the minimal interval spanning the
range. Range queries then continue to use hand-over-hand locking to traverse through all records
within the range. The use of hand-over-hand locking ensures that range queries are atomic
(linearizable), i.e., return a consistent view of the data structure.

This pessimistic lock-based algorithm is denoted \domSkiplist.
As in previous data structures, we  apply the lock-removal transformation to the
reference implementation to get a semi-optimistic algorithm, which we call
\autoSkiplist.

We also applied Deuce to the skip-list sequential implementation. The resulting algorithm is denoted \stmSkiplist.

Our approach is also compared to the aforementioned
state-of-the-art data structures that support range queries. Specifically,
we compare \autoSkiplist to 
\begin{description}
\setlength{\itemsep}{0pt}
\setlength{\parskip}{0pt}
\item[\skiplist] The non-blocking Java skip-list which supports \emph{non}-linarizable range queries through iterators.
\item[\kary] A linearizable, non-blocking $k$-ary search tree
that supports range queries~\cite{BrownA12}\footnote{\url{http://www.cs.toronto.edu/~tabrown/kstrq}}.
%\item[\bronson] provides atomic range queries by traversing a clone of the
%original tree that is lazily generated
\end{description}
To ensure a fair comparison (following~\cite{BrownA12}) the range query operation in all implementations return an array of keys.
For \skiplist this means projecting a subset of the keys, iterating over them, and then copying each of these keys into an array. This does not include a snapshot, so range queries are
not always linearizable.

Like many data structure libraries, \friendly and \danaAVL
do not support atomic range queries, and
there is no straightforward way to add them.

The evaluation of range queries focuses on a mixed workload, where half the
threads are dedicated to performing range queries, and the other half perform a
mix of insert and delete operations (50\% each).
This mix allows us to evaluate both the performance of the range queries,
and their impact on concurrent updates.
%\eshcar{what about a workload that includes only 100\% range
%queries?}
We experiment with queries with large ranges varying between $1000$ to $2000$ keys
(Figure~\ref{evaluation:results:skiplist1000}) and small ranges
between $10$ to $20$
keys (Figure~\ref{evaluation:results:skiplist}).


% \begin{figure*}
% \begin{center}
% 
\begin{tikzpicture}
\begin{axis}[mystyle,skiplist,
 ylabel = { million ops/sec},
 %title={\textbf{0\% read-only}}]
 ]
\addplot [black,mark=square*] table [x={threads}, y={TimeoutSkiplist}]
{results/range.txt}; 
\addplot [darkspringgreen,mark=triangle] table [x={threads}, y={Bronson}]
{results/range.txt}; 
\addplot [coralred,mark=asterisk] table [x={threads}, y={JavaSkipList}]
{results/range.txt};
\addplot [blues4,mark=x] table [x={threads}, y={DominationLockingSkiplist}]
{results/range.txt};
\end{axis}
\end{tikzpicture}

%\ref{skiplistLegened}



% \end{center}
% \caption{Throughput of skip-list operations.}
% \label{evaluation:results:range}
% \end{figure*}


\begin{figure*}
\begin{center}
\begin{tikzpicture}
\begin{axis}[mystyle,unbalanced,
title={\textbf{write-dominated}},
ylabel = { million ops/sec}]
\addplot [black,mark=square*] table [x={threads}, y={LockRemovalTree}]
{results/trees-i1000000-u100.log}; 
\addplot [orange,mark=diamond] table [x={threads}, y={LogicalOrderingTree}]
{results/trees-i1000000-u100.log}; 
\addplot [blues5,mark=10-pointed star] table [x={threads}, y={NonBlockingTorontoBSTMap}]
{results/trees-i1000000-u100.log}; 
\addplot [magenta,mark=+] table [x={threads}, y={STMBinaryTree}]
{results/trees-i1000000-u100.log};
\addplot [blues4,mark=x] table [x={threads}, y={DominationLockingTree}]
{results/trees-i1000000-u100.log};
\end{axis}
\end{tikzpicture}
\begin{tikzpicture}
\begin{axis}[mystyle,unbalanced,title={\textbf{mixed workload}}]
\addplot [black,mark=square*] table [x={threads}, y={LockRemovalTree}]
{results/trees-i1000000-u50.log}; 
\addplot [orange,mark=diamond] table [x={threads}, y={LogicalOrderingTree}]
{results/trees-i1000000-u50.log}; 
\addplot [blues5,mark=10-pointed star] table [x={threads}, y={NonBlockingTorontoBSTMap}]
{results/trees-i1000000-u50.log}; 
\addplot [magenta,mark=+] table [x={threads}, y={STMBinaryTree}]
{results/trees-i1000000-u50.log};
\addplot [blues4,mark=x] table [x={threads}, y={DominationLockingTree}]
{results/trees-i1000000-u50.log};
\end{axis}
\end{tikzpicture}
\begin{tikzpicture}
\begin{axis}[mystyle,unbalanced,
title= { \textbf{read-only}}]
\addplot [black,mark=square*] table [x={threads}, y={LockRemovalTree}]
{results/trees-i1000000-u0.log};
\addplot [orange,mark=diamond] table [x={threads}, y={LogicalOrderingTree}]
{results/trees-i1000000-u0.log};  
\addplot [blues5,mark=10-pointed star] table [x={threads}, y={NonBlockingTorontoBSTMap}]
{results/trees-i1000000-u0.log}; 
\addplot [magenta,mark=+] table [x={threads}, y={STMBinaryTree}]
{results/trees-i1000000-u0.log};
\addplot [blues4,mark=x] table [x={threads}, y={DominationLockingTree}]
{results/trees-i1000000-u0.log};
\end{axis}
\end{tikzpicture}


\ref{unbalancedLegened}
\end{center}
\caption{Throughput of unbalanced data structures.}
\label{evaluation:results:unbalanced}
\end{figure*}

\Xomit{
\begin{figure*}
\begin{center}
\begin{tikzpicture}
\begin{axis}[mystyle,unbalanced,
title={\textbf{write-dominated}},
ylabel = { million ops/sec}]
\addplot [black,mark=square*] table [x={threads}, y={LockRemovalTree}]
{results/trees-i1000000-u100.log}; 
\addplot [orange,mark=diamond] table [x={threads}, y={LogicalOrderingTree}]
{results/trees-i10000-u100.log}; 
\addplot [darkspringgreen,mark=triangle] table [x={threads}, y={NonBlockingTorontoBSTMap}]
{results/trees-i10000-u100.log}; 
\addplot [magenta,mark=+] table [x={threads}, y={STMBinaryTree}]
{results/trees-i10000-u100.log};
\addplot [blues4,mark=x] table [x={threads}, y={DominationLockingTree}]
{results/trees-i10000-u100.log};
\end{axis}
\end{tikzpicture}
\begin{tikzpicture}
\begin{axis}[mystyle,unbalanced,title={\textbf{mixed workload}}]
\addplot [black,mark=square*] table [x={threads}, y={LockRemovalTree}]
{results/trees-i10000-u50.log}; 
\addplot [orange,mark=diamond] table [x={threads}, y={LogicalOrderingTree}]
{results/trees-i10000-u50.log}; 
\addplot [darkspringgreen,mark=triangle] table [x={threads}, y={NonBlockingTorontoBSTMap}]
{results/trees-i10000-u50.log}; 
\addplot [magenta,mark=+] table [x={threads}, y={STMBinaryTree}]
{results/trees-i10000-u50.log};
\addplot [blues4,mark=x] table [x={threads}, y={DominationLockingTree}]
{results/trees-i10000-u50.log};
\end{axis}
\end{tikzpicture}
\begin{tikzpicture}
\begin{axis}[mystyle,unbalanced,
title= { \textbf{read-only}}]
\addplot [black,mark=square*] table [x={threads}, y={LockRemovalTree}]
{results/trees-i10000-u0.log};
\addplot [orange,mark=diamond] table [x={threads}, y={LogicalOrderingTree}]
{results/trees-i10000-u0.log};  
\addplot [darkspringgreen,mark=triangle] table [x={threads}, y={NonBlockingTorontoBSTMap}]
{results/trees-i10000-u0.log}; 
\addplot [magenta,mark=+] table [x={threads}, y={STMBinaryTree}]
{results/trees-i10000-u0.log};
\addplot [blues4,mark=x] table [x={threads}, y={DominationLockingTree}]
{results/trees-i10000-u0.log};
\end{axis}
\end{tikzpicture}


\ref{unbalancedLegened}
\end{center}
\caption{Throughput of unbalanced data structures. Small Tree. }
\label{evaluation:results:unbalanced}
\end{figure*}
}

\begin{figure*}
\begin{center}

\begin{tikzpicture}
\begin{axis}[mystyle,balanced,
 ylabel = { million ops/sec},
 title={\textbf{write-dominated}}]
\addplot [black,mark=square*] table [x={threads}, y={LockRemovalTreap}]
{results/trees-i1000000-u100.log}; 
\addplot [orange,mark=diamond] table [x={threads}, y={LogicalOrderingAVL}]
{results/trees-i1000000-u100.log}; 
\addplot [darkspringgreen,mark=triangle] table [x={threads}, y={LockBasedStanfordTreeMap}]
{results/trees-i1000000-u100.log}; 
\addplot [coralred,mark=asterisk] table [x={threads}, y={LockBasedFriendlyTreeMap}]
{results/trees-i1000000-u100.log};
\addplot [magenta,mark=+] table [x={threads}, y={STMTreap}]
{results/trees-i1000000-u100.log};
\addplot [blues4,mark=x] table [x={threads}, y={DominationLockingTreap}]
{results/trees-i1000000-u100.log};
\addplot [black,mark=square] table [x={threads}, y={OptLockRemovalTreap}]
{results/trees-i1000000-u100.log}; 
\end{axis}
\end{tikzpicture}
\begin{tikzpicture}
\begin{axis}[mystyle,balanced,title={\textbf{mixed workload}}]
\addplot [black,mark=square*] table [x={threads}, y={LockRemovalTreap}]
{results/trees-i1000000-u50.log}; 
\addplot [orange,mark=diamond] table [x={threads}, y={LogicalOrderingAVL}]
{results/trees-i1000000-u50.log}; 
\addplot [darkspringgreen,mark=triangle] table [x={threads}, y={LockBasedStanfordTreeMap}]
{results/trees-i1000000-u50.log}; 
\addplot [coralred,mark=asterisk] table [x={threads}, y={LockBasedFriendlyTreeMap}]
{results/trees-i1000000-u50.log};
\addplot [magenta,mark=+] table [x={threads}, y={STMTreap}]
{results/trees-i1000000-u50.log};
\addplot [blues4,mark=x] table [x={threads}, y={DominationLockingTreap}]
{results/trees-i1000000-u50.log};
\addplot [black,mark=square] table [x={threads}, y={OptLockRemovalTreap}]
{results/trees-i1000000-u50.log}; 
\end{axis}
\end{tikzpicture}
\begin{tikzpicture}
\begin{axis}[mystyle, 
				 title={\textbf{read-only}},]
\addplot [black,mark=square*] table [x={threads}, y={LockRemovalTreap}]
{results/trees-i1000000-u0.log}; 
\addplot [orange,mark=diamond] table [x={threads}, y={LogicalOrderingAVL}]
{results/trees-i1000000-u0.log}; 
\addplot [darkspringgreen,mark=triangle] table [x={threads}, y={LockBasedStanfordTreeMap}]
{results/trees-i1000000-u0.log}; 
\addplot [coralred,mark=asterisk] table [x={threads}, y={LockBasedFriendlyTreeMap}]
{results/trees-i1000000-u0.log};
\addplot [magenta,mark=+] table [x={threads}, y={STMTreap}]
{results/trees-i1000000-u0.log};
\addplot [blues4,mark=x] table [x={threads}, y={DominationLockingTreap}]
{results/trees-i1000000-u0.log};
\end{axis}
\end{tikzpicture}


\ref{balancedLegened}



\end{center}
\caption{Throughput of balanced data
structures.}
\label{evaluation:results:balanced}
\end{figure*}

\Xomit{
\begin{figure*}
\begin{center}

\begin{tikzpicture}
\begin{axis}[mystyle,balanced,
 ylabel = { million ops/sec},
 title={\textbf{write-dominated}}]
\addplot [black,mark=square*] table [x={threads}, y={LockRemovalTreap}]
{results/trees-i10000-u100.log}; 
\addplot [orange,mark=diamond] table [x={threads}, y={LogicalOrderingAVL}]
{results/trees-i10000-u100.log}; 
\addplot [darkspringgreen,mark=triangle] table [x={threads}, y={LockBasedStanfordTreeMap}]
{results/trees-i10000-u100.log}; 
\addplot [coralred,mark=asterisk] table [x={threads}, y={LockBasedFriendlyTreeMap}]
{results/trees-i10000-u100.log};
\addplot [magenta,mark=+] table [x={threads}, y={STMTreap}]
{results/trees-i10000-u100.log};
\addplot [blues4,mark=x] table [x={threads}, y={DominationLockingTreap}]
{results/trees-i10000-u100.log};
\end{axis}
\end{tikzpicture}
\begin{tikzpicture}
\begin{axis}[mystyle,balanced,title={\textbf{mixed workload}}]
\addplot [black,mark=square*] table [x={threads}, y={LockRemovalTreap}]
{results/trees-i1000000-u50.log}; 
\addplot [orange,mark=diamond] table [x={threads}, y={LogicalOrderingAVL}]
{results/trees-i10000-u50.log}; 
\addplot [darkspringgreen,mark=triangle] table [x={threads}, y={LockBasedStanfordTreeMap}]
{results/trees-i10000-u50.log}; 
\addplot [coralred,mark=asterisk] table [x={threads}, y={LockBasedFriendlyTreeMap}]
{results/trees-i10000-u50.log};
\addplot [magenta,mark=+] table [x={threads}, y={STMTreap}]
{results/trees-i10000-u50.log};
\addplot [blues4,mark=x] table [x={threads}, y={DominationLockingTreap}]
{results/trees-i10000-u50.log};
\end{axis}
\end{tikzpicture}
\begin{tikzpicture}
\begin{axis}[mystyle,balanced, 
				 title={\textbf{read-only}},]
\addplot [black,mark=square*] table [x={threads}, y={LockRemovalTreap}]
{results/trees-i10000-u0.log}; 
\addplot [orange,mark=diamond] table [x={threads}, y={LogicalOrderingAVL}]
{results/trees-i10000-u0.log}; 
\addplot [darkspringgreen,mark=triangle] table [x={threads}, y={LockBasedStanfordTreeMap}]
{results/trees-i10000-u0.log}; 
\addplot [coralred,mark=asterisk] table [x={threads}, y={LockBasedFriendlyTreeMap}]
{results/trees-i10000-u0.log};
\addplot [magenta,mark=+] table [x={threads}, y={STMTreap}]
{results/trees-i10000-u0.log};
\addplot [blues4,mark=x] table [x={threads}, y={DominationLockingTreap}]
{results/trees-i10000-u0.log};
\end{axis}
\end{tikzpicture}


\ref{balancedLegened}



\end{center}
\caption{Throughput of balanced data
structures. Small Tree.}
\label{evaluation:results:balanced}
\end{figure*}
}

\begin{figure*}
	\begin{center}
	\begin{subfigure}[t]{.35\textwidth}
		\caption{Range queries}
		
\begin{tikzpicture}
\begin{axis}[mystyle,skiplist1000,
 ylabel = { million ops/sec},
 %title={\textbf{0\% read-only}}]
 ]
\addplot [black,mark=square*] table [x={threads}, y={TimeoutSkiplist}]
{results/range1000.txt}; 
\addplot [darkspringgreen,mark=triangle] table [x={threads}, y={Bronson}]
{results/range1000.txt}; 
\addplot [coralred,mark=asterisk] table [x={threads}, y={JavaSkipList}]
{results/range1000.txt};
\addplot [blues4,mark=x] table [x={threads}, y={DominationLockingSkiplist}]
{results/range1000.txt};
\end{axis}
\end{tikzpicture}

%\ref{skiplistLegened}



		\label{evaluation:results:range1000}
	\end{subfigure}
	\quad\quad
	\begin{subfigure}[t]{.35\textwidth}
		\caption{Insert and delete operations}
		
\begin{tikzpicture}
\begin{axis}[mystyle,skiplist,
 ylabel = { million ops/sec},
 %title={\textbf{0\% read-only}}]
 ]
\addplot [black,mark=square*] table [x={threads}, y={TimeoutSkiplist}]
{results/update1000.txt}; 
\addplot [darkspringgreen,mark=triangle] table [x={threads}, y={Bronson}]
{results/update1000.txt}; 
\addplot [coralred,mark=asterisk] table [x={threads}, y={JavaSkipList}]
{results/update1000.txt};
\addplot [blues4,mark=x] table [x={threads}, y={DominationLockingSkiplist}]
{results/update1000.txt};
\end{axis}
\end{tikzpicture}

%\ref{skiplistLegened}



		\label{evaluation:results:update1000}
	\end{subfigure}
	\ref{skiplistLegened1000}
	\end{center}
\caption{Half the threads execute large range queries $[1000,2000]$
and half the threads execute insert and delete operations.}
\label{evaluation:results:skiplist1000}
\end{figure*}


\begin{figure*}
	\begin{center}
	\begin{subfigure}[t]{.35\textwidth}
		\caption{Range queries}
		
\begin{tikzpicture}
\begin{axis}[mystyle,skiplist,
 ylabel = { million ops/sec},
 %title={\textbf{0\% read-only}}]
 ]
\addplot [black,mark=square*] table [x={threads}, y={TimeoutSkiplist}]
{results/range.txt}; 
\addplot [darkspringgreen,mark=triangle] table [x={threads}, y={Bronson}]
{results/range.txt}; 
\addplot [coralred,mark=asterisk] table [x={threads}, y={JavaSkipList}]
{results/range.txt};
\addplot [blues4,mark=x] table [x={threads}, y={DominationLockingSkiplist}]
{results/range.txt};
\end{axis}
\end{tikzpicture}

%\ref{skiplistLegened}



		\label{evaluation:results:range}
	\end{subfigure}
	\quad\quad
	\begin{subfigure}[t]{.35\textwidth}
		\caption{Insert and delete operations}
		
\begin{tikzpicture}
\begin{axis}[mystyle,skiplist,
% ylabel = { million ops/sec},
 %title={\textbf{0\% read-only}}]
 ]
\addplot [black,mark=square*] table [x={threads}, y={LockRemovalSkipList}]
{results/update1.txt}; 
\addplot [violet,mark=o] table [x={threads}, y={LockFreeKSTRQ}]
{results/update1.txt}; 
\addplot [teal,mark=pentagon] table [x={threads}, y={LockFreeJavaSkipList}]
{results/update1.txt};
\addplot [magenta,mark=+] table [x={threads}, y={STMSkipList}]
{results/update1.txt};
\addplot [blues4,mark=x] table [x={threads}, y={DominationLockingSkipList}]
{results/update1.txt};
\end{axis}
\end{tikzpicture}

%\ref{skiplistLegened}



		\label{evaluation:results:update}
	\end{subfigure}
	\ref{skiplistLegened}
	\end{center}
\caption{Half the threads execute small range queries$[10,20]$
and half the threads execute insert and delete operations.}
\label{evaluation:results:skiplist}
\end{figure*}

\paragraph{Results}
We measure the throughput of range queries and update operations separately.
The overall number of range queries executed per second is reported
in Figures~\ref{evaluation:results:range1000}
and~\ref{evaluation:results:range}, and the overall number of update (insert
and delete) operations executed per second is reported in
Figures~\ref{evaluation:results:update1000} and~\ref{evaluation:results:update}.

Again, we see that our transformed code outperforms
  to the fully-pessimistic fine-grain one. We believe that,
as in the
domination locking versions of the tree data structures, holding a lock on the head sentinel of the skip list in
\domSkiplist, even for short periods, imposes a major performance penalty,
which is eliminated by our  semi-optimistic approach.

It is superior also to the STM implementation
improving throughput by
up to three orders of magnitude -- 500x for large range queries, 2x for small
ranges, and 10x for update operations, regardless of the range size. The improvement in update operations can be attributed to lack of contention on a centralized object like the global version in \stmSkiplist. \eshcar{explain difference in range queries}


The simplicity of the non-linearizable implementation of \skiplist allows it to perform better than \autoSkiplist on range queries.  However, the performance of
\autoSkiplist is almost identical to that of \skiplist --and even better in some cases, when comparing the update operations.
%despite the fact that \skiplist performs inconsistent iterations.

\Xomit{
Figure~\ref{evaluation:results:range1000}
shows that, for large ranges, \bronson scales well up to $8$ threads, outperforming
all other implementations, but at $32$ threads its performance deteriorates.
This might be because this implementation is
optimized for full scans or very large range queries running sequentially.
The overhead of initiating a clone of the data structure per range query, which
involves waiting for all
pending update operations to complete, hampers scalability when
a number of queries are executed in parallel. This effect is even more pronounced when the
ranges are small (Figure~\ref{evaluation:results:range}), and the throughput of
\bronson flattens out.

In addition, the lazy cloning required to support
range queries imposes an overhead on  update operations. Copying each node
during downward traversal might be the main impediment preventing these
operations from scaling when the ranges are small (Figure~\ref{evaluation:results:update}) as well as when
they are large (Figure~\ref{evaluation:results:update1000}). In contrast,
the update operations in \autoSkiplist and \skiplist do not take any special
measures for the benefit of concurrent queries (which validate themselves in
\autoSkiplist, and are not atomic in \skiplist), and hence continue to perform well in
their presence.
}

\section{Related Work}\label{sec:related}
\paragraph{Concurrent Data Structures}
Many sophisticated concurrent data structures (e.g., \cite{ArbelA2014,DrachslerVY2014,NatarajanM2014,BrownER2014,CrainGR2013,BraginskyP2012,
AfekKKMT2012,EllenFRB2010,BronsonCCO2010,HerlihyLLS2007,Michael:1996})
were developed and used in concurrent software systems~\cite{Ohad:OOPSLA11}.
Implementing efficient synchronization for such data structures is considered a challenging and error-prone task~\cite{Ohad:OOPSLA11,Doh:SPAA04,Jin:2012}.
As a result, concurrent data structures are manually implemented by concurrency experts.
This paper shows that (in some cases) an automatic algorithm can produce synchronization which is comparable to synchronization implemented by experts.

\paragraph{Locking Protocols}
Locking protocols are used in databases and shared memory systems to guarantee correctness
of concurrently executing transactions~\cite{Weikum:2001,BHG:Book87}.
Our approach can be seen as a way to extend many existing locking protocols by combining them with an optimistic concurrency control.
In particular, our approach extends the following locking protocols:
two-phase~\cite{Eswaran:1976}, tree locking~\cite{SilberschatzK1980}, DAG locking~\cite{CH:PODS95} and domination locking~\cite{Gueta2011}.
We demonstrate this by showing that extending the  domination locking protocol enables producing efficient concurrency control for
dynamic data structures.


\paragraph{Lock Inference Algorithms}
There has been a lot of work on automatically inferring locks for transactions.
Most of the algorithms in the literature infer locks for following the two-phase
locking protocol~\cite{MZGB:POPL06,Emmi06POPL,gudka2012lock,CCG:PLDI08,HFP:TRANSACT06,CGE:CC08}.
Our approach can potentially be used to optimized the synchronization produced by these algorithms.
For example, for the algorithms that employ a two-phase variant in which all locks are acquired at the beginning of transactions (e.g.,~\cite{gudka2012lock,CCG:PLDI08}),
our approach can be used to defer the locking (e.g., to just before the first write operation) and to eliminate some of the locking operations.


\paragraph{Transactional Memory}
Transactional memory approaches (TMs) dynamically resolve inconsistencies
and deadlocks by rolling back partially completed transactions.
%
Unfortunately, in spite of a lot of effort and many TM implementations (see~\cite{HLR:SLCA2010}), existing TMs
have not been widely adopted due to various concerns~\cite{DuffyTM2010,Cascaval:2008,mckenneyParallel}, including high runtime overhead,
poor performance and limited ability to handle irreversible operations.
In particular, modern concurrent programs (and concurrent data structures) are typically based on hand-crafted synchronization, rather than  on a TM approach~\cite{Ohad:OOPSLA11}.

In a sense, our approach can be seen as a specialized TM approach that can be practically used to handle concurrent data structure.


%\paragraph{Lock Elision for Read-Only Transactions}
\paragraph{Lock Elision}
Our approach is inspired by the idea of \emph{sequential locks}~\cite{mckenneyParallel} and the approach presented in~\cite{Nakaike:2010}.
But  in contrast to the approaches in \cite{mckenneyParallel,Nakaike:2010} which are designed to handle read-only transactions,
our approach handles read-only prefixes of transactions that update the shared memory. 
Moreover, using these approaches for a highly-contended data structure (as in Section~\ref{sec:eval}) will provide limited performance, 
because each update transaction causes all the read-only transactions to abort.

There are some transactional memory techniques to elide locks from arbitrary critical sections (e.g.,~\cite{Rajwar:2002:TLE:635508.605399,Roy:2009:RSS:1519065.1519094,Afek:2014:SHL:2611462.2611482}).
In these techniques a transaction executes the critical section speculatively without acquiring the lock.
When a transaction is aborted, it can acquire the lock and execute the critical section non-speculatively.
In contrast to our approach, these techniques cannot combine speculative and non-speculative execution of the same transaction.







\section{Discussion}\label{sec:discussion}

%We have made the case that automatic synchronization can be a viable approach for producing scalable concurrent algorithms from %legacy sequential code.
The development of scalable concurrent programs today
heavily relies on custom-tailored implementations, which require painstaking correctness proofs.
In this paper, we have shown a relatively simple  transformation that can facilitate this labor-intensive process, and
thus make scalable synchronization more readily available.
The input for our transformation is a conventional lock-based concurrent program, which may be either constructed manually or
synthesized from sequential code. Our source-to-source transformation then makes judicious use of optimism in order to
eliminate principal concurrency bottlenecks in the given program and improve its scalability. 

We have illustrated our method for a number of data structures -- unbalanced and balanced search trees,  as well as
skip lists supporting range queries. 
In all cases, the transformed code performed significantly better than the original
lock-based one. It also scaled comparably  to hand-crafted
implementations that took considerably more effort to produce.
Moreover, extending the synthesized code with new functionalities
such as range queries was immediate. 
In these examples, we have manually applied our transformation (according to the algorithm presented in Section~\ref{sec:algorithm}). An interesting direction for future work would be to create a tool that automatically applies our transformation at compile time.

Our approach makes use of a common pattern in data structures, where an operation typically begins with a long read-only traversal, followed by a handful of (usually local) modifications.
A promising direction for future work  is to try and
exploit similar patterns in order to parallelize or remove locks in other types of code (not data structures), for example, programs that rely on two-phase locking.
Furthermore, for programs that follow different patterns, other combinations of optimism and pessimism may prove effective.

Finally, there still remains a gap between the performance achievable by manually optimized solutions and what we could achieve automatically. Our algorithm induces inherent overhead for tracking all operations in the read-only phase for later verification.
In specific data structures, these checks might be redundant, but it is difficult  to detect this automatically. We believe that
it may well be possible to enhance  transformations such as ours with computer-assisted optimizations. For example, a programmer may provide hints regarding certain
invariants that are always preserved in the code, in order to eliminate the need for tracking some values for later
validation. Such optimizations have the potential to bridge the remaining performance gap, while requiring far less work
for proving correctness -- instead of proving that the entire construction is correct, the developer would only need to
prove that her program maintains the specific invariants used.

%\section{Appendix Title}

%This is the text of the appendix, if you need one.

%\acks

\bibliography{myRef}
\bibliographystyle{abbrv}
%

\appendix

\section{Formal Correctness Proof}\label{sec:formal-proof}

We now formalize the correctness arguments made in Section~\ref{sec:proof}. 
First we define our model and the correctness properties of the algorithm for
which we provide the proof.

\paragraph{Model}

We consider an asynchronous shared memory model, where independent threads
interact via shared memory objects. 
Every thread executes a sequence of operations, each of which is invoked with certain parameters and returns a response.
An operation's execution consists of a sequence of primitive \emph{steps}, beginning with an \emph{invoke} step, followed by
atomic accesses to shared objects, and ending with a \emph{return} step. Steps also modify the executing thread's local variables.

A \emph{configuration} is an assignment of values to all shared and local variables. Thus, each step takes the system from one
configuration to another. Steps are deterministically defined by the data structure's protocol and the current configuration.
In the \emph{initial configuration}, each variable holds its initial value.

An \emph{execution} is an alternating sequence of configurations and steps,
$C_0,s_1,C_1, \ldots,s_i,C_i,\ldots,$
where $C_0$ is an initial configuration,
and each configuration $C_i$ is the result of
executing step $s_i$ on configuration $C_{i-1}$.
We only consider finite executions in this paper.
An execution is \emph{sequential} if steps of different operations are not interleaved.
In other words, a sequential execution is a sequence of operation executions.

Two executions are \emph{indistinguishable} to a set of operations if each
operation in the set executes the same steps on shared objects, and
receives the same value from those objects, in both executions. A step $\tau$
by operation $op$ is \emph{invisible} to all other operations 
if the executions with and without $\tau$ are indistinguishable to
$\op\setminus \{op\}$. For example, read steps are invisible.

\paragraph{Correctness}

The correctness of a data structure is defined in terms of its external behavior, as reflected in values returned by invoked operations.
Correctness of a code transformation is proven by showing that the synthesized code's executions are equivalent to ones of the original code,
where two executions are  \emph{equivalent} if every thread invokes the same
operations in the same order  in both executions, and gets the same result for each operation. More formally, we say in this paper that a code transformation is \emph{correct} if every execution of the transformed code
is equivalent to some execution of the original code.

The widely-used correctness criterion of serializability relies on equivalence to sequential executions in order to
link a data structure's behavior under concurrency to its sequentially specified behavior. Since equivalence is transitive,
we get that any code transformation satisfying our correctness notion, when applied to serializable code, yields code that is also serializable.
If the code transformation further ensures the real-time order of operations (i.e., operations that do not overlap appear in the same order in 
executions of the transformed and original code), then linearizability (atomicity) is also invariant under the transformation.
Another important aspect of correctness is preserving the progress conditions of the original code, for example, deadlock-freedom.

In this paper, we are not concerned with internal consistency (as required e.g., by opacity~\cite{GuerraouiK2008} or the validity notion of~\cite{LevAriCK2014}),
which restricts the configurations an operation might see during its execution.
This is because our code transformation uses timeouts and exception handlers to overcome unexpected behavior that may arise when a thread sees an inconsistent view of global variables (similar to~\cite{Nakaike:2010}).


\paragraph{Formal Proof}
We consider a finite execution $\pi$ of the transformed
code, and find an equivalent execution of the original lock-based
code.
Each operation in $\pi$ is an interleaved sequence of read-only phases and validation phases followed by a (single) update phase -- or a prefix of such pattern.
For each operation in $\pi$ we consider its \emph{successful validation}, i.e.,
the last (successful) execution of a validation phase when switching from the read-only
phase to the update phase. Each operation executes at most one successful 
validation. The (complete) read-only phase preceding the successful
validation phase is called \emph{successful read-only
phase} .
Note that each operation executes at most one successful read-only
phase.
Towards proving equivalence to the original code execution, for each operation $op$, we remove the prefix of $op$ that precedes the successful read-only phase.
This includes completely removing operations that have no successful read-only phase.
We call the resulting execution $\hat{\pi}$.
%Finally, we remove all validation phases executed during the successful
%read-only phase that are not the unique successful  validation phase of
%the operation.
These prefixes include read steps as well as tryLock and
unlock steps.
Removing read steps is invisible to other processes. Since we remove all tryLock steps that have failed to acquire locks, 
all remaining tryLock are successful also in $\hat{\pi}$. In addition, since the operation discards all local (private) state when restarting the read-only phase, $\pi$ and $\hat{\pi}$ are indistiguishable to all operations that have completed in $\pi$.
\begin{claim}
\label{claim:pipihat}
$\pi$ and $\hat{\pi}$ are equivalent.
\end{claim}

Denote by $e_1, e_2, \ldots, e_k$ the sequence of the first steps of
the read set validation in the execution of successful validation
phases, by their order in $\hat{\pi}$, where $e_i$ is a step of the operation $op_{i}$ executed by process $p_{i}$.
(Possibly $p_i=p_j$ for $j \neq i$).

For every operation $op_{i}$, consider the partition of $\hat{\pi}$ to
the following intervals $\hat{\pi}=\alpha_i\beta_i\gamma_i$, such that
$\alpha_i$ includes the execution interval of $op_{i}$'s (successful) read-only phase
(denote $op_{i}$'s read set $rs_{i}$); $\beta_i=\beta_{i_1}\beta_{i_2}$, is the
minimal execution interval of $op_{i}$'s successful validation phase;
in $\beta_{i_1}$, $op_{i}$ acquires 
locks on its lock set, denoted $ls_{i}$; 
$e_i$ is the first step of $\beta_{i_2}$, namely the read set validation
interval.

Let \op\ be the set of operations in $\hat{\pi}$.
The next claim follows from the fact that the validation phase of $op_{i}$
in $\beta_i$ is successful, and includes locks and versions
re-validation: 
%\eshcar{need to prove these? or are these clear from the alg description?}

\begin{claim}
\label{claim:locks}
No operation in $\op\setminus\{op_{i}\}$ holds a lock in
$\alpha_i\beta_{i_1}$ that is associated with an object $obj$ in $rs_{i}$ after $op_{i}$'s first
read of $obj$ in $\alpha_i$.
\end{claim}


We next project
object versions out of $\hat{\pi}$'s configurations, and remove all accesses (reads and writes) to object versions.
That is, we replace steps that access versions with local steps that modify the operation's local memory only.
Note that we get an execution with exactly the same invocations, responses, local states, and shared object states, but without 
versions. 
We call the resulting execution $\pi'$.

%Essentially, $\pi'$ is a projection of $\pi$
%excluding versions and all prefixes of the operations preceding their (single)
%successful read-only phase. 
%%, and all failed validation attempts.
%Therefore, all operations that returned a value in $\pi'$ return the same values as in $\pi$:
\begin{claim}
\label{claim:pihatpitag}
$\pi'$ and $\hat{\pi}$ are equivalent.
\end{claim}

Our main lemma constructs the execution of a fully-pessimistic locking code. 
The core idea is to replace the optimistic read-only phase
and validation phase of each operation with a solo execution of the
pessimistic lock-based read phase taking
place at the point where all objects in the lock set are locked.
\begin{lemma}
\label{lemma:pitagtag}
There is an execution of lock-based algorithm that is equivalent to $\pi'$.
\end{lemma}
\begin{proof}
We start with the execution $\pi_0=\pi'$.
For every $i \geq 0$, we show how to perturb $\pi_i$ to
obtain an execution $\pi_{i+1}$. For each $j\geq i+1$, let
  $\beta_{j}^{'}=\beta_{j_1}^{'}\beta_{j_2}^{'}$ be the minimal interval
  containing $op_{j}$'s validation phase in $\pi_{i+1}$, where $\beta_{j_1}^{'}$
  is the minimal interval containing $op_{j}$'s tryLock phase.
  Denote the configuration between $\beta_{j_1}^{'}$ and $\beta_{j_2}^{'}$
  $C_{j}$. In $\pi_{i+1}$ the following conditions are
satisfied:
\begin{enumerate}
  \item \label{cond:lp} The operations $op_{1},\ldots,op_{i}$ follow the
  fully-pessimistic locking algorithm, while the rest of the operations
  $\opt_{i+1}=\op\setminus\{op_{1},\ldots,op_{i}\}$ proceed according to our
  semi-optimistic algorithm.
  \item \label{cond:locks} 
  For $j\geq i+1$, no operation in $\op\setminus\{op_{j}\}$ holds a lock in
  $C_{j}$ that is associated with an object $obj$ in $rs_{j}$.
  \item \label{cond:writes} 
  For $j\geq i+1$, no operation in
  $\op\setminus\{op_{j}\}$ writes to an object $obj$ in $rs_{j}$ after
  $op_{j}$ first read $obj$ before $C_{j}$.
  \item \label{cond:trylocks} 
  For $j\geq i+1$, all try-lock steps by $op_j$ are invisible to
  $\op\setminus\{op_{j}\}$.
  %after $op_{j}$'s last read $obj$ before $\beta_j^{'}$
  \item \label{cond:equiv} $\pi'$ and $\pi_{i+1}$ are equivalent.
\end{enumerate}

For $\opt_{k+1}=\emptyset$, we get an execution where all operations follow the
pessimistic locking algorithm, and by Condition~\ref{cond:equiv} $\pi_{k+1}$ is
equivalent to $\pi'$ and we are done.

The proof is by induction on $i$. For the base case we consider
the execution $\pi'=\pi_0$. Condition~\ref{cond:lp} holds since none of
the operations in this execution follow the full locking algorithm.
Conditions~\ref{cond:locks} and~\ref{cond:writes} hold by
Claim~\ref{claim:locks}, 
and since accesses to objects (other than versions) are similar in $\hat{\pi}$ and
$\pi'$. Condition~\ref{cond:trylocks} holds since by construction, in $\pi'$ every step accessing an object, either
for locking it or for validating it is not locked, finds the object not locked.
.
Condition~\ref{cond:equiv} vacuously holds since $\pi'$ and $\pi_0$ are the
same execution.

For the induction step, assume $\opt_i \neq \emptyset$ and
the execution
$\pi_i=\alpha_i^{'}\beta_{i_1}^{'}\beta_{i_2}^{'}\gamma_i^{'}$ satisfies
the above conditions.
We replace $\pi_i$ with
$\pi_{i+1}=\alpha_i^{''}\beta_{i_1}^{''}\delta_i\beta_{i_2}^{''}\gamma_i^{'}$,
such that $\alpha_i^{''}$, $\beta_{i_1}^{''}$, and $\beta_{i_2}^{''}$ are the
projection of $\alpha_i^{'}$, $\beta_{i_1}^{'}$ and $\beta_{i_2}^{'}$, excluding
the steps by $op_{i}$, while $\delta_i$ is a $p_{i}$-only execution
interval in which $p_{i}$ follows the locking algorithm while
reading $rs_{i}$; after $\delta_{i}$, $p_{i}$ holds the locks on all
objects in $ls_{i}$, and holds no lock on other objects. 
In other words, we replace the optimistic read-only phase and validation phase
of $op_{i}$ with an execution of the original
locking algorithm, taking place at $C_{j}$.
%at the point just before the read set validation starts.

By Condition~\ref{cond:locks} of the induction hypothesis no operation holds 
locks associated with objects in the read set of $op_{i}$ in $C_{i}$, therefore,
$p_{i}$ can acquire the locks on these objects while executing $\delta_{i}$.
By Condition~\ref{cond:writes} of the induction hypothesis no
operation writes to an object $obj$ in $rs_{i}$ after
$op_{i}$ first read $obj$ before $C_{i}$, hence $op_{i}$ reads the same
values in its read set in $\pi_i$ and $\pi_{i+1}$. After $\delta_{i}$,
$op_{i}$ holds the locks on all objects in $ls_i$, hence it can continue with
the execution of the locking algorithm.

In $\alpha_i^{''}\beta_{i_1}^{''}$ we only removed read steps and tryLock steps
by $op_{i}$ that are invisible to all other operations, by
Condition~\ref{cond:trylocks}. Therefore, the executions
$\alpha_i^{'}\beta_{i_1}^{'}$, ending with configuration $C'$, 
and $\alpha_i^{''}\beta_{i_1}^{''}\delta_i$, ending with configuration $C''$, 
are indistinguishable to all operations in $\op\setminus\{op_{i}\}$. 
In addition, in $\beta_{i_2}^{''}$ we only removed invisible read steps.
The values of all shared objects and locks are the same in $C'$ and $C''$,
hence the executions $\alpha_i^{'}\beta_{i_1}^{'}\beta_{i_2}^{'}\gamma_i^{'}$
and $\alpha_i^{''}\beta_{i_1}^{''}\delta_i\beta_{i_2}^{''}\gamma_i^{'}$ are
indistinguishable to all operations in $\op\setminus\{op_{i}\}$. 

The indistinguishability and the induction hypothesis imply that Conditions~\ref{cond:locks},~\ref{cond:writes},~\ref{cond:trylocks} hold.
In addition, this implies that (1)~the projection of the execution $\pi_{i+1}$ on $op_{i}$
follows the full pessimistic locking algorithm satisfying Condition~\ref{cond:lp}, and
(2)~all operations return the same value in $\pi_{i+1}$ as in $\pi'$, which
means Condition~\ref{cond:equiv} holds.

%It is left to show that $\pi_{i+1}$ satisfies
%Conditions~\ref{cond:locks},~\ref{cond:writes},~\ref{cond:trylocks}.
%This is straightforward from the induction hypothesis and the fact that only
%$op_{i}$ changed its excution in the last iteration, and specifically
%removed all its try-lock steps, and since $\delta_i$ precedes $C_{j}$ for all
%$j\geq i+1$ in $\pi_{i+1}$.
\end{proof}

By Lemma~\ref{lemma:pitagtag}, Claim~\ref{claim:pipihat} and Claim~\ref{claim:pihatpitag} we conclude the following
theorem:
\begin{theorem}
Every execution of the transformed code is equivalent to an
execution of the original locking code.
\end{theorem}

%\section{Additional Results}
\label{sec:appendix:results}

\begin{figure*}
\begin{center}
\begin{tikzpicture}
\begin{axis}[mystyle,unbalanced,
title={\textbf{write-dominated}},
ylabel = { million ops/sec}]
\addplot [black,mark=square*] table [x={threads}, y={LockRemovalTree}]
{results/trees-i1000000-u100.log}; 
\addplot [orange,mark=diamond] table [x={threads}, y={LogicalOrderingTree}]
{results/trees-i10000-u100.log}; 
\addplot [darkspringgreen,mark=triangle] table [x={threads}, y={NonBlockingTorontoBSTMap}]
{results/trees-i10000-u100.log}; 
\addplot [magenta,mark=+] table [x={threads}, y={STMBinaryTree}]
{results/trees-i10000-u100.log};
\addplot [blues4,mark=x] table [x={threads}, y={DominationLockingTree}]
{results/trees-i10000-u100.log};
\end{axis}
\end{tikzpicture}
\begin{tikzpicture}
\begin{axis}[mystyle,unbalanced,title={\textbf{mixed workload}}]
\addplot [black,mark=square*] table [x={threads}, y={LockRemovalTree}]
{results/trees-i10000-u50.log}; 
\addplot [orange,mark=diamond] table [x={threads}, y={LogicalOrderingTree}]
{results/trees-i10000-u50.log}; 
\addplot [darkspringgreen,mark=triangle] table [x={threads}, y={NonBlockingTorontoBSTMap}]
{results/trees-i10000-u50.log}; 
\addplot [magenta,mark=+] table [x={threads}, y={STMBinaryTree}]
{results/trees-i10000-u50.log};
\addplot [blues4,mark=x] table [x={threads}, y={DominationLockingTree}]
{results/trees-i10000-u50.log};
\end{axis}
\end{tikzpicture}
\begin{tikzpicture}
\begin{axis}[mystyle,unbalanced,
title= { \textbf{read-only}}]
\addplot [black,mark=square*] table [x={threads}, y={LockRemovalTree}]
{results/trees-i10000-u0.log};
\addplot [orange,mark=diamond] table [x={threads}, y={LogicalOrderingTree}]
{results/trees-i10000-u0.log};  
\addplot [darkspringgreen,mark=triangle] table [x={threads}, y={NonBlockingTorontoBSTMap}]
{results/trees-i10000-u0.log}; 
\addplot [magenta,mark=+] table [x={threads}, y={STMBinaryTree}]
{results/trees-i10000-u0.log};
\addplot [blues4,mark=x] table [x={threads}, y={DominationLockingTree}]
{results/trees-i10000-u0.log};
\end{axis}
\end{tikzpicture}


\ref{unbalancedLegened}
\end{center}
\caption{Throughput of unbalanced data structures. Small Tree. }
\label{evaluation:results:unbalancedsmall}
\end{figure*}

\begin{figure*}
\begin{center}

\begin{tikzpicture}
\begin{axis}[mystyle,balanced,
 ylabel = { million ops/sec},
 title={\textbf{write-dominated}}]
\addplot [black,mark=square*] table [x={threads}, y={LockRemovalTreap}]
{results/trees-i10000-u100.log}; 
\addplot [orange,mark=diamond] table [x={threads}, y={LogicalOrderingAVL}]
{results/trees-i10000-u100.log}; 
\addplot [darkspringgreen,mark=triangle] table [x={threads}, y={LockBasedStanfordTreeMap}]
{results/trees-i10000-u100.log}; 
\addplot [coralred,mark=asterisk] table [x={threads}, y={LockBasedFriendlyTreeMap}]
{results/trees-i10000-u100.log};
\addplot [magenta,mark=+] table [x={threads}, y={STMTreap}]
{results/trees-i10000-u100.log};
\addplot [blues4,mark=x] table [x={threads}, y={DominationLockingTreap}]
{results/trees-i10000-u100.log};
\end{axis}
\end{tikzpicture}
\begin{tikzpicture}
\begin{axis}[mystyle,balanced,title={\textbf{mixed workload}}]
\addplot [black,mark=square*] table [x={threads}, y={LockRemovalTreap}]
{results/trees-i1000000-u50.log}; 
\addplot [orange,mark=diamond] table [x={threads}, y={LogicalOrderingAVL}]
{results/trees-i10000-u50.log}; 
\addplot [darkspringgreen,mark=triangle] table [x={threads}, y={LockBasedStanfordTreeMap}]
{results/trees-i10000-u50.log}; 
\addplot [coralred,mark=asterisk] table [x={threads}, y={LockBasedFriendlyTreeMap}]
{results/trees-i10000-u50.log};
\addplot [magenta,mark=+] table [x={threads}, y={STMTreap}]
{results/trees-i10000-u50.log};
\addplot [blues4,mark=x] table [x={threads}, y={DominationLockingTreap}]
{results/trees-i10000-u50.log};
\end{axis}
\end{tikzpicture}
\begin{tikzpicture}
\begin{axis}[mystyle,balanced, 
				 title={\textbf{read-only}},]
\addplot [black,mark=square*] table [x={threads}, y={LockRemovalTreap}]
{results/trees-i10000-u0.log}; 
\addplot [orange,mark=diamond] table [x={threads}, y={LogicalOrderingAVL}]
{results/trees-i10000-u0.log}; 
\addplot [darkspringgreen,mark=triangle] table [x={threads}, y={LockBasedStanfordTreeMap}]
{results/trees-i10000-u0.log}; 
\addplot [coralred,mark=asterisk] table [x={threads}, y={LockBasedFriendlyTreeMap}]
{results/trees-i10000-u0.log};
\addplot [magenta,mark=+] table [x={threads}, y={STMTreap}]
{results/trees-i10000-u0.log};
\addplot [blues4,mark=x] table [x={threads}, y={DominationLockingTreap}]
{results/trees-i10000-u0.log};
\end{axis}
\end{tikzpicture}


\ref{balancedLegened}



\end{center}
\caption{Throughput of balanced data
structures. Small Tree.}
\label{evaluation:results:balancedsmall}
\end{figure*}

\begin{figure*}
	\begin{center}
	\begin{subfigure}[t]{.35\textwidth}
		\caption{Range queries}
		
\begin{tikzpicture}
\begin{axis}[mystyle,skiplist1000,
 ylabel = { million ops/sec},
 %title={\textbf{0\% read-only}}]
 ]
\addplot [black,mark=square*] table [x={threads}, y={TimeoutSkiplist}]
{results/range1000.txt}; 
\addplot [darkspringgreen,mark=triangle] table [x={threads}, y={Bronson}]
{results/range1000.txt}; 
\addplot [coralred,mark=asterisk] table [x={threads}, y={JavaSkipList}]
{results/range1000.txt};
\addplot [blues4,mark=x] table [x={threads}, y={DominationLockingSkiplist}]
{results/range1000.txt};
\end{axis}
\end{tikzpicture}

%\ref{skiplistLegened}



		\label{evaluation:results:range1000}
	\end{subfigure}
	\quad\quad
	\begin{subfigure}[t]{.35\textwidth}
		\caption{Insert and delete operations}
		
\begin{tikzpicture}
\begin{axis}[mystyle,skiplist,
 ylabel = { million ops/sec},
 %title={\textbf{0\% read-only}}]
 ]
\addplot [black,mark=square*] table [x={threads}, y={TimeoutSkiplist}]
{results/update1000.txt}; 
\addplot [darkspringgreen,mark=triangle] table [x={threads}, y={Bronson}]
{results/update1000.txt}; 
\addplot [coralred,mark=asterisk] table [x={threads}, y={JavaSkipList}]
{results/update1000.txt};
\addplot [blues4,mark=x] table [x={threads}, y={DominationLockingSkiplist}]
{results/update1000.txt};
\end{axis}
\end{tikzpicture}

%\ref{skiplistLegened}



		\label{evaluation:results:update1000}
	\end{subfigure}
	\ref{skiplistLegened1000}
	\end{center}
\caption{Half the threads execute large range queries $[1000,2000]$
and half the threads execute insert and delete operations.}
\label{evaluation:results:skiplist1000}
\end{figure*}


\end{document}
