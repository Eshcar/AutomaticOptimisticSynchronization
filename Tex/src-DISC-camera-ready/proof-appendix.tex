
\section{Formal Correctness Proof}\label{sec:formal-proof}

We now formalize the correctness arguments made in Section~\ref{sec:proof}. 
First we define our model and the correctness properties of the algorithm for
which we provide the proof.

\paragraph{Model}

We consider an asynchronous shared memory model, where independent threads
interact via shared memory objects. 
Every thread executes a sequence of operations, each of which is invoked with certain parameters and returns a response.
An operation's execution consists of a sequence of primitive \emph{steps}, beginning with an \emph{invoke} step, followed by
atomic accesses to shared objects, and ending with a \emph{return} step. Steps also modify the executing thread's local variables.

A \emph{configuration} is an assignment of values to all shared and local variables. Thus, each step takes the system from one
configuration to another. Steps are deterministically defined by the data structure's protocol and the current configuration.
In the \emph{initial configuration}, each variable holds its initial value.

An \emph{execution} is an alternating sequence of configurations and steps,
$C_0,s_1,C_1, \ldots,s_i,C_i,\ldots,$
where $C_0$ is an initial configuration,
and each configuration $C_i$ is the result of
executing step $s_i$ on configuration $C_{i-1}$.
We only consider finite executions in this paper.
An execution is \emph{sequential} if steps of different operations are not interleaved.
In other words, a sequential execution is a sequence of operation executions.

Two executions are \emph{indistinguishable} to a set of operations if each
operation in the set executes the same steps on shared objects, and
gets the same value from those objects, in both executions. A step $\tau$
by operation $op$ is \emph{invisible} to all other operations 
if the executions with and without $\tau$ are indistinguishable to
$\op\setminus \{op\}$. For example, read steps are invisible.

\paragraph{Correctness}

The correctness of a data structure is defined in terms of its external behavior, as reflected in values returned by invoked operations.
Correctness of a code transformation is proven by showing that the synthesized code's executions are equivalent to ones of the original code,
where two executions are  \emph{equivalent} if when considering operations that have completed every thread invokes the same
operations in the same order  in both executions, and gets the same result for each operation. More formally, we say in this paper that a code transformation is \emph{correct} if every execution of the transformed code
is equivalent to some execution of the original code.

The widely-used correctness criterion of serializability relies on equivalence to sequential executions in order to
link a data structure's behavior under concurrency to its sequentially specified behavior. Since equivalence is transitive,
we get that any code transformation satisfying our correctness notion, when applied to serializable code, yields code that is also serializable.
If the code transformation further ensures the real-time order of operations (i.e., operations that do not overlap appear in the same order in 
executions of the transformed and original code), then linearizability (atomicity) is also invariant under the transformation.
Another important aspect of correctness is preserving the progress conditions of the original code, for example, deadlock-freedom.

In this paper, we are not concerned with internal consistency (as required e.g., by opacity~\cite{GuerraouiK2008} or the validity notion of~\cite{LevAriCK2014}),
which restricts the configurations an operation might see during its execution.
This is because our code transformation uses timeouts and exception handlers to overcome unexpected behavior that may arise when a thread sees an inconsistent view of global variables (similar to~\cite{Nakaike:2010}).


\paragraph{Formal Proof}
We consider a finite execution $\pi$ of the transformed
code, and find an equivalent execution of the original lock-based
code.
Each operation in $\pi$ is an interleaved sequence of read-only and validation phases followed by a (single) update phase, or a prefix of such pattern.
For each operation in $\pi$ we consider its \emph{successful validation}, i.e.,
the last (successful) execution of a validation phase before switching to the update phase. Each operation executes at most one successful 
validation. The read-only phase preceding the successful
validation phase is called \emph{successful read-only
phase} .
Each operation executes at most one successful read-only
phase.
Towards proving equivalence to the original code execution, for each operation $op$, we remove the prefix of $op$ that precedes the successful read-only phase.
This includes completely removing operations that have no successful read-only phase.
We call the resulting execution $\hat{\pi}$.
%Finally, we remove all validation phases executed during the successful
%read-only phase that are not the unique successful  validation phase of
%the operation.
The removed prefixes include read steps as well as tryLock and
unlock steps.
Removing read steps is invisible to other processes. Since we only remove steps that acquire locks all remaining locking steps in $\hat{\pi}$ have the same affect and get the same response (success or failure) as in $\pi$. Finally, since the operation discards all local (private) state when restarting a read-only phase, $\pi$ and $\hat{\pi}$ are indistiguishable to all operations that have completed in $\pi$.
\begin{claim}
\label{claim:pipihat}
$\pi$ and $\hat{\pi}$ are equivalent.
\end{claim}

Denote by $e_1, e_2, \ldots, e_k$ the sequence of the first steps of
the read set validation in the execution of successful validation
phases, by their order in $\hat{\pi}$, where $e_i$ is a step of the operation $op_{i}$ executed by process $p_{i}$.
(Possibly $p_i=p_j$ for $j \neq i$).

Let \op\ be the set of operations in $\hat{\pi}$.
For every operation $op_{i}$ in \op, consider the partition of $\hat{\pi}$ to
the following intervals $\hat{\pi}=\alpha_i\beta_i\gamma_i$, such that
$\alpha_i$ includes the execution interval of $op_{i}$'s (successful) read-only phase
(denote $op_{i}$'s read set $rs_{i}$); $\beta_i=\beta_{i_1}\beta_{i_2}$, is the
minimal execution interval of $op_{i}$'s successful validation phase;
in $\beta_{i_1}$, $op_{i}$ acquires 
locks on its lock set, denoted $ls_{i}$; 
$e_i$ is the first step of $\beta_{i_2}$, namely the read set validation
interval.

%The next claim follows from the fact that the validation phase of $op_{i}$
%in $\beta_i$ is successful, and includes locks and versions
%re-validation: 
%\eshcar{need to prove these? or are these clear from the alg description?}

\begin{claim}
\label{claim:locks}
No operation in $\op\setminus\{op_{i}\}$ holds a lock in
$\alpha_i\beta_{i_1}$ that is associated with an object $obj$ in $rs_{i}$ after $op_{i}$'s first
read of $obj$ in $\alpha_i$.
\end{claim}
\begin{proof}
Let $lck$ be the lock associated with $obj$. Before reading $obj$ the first time in $\alpha_i$ $op_i$ records the version number of $lck$ (line~\ref{code:track:getVersion} in function \emph{track}) and checks that $lck$ is not held by any other thread (line~\ref{code:track:verifyUnlocked} in function \emph{track}).

We assume the function \emph{isLockedByAnother} imposes a memory fence and that at the beginning of the validation phase there is a read fence. If another thread acquired $lck$ after the first read and did not release it, this is discovered during validation (line~\ref{code:validate:verifyUnlocked} in function \emph{validateReadSet}). If another thread acquired $lck$ after the first read--and therefore after reading the version the first time---and did release the lock, then this thread increased the version number of $lck$ before releasing it. The fencing guarantees that $op_i$ observes the version number has changed (line~\ref{code:validate:verifyVersion} in function \emph{validateReadSet}). Since the validation phase of $op_i$ is successful no operation other than $op_i$ holds $lck$ in
$\alpha_i\beta_{i_1}$.
\end{proof}

We next project
object versions out of $\hat{\pi}$'s configurations, and remove all accesses (reads and writes) to object versions.
That is, we replace steps that access versions with local steps that modify the operation's local memory only.
Note that we get an execution with exactly the same invocations, responses, local states, and shared object states, but without 
versions. 
We call the resulting execution $\pi'$.

%Essentially, $\pi'$ is a projection of $\pi$
%excluding versions and all prefixes of the operations preceding their (single)
%successful read-only phase. 
%%, and all failed validation attempts.
%Therefore, all operations that returned a value in $\pi'$ return the same values as in $\pi$:
\begin{claim}
\label{claim:pihatpitag}
$\pi'$ and $\hat{\pi}$ are equivalent.
\end{claim}

Our main lemma constructs the execution of a fully-pessimistic locking code. 
The core idea is to replace the optimistic read-only phase
and validation phase of each operation with a solo execution of the
pessimistic lock-based read phase taking
place at the point where all objects in the lock set are locked.
\begin{lemma}
\label{lemma:pitagtag}
There is an execution of lock-based algorithm that is equivalent to $\pi'$.
\end{lemma}
\begin{proof}
We start with the execution $\pi_0=\pi'$.
For every $i \geq 0$, we show how to perturb $\pi_i$ to
obtain an execution $\pi_{i+1}$. 
We consider the operations $\op=op_1, \ldots, \op_k$ as defined above by the steps $e_1, e_2, \ldots, e_k$ .
For an operation $op_j$ such that $j\geq i+1$ in $\pi_i$, let
  $\beta_{j}^{'}=\beta_{j_1}^{'}\beta_{j_2}^{'}$ be the minimal interval
  containing $op_{j}$'s validation phase, where $\beta_{j_1}^{'}$
  is the minimal interval containing $op_{j}$'s tryLock phase.
  Denote the configuration between $\beta_{j_1}^{'}$ and $\beta_{j_2}^{'}$
  $C_{j}$. In $\pi_{i}$ the following conditions are
satisfied:
\begin{enumerate}
  \item \label{cond:lp} The operations $op_{1},\ldots,op_{i}$ execute the
  fully-pessimistic locking algorithm, while the rest of the operations
  $\opt_{i}=\op\setminus\{op_{1},\ldots,op_{i-1}\}$ execute our
  semi-optimistic algorithm.
  \item \label{cond:locks} 
  For $j\geq i+1$, no operation in $\op\setminus\{op_{j}\}$ holds a lock in
  $C_{j}$ that is associated with an object $obj$ in $rs_{j}$.
  \item \label{cond:writes} 
  For $j\geq i+1$, no operation in
  $\op\setminus\{op_{j}\}$ writes to an object $obj$ in $rs_{j}$ after
  $op_{j}$ first read $obj$ before $C_{j}$.
  \item \label{cond:trylocks} 
  For $j\geq i+1$, all try-lock steps by $op_j$ are invisible to
  $\op\setminus\{op_{j}\}$.
  %after $op_{j}$'s last read $obj$ before $\beta_j^{'}$
  \item \label{cond:equiv} $\pi'$ and $\pi_{i}$ are equivalent.
\end{enumerate}

For $\opt_{k}=\emptyset$, we get an execution where all operations execute the
pessimistic locking algorithm, and by Condition~\ref{cond:equiv} $\pi_{k+1}$ is
equivalent to $\pi'$ and we are done.

The proof is by induction on $i$. For the base case we consider
the execution $\pi_0$. Condition~\ref{cond:lp} holds since none of
the operations in this execution execute the full locking algorithm.
Conditions~\ref{cond:locks} and~\ref{cond:writes} hold by
Claim~\ref{claim:locks}, 
and since accesses to objects (other than versions) are similar in $\hat{\pi}$ and
$\pi_0$. Condition~\ref{cond:trylocks} holds since by construction, in $\pi_0$ every step accessing an object, either
for locking it or for validating it is not locked, finds the object not locked.
Condition~\ref{cond:equiv} vacuously holds since $\pi'$ and $\pi_0$ are the
same execution.

For the induction step, assume $\opt_i \neq \emptyset$ and
the execution
$\pi_i$ satisfies
the above conditions.
We consider $op_i \in \opt_i$ which partitions $\pi_i$ to $\alpha_i^{'}\beta_{i_1}^{'}\beta_{i_2}^{'}\gamma_i^{'}$. We replace $\pi_i$ with
$\pi_{i+1}=\alpha_i^{''}\beta_{i_1}^{''}\delta_i\beta_{i_2}^{''}\gamma_i^{'}$,
such that $\alpha_i^{''}$, $\beta_{i_1}^{''}$, and $\beta_{i_2}^{''}$ are the
projection of $\alpha_i^{'}$, $\beta_{i_1}^{'}$ and $\beta_{i_2}^{'}$, excluding
the steps by $op_{i}$, while $\delta_i$ is a $p_{i}$-only execution
interval in which $p_{i}$ follows the locking algorithm while
reading $rs_{i}$; after $\delta_{i}$, $p_{i}$ holds the locks on all
objects in $ls_{i}$, and holds no lock on other objects. 
In other words, we replace the optimistic read-only phase and validation phase
of $op_{i}$ with an execution of the original
locking algorithm, taking place at $C_{j}$.
%at the point just before the read set validation starts.

By Condition~\ref{cond:locks} of the induction hypothesis no operation holds 
locks associated with objects in the read set of $op_{i}$ in $C_{i}$, therefore,
$p_{i}$ can acquire the locks on these objects while executing $\delta_{i}$.
By Condition~\ref{cond:writes} of the induction hypothesis no
operation writes to an object $obj$ in $rs_{i}$ after
$op_{i}$ first read $obj$ before $C_{i}$, hence $op_{i}$ reads the same
values in its read set in $\pi_i$ and $\pi_{i+1}$. After $\delta_{i}$,
$op_{i}$ holds the locks on all objects in $ls_i$, hence it can continue with
the execution of the locking algorithm.
This implies that the projection of the execution $\pi_{i+1}$ on $op_{i}$
follows the full pessimistic locking algorithm satisfying Condition~\ref{cond:lp}.

In $\alpha_i^{''}\beta_{i_1}^{''}$ we only removed read steps and tryLock steps
by $op_{i}$ that are invisible to all other operations, by
Condition~\ref{cond:trylocks}. Therefore, the executions
$\alpha_i^{'}\beta_{i_1}^{'}$, ending with configuration $C'$, 
and $\alpha_i^{''}\beta_{i_1}^{''}\delta_i$, ending with configuration $C''$, 
are indistinguishable to all operations in $\op\setminus\{op_{i}\}$. 
In addition, in $\beta_{i_2}^{''}$ we only removed invisible read steps.
The values of all shared objects and locks are the same in $C'$ and $C''$,
hence the executions $\alpha_i^{'}\beta_{i_1}^{'}\beta_{i_2}^{'}\gamma_i^{'}$
and $\alpha_i^{''}\beta_{i_1}^{''}\delta_i\beta_{i_2}^{''}\gamma_i^{'}$ are
indistinguishable to all operations in $\op\setminus\{op_{i}\}$. 

The indistinguishability and the induction hypothesis imply that Conditions~\ref{cond:locks},~\ref{cond:writes},~\ref{cond:trylocks} hold.
In addition, this implies that all completed operations return the same value in $\pi_{i+1}$ and $\pi'$, which
means Condition~\ref{cond:equiv} holds.

%It is left to show that $\pi_{i+1}$ satisfies
%Conditions~\ref{cond:locks},~\ref{cond:writes},~\ref{cond:trylocks}.
%This is straightforward from the induction hypothesis and the fact that only
%$op_{i}$ changed its excution in the last iteration, and specifically
%removed all its try-lock steps, and since $\delta_i$ precedes $C_{j}$ for all
%$j\geq i+1$ in $\pi_{i+1}$.
\end{proof}

By Lemma~\ref{lemma:pitagtag}, Claim~\ref{claim:pipihat} and Claim~\ref{claim:pihatpitag} we conclude the following
theorem:
\begin{theorem}
Every execution of the transformed code is equivalent to an
execution of the original locking code.
\end{theorem}
