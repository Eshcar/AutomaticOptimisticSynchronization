\renewcommand{\ttdefault}{pcr}
%\algrenewcommand\textkeyword{\texttt}
\algrenewcommand\algorithmicif{\texttt{if}}
\algrenewcommand\algorithmicthen{\texttt{then}}
\algrenewcommand\algorithmicfunction{\textsc{Function}}
\algrenewcommand\algorithmicforall{\texttt{for all}}
\algrenewcommand\algorithmicdo{\texttt{do}}
\algrenewcommand\textproc{\textit}



\section{Automatic Transformation}\label{sec:algorithm}

We present an automatic source-to-source transformation, whose
goal is to optimize the code of a given data structure implemented using lock-based concurrency control.
In Section~\ref{ssec:locks}, we detail our assumptions about the given code and the locks it uses.

The transformation produces a combination of pessimistic and optimistic synchronization
by replacing part of the locking (in the given code) with optimistic concurrency control.
Section~\ref{ssec:overview} overviews our general approach to combining optimism and pessimism.

The synthesized code consists of three phases - an optimistic read-only phase, a validation phase,
and a pessimistic update phase. The transformation first partitions the given code into a read-only phase and
an update phase, then instruments each of these phases separately and adds the validation phase between the two.
Section~\ref{ssec:transformation} describes how the code of each of the three phases is produced, whereas
Section~\ref{ssec:extendedTran} explains how the code is partitioned into two phases.

\subsection{Lock-based Data Structures}\label{ssec:locks}

A \emph{data structure} defines a set of \emph{operations} that may be invoked by
clients of the data structure, potentially concurrently.
We assume that the code of each operation is represented by a separate \emph{control-flow graph (CFG)}~\cite{Nielson:1999}.

Operations have parameters and local variables, which are private to the invocation, and are thus thread-local.
%
The operations interact via \emph{shared memory variables}, which are also called \emph{shared objects}.
%
Each shared object supports atomic \emph{read} (load) and \emph{write} (store) instructions.
%
%Below, we extend shared objects to also support locks.

In addition, each shared object is associated with a unique lock, and supports
atomic \emph{lock}, \emph{tryLock}, \emph{unlock} and \emph{isLockedByAnotherThread} instructions.
Locks are exclusive (i.e., a lock can be held by at most one thread at a time).
The execution of a thread
trying to acquire a lock (by a \emph{lock} instruction) which is
held by another thread is blocked until a time when the
lock is available (i.e., is not held by any thread).
The other instructions never block.
The \emph{tryLock} instruction returns \emph{false} if the lock is currently held by another thread, otherwise it acquires the lock and returns \emph{true}.
The \emph{isLockedByAnotherThread} instruction returns \emph{true}, if and only if, the lock is currently held by another thread.

We assume that in the given code every (read
or write) access by an operation to a shared object is performed when the
executing thread holds the lock on that object.
We also assume that the locking in the given code only uses the \emph{lock} and \emph{unlock} instructions.
%(i.e., the other locking instructions are not used).

\subsection{Combining Optimism and Pessimism}\label{ssec:overview}

% Optimism
Generally speaking, optimistic concurrency control is a form of lock-free synchronization, which accesses shared variables without locks in the hope that they will not be modified by others before the end of the operation (or more generally, the transaction). To verify the latter, optimistic concurrency control relies on \emph{validation}, which is typically implemented using version numbers. If validation fails, the operation restarts. Optimistic execution of update operations requires either performing roll-back (reverting variables to their old values) upon validation failure, or deferring writes to commit time; both approaches induce significant overhead~\cite{Cascaval:2008}. We therefore refrain from speculative shared memory updates.



%The main idea:
The main idea behind our approach is judicious use of optimistic synchronization for reading
shared variables without locks, but only as long as the operation does not update shared state. Once an operation
writes to shared memory, we revert to pessimistic (lock-based) synchronization. In
other words, we rely on validation at the end of the read-only prefix of an operation in order to render redundant
locks that would have been acquired and freed before the first update.
This scheme is particularly suitable for data structures,
since the common behavior of their operations
is to first traverse the data structure, and then
perform modifications.

Conceptually, our approach divides an operation into three phases: an optimistic \emph{read-only phase},
a pessimistic \emph{update phase} and a \emph{validation phase} that conjoins them.
The read-only phase traverses the data structure without taking any locks, while maintaining sufficient information to later ensure the correctness of the traversal.
The update phase uses the original pessimistic (lock-based) synchronization.
The validation phase bridges between the optimistic and pessimistic ones.
It first locks the objects for which a lock would have been held at this point by
the original locking code, and then validates the correctness
of the read-only phase. This allows the
update phase to run as if an execution of the original pessimistic synchronization
took place. If the validation fails, the operation
restarts. In order to avoid livelock, we set a threshold on the number of restarts.
If the threshold is exceeded, the code falls back on pessimistic execution.
Note that it is safe to do so, since our semi-optimistic code is compatible
with the fully pessimistic one.


\subsection{Transforming the Code Phases}\label{ssec:transformation}

\newcommand{\spOne}{\hspace{-3mm}\ }
\newcommand{\spZero}{\hspace{-3mm}}
\begin{figure*}
\scriptsize
	\begin{center}
	\begin{subfigure}[t]{.45\textwidth}
		\begin{algorithmic}[1]{}
		{\ttfamily
			\Function{addThird}{List list, Node new} \label{code:begin}
			\Statex ----------------------------
			\State                               \label{code:beginRead}
            \State{\spOne}\textbf{list.lock()}
			\State{\spOne}Node prev = list.head
			\State{\spOne}\textbf{prev.lock()}
            \State{\spOne}\textbf{list.unlock()}
			\State{\spOne}Node succ = prev.next
			\State{\spOne}\textbf{succ.lock()}
			\State{\spOne}\textbf{prev.unlock()}
			\State{\spZero}prev = succ
			\State{\spZero}succ = succ.next
			\State{\spZero}\textbf{succ.lock()}  \label{code:endRead}
			\Statex ----------------------------
			\State                               \label{code:beginValidation}
			\State
			\State
			\State
			\State
			\State
            \State                               \label{code:endValidation}
			\Statex ----------------------------
			\State{\spZero}prev.next = new       \label{code:beginUpdate}
			\State{\spZero}\textbf{new.lock()}
			\State{\spZero}new.next = succ
            \State
			\State{\spZero}\textbf{prev.unlock()}
            \State
			\State{\spZero}\textbf{new.unlock()}
            \State
			\State{\spZero}\textbf{succ.unlock()}  \label{code:endUpdate}
			\EndFunction
			}
		\end{algorithmic}
		\caption{Code with original locking} \label{figure:transformation:before}
	\end{subfigure}
	\begin{subfigure}[t]{.45\textwidth}
		\begin{algorithmic}[1]{}
		{\ttfamily
			\Function{addThird}{List list, Node new} \label{code:begin}
			\Statex ----------------------------
			\Comment{\textrm{read-only phase}}
            \State{\spOne}\textbf{lockedSet.init(), readSet.init()} \label{code:initSets}
            \State{\spOne}\textbf{if !track(list)  then {goto} \ref{code:begin}} \label{code:readGhaseGoto0}
			\State{\spOne}Node prev = list.head
			\State{\spOne}\textbf{if !track(prev)  then {goto} \ref{code:begin}} \label{code:readGhaseGoto1}
            \State{\spOne}\textbf{lockedSet.remove(list)} \label{code:lockedSet:remove1}
			\State{\spOne}Node succ = prev.next
			\State{\spOne}\textbf{if !track(succ) then {goto} \ref{code:begin}}  \label{code:readGhaseGoto2}
			\State{\spOne}\textbf{lockedSet.remove(prev)} \label{code:lockedSet:remove2}
			\State{\spOne}prev = succ
			\State{\spOne}succ = succ.next
			\State{\spZero}\textbf{if !track(succ) then {goto} \ref{code:begin}} \label{code:readGhaseGoto3}
			\Statex ----------------------------
			\Comment{\textrm{validation phase}}
			\State{\spZero}\textbf{for all obj in lockedSet do} \label{code:validateLockedSet}	
            \State{\spZero}\ \ \textbf{if !obj.tryLock() then}
            \State{\spZero}\ \ \ \ \ \textbf{unlockAll()}
            \State{\spZero}\ \ \ \ \ \textbf{{goto} \ref{code:begin}} \label{code:validateGoto1}
			\State{\spZero}\textbf{if !validateReadSet() then} 		\label{code:validateReadSet}
				\State{\spZero}\ \ \textbf{unlockAll()}
				\State{\spZero}\ \ \textbf{{goto} \ref{code:begin}} \label{code:validateGoto2}
				%\Comment Restart Operation
			\Statex ----------------------------
			\Comment{\textrm{update phase}}
			\State{\spZero}prev.next = new
			\State{\spZero}\textbf{new.lock()}
			\State{\spZero}new.next = succ			
			\State{\spZero}\textbf{prev.version++}
			\State{\spZero}\textbf{prev.unlock()}
			\State{\spZero}\textbf{new.version++}
			\State{\spZero}\textbf{new.unlock()}
			\State{\spZero}\textbf{succ.version++}
			\State{\spZero}\textbf{succ.unlock()}

			\EndFunction
			}
		\end{algorithmic}
		\caption{The code produced by our automatic transformation}\label{figure:transformation:after}
	\end{subfigure}
	%\bigskip
	%\hline
	\end{center}
\vspace{-4mm}
	\caption{Code example.
	The synchronization code is in bold.
			\label{figure:transformation}}
\end{figure*}

In this section we assume that the code had already been partitioned into a read-phase and an update phase using
the transformation described in Section~\ref{ssec:extendedTran}, and
describe how we synthesize the code for each of the phases.
We illustrate the transformation for a simple code snippet that adds a new element as the third node in a linked list.
The original and transformed code are provided in Figure \ref{figure:transformation}. The latter uses
the tracking and validation functions in Figures \ref{figure::track} and
\ref{figure::validate}, resp.

\paragraph{Code Phases in the CFG}
In this section we assume that
control-flow-graph (CFG) of each operation has been partitioned (by the transformation in the next section)
into two subgraphs corresponding to two phases --
read-only phase $C_r$, and update phase $C_u$. The partitioning satisfies the following:
(1)~every execution of the operation starts in $C_r$ and ends in $C_u$; (2)~there is no edge from $C_u$ to  $C_r$;
and (3) $C_r$ does not contain instructions that write to shared objects\footnote{
Note that $C_r$ may contain \emph{lock} and \emph{unlock} instructions (even though their implementation may write to shared memory);
only \emph{non-locking write instructions} cannot appear in $C_r$.}.
%
In the example of Figure~\ref{figure:transformation:before}, $C_r$ is the code in lines \ref{code:beginRead}-\ref{code:endRead},
and $C_u$ is the code in lines \ref{code:beginUpdate}-\ref{code:endUpdate}.

\paragraph{Version Numbers}
Our transformation instruments each object $o$ with an additional field \emph{version}.
%Later we will show that
If $o$ is not locked, then this field  represents the current version number of $o$.
Version numbers are used to validate the correctness of the optimistic execution of the read-only phase.
Note that each object has its own version --- i.e., version numbers of different objects are independent of each other.

\paragraph{Read-only Phase}
In this phase, we replace all the lock and unlock instructions with synchronization that avoids writing to shared memory.
During this phase, our synchronization maintains two thread-local multi-sets: \emph{lockedSet} and \emph{readSet}.
The \emph{lockedSet} is used to track the objects that were supposed to be locked by the original synchronization.
Using a multi-set allows for reentrant locks \guy{TBD: at this point, it may be hard to understand why we need multi-set} .
The \emph{readSet} is used to track versions of all objects read by the
operation, in order to allow us to later validate that the operation has observed a consistent view of shared memory.

At the beginning of the read-only phase, we insert code that initializes \emph{lockedSet} and \emph{readSet} to be empty (see  line~\ref{code:initSets} of Figure~\ref{figure:transformation:after}).
We replace every lock and unlock instruction with the corresponding code in Table~\ref{Ta:readOnlyTransformation}.
A lock instruction on object $o$ is replaced with code that tracks the object and its version in
thread-local variables \emph{lockedSet} and \emph{readSet} (see Figure~\ref{figure::track}).
An unlock instruction on object $o$ is replaced with code that removes $o$ from \emph{lockedSet}.
An example for a transformed code is shown in lines \ref{code:beginRead}-\ref{code:endRead} of Figure~\ref{figure:transformation:after}.

In Figure~\ref{figure::track} we use an eager validation scheme. 
If the object already exists in \emph{readSet}, we check that its current version is equal
to the version in \emph{readSet}; if the versions are different  the operation restarts from the beginning (line~\ref{code:track:verifyVersion}).
Similarly, it is also checked to be unlocked; if it is locked, the operation restarts from the beginning (line~\ref{code:track:verifyUnlocked}).



\begin{table}
\scriptsize
\ttfamily
{\tt
\begin{center}
\begin{tabular}{|l|l|}
\hline
\textbf{Original Locking Instruction} & \textbf{Transformed Code}\\
\hline
\textit{x.lock()}&
\textit{if !track(x) then goto $S$}
\\
\hline
\textit{x.unlock()}&
\textit{lockedSet.remove(x)}
\\
\hline
\end{tabular}
\end{center}
}
\caption{Transformation for read-only phase:
each locking instruction (left side) is replaced with the corresponding code on the right;
 $S$  denotes the beginning of the operation.
}
\label{Ta:readOnlyTransformation}
\end{table}

\begin{figure}
\scriptsize
\begin{algorithmic}[1]{}
		{\ttfamily
		\Function{track}{obj}
		\State{}lockedSet.add(obj) \label{code:lockedSet:add}
			\State long ver = obj.version \label{code:track:getVersion}
			\State if {$\exists$<o,v>$\in$lockedSet such that o=obj and v!=ver} then return false \label{code:track:verifyVersion}
			\State if {obj.isLockedByAnotherThread()} then return false \label{code:track:verifyUnlocked}
			\State readSet.add(<obj,ver>)
			\State retrun true
		\EndFunction
		}
\end{algorithmic}
\caption{In read-only phase, locking is replaced by
tracking locks and read
objects' versions.
\label{figure::track}}
\end{figure}


\paragraph{Inconsistent Views}
%Other than when reading objects already in the read set, the read phase does not validate past reads during its executions ---
Other than in Figure~\ref{figure::track}, the read phase does not validate past reads during its executions ---
as a result, it may observe an inconsistent state of shared memory.
%
The function \emph{validateReadSet} in Figure~\ref{figure::validate} can be used to validate past reads: it returns \emph{true} if and only if the objects in the read set have not been updated.
%
The function checks that each object in the read set is not locked by another thread,
and that the object's current version matches the version saved in the
read set.
This check guarantees that the object was not locked from the time it was read until
the time it was validated.
Since operations write only to
locked nodes, it follows that the object was not changed.
This read set validation can be viewed as a double collect~\cite{Afek:1993:ASS:153724.153741}
of all objects accessed by the read-only phase.

In order to avoid infinite loops (in the read-only phase) that might occur due to inconsistent reads~\cite{HLR:SLCA2010}, a timeout is set.
If the timeout expires before the read-only phase is completed, read set
validation takes place (by invoking the function \emph{validateReadSet}). If the validation fails, the operation is restarted.
This is realized by inserting code that examines the timeout in each backward edge in the CFG of the read-only phase.

Similarly, inconsistent views may lead to spurious exceptions in the read-only phase. These are manifested as branches that
may lead outside of $C_r$ (end even outside of the operation's code altogether). We instrument such edges in the CFG, and
perform validation. Here too, if the validation fails, the operation is restarted. Otherwise, the exception is handled as in
the original code.


\paragraph{Validation Phase}
The code of the validation phase is inserted in each edge from $C_r$ to $C_u$ (i.e., this code is invoked between the read-only phase and the update phase).
This code is shown in lines \ref{code:beginValidation}-\ref{code:endValidation} of Figure~\ref{figure:transformation:after}.
It locks the objects in \emph{lockedSet} and validates the objects in \emph{readSet}.
To avoid deadlocks, the locks are acquired using a tryLock
instruction.
If a tryLock fails, the code unlocks  all
previously acquired locks and restarts from the beginning
(lines \ref{code:validateLockedSet}-\ref{code:validateGoto1}).
Similarity, the operation is restarted if the validation fails (lines \ref{code:validateReadSet}-\ref{code:validateGoto2}).

\paragraph{Update Phase}
In this phase our transformation preserves the original locking while maintaining the versions of the objects, i.e., the version of an object $o$ is incremented every time $o$ is unlocked.
Here, before each unlock instruction \emph{\ttfamily x.unlock()} we insert the code \emph{\ttfamily x.version++} .
An example is shown in lines \ref{code:beginUpdate}-\ref{code:endUpdate} of Figure~\ref{figure:transformation:after}.






\begin{figure}
\scriptsize
\begin{algorithmic}[1]{}
		{\ttfamily
		\Function{validateReadSet}{}()
		\ForAll {<obj,ver> in readSet}
			\If{obj.isLockedByAnotherThread()}
			\State return false \Comment{\textrm{validation failed (due to a locked object)}}
			\EndIf
			\If{obj.version != ver}
				\State return false \Comment{\textrm{validation failed (due to a different version)}}
			\EndIf
		\EndFor
		\State retrun true \Comment{\textrm{validation succeed}}
		\EndFunction
		}
\end{algorithmic}
\caption{Read set validation.\label{figure::validate}}
\end{figure}

\subsection{Phase Partitioning}\label{ssec:extendedTran}



\begin{figure*}
\scriptsize
	\begin{center}
	\begin{subfigure}[t]{.3\textwidth}
		\begin{algorithmic}[0]{}
		{\ttfamily
			\Function{foo}{X x, int i} \label{codeXXX:aaaaa}
            \Statex --------------------
            \State\hspace{-3mm}{1 :\ x.lock()}
            \State\hspace{-3mm}{2 :\ if i>7  then}
            \State\hspace{-3mm}{3 :\ \ \ x.f2 = i}
            \State\hspace{-3mm}{4 :\ temp = x.f1 + x.f2}
            \State\hspace{-3mm}{5 :\ x.unlock()}
            \State\hspace{-3mm}{6 :\ return temp}
            \Statex
            \State
            \State
            \State
            \State
            \State
            \State
			\EndFunction
			}
		\end{algorithmic}
		\caption{Original code.} \label{figure:autoPartitioning:step1}
	\end{subfigure}
	\begin{subfigure}[t]{.3\textwidth}
		\begin{algorithmic}[0]{}
		{\ttfamily
			\Function{foo}{X x, int i} \label{codeXXX:aaaaa}
            \Statex -------------------- \Comment{\textrm{original code $C$}}
            \State\hspace{-3mm}{1 :\ x.lock()}
            \State\hspace{-3mm}{2 :\ if i>7  then}
            \State\hspace{-3mm}{3 :\ \ \ x.f2 = i}
            \State\hspace{-3mm}{4 :\ temp = x.f1 + x.f2}
            \State\hspace{-3mm}{5 :\ x.unlock()}
            \State\hspace{-3mm}{6 :\ return temp}
            \Statex -------------------- \Comment{\textrm{clone $C_u$}}
            \State\hspace{-3mm}{1':\ x.lock()}
            \State\hspace{-3mm}{2':\ if x.f1>i  then}
            \State\hspace{-3mm}{3':\ \ \ x.f2 = i}
            \State\hspace{-3mm}{4':\ temp = x.f1 + x.f2}
            \State\hspace{-3mm}{5':\ x.unlock()}
            \State\hspace{-3mm}{6':\ return temp}
			\EndFunction
			}
		\end{algorithmic}
		\caption{The code after \emph{CFG cloning}.}
 \label{figure:autoPartitioning:step2}
	\end{subfigure}
	\begin{subfigure}[t]{.35\textwidth}
		\begin{algorithmic}[0]{}
		{\ttfamily
			\Function{foo}{X x, int i} \label{codeXXX:aaaaa}
            \Statex -------------------- \Comment{\textrm{read-only phase $C_r$}}
            \State\hspace{-3mm}{1 :\ x.lock()}
            \State\hspace{-3mm}{2 :\ if i>7  then}
            \State\hspace{-3mm}{3 :\ \ \ \underline{\textbf{goto 3'}}}
            \State\hspace{-3mm}{4 :\ temp = x.f1 + x.f2}
            \State\hspace{-3mm}{5 :\ x.unlock()}
            \State\hspace{-3mm}{6 :\ \underline{\textbf{goto 6'}}}
            \Statex -------------------- \Comment{\textrm{update phase $C_u$}}
            \State\hspace{-3mm}{1' :\ x.lock()}
            \State\hspace{-3mm}{2':\ if x.f1>i  then}
            \State\hspace{-3mm}{3':\ \ \ x.f2 = i}
            \State\hspace{-3mm}{4':\ temp = x.f1 + x.f2}
            \State\hspace{-3mm}{5':\ x.unlock()}
            \State\hspace{-3mm}{6':\ return temp}
			\EndFunction
			}
		\end{algorithmic}
		\caption{The code after \emph{phase separation}.
        %Lines 1-6 are the read-only phase; and lines 1'-6' are the update phase.
} \label{figure:autoPartitioning:step3}
	\end{subfigure}

	\end{center}
\vspace{-4mm}
	\caption{Example for automatic code partitioning to read-only and update phases.}
			\label{figure:autoPartitioning}
\end{figure*}

We now explain our automatic technique to partition code $C$ to read-only and update phases $C_r$ and $C_u$, resp.
By slight abuse of terminology, we use the same notation ($C$, $C_r$, or $C_u$) both for the code section and
for its corresponding CFG.
%
We demonstrate our technique for the code in Figure~\ref{figure:autoPartitioning:step1}.
Notice that here, the code location where the shared memory is updated depends on the runtime value of the parameter $i$: if $i>7$ then the code writes to shared memory in line $3$, and otherwise the code does not update shared memory.

\noindent Our technique is realized by the following two steps:
\begin{description}
  \item [CFG cloning:]
Given code $C$ of a data structure operation, we create a clone $C_u$ of $C$, and concatenate $C_u$ to the end of $C$.
For example, Figure~\ref{figure:autoPartitioning:step2} shows the code produced from Figure~\ref{figure:autoPartitioning:step1}.
Each location $\ell$ in $C$ has a matching location in its clone, denoted by $\ell'$
(e.g, in Figure~\ref{figure:autoPartitioning:step2}, the matching location of line $5$ is line $5'$).
  \item [Phase separation:]
Next, we modify the original code $C$ to execute only the read-only phase, and branch to $C_u$ when the
update phase begins. Our basic phase partitioning branches out to the update phase at the time of the first update in $C_r$.
To this end, we replace every shared memory write instruction
at location $l \in C$ with the instruction \emph{goto l'}.
Likewise, every \emph{return instruction} (or any other instruction that exits from the operation)
at location $l \in C$ is replaced with the instruction \emph{goto l'}.
The transformed version of $C$ is denoted $C_r$.
Figure~\ref{figure:autoPartitioning:step3} shows the code produced from Figure~\ref{figure:autoPartitioning:step2}.
\end{description}


For example, in Figure~\ref{figure:autoPartitioning:step3}, lines 1-6 are the read-only phase; and lines 1'-6' are the update phase.

It is easy to see that this new code is equivalent to the original one (in terms of external behavior), and it satisfies the assumption from Section~\ref{ssec:transformation}. Note further that $C_u$ consists of the complete original code, including
the read-only prefix, which is usually skipped. This allows us to separate the phases differently by branching to
the update phase earlier than the first write, and in particular, to fall back on executing the
operation pessimistically, as we now explain.

\paragraph{Early Phase Separation}
Note that, using our transformation, the shared state at the end of the validation phase
is identical to the state that would have been reached had the code been executed pessimistically from
the outset. Hence, the three-phase version of the code is compatible with the instrumented
pessimistic version. This means that if the optimistic phase is unsuccessful for any reason, we can always
fall back on the pessimistic version. Moreover, we can switch from optimistic to pessimistic synchronization
\emph{at any point} during the read phase.
We use this property in two ways, as we now describe.

First, we avoid livelocks by limiting the number of restarts due to conflicts:
The validation phase tracks the number of restarts in a thread-local variable.
If this number exceeds a certain threshold, we branch to $S'$, i.e., the location in the clone $C_u$ that matches
the beginning of the operation $S$.

Second, this property offers the optimistic implementation the liberty of
failing spuriously, even in the absence of conflicts, because it can always fall back on the safe pessimistic version
of the code. 
We take advantage of this liberty, and implement the \emph{lockedSet} and \emph{readSet} by using constant size arrays.
%Our implementation takes advantage of this liberty, and uses constant size arrays for the \emph{lockedSet} and \emph{readSet}.
In case either of these arrays becomes full, we cannot proceed with the optimistic version, but also
do not need to start the operation anew.
Instead, we immediately perform the validation phase, which, if successful, branches to the location in $C_u$
that matches the current code location, after having acquired all the needed locks.
