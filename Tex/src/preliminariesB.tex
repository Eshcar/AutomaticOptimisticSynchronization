\section{Model and Definitions}\label{sec:model}



\paragraph{Shared Memory Data Structures}

A \emph{data structure} defines a set of \emph{operations} that may be invoked by
clients of the data structure, potentially concurrently.
%
Operations have parameters and local variables, which are private to the invocation of the operations.
(Thus, these are thread-local variables.)
%
We assume that the code of each operation is represented by a separate control-flow graph (CFG).

The operations of a data structure interact via a shared memory which is composed of a set of \emph{shared objects}.
%
Each shared object supports atomic \emph{read} (load) and \emph{write} (store) instructions.
%
Below, we extend shared objects to also support locks.

Every thread executes a sequence of operations, each of which is invoked with certain parameters and returns a response.
An operation's execution consists of a sequence of primitive \emph{steps}, beginning with an \emph{invoke} step, followed by
atomic accesses to shared objects, and ending with a \emph{return} step. Steps also modify the executing thread's local variables.

A \emph{configuration} is an assignment of values to all shared objects and local variables. Thus, each step takes the system from one
configuration to another. Steps are deterministically defined by the data structure's code and the current configuration.
We assume that each data structure has a single \emph{initial configuration}.

An \emph{execution} is an alternating sequence of configurations and steps,
$C_0,s_1,C_1, \ldots,s_i,C_i,\ldots,$
where $C_0$ is the initial configuration,
and each configuration $C_i$ is the result of
executing step $s_i$ on configuration $C_{i-1}$.
We only consider finite executions in this paper.
An execution is \emph{sequential} if steps of different operations are not interleaved.
In other words, a sequential execution is a sequence of operation executions.

\paragraph{Locks}
Each shared object serves as a lock for itself.
It supports atomic \emph{lock}, \emph{tryLock}, \emph{unlock} and \emph{isLockedByAnotherThread} instructions.
Locks are exclusive (i.e., a lock can be held by at most one thread at a time).
The execution of a thread
trying to acquire a lock (by a \emph{lock} instruction) which is
held by another thread is blocked until a time when the
lock is available (i.e., is not held by any thread).
The other instructions never block the execution.
The \emph{tryLock} instruction returns \emph{false} if the lock is currently held by another thread, otherwise it acquires the lock and returns \emph{true}.
The \emph{isLockedByAnotherThread} instruction returns \emph{true}, if and only if, the lock is currently held by another thread.

We assume that in the given code every (read
or write) access by an operation to a shared object is performed when the
executing thread holds the lock on that object.
We also assume that the locking in the given code only uses \emph{lock} and \emph{unlock} instructions 
(i.e., the other locking instructions are not used).



\paragraph{Correctness}

The correctness of a data structure is defined in terms of its external behavior, as reflected in values returned by invoked operations.
Correctness of a code transformation is proven by showing that the synthesized code's executions are equivalent to ones of the original code,
where two executions are  \emph{equivalent} if every thread invokes the same
operations in the same order  in both executions, and gets the same result for each operation. More formally, we say in this paper that a code transformation is \emph{correct} if every execution of the transformed code
is equivalent to some execution of the original code.

The widely-used correctness criterion of serializability relies on equivalence to sequential executions in order to
link a data structure's behavior under concurrency to its sequentially specified behavior. Since equivalence is transitive,
we get that any code transformation satisfying our correctness notion, when applied to serializable code, yields code that is also serializable.
If the code transformation further ensures the real-time order of operations (i.e., operations that do not overlap appear in the same order in
executions of the transformed and original code), then linearizability (atomicity) is also invariant under the transformation.
Another important aspect of correctness is preserving the progress conditions of the original code, for example, deadlock-freedom.

In this paper, we are not concerned with internal consistency (as required e.g., by opacity~\cite{GuerraouiK2008} or the validity notion of~\cite{LevAriCK2014}),
which restricts the configurations an operation might see during its execution.
This is because our code transformation uses timeouts and exception handlers to overcome unexpected behavior that may arise when a thread sees an inconsistent view of global variables (similar to~\cite{Nakaike:2010}).



