\section{Evaluation}
In order to apply our approach to a data structure, a
''black-box'' pessimistic locking is required. One such 
approach was presented in \cite{Gueta2011}, automatically 
applying domination locking protocol to forest based data structures.  
We applied domination locking followed by our optimistic 
transformation to two tree based data structures, 
a simple unbalanced binary search tree 
and a treap (randomized binary search tree) \cite{AragonS1989}.

\paragraph{Setup}
We compared the performance of our automatic implementations, 
the automatic binary search tree (\autoTree) and the automatic
treap (\autoTreap), to the following custom tailored approaches: 
\begin{itemize}
\item \danaTree - The locked-based 
				unbalanced tree of Drachsler at al.\cite{DrachslerVY2014}. 
\item \danaAVL - The locked-based relaxed balanced AVL tree of 
				Drachsler et al.\cite{DrachslerVY2014}.
\item \bronson - The locked based relaxed balanced AVL tree
				of Bronson et al.\cite{BronsonCCO2010}.
\item \skiplist - The non-blocking skip-list by Doug 
				Lea included in the 
				the Java standard library.
\end{itemize}

We also compared our algorithms to previous automatic approaches, 
mainly global locking (\globalTree, \globalTreap) 
and domination locking (\domTree, \domTreap). 

%TODO 
We ran our experiments on a \ldots

We evaluated the performance on a variety of workloads, 
each workload is defined by the percentage of read-only
operations (\getOP queries) and the remaining operations 
are divided equally between insert and delete operations.
Our workloads include heavy read-only workloads
(100\%,70\% \getOP operations), medium read-only workload 
(50\% \getOP operations) and update only workload
(0\% \getOP operations). 

We used two key ranges $[0,2\cdot10^5]$ and $[0,2\cdot10^6]$,
for each range, the tree was pre-filled until the tree size was 
within 5\% of half the key range.   

We ran five seconds trials measuring the total throughput
(number of operations per second) of all threads.
During the trial, each thread continuously executed randomly
chosen operations according to the workload distribution 
using uniformly random keys from the key range.  
We ran every trial 7 times, we report the average throughput
while eliminating outliers.

\paragraph{Results} Figure \ref{evaluation:results:unbalanced} 
reports the throughput of unbalanced data structures and Figure 
\ref{evaluation:results:balanced} reports
the throughput of the balanced data structures. 


\begin{figure*}
\begin{center}
\begin{tikzpicture}
\begin{axis}[mystyle,unbalanced]
\addplot [blues3,mark=square*] table [x={threads}, y={TimeoutBinaryTree}]
{results/100C20K.txt}; 
\addplot [orange,mark=diamond*] table [x={threads}, y={Danaunbalanced}]
{results/100C20K.txt}; 
\addplot [gray,mark=asterisk] table [x={threads}, y={GlobalLockBinaryTree}]
{results/100C20K.txt};
\addplot [black,mark=x] table [x={threads}, y={DominationLockingBinaryTree}]
{results/100C20K.txt};
\end{axis}
\end{tikzpicture}
\begin{tikzpicture}
\begin{axis}[mystyle,unbalanced]
\addplot [blues3,mark=square*] table [x={threads}, y={TimeoutBinaryTree}]
{results/100C200K.txt}; 
\addplot [orange,mark=diamond*] table [x={threads}, y={Danaunbalanced}]
{results/100C200K.txt}; 
\addplot [gray,mark=asterisk] table [x={threads}, y={GlobalLockBinaryTree}]
{results/100C200K.txt};
\addplot [black,mark=x] table [x={threads}, y={DominationLockingBinaryTree}]
{results/100C200K.txt};
\end{axis}
\end{tikzpicture}
\begin{tikzpicture}
\begin{axis}[mystyle,unbalanced]
\addplot [blues3,mark=square*] table [x={threads}, y={TimeoutBinaryTree}]
{results/100C2000K.txt}; 
\addplot [orange,mark=diamond*] table [x={threads}, y={Danaunbalanced}]
{results/100C2000K.txt}; 
\addplot [gray,mark=asterisk] table [x={threads}, y={GlobalLockBinaryTree}]
{results/100C2000K.txt};
\addplot [black,mark=x] table [x={threads}, y={DominationLockingBinaryTree}]
{results/100C2000K.txt};
%\addplot  [magenta,mark=pentagon] table [x={threads}, y={rb}]
% {results/100C200K.txt};
\end{axis}
\end{tikzpicture}

\begin{tikzpicture}
\begin{axis}[mystyle,unbalanced]
\addplot [blues3,mark=square*] table [x={threads}, y={TimeoutBinaryTree}]
{results/70C20K.txt}; 
\addplot [orange,mark=diamond*] table [x={threads}, y={Danaunbalanced}]
{results/70C20K.txt}; 
\addplot [gray,mark=asterisk] table [x={threads}, y={GlobalLockBinaryTree}]
{results/70C20K.txt};
\addplot [black,mark=x] table [x={threads}, y={DominationLockingBinaryTree}]
{results/70C20K.txt};
\end{axis}
\end{tikzpicture}
\begin{tikzpicture}
\begin{axis}[mystyle,unbalanced]
\addplot [blues3,mark=square*] table [x={threads}, y={TimeoutBinaryTree}]
{results/70C200K.txt}; 
\addplot [orange,mark=diamond*] table [x={threads}, y={Danaunbalanced}]
{results/70C200K.txt}; 
\addplot [gray,mark=asterisk] table [x={threads}, y={GlobalLockBinaryTree}]
{results/70C200K.txt};
\addplot [black,mark=x] table [x={threads}, y={DominationLockingBinaryTree}]
{results/70C200K.txt};
\end{axis}
\end{tikzpicture}
\begin{tikzpicture}
\begin{axis}[mystyle,unbalanced]
\addplot [blues3,mark=square*] table [x={threads}, y={TimeoutBinaryTree}]
{results/70C2000K.txt}; 
\addplot [orange,mark=diamond*] table [x={threads}, y={Danaunbalanced}]
{results/70C2000K.txt}; 
\addplot [gray,mark=asterisk] table [x={threads}, y={GlobalLockBinaryTree}]
{results/70C2000K.txt};
\addplot [black,mark=x] table [x={threads}, y={DominationLockingBinaryTree}]
{results/70C2000K.txt};
%\addplot  [magenta,mark=pentagon] table [x={threads}, y={rb}]
% {results/100C200K.txt};
\end{axis}
\end{tikzpicture}

\begin{tikzpicture}
\begin{axis}[mystyle,unbalanced]
\addplot [blues3,mark=square*] table [x={threads}, y={TimeoutBinaryTree}]
{results/50C20K.txt}; 
\addplot [orange,mark=diamond*] table [x={threads}, y={Danaunbalanced}]
{results/50C20K.txt}; 
\addplot [gray,mark=asterisk] table [x={threads}, y={GlobalLockBinaryTree}]
{results/50C20K.txt};
\addplot [black,mark=x] table [x={threads}, y={DominationLockingBinaryTree}]
{results/50C20K.txt};
\end{axis}
\end{tikzpicture}
\begin{tikzpicture}
\begin{axis}[mystyle,unbalanced]
\addplot [blues3,mark=square*] table [x={threads}, y={TimeoutBinaryTree}]
{results/50C200K.txt}; 
\addplot [orange,mark=diamond*] table [x={threads}, y={Danaunbalanced}]
{results/50C200K.txt}; 
\addplot [gray,mark=asterisk] table [x={threads}, y={GlobalLockBinaryTree}]
{results/50C200K.txt};
\addplot [black,mark=x] table [x={threads}, y={DominationLockingBinaryTree}]
{results/50C200K.txt};
\end{axis}
\end{tikzpicture}
\begin{tikzpicture}
\begin{axis}[mystyle,unbalanced]
\addplot [blues3,mark=square*] table [x={threads}, y={TimeoutBinaryTree}]
{results/50C2000K.txt}; 
\addplot [orange,mark=diamond*] table [x={threads}, y={Danaunbalanced}]
{results/50C2000K.txt}; 
\addplot [gray,mark=asterisk] table [x={threads}, y={GlobalLockBinaryTree}]
{results/50C2000K.txt};
\addplot [black,mark=x] table [x={threads}, y={DominationLockingBinaryTree}]
{results/50C2000K.txt};
%\addplot  [magenta,mark=pentagon] table [x={threads}, y={rb}]
% {results/100C200K.txt};
\end{axis}
\end{tikzpicture}

\begin{tikzpicture}
\begin{axis}[mystyle,unbalanced]
\addplot [blues3,mark=square*] table [x={threads}, y={TimeoutBinaryTree}]
{results/0C20K.txt}; 
\addplot [orange,mark=diamond*] table [x={threads}, y={Danaunbalanced}]
{results/0C20K.txt}; 
\addplot [gray,mark=asterisk] table [x={threads}, y={GlobalLockBinaryTree}]
{results/0C20K.txt};
\addplot [black,mark=x] table [x={threads}, y={DominationLockingBinaryTree}]
{results/0C20K.txt};
\end{axis}
\end{tikzpicture}
\begin{tikzpicture}
\begin{axis}[mystyle,unbalanced]
\addplot [blues3,mark=square*] table [x={threads}, y={TimeoutBinaryTree}]
{results/0C200K.txt}; 
\addplot [orange,mark=diamond*] table [x={threads}, y={Danaunbalanced}]
{results/0C200K.txt}; 
\addplot [gray,mark=asterisk] table [x={threads}, y={GlobalLockBinaryTree}]
{results/0C200K.txt};
\addplot [black,mark=x] table [x={threads}, y={DominationLockingBinaryTree}]
{results/0C200K.txt};
\end{axis}
\end{tikzpicture}
\begin{tikzpicture}
\begin{axis}[mystyle,unbalanced]
\addplot [blues3,mark=square*] table [x={threads}, y={TimeoutBinaryTree}]
{results/0C2000K.txt}; 
\addplot [orange,mark=diamond*] table [x={threads}, y={Danaunbalanced}]
{results/0C2000K.txt}; 
\addplot [gray,mark=asterisk] table [x={threads}, y={GlobalLockBinaryTree}]
{results/0C2000K.txt};
\addplot [black,mark=x] table [x={threads}, y={DominationLockingBinaryTree}]
{results/0C2000K.txt};
%\addplot  [magenta,mark=pentagon] table [x={threads}, y={rb}]
% {results/100C200K.txt};
\end{axis}
\end{tikzpicture}

\ref{unbalancedLegened}
\end{center}
\caption{Throughput of unbalanced data
structures.\label{evaluation:results:unbalanced}}
\end{figure*}


\begin{figure*}
\begin{center}
\begin{tikzpicture}
\begin{axis}[mystyle,balanced,
				title=100\% read-only operations,
				ylabel = {Key range [$0,2\cdot10^4$]},
				]
\addplot [blues3,mark=square*] table [x={threads}, y={TimeoutTreap}]
{results/100C20K.txt}; 
\addplot [orange,mark=diamond*] table [x={threads}, y={DanaAVL}]
{results/100C20K.txt}; 
\addplot [darkspringgreen,mark=triangle*] table [x={threads}, y={Bronson}]
{results/100C20K.txt}; 
\addplot [coralred,mark=*] table [x={threads}, y={JavaSkipList}]
{results/100C20K.txt};
\addplot [gray,mark=asterisk] table [x={threads}, y={GlobalLockTreap}]
{results/100C20K.txt};
\addplot [black,mark=x] table [x={threads}, y={DominationLockingTreap}]
{results/100C20K.txt};
\end{axis}
\end{tikzpicture}
\begin{tikzpicture}
\begin{axis}[mystyle,balanced,title=50\% read-only operations]
\addplot [blues3,mark=square*] table [x={threads}, y={TimeoutTreap}]
{results/50C20K.txt}; 
\addplot [orange,mark=diamond*] table [x={threads}, y={DanaAVL}]
{results/50C20K.txt}; 
\addplot [darkspringgreen,mark=triangle*] table [x={threads}, y={Bronson}]
{results/50C20K.txt}; 
\addplot [coralred,mark=*] table [x={threads}, y={JavaSkipList}]
{results/50C20K.txt};
\addplot [gray,mark=asterisk] table [x={threads}, y={GlobalLockTreap}]
{results/50C20K.txt};
\addplot [black,mark=x] table [x={threads}, y={DominationLockingTreap}]
{results/50C20K.txt};
\end{axis}
\end{tikzpicture}
\begin{tikzpicture}
\begin{axis}[mystyle,balanced,title=0\% read-only operations]
\addplot [blues3,mark=square*] table [x={threads}, y={TimeoutTreap}]
{results/0C20K.txt}; 
\addplot [orange,mark=diamond*] table [x={threads}, y={DanaAVL}]
{results/0C20K.txt}; 
\addplot [darkspringgreen,mark=triangle*] table [x={threads}, y={Bronson}]
{results/0C20K.txt}; 
\addplot [coralred,mark=*] table [x={threads}, y={JavaSkipList}]
{results/0C20K.txt};
\addplot [gray,mark=asterisk] table [x={threads}, y={GlobalLockTreap}]
{results/0C20K.txt};
\addplot [black,mark=x] table [x={threads}, y={DominationLockingTreap}]
{results/0C20K.txt};
\end{axis}
\end{tikzpicture}


\begin{tikzpicture}
\begin{axis}[mystyle,balanced, 
				ylabel = {Key range [$0,2\cdot10^6$]}, ]
\addplot [blues3,mark=square*] table [x={threads}, y={TimeoutTreap}]
{results/100C2000K.txt}; 
\addplot [orange,mark=diamond*] table [x={threads}, y={DanaAVL}]
{results/100C2000K.txt}; 
\addplot [darkspringgreen,mark=triangle*] table [x={threads}, y={Bronson}]
{results/100C2000K.txt}; 
\addplot [coralred,mark=*] table [x={threads}, y={JavaSkipList}]
{results/100C2000K.txt};
\addplot [gray,mark=asterisk] table [x={threads}, y={GlobalLockTreap}]
{results/100C2000K.txt};
\addplot [black,mark=x] table [x={threads}, y={DominationLockingTreap}]
{results/100C2000K.txt};
\end{axis}
\end{tikzpicture}
\begin{tikzpicture}
\begin{axis}[mystyle,balanced]
\addplot [blues3,mark=square*] table [x={threads}, y={TimeoutTreap}]
{results/50C2000K.txt}; 
\addplot [orange,mark=diamond*] table [x={threads}, y={DanaAVL}]
{results/50C2000K.txt}; 
\addplot [darkspringgreen,mark=triangle*] table [x={threads}, y={Bronson}]
{results/50C2000K.txt}; 
\addplot [coralred,mark=*] table [x={threads}, y={JavaSkipList}]
{results/50C2000K.txt};
\addplot [gray,mark=asterisk] table [x={threads}, y={GlobalLockTreap}]
{results/50C2000K.txt};
\addplot [black,mark=x] table [x={threads}, y={DominationLockingTreap}]
{results/50C2000K.txt};
\end{axis}
\end{tikzpicture}
\begin{tikzpicture}
\begin{axis}[mystyle,balanced]
\addplot [blues3,mark=square*] table [x={threads}, y={TimeoutTreap}]
{results/0C2000K.txt}; 
\addplot [orange,mark=diamond*] table [x={threads}, y={DanaAVL}]
{results/0C2000K.txt}; 
\addplot [darkspringgreen,mark=triangle*] table [x={threads}, y={Bronson}]
{results/0C2000K.txt}; 
\addplot [coralred,mark=*] table [x={threads}, y={JavaSkipList}]
{results/0C2000K.txt};
\addplot [gray,mark=asterisk] table [x={threads}, y={GlobalLockTreap}]
{results/0C2000K.txt};
\addplot [black,mark=x] table [x={threads}, y={DominationLockingTreap}]
{results/0C2000K.txt};
\end{axis}
\end{tikzpicture}
\ref{balancedLegened}
\end{center}
\caption{Throughput of balanced data
structures.\label{evaluation:results:balanced}}
\end{figure*}