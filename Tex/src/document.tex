%%This is a very basic article template.
%%There is just one section and two subsections.

\title{Automatic Optimistic Synchronization for Dynamic Data Structures}
\documentclass{article}
\usepackage{xspace}

%-------------Theorem Definitions ---------------%
\newtheorem{theorem}{Theorem}[section]
\newtheorem{lemma}[theorem]{Lemma}
\newtheorem{proposition}[theorem]{Proposition}
\newtheorem{corollary}[theorem]{Corollary}

\newenvironment{proof}[1][Proof]{\begin{trivlist}
\item[\hskip \labelsep {\bfseries #1}]}{\qedsymb\end{trivlist}}
\newenvironment{definition}[1][Definition]{\begin{trivlist}
\item[\hskip \labelsep {\bfseries #1}]}{\end{trivlist}}
\newenvironment{example}[1][Example]{\begin{trivlist}
\item[\hskip \labelsep {\bfseries #1}]}{\end{trivlist}}
\newenvironment{remark}[1][Remark]{\begin{trivlist}
\item[\hskip \labelsep {\bfseries #1}]}{\end{trivlist}}

\newcommand{\qed}{\nobreak \ifvmode \relax \else
      \ifdim\lastskip<1.5em \hskip-\lastskip
      \hskip1.5em plus0em minus0.5em \fi \nobreak
      \vrule height0.75em width0.5em depth0.25em\fi}
\newcommand{\qedsymb}{\hfill{\rule{2mm}{2mm}}}

%--------------------------------------------------%

\newcommand{\code}[1]{\textsf{#1}}
\newcommand{\readV}{\code{read\_version}\xspace} 
\newcommand{\readSet}{\code{read\_set}\xspace}
\newcommand{\writeV}{\code{write\_version}\xspace}

\newcommand{\reqI}{\textbf{LPR1}\xspace}
\newcommand{\reqII}{\textbf{LPR2}\xspace}

\begin{document}
\maketitle

\section{Introduction}

\section{Preliminaries}
%A \emph{concurrent system} (module) consists of a set of threads, 
A \emph{module} defines a set of \emph{abstract data types} and a 
set of \emph{operations} that may be invoked by
clients of the module.
Each operation is invoked with possible parameters
and returns with a response. 
The invocation is a local step of a thread, 
followed by a execution of a sequence of atomic steps. 
Atomic step is either a computation on local 
variables or a primitive operation on shared variable.
Read and write to shared memory are 
denoted load and store.  
%a load (``\code{x=y.f}'') or 
%a store (``\code{x.f=y}'') operation on shared variables.
The parameters and local variables 
of an operation are private to the
invocation of the operation (thread local). 
There are no static or global variables shared by  
different invocations of the operations. 

A \emph{configuration} is an instantaneous representation 
of the system, including the state of the shared memory
and the local variables. In the \emph{initial configuration} 
all variables hold an initial value. 
An \emph{execution} is an alternating sequence of 
configurations and steps,
$C_0,s_1,\ldots,s_i,C_i,\ldots,$ 
where $C_0$ is the initial configuration,
and each configuration $C_i$ is the result of
executing step $s_i$ on configuration $C_{i-1}$.
 
An execution is \emph{non-interleaved} 
(abbreviated NI-execution) if primitive operation of 
different operations are not interleaved, i.e., 
for every pair of operations invocations $p_i \neq p_j$ 
either all primitive operations of $p_i$ come before any 
primitive operation of $p_j$, or vice versa.
 
Given an execution, we say that two primitive operations 
\emph{conflict} if (i) they are executed by two different threads, 
(ii) they access some common global variable or a heap allocated object 
(iii) at least one of the conflicting instruction is a write.


%Let $\pi$ be an execution and $op_i,op_j$ are primitive operations in $\pi$ the
% conflict relation of $\pi$ is: \\ $conf(\pi)=\{(op_i,op_j)\mid op_i$ and $op_j$ are in conflict and $op_i<_{\pi}op_j\}$\\

%Let $\pi$ be an execution the conflict graph $G(\pi)=(V,E)$ is defined by:\\ $V
% = \{$ all operations (threads) of $\pi\}$\\
%$E = \{ (t_i,t_j) \mid t_i\neq t_j$ and $\exists op_i\in t_i, \exists op_j\in
% t_j$ such that $(op_i,op_j)\in conf(\pi) \}$\\

Executions $\pi_1$ and $\pi_2$ are \emph{conflict-equivalent} 
if they include the same primitive operations and they 
agree on the order between conflicting operations. 
An execution is \emph{conflict-serializable} if it is 
conflict-equivalent with a non-interleaved execution.


%TODO: more definitions about executions
%TODO: hidden and exposed! 

%DOMINATION LOCKING
%\emph{Domination Locking Protocol} is a conflict serializable
%locking protocol presented in \cite{Gueta2011}. 
%It requires a fixed total order $\leq$ on all heap objects. 
%An execution satisfies the the Domination Locking protocol,
%with respect to $\leq$ if it satisfies the following conditions:


\section{Automatic Optimistic Synchronization}
%The main idea: 
We address the problem of adding an 
optimism to a pessimistic locking protocol. 
Given a sequential implementation and some pessimistic
locking protocol, that performs store steps 
only to locked objects,
we add optimistic synchronization. 
The idea of optimistic synchronization is to load 
shared variables without locks and start using the 
locking protocol only when the operation reaches a store step.
In practice our algorithm shows that there is no need to 
acquire locks if they are freed before any change is made.    
%In order to implement such synchronization scheme, 
%we divide the operation to three \emph{phases}:

This optimistic scheme separates the operation to three
\emph{phases}, an optimistic \emph{read phase},
a pessimistic \emph{read-write phase} and a 
\emph{validation phase} that connect them. 



\subsection{Requirements on The Locking Protocol}
In order for our algorithm to work, the initial locking
protocol must have one of the following properties: 

\begin{description}
\item[\reqI] The protocols allows early lock release. 
\item[\reqII] The protocols requires that all locks are acquired 
before the first store step.    
\end{description}
    

For example the \emph{two phase locking (2PL)} protocol 
does not allow early lock release (no locks can be acquired after a 
lock was released), also, it does not require acquiring all locks
before the first store operation. However, a stricter version 
of 2PL locking that requires \reqII can be used to achieve optimism. 
The resulting optimistic protocol would be similar to 
\emph{Transactional Locking 2}\cite{DiceSS2006}.   



\subsection{Detailed Algorithm}
Each objects maintains a counter, incremented every time the 
object is locked. This counter is used to validate the correctness
of the optimistic read phase. 

\begin{description}
  \item[Read Phase] Stats at the beginning of the operation and
  ends at any point before the first store operation. 
  During this phase the operation maintains a \readSet, 
  containing references to objects loaded and the local 
  version when it was read. The local versions are incremented 
  during the read-write phase, when the object is locked. 
  Incrementing the version is not atomic with the lock, thus, 
  the object is also checked to be unlocked. 
  The read phase does not validate reads, 
  in order to avoid infinite loops, a timeout is set. 
  If the operation reaches the timeout, 
  a \readSet validation takes place, if it fails 
  the operation restarts from the beginning.
  A pseudo code of the transformation can be found in \ref{}. 
  
  \item[Validation Phase] This phase connects the read phase
  with the read-write phase. It has two requirements: (i) lock 
  local variables of the operation and (ii) ensure that the 
  values read during the read phase are consistent, i.e.
  as if the values were read while executing the locking 
  protocol. To avoid deadlocks, the locks are acquired using 
  a \code{try\_lock} operation, if the \code{try\_lock} fails, 
  the operation restarts from the beginning. Next, the \readSet 
  is validated, if the validation fails the operation restarts from the
  beginning. During the \readSet validation, each reference saved 
  in the \readSet is checked to be unlocked and that the current 
  version matches the version saved in the \readSet. 
  A pseudo code of the \readSet validation can be found in \ref{}. 
  
  \item[Read-Write Phase] This phase enforces the locking protocol
  while maintaining the local versions, i.e., the local version of 
  an object is incremented every time it is locked.  
  Once the read-write phase begins, the operation is guaranteed to to 
  finish without restarts.  
\end{description}

The use of timeout does not guarantee
\emph{opacity}~\cite{GuerraouiK2008} or 
\emph{validity}\cite{LevAriCK2014}. 

%The \readSet validation first checks that the node is unlocked,
%(or locked by the current operation), then it checks that the 
%current version is equal to the version saved in the \readSet. 
 
\section{Algorithm's Correctness} 
We will prove that if the original locking protocol is 
conflict-serializable then our algorithm is conflict-serializable.

Let $\pi$ be an execution of our optimistic automation on a 
sequential algorithm. We will construct an execution $\pi_{LP}$ 
which is an execution following the original locking protocol. 
We will prove that both executions are conflict-equivalent. 
Since any execution of the original locking protocol
is conflict-serializable, then $\pi$ is conflict-serializable. 

Let $p_1,p_2,\ldots,p_n$ be the operations $\in\pi$ ordered by the 
order of execution of the first step of a successful \readSet 
validation. (If some operation does not have such point we omit it).
Let $\pi_{LP} = \pi_{lp1},\pi_{1},\ldots,\pi_{lpi},\pi_{i}$ where 
$\pi_{lpi}$ is a $p_i$-only execution of original locking protocol 
until $p_i$ holds locks only on the local variable locked
in the validation phase of $p_i \in \pi$, and $\pi_i$ is
the interval of $\pi$ starting from the return from the validation of
$pi$ until the first step of the successful \readSet validation of 
$p_{i+1}$ that includes only the operations by $\{p_1,\ldots,p_i\}$.
In other words, we replace the read-phase and validation phase with 
an execution of the original locking protocol, 
taking place at the point just before the \readSet validation starts. 

%TODO connect the requirement on the locking protocol to the construction.
\begin{lemma}
The construction of $\pi_{LP}$ is feasible.  
\end{lemma}
\begin{proof}
Proof by induction on $p_1,p_2,\ldots,p_n$. Base case is immediate. 

Let $\pi' = \pi_{lp1},\pi_{1},\ldots,\pi_{lpk-1},\pi_{k-1}$ be the feasible
construction so far, and let $p_{k}$ be the next operation to be 
added. 

Assume by contradiction that $\pi'\cdot\pi_{lpk}\cdot\pi_{k}$ 
cannot be constructed, thus, some object $v$ that $p_{k}$ locks 
in $\pi_{lpk}$ is already locked 
by $p_j \in \{p_1,p_2,\ldots,p_{k-1}\}$ in
the last configuration of $\pi'$. 
If $p_j$ locked $v$ before $p_k$ read $v$ for the first time, 
then $v$ was locked during the read phase of $p_k$, 
in contradiction to $p_k$ reaching its validation. 
Otherwise, $p_j$ locked $v$ after $p_k$ read $v$. 
If $v$ is still locked during the validation of $p_k$ then 
the validation will fail, contradiction. Alternatively, $v$ 
was unlocked by $p_j$ before $p_k$ validated $v$, 
its version incremented to a version bigger 
than the local version read by $p_k$, 
contradicting the successful validation of $p_k$.  
\end{proof}

\begin{lemma}
$\pi_{LP}$ is conflict-equivalent to $\pi$
\end{lemma}
\begin{proof}
Each operation performs a double collect on all the values it reads. 
The first collect is the read phase and the second is the \readSet 
validation of the validation phase. Since validation was successful, 
both collect are identical, meaning that the values of the \readSet
do not change from the return of the last read of the read phase,
until the first read of the \readSet validation. Therefore, executing 
the original locking of $p_k$ after $\pi' =
\pi_{lp1},\pi_{1},\ldots,\pi_{lpk-1},\pi_{k-1}$ is conflict-equivalent
to the original read phase. The read-write phase remains unchanged, 
maintaining conflict-equivalence to $\pi$.
\end{proof}


\bibliography{myRef}
\bibliographystyle{plain}
\end{document}
