%%This is a very basic article template.
%%There is just one section and two subsections.

\title{Scalable Automatic Synchronization Combining Optimism with Pessimism}
\documentclass{article}
\usepackage{xspace}
\usepackage{algorithm}
\usepackage[noend]{algpseudocode}
\usepackage{caption}
\usepackage{subcaption}

%-------------Theorem Definitions ---------------%
\newtheorem{theorem}{Theorem}[section]
\newtheorem{lemma}[theorem]{Lemma}
\newtheorem{proposition}[theorem]{Proposition}
\newtheorem{corollary}[theorem]{Corollary}

\newenvironment{proof}[1][Proof]{\begin{trivlist}
\item[\hskip \labelsep {\bfseries #1}]}{\qedsymb\end{trivlist}}
\newenvironment{definition}[1][Definition]{\begin{trivlist}
\item[\hskip \labelsep {\bfseries #1}]}{\end{trivlist}}
\newenvironment{example}[1][Example]{\begin{trivlist}
\item[\hskip \labelsep {\bfseries #1}]}{\end{trivlist}}
\newenvironment{remark}[1][Remark]{\begin{trivlist}
\item[\hskip \labelsep {\bfseries #1}]}{\end{trivlist}}

\newcommand{\qed}{\nobreak \ifvmode \relax \else
      \ifdim\lastskip<1.5em \hskip-\lastskip
      \hskip1.5em plus0em minus0.5em \fi \nobreak
      \vrule height0.75em width0.5em depth0.25em\fi}
\newcommand{\qedsymb}{\hfill{\rule{2mm}{2mm}}}

%--------------------------------------------------%

\newcommand{\code}[1]{\textsf{#1}}
\newcommand{\readV}{\code{read\_version}\xspace} 
\newcommand{\readSet}{\code{read\_set}\xspace}
\newcommand{\writeV}{\code{write\_version}\xspace}

\newcommand{\reqI}{\textbf{LPR1}\xspace}
\newcommand{\reqII}{\textbf{LPR2}\xspace}

\begin{document}
\maketitle

\abstract{We present an \emph{automatic} approach for 
parallelizing sequential data structures in a way that 
is both safe and scalable. While there exist pessimistic 
transformations that make code thread-safe by adding 
(either global or fine-grained) locks, this approach is 
limited in its performance due to synchronization bottlenecks, 
for example, locking the root in a tree data structure. 
In this paper, we improve the performance and scalability 
of such synthesized code by reducing bottlenecks. 
Specifically, we present an automatic approach to eliminate 
many of the locking steps, relying instead on optimistic 
partial traversals of the data structure. 
We realize our approach for tree data structures, 
using the domination locking technique. 
The resulting code scales well, 
significantly outperforms pessimistic approaches, 
and achieves performance close to those of 
custom-tailored concurrent data structures. 
Our work thus shows the promise that automated approaches 
bear for overcoming the difficulty involved in manually 
hand crafting concurrent data structures. 
}

\section{Introduction} \label{sec:intro}

\subsection{Automatic Lock Removal}
% Parallelizing data strucutres is important for performance
The steady increase in the number of cores in today's computers is driving software developers to allow more and more parallelism.
An important focal point for such efforts is scaling the concurrency of shared data structures, which are often a principal friction point among threads.
%; recent work has illustrated that improved data structure concurrency can lead to
%benefits in overall system performance~\cite{clsm-poster}.
%It is therefore not surprising that many recent works have been dedicated to developing scalable concurrent data
Many recent works have been dedicated to developing scalable concurrent data
structures (e.g.,~\cite{ArbelA2014,DrachslerVY2014,NatarajanM2014,BrownER2014,CrainGR2013,BraginskyP2012,
AfekKKMT2012,EllenFRB2010,BronsonCCO2010,HerlihyLLS2007,fraser2004practical,Michael:1996}),
some of which are widely used in real-world systems~\cite{Ohad:OOPSLA11}.

% They are difficult to build and of resticted use
Each of these projects generally focuses on a single data
structure (for example, a binary search tree~\cite{BronsonCCO2010} or a queue~\cite{Michael:1996}) and manually optimizes its implementation. These data structures are developed by concurrency experts, typically PhDs or PhD candidates.
Proving the correctness of such custom-tailored data structures is painstaking;
for example, the proofs of \cite{BraginskyP2012,EllenFRB2010} are $31$ and $20$ pages long
respectively.
The rationale behind dedicating so much effort to one data structure is that it is
generic and can be used by many applications. Nevertheless,  systems often use data structures in unique ways
that necessitate changing or extending their code (e.g.,~\cite{levelDB,jmonkey,OhadThesis,zyulkyarov2009atomic}), which limits the usability of custom-tailored
implementations. Hence, the return-on-investment for such endeavors may be suboptimal.
Here, we propose an approach to facilitate this labor-intensive process by automatic means,
making scalable synchronization more readily available.

% We give a transformation
Specifically, we present in Section~\ref{sec:algorithm} a source-to-source
code transformation that takes a lock-based concurrent data structure implementation as its input
and generates more scalable code for the same data structure via judicious use of optimism.
%%%Section~\ref{sec:model} details our model and assumptions, and
%%%Section~\ref{sec:algorithm} specifies the transformation.
Our approach combines optimism and pessimism in a new, practical, way.
In striking the balance between the two, we exploit the common access pattern in data structure operations,
(for example, tree insertion or removal), which typically begin by traversing the data structure (to the insertion or removal point), and then perform (mostly) local updates at that location.
Our transformation replaces locking steps in the initial read-only traversal of each operation with
optimistic synchronization, whereas the update phase employs the original lock-based synchronization.
Our work may thus be seen as a form of software lock elision for read-only operation prefixes.

% Best of both worlds
Combining optimism and pessimism allows us to achieve ``the best of both worlds'' -- while the
optimistic traversal increases concurrency and eliminates bottlenecks,
the use of pessimistic updates saves the overhead associated with speculative or deferred shared
memory updates, (as in software transactional memory~\cite{HLR:SLCA2010}).
The partially-optimistic execution is compatible with the original code, which permits us to re-execute operations
pessimistically when too many conflicts occur, avoiding livelocks.
%Furthermore, it allows for code optimizations
%that make the optimistic execution fail in some conflict-free cases (for example, when too many items would have been locked
%by the original code), since we can always fallback upon lock-based execution.

% Properties of our transformation
We show in Section~\ref{sec:proof} that our transformation preserves the external behaviour of the original lock-based code.
%all essential properties of the original code: serializability, linearizability, and deadlock-freedom.
In other words, if the original code is correct (in the sense of serializability, linearizability, and deadlock-freedom), so is the
transformed version. Moreover, the transformation
%does not introduce any new accesses to shared data; in particular, it
refrains from introducing a shared global clock (as used in some transactional memory systems~\cite{DBLP:conf/eurosys/ShalevS06}) or other
shared data structures. Thus, if the original code is \emph{disjoint access parallel}~\cite{Israeli:1994:DIS:197917.198079}, i.e., threads
that access disjoint (abstract) data objects do not contend on (low level) shared memory locations, then this
property holds also for the transformed code.

\subsection{Fully Automatic Parallelization}
% Automatic parallelization
One important use case for our transformation is to apply it in conjunction with automatic lock-based
parallelization mechanisms~\cite{Gueta2011,MZGB:POPL06}.
The latter automatically instrument sequential code (at compile time)
and add fine-grained lock and unlock instructions that ensure its safety in concurrent executions.
%For example, \emph{domination locking}~\cite{Gueta2011} is applicable to trees or forests
%(including binary search trees, b-trees, treaps~\cite{AragonS1989}, and self adjusting heaps~\cite{Sleator:SAH1986:SAH}); it
%employs a variant of hand-over-hand locking~\cite{SilberschatzK1980},
%acquiring and releasing locks as it goes down the tree.
%%Other approaches are applicable to DAGS~\cite{dag-locking}.
Our evaluation shows that, by themselves, solutions of this sort may scale poorly.
%~\cite{Gueta2011}.
This is due to synchronization bottlenecks, e.g., the root of a tree,
which is locked by all operations.
By subsequently applying our transformation, one can optimize
the lock-based code they produce, yielding a \emph{fully automatic approach to
scalable parallelization of sequential code}.

%% Why not read-write locks
%It is worth noting that the aforementioned mechanisms synthesize code that uses conventional symmetric locks,
%which is the type of locks handled by our transformation. We are not aware of any automatic transformation
%inserting read-write locks. We further note that read-write locks
%are more costly than conventional ones, and, moreover, threads using
%read locks contend on the shared memory locations employed by the lock's implementation~\cite{xxx}.
%In contrast, with our transformation, threads executing the optimistic phase do not contend on locks,
%and are completely invisible to other threads.

\subsection{Evaluation}
In Section~\ref{sec:eval} we evaluate our transformation by generating three data structures-- an unbalanced search tree, a treap
(randomized balanced search tree),
and a skip list that supports range queries. We synthesize the first two from sequential implementations using the algorithm of~\cite{Gueta2011}, followed by our transformation.
For the skip list, we manually add fine-grained locks (in a straightforward manner), and then apply our transformation.
All examples are implemented in Java. We evaluate the scalability of the resulting code
in a range of workload scenarios on a $32$-core machine.
In all cases, the lock-based implementations do not scale --
their throughput remains flat as the number of running threads increases. In contrast, the code generated by our transformation
is scalable, and its throughput continues to grow with the number of threads.

We compare our synthesized code to state-of-the-art data structures
that were hand-crafted by experts in the field~\cite{DrachslerVY2014,BronsonCCO2010,ConcurrentSkipList,fraser2004practical}.
%and resulted in publications in leading venues~\cite{DrachslerVY2014,BronsonCCO2010}.
Our results show that the
fully-automatically generated code for the search tree and treap
achieves comparable performance to that of
custom-tailored solutions.

We also consider a data structure that supports range queries, which are required by
many applications (e.g.,~\cite{levelDB,FerroJKRY14}). To this end we implement a skip list -- a data structure that naturally supports range queries.
While range queries implemented using the iterators available in the Java concurrency library's skip list~\cite{ConcurrentSkipList} perform
somewhat better than ones in our synthesized code, it is important to note that these iterators are \emph{not}
linearizable (atomic), and only support so-called weak consistency, whereas range queries in our implementation are linearizable.
A notable previous linearizable implementation is due to Bronson et al.~\cite{BronsonCCO2010},
and it is significantly out-performed by our synthesized code.
%
This illustrates the benefit of the broad applicability
of our automatic approach compared to specific custom-tailored implementations.

Note that the alternative automatic approach we compare with is domination locking~\cite{Gueta2011} which scales poorly.
%\Idit{Say something about transactional memory.}
%The only other automatic parallelization approach we are familiar with is software transactional memory~\cite{xxx},
%which has been shown to perform even worse than a global lock~\cite{}, hence we do not empirically
%compare our solution with it.
Software transactional memory is another approach for automatic parallelization,
yet due to the significant overhead associated with this approach~\cite{Cascaval:2008,DuffyTM2010}, it is often complemented
by custom-tailored data structure implementations whose operations
can be called from within transactions~\cite{Herlihy:2008,Koskinen:2010,NathanBronson11}.
%
Further discussion of related work appears in Section~\ref{sec:related}.

To conclude, this paper demonstrates that automatic synchronization, based on a careful combination of optimistic and
pessimistic concurrency control, is a promising approach for bringing legacy code to emerging computer architectures.
While this paper illustrates the method for tree and skip list data structures, we believe that the general direction may be more broadly applicable, and maybe used with a variety of locking schemes, such as two phase locking.
Section~\ref{sec:discussion} concludes the paper and touches on some directions for future work. 


\section{Model and  Definitions}


\subsection{Shared Memory Data Structures}

We consider an asynchronous shared memory model, where independent threads interact via shared memory objects.
For the sake of our discussion, we do not distinguish among different types of shared memory (e.g., global or heap-allocated). 
In addition, each thread has access to \emph{local} (thread-local) memory.

A \emph{data structure} is an abstract data type exporting a set of \emph{operations}.
A data structure is implemented from a collection of primitive shared \emph{objects} supporting atomic load (read) and store (write) operations.
In Section~\ref{ssec:locking}, we extend the allowed primitive variables to also include locks.  

Each thread executes a sequence of operations, each of which is invoked with certain parameters and returns a response.
An operation's execution consists of a sequence of primitive \emph{steps}, beginning with an \emph{invoke} step, followed by 
atomic accesses to shared objects, and ending with a \emph{return} step. Steps also modify the executing thread's local variables.
A \emph{configuration} is an assignment of values to all shared and local variables. Thus, each step takes the system from one 
configuration to another. Steps are deterministically defined by the data structure's protocol and the current configuration. 
In the \emph{initial configuration}, each variable holds an initial value. 

An \emph{execution} is an alternating sequence of configurations and steps,
$C_0,s_1,\ldots,s_i,C_i,\ldots,$ 
where $C_0$ is an initial configuration,
and each configuration $C_i$ is the result of
executing step $s_i$ on configuration $C_{i-1}$.
An execution is \emph{sequential} if steps of different operations are not interleaved. 
In other words, a sequential execution is a sequence of operation executions.

\subsection{Locking Protocols}
\label{ssec:locking}

A \emph{lock} is a primitive type that supports atomic \emph{lock}, \emph{try\_lock}, and \emph{unlock} operations, 
where try\_lock is a non-blocking attempt to acquire a lock that may fail. 

We assume in this paper a \emph{locking protocol}, which transforms a sequential data-structure to a correct concurrent one by adding locks;
correctness is defined in Section~\ref{sec:linearizable} below.
Examples of such protocols are Tree Locking~\cite{tree-locking}, and Domination Locking~\cite{domination-locking}. 
The locking protocol associates a lock with every primitive shared object used by the data structure, and instruments the sequential code 
by adding lock and unlock operations. Intuitively, such protocols automatically perform some sort of ``hand-over-hand'' locking, acquiring locks as
they traverse a linked-list or tree, and releasing locks on previously traversed nodes. They are typically restricted to tree-like data structures.


We assume that the resulting code (obtained by adding the locks) satisfies the following properties:

\begin{itemize}
\item Every (load or store) access by an operation to a shared object is performed when the executing thread holds the lock on that object.
\item The protocol is deadlock-free, i.e., locking  does not introduce deadlocks.
\item The protocol allows early lock release in the sense that  it never needs to hold a lock on an object that it no longer has a pointer to.
 \end{itemize}

The last condition means that even if the protocol holds a lock on an object it no longer holds a pointer to, it is safe to unlock it at this point (and re-acquire the lock later if it is accessed again ), in the sense that correctness, as defined below, is not breached.  


\subsection{Correctness}
\label{ssec:linearizable} 

The correctness of a data structure is defined in terms of its external behavior, as reflected in values returned by invoked operations. 
This is captured by the notion of a \emph{history} -- the history of an execution $\sigma$ is the subsequence  of $\sigma$ consisting 
of invoke and return steps. The widely-used correctness criteria of linearizability and serializability link the data structure's behavior under concurrency to its allowed behavior in sequential executions. The latter is defined by a \emph{sequential specification}, which is a set of its allowed sequential histories. 

A history $H$ is \emph{linearizable}
\cite{Herlihy:1990:LCC:78969.78972} if there exists $H'$ that can be created by adding zero or more 
return steps to $H$, and there is a sequential permutation $\pi$ of complete($H'$), 
such that: (1) $\pi$ belongs to the sequential specification; and 
(2) every pair of operations that are not interleaved in $\sigma$, appear in the same order in $\sigma$ and in $\pi$. 
A data structure  is \emph{linearizable}, also called \emph{atomic}, if the histories of all of its executions are linearizable.

A history is \emph{serializable} if it satisifes property (1) above. That is, the real-time order of operations is not required.
A data structure  is \emph{serializable} the histories of all of its executions are serializable.

In this paper, we are not concerned with internal consistency (as required e.g., by opacity~\cite{opacity} or the validity notion of~\cite{Kfir}), which 
restricts the configurations an operation might see during its execution. 
This is because our code transformation deals with inconsistencies that may arise when a thread sees an inconsistent view of global variables using 
timouts and exception handlers.   



\section{\ldots}
%The main idea: 
We address the problem of adding an 
optimism to a pessimistic locking protocol. 
Given a sequential implementation and some pessimistic
locking protocol, that performs store steps 
only to locked objects,
we add optimistic synchronization. 
The idea of optimistic synchronization is to load 
shared variables without locks and start using the 
locking protocol only when the operation reaches a store step.
In practice our algorithm shows that there is no need to 
acquire locks if they are freed before any change is made.    
%In order to implement such synchronization scheme, 
%we divide the operation to three \emph{phases}:

This optimistic scheme separates the operation to three
\emph{phases}, an optimistic \emph{read phase},
a pessimistic \emph{read-write phase} and a 
\emph{validation phase} that connect them. 

\subsection{Detailed Algorithm}
Each objects maintains a counter, incremented every time the 
object is locked. This counter is used to validate the correctness
of the optimistic read phase. 

\begin{description}
  \item[Read Phase] Stats at the beginning of the operation and
  ends at any point before the first store operation. 
  During this phase the operation maintains a \readSet, 
  containing references to objects loaded and the local 
  version when it was read. The local versions are incremented 
  during the read-write phase, when the object is locked. 
  Incrementing the version is not atomic with the lock, thus, 
  the object is also checked to be unlocked. 
  The read phase does not validate reads, 
  in order to avoid infinite loops, a timeout is set. 
  If the operation reaches the timeout, 
  a \readSet validation takes place, if it fails 
  the operation restarts from the beginning.
  A pseudo code of the transformation can be found in 
  Figure \ref{figure:readPhaseTransformation}. 
  
  \item[Validation Phase] This phase connects the read phase
  with the read-write phase. It has two requirements: (i) lock 
  local variables of the operation and (ii) ensure that the 
  values read during the read phase are consistent, i.e.
  as if the values were read while executing the locking 
  protocol. To avoid deadlocks, the locks are acquired using 
  a \code{try\_lock} operation, if the \code{try\_lock} fails, 
  the operation restarts from the beginning. Next, the \readSet 
  is validated, if the validation fails the operation restarts from the
  beginning. During the \readSet validation, each reference saved 
  in the \readSet is checked to be unlocked and that the current 
  version matches the version saved in the \readSet. 
  A pseudo code of the \readSet validation can be found in
  Figure \ref{figure:readSetValidation}.
  
  \item[Read-Write Phase] This phase enforces the locking protocol
  while maintaining the local versions, i.e., the local version of 
  an object is incremented every time it is locked.  
  Once the read-write phase begins, the operation is guaranteed to to 
  finish without restarts.  
\end{description}

The use of timeout does not guarantee
\emph{opacity}~\cite{GuerraouiK2008} or 
\emph{validity}\cite{LevAriCK2014}. 

%The \readSet validation first checks that the node is unlocked,
%(or locked by the current operation), then it checks that the 
%current version is equal to the version saved in the \readSet. 
 
%ALGORITHM CODE

\begin{figure}

	\begin{subfigure}{.35\textwidth}
		\begin{algorithmic}[1]{}
			\Function{foo}{\ldots} \label{code:begin}
			\Statex \ldots
			\State x = ptrExp \label{code:readRef}
			\Statex \ldots
			\EndFunction
		\end{algorithmic}
	\end{subfigure}
    \begin{subfigure}{.60\textwidth}
		\begin{algorithmic}[1]{}
			\State temp = ptrExp
			\State version = temp.getVersion()
			\If{y.isLocked()}
				\State \textbf{goto} \ref{code:begin} 
				\Comment Restart Operation
			\EndIf
			\State readSet.add(temp,version) 
			\State x = temp
		\end{algorithmic}
	\end{subfigure}
	\caption{Read phase transformation, the code in line \ref{code:readRef} 
	is replaced with the code on the right. 
			\label{figure:readPhaseTransformation}}
\end{figure}


\begin{figure}
	\begin{algorithmic}[1]{}
	\Function{foo}{\ldots}
			\Statex \ldots
			\Statex \Comment begin \readSet validation
		\ForAll{ (ref,version)$\in$ readSet}
			\If{ref.isLocked() \textbf{and} ref.lockedBy!= self} 
				\State \textbf{goto} \ref{code:begin} 
				\Comment Restart Operation
			\EndIf
			\If{version!= ref.getVersion()}
				\State \textbf{goto} \ref{code:begin} 
				\Comment Restart Operation
			\EndIf
		\EndFor
		\Statex \ldots
		\EndFunction
	\end{algorithmic}
	\caption{\readSet validation \label{figure:readSetValidation}}
\end{figure}
%END CODE 
\section{Algorithm's Correctness} 

We will prove that if the original locking protocol is 
conflict-serializable then our algorithm is conflict-serializable.

Let $\pi$ be an execution of our optimistic automation on a 
sequential algorithm. We will construct an execution $\pi_{LP}$ 
which is an execution following the original locking protocol. 
We will prove that both executions are conflict-equivalent. 
Since any execution of the original locking protocol
is conflict-serializable, then $\pi$ is conflict-serializable. 

Let $p_1,p_2,\ldots,p_n$ be the operations $\in\pi$ ordered by the 
order of execution of the first step of a successful \readSet 
validation. (If some operation does not have such point we omit it).
Let $\pi_{LP} = \pi_{lp1},\pi_{1},\ldots,\pi_{lpi},\pi_{i}$ where 
$\pi_{lpi}$ is a $p_i$-only execution of original locking protocol 
until $p_i$ holds locks only on the local variable locked
in the validation phase of $p_i \in \pi$, and $\pi_i$ is
the interval of $\pi$ starting from the return from the validation of
$pi$ until the first step of the successful \readSet validation of 
$p_{i+1}$ that includes only the operations by $\{p_1,\ldots,p_i\}$.
In other words, we replace the read-phase and validation phase with 
an execution of the original locking protocol, 
taking place at the point just before the \readSet validation starts. 

%TODO connect the requirement on the locking protocol to the construction.
\begin{lemma}
The construction of $\pi_{LP}$ is feasible.  
\end{lemma}
\begin{proof}
Proof by induction on $p_1,p_2,\ldots,p_n$. Base case is immediate. 

Let $\pi' = \pi_{lp1},\pi_{1},\ldots,\pi_{lpk-1},\pi_{k-1}$ be the feasible
construction so far, and let $p_{k}$ be the next operation to be 
added. 

Assume by contradiction that $\pi'\cdot\pi_{lpk}\cdot\pi_{k}$ 
cannot be constructed, thus, some object $v$ that $p_{k}$ locks 
in $\pi_{lpk}$ is already locked 
by $p_j \in \{p_1,p_2,\ldots,p_{k-1}\}$ in
the last configuration of $\pi'$. 
If $p_j$ locked $v$ before $p_k$ read $v$ for the first time, 
then $v$ was locked during the read phase of $p_k$, 
in contradiction to $p_k$ reaching its validation. 
Otherwise, $p_j$ locked $v$ after $p_k$ read $v$. 
If $v$ is still locked during the validation of $p_k$ then 
the validation will fail, contradiction. Alternatively, $v$ 
was unlocked by $p_j$ before $p_k$ validated $v$, 
its version incremented to a version bigger 
than the local version read by $p_k$, 
contradicting the successful validation of $p_k$.  
\end{proof}

\begin{lemma}
$\pi_{LP}$ is conflict-equivalent to $\pi$
\end{lemma}
\begin{proof}
Each operation performs a double collect on all the values it reads. 
The first collect is the read phase and the second is the \readSet 
validation of the validation phase. Since validation was successful, 
both collect are identical, meaning that the values of the \readSet
do not change from the return of the last read of the read phase,
until the first read of the \readSet validation. Therefore, executing 
the original locking of $p_k$ after $\pi' =
\pi_{lp1},\pi_{1},\ldots,\pi_{lpk-1},\pi_{k-1}$ is conflict-equivalent
to the original read phase. The read-write phase remains unchanged, 
maintaining conflict-equivalence to $\pi$.
\end{proof}
\section{Evaluation}\label{sec:eval}
In order to apply our approach to a data structure, a
''black-box'' pessimistic locking is required. One such 
approach was presented in \cite{Gueta2011}, automatically 
applying domination locking protocol to forest based data structures.  
We applied domination locking followed by our optimistic 
transformation to two tree based data structures, 
a simple unbalanced binary search tree 
and a treap (randomized binary search tree) \cite{AragonS1989}.

\paragraph{Setup}
We compared the performance of our automatic implementations, 
the automatic binary search tree (\autoTree) and the automatic
treap (\autoTreap), to the following custom tailored approaches: 
\begin{itemize}
\item \danaTree - The locked-based 
				unbalanced tree of Drachsler at al.\cite{DrachslerVY2014}. 
\item \danaAVL - The locked-based relaxed balanced AVL tree of 
				Drachsler et al.\cite{DrachslerVY2014}.
\item \bronson - The locked based relaxed balanced AVL tree
				of Bronson et al.\cite{BronsonCCO2010}.
\item \skiplist - The non-blocking skip-list by Doug 
				Lea included in the 
				the Java standard library.
\end{itemize}

We also compared our algorithms to previous automatic approaches, 
mainly global locking (\globalTree, \globalTreap) 
and domination locking (\domTree, \domTreap). 

%TODO 
We ran our experiments on a \ldots

We evaluated the performance on a variety of workloads, 
each workload is defined by the percentage of read-only
operations (\getOP queries) and the remaining operations 
are divided equally between insert and delete operations.
Our workloads include heavy read-only workloads
(100\%,70\% \getOP operations), medium read-only workload 
(50\% \getOP operations) and update only workload
(0\% \getOP operations). 

We used two key ranges $[0,2\cdot10^5]$ and $[0,2\cdot10^6]$,
for each range, the tree was pre-filled until the tree size was 
within 5\% of half the key range.   

We ran five seconds trials measuring the total throughput
(number of operations per second) of all threads.
During the trial, each thread continuously executed randomly
chosen operations according to the workload distribution 
using uniformly random keys from the key range.  
We ran every trial 7 times, we report the average throughput
while eliminating outliers.

\paragraph{Results} Figure \ref{evaluation:results:unbalanced} 
reports the throughput of unbalanced data structures and Figure 
\ref{evaluation:results:balanced} reports
the throughput of the balanced data structures. 


\begin{figure*}
\begin{center}
\begin{tikzpicture}
\begin{axis}[mystyle,unbalanced]
\addplot [blues3,mark=square*] table [x={threads}, y={TimeoutBinaryTree}]
{results/100C20K.txt}; 
\addplot [orange,mark=diamond*] table [x={threads}, y={Danaunbalanced}]
{results/100C20K.txt}; 
\addplot [gray,mark=asterisk] table [x={threads}, y={GlobalLockBinaryTree}]
{results/100C20K.txt};
\addplot [black,mark=x] table [x={threads}, y={DominationLockingBinaryTree}]
{results/100C20K.txt};
\end{axis}
\end{tikzpicture}
\begin{tikzpicture}
\begin{axis}[mystyle,unbalanced]
\addplot [blues3,mark=square*] table [x={threads}, y={TimeoutBinaryTree}]
{results/100C200K.txt}; 
\addplot [orange,mark=diamond*] table [x={threads}, y={Danaunbalanced}]
{results/100C200K.txt}; 
\addplot [gray,mark=asterisk] table [x={threads}, y={GlobalLockBinaryTree}]
{results/100C200K.txt};
\addplot [black,mark=x] table [x={threads}, y={DominationLockingBinaryTree}]
{results/100C200K.txt};
\end{axis}
\end{tikzpicture}
\begin{tikzpicture}
\begin{axis}[mystyle,unbalanced]
\addplot [blues3,mark=square*] table [x={threads}, y={TimeoutBinaryTree}]
{results/100C2000K.txt}; 
\addplot [orange,mark=diamond*] table [x={threads}, y={Danaunbalanced}]
{results/100C2000K.txt}; 
\addplot [gray,mark=asterisk] table [x={threads}, y={GlobalLockBinaryTree}]
{results/100C2000K.txt};
\addplot [black,mark=x] table [x={threads}, y={DominationLockingBinaryTree}]
{results/100C2000K.txt};
%\addplot  [magenta,mark=pentagon] table [x={threads}, y={rb}]
% {results/100C200K.txt};
\end{axis}
\end{tikzpicture}

\begin{tikzpicture}
\begin{axis}[mystyle,unbalanced]
\addplot [blues3,mark=square*] table [x={threads}, y={TimeoutBinaryTree}]
{results/70C20K.txt}; 
\addplot [orange,mark=diamond*] table [x={threads}, y={Danaunbalanced}]
{results/70C20K.txt}; 
\addplot [gray,mark=asterisk] table [x={threads}, y={GlobalLockBinaryTree}]
{results/70C20K.txt};
\addplot [black,mark=x] table [x={threads}, y={DominationLockingBinaryTree}]
{results/70C20K.txt};
\end{axis}
\end{tikzpicture}
\begin{tikzpicture}
\begin{axis}[mystyle,unbalanced]
\addplot [blues3,mark=square*] table [x={threads}, y={TimeoutBinaryTree}]
{results/70C200K.txt}; 
\addplot [orange,mark=diamond*] table [x={threads}, y={Danaunbalanced}]
{results/70C200K.txt}; 
\addplot [gray,mark=asterisk] table [x={threads}, y={GlobalLockBinaryTree}]
{results/70C200K.txt};
\addplot [black,mark=x] table [x={threads}, y={DominationLockingBinaryTree}]
{results/70C200K.txt};
\end{axis}
\end{tikzpicture}
\begin{tikzpicture}
\begin{axis}[mystyle,unbalanced]
\addplot [blues3,mark=square*] table [x={threads}, y={TimeoutBinaryTree}]
{results/70C2000K.txt}; 
\addplot [orange,mark=diamond*] table [x={threads}, y={Danaunbalanced}]
{results/70C2000K.txt}; 
\addplot [gray,mark=asterisk] table [x={threads}, y={GlobalLockBinaryTree}]
{results/70C2000K.txt};
\addplot [black,mark=x] table [x={threads}, y={DominationLockingBinaryTree}]
{results/70C2000K.txt};
%\addplot  [magenta,mark=pentagon] table [x={threads}, y={rb}]
% {results/100C200K.txt};
\end{axis}
\end{tikzpicture}

\begin{tikzpicture}
\begin{axis}[mystyle,unbalanced]
\addplot [blues3,mark=square*] table [x={threads}, y={TimeoutBinaryTree}]
{results/50C20K.txt}; 
\addplot [orange,mark=diamond*] table [x={threads}, y={Danaunbalanced}]
{results/50C20K.txt}; 
\addplot [gray,mark=asterisk] table [x={threads}, y={GlobalLockBinaryTree}]
{results/50C20K.txt};
\addplot [black,mark=x] table [x={threads}, y={DominationLockingBinaryTree}]
{results/50C20K.txt};
\end{axis}
\end{tikzpicture}
\begin{tikzpicture}
\begin{axis}[mystyle,unbalanced]
\addplot [blues3,mark=square*] table [x={threads}, y={TimeoutBinaryTree}]
{results/50C200K.txt}; 
\addplot [orange,mark=diamond*] table [x={threads}, y={Danaunbalanced}]
{results/50C200K.txt}; 
\addplot [gray,mark=asterisk] table [x={threads}, y={GlobalLockBinaryTree}]
{results/50C200K.txt};
\addplot [black,mark=x] table [x={threads}, y={DominationLockingBinaryTree}]
{results/50C200K.txt};
\end{axis}
\end{tikzpicture}
\begin{tikzpicture}
\begin{axis}[mystyle,unbalanced]
\addplot [blues3,mark=square*] table [x={threads}, y={TimeoutBinaryTree}]
{results/50C2000K.txt}; 
\addplot [orange,mark=diamond*] table [x={threads}, y={Danaunbalanced}]
{results/50C2000K.txt}; 
\addplot [gray,mark=asterisk] table [x={threads}, y={GlobalLockBinaryTree}]
{results/50C2000K.txt};
\addplot [black,mark=x] table [x={threads}, y={DominationLockingBinaryTree}]
{results/50C2000K.txt};
%\addplot  [magenta,mark=pentagon] table [x={threads}, y={rb}]
% {results/100C200K.txt};
\end{axis}
\end{tikzpicture}

\begin{tikzpicture}
\begin{axis}[mystyle,unbalanced]
\addplot [blues3,mark=square*] table [x={threads}, y={TimeoutBinaryTree}]
{results/0C20K.txt}; 
\addplot [orange,mark=diamond*] table [x={threads}, y={Danaunbalanced}]
{results/0C20K.txt}; 
\addplot [gray,mark=asterisk] table [x={threads}, y={GlobalLockBinaryTree}]
{results/0C20K.txt};
\addplot [black,mark=x] table [x={threads}, y={DominationLockingBinaryTree}]
{results/0C20K.txt};
\end{axis}
\end{tikzpicture}
\begin{tikzpicture}
\begin{axis}[mystyle,unbalanced]
\addplot [blues3,mark=square*] table [x={threads}, y={TimeoutBinaryTree}]
{results/0C200K.txt}; 
\addplot [orange,mark=diamond*] table [x={threads}, y={Danaunbalanced}]
{results/0C200K.txt}; 
\addplot [gray,mark=asterisk] table [x={threads}, y={GlobalLockBinaryTree}]
{results/0C200K.txt};
\addplot [black,mark=x] table [x={threads}, y={DominationLockingBinaryTree}]
{results/0C200K.txt};
\end{axis}
\end{tikzpicture}
\begin{tikzpicture}
\begin{axis}[mystyle,unbalanced]
\addplot [blues3,mark=square*] table [x={threads}, y={TimeoutBinaryTree}]
{results/0C2000K.txt}; 
\addplot [orange,mark=diamond*] table [x={threads}, y={Danaunbalanced}]
{results/0C2000K.txt}; 
\addplot [gray,mark=asterisk] table [x={threads}, y={GlobalLockBinaryTree}]
{results/0C2000K.txt};
\addplot [black,mark=x] table [x={threads}, y={DominationLockingBinaryTree}]
{results/0C2000K.txt};
%\addplot  [magenta,mark=pentagon] table [x={threads}, y={rb}]
% {results/100C200K.txt};
\end{axis}
\end{tikzpicture}

\ref{unbalancedLegened}
\end{center}
\caption{Throughput of unbalanced data
structures.\label{evaluation:results:unbalanced}}
\end{figure*}


\begin{figure*}
\begin{center}
\begin{tikzpicture}
\begin{axis}[mystyle,balanced,
				title=100\% read-only operations,
				ylabel = {Key range [$0,2\cdot10^4$]},
				]
\addplot [blues3,mark=square*] table [x={threads}, y={TimeoutTreap}]
{results/100C20K.txt}; 
\addplot [orange,mark=diamond*] table [x={threads}, y={DanaAVL}]
{results/100C20K.txt}; 
\addplot [darkspringgreen,mark=triangle*] table [x={threads}, y={Bronson}]
{results/100C20K.txt}; 
\addplot [coralred,mark=*] table [x={threads}, y={JavaSkipList}]
{results/100C20K.txt};
\addplot [gray,mark=asterisk] table [x={threads}, y={GlobalLockTreap}]
{results/100C20K.txt};
\addplot [black,mark=x] table [x={threads}, y={DominationLockingTreap}]
{results/100C20K.txt};
\end{axis}
\end{tikzpicture}
\begin{tikzpicture}
\begin{axis}[mystyle,balanced,title=50\% read-only operations]
\addplot [blues3,mark=square*] table [x={threads}, y={TimeoutTreap}]
{results/50C20K.txt}; 
\addplot [orange,mark=diamond*] table [x={threads}, y={DanaAVL}]
{results/50C20K.txt}; 
\addplot [darkspringgreen,mark=triangle*] table [x={threads}, y={Bronson}]
{results/50C20K.txt}; 
\addplot [coralred,mark=*] table [x={threads}, y={JavaSkipList}]
{results/50C20K.txt};
\addplot [gray,mark=asterisk] table [x={threads}, y={GlobalLockTreap}]
{results/50C20K.txt};
\addplot [black,mark=x] table [x={threads}, y={DominationLockingTreap}]
{results/50C20K.txt};
\end{axis}
\end{tikzpicture}
\begin{tikzpicture}
\begin{axis}[mystyle,balanced,title=0\% read-only operations]
\addplot [blues3,mark=square*] table [x={threads}, y={TimeoutTreap}]
{results/0C20K.txt}; 
\addplot [orange,mark=diamond*] table [x={threads}, y={DanaAVL}]
{results/0C20K.txt}; 
\addplot [darkspringgreen,mark=triangle*] table [x={threads}, y={Bronson}]
{results/0C20K.txt}; 
\addplot [coralred,mark=*] table [x={threads}, y={JavaSkipList}]
{results/0C20K.txt};
\addplot [gray,mark=asterisk] table [x={threads}, y={GlobalLockTreap}]
{results/0C20K.txt};
\addplot [black,mark=x] table [x={threads}, y={DominationLockingTreap}]
{results/0C20K.txt};
\end{axis}
\end{tikzpicture}


\begin{tikzpicture}
\begin{axis}[mystyle,balanced, 
				ylabel = {Key range [$0,2\cdot10^6$]}, ]
\addplot [blues3,mark=square*] table [x={threads}, y={TimeoutTreap}]
{results/100C2000K.txt}; 
\addplot [orange,mark=diamond*] table [x={threads}, y={DanaAVL}]
{results/100C2000K.txt}; 
\addplot [darkspringgreen,mark=triangle*] table [x={threads}, y={Bronson}]
{results/100C2000K.txt}; 
\addplot [coralred,mark=*] table [x={threads}, y={JavaSkipList}]
{results/100C2000K.txt};
\addplot [gray,mark=asterisk] table [x={threads}, y={GlobalLockTreap}]
{results/100C2000K.txt};
\addplot [black,mark=x] table [x={threads}, y={DominationLockingTreap}]
{results/100C2000K.txt};
\end{axis}
\end{tikzpicture}
\begin{tikzpicture}
\begin{axis}[mystyle,balanced]
\addplot [blues3,mark=square*] table [x={threads}, y={TimeoutTreap}]
{results/50C2000K.txt}; 
\addplot [orange,mark=diamond*] table [x={threads}, y={DanaAVL}]
{results/50C2000K.txt}; 
\addplot [darkspringgreen,mark=triangle*] table [x={threads}, y={Bronson}]
{results/50C2000K.txt}; 
\addplot [coralred,mark=*] table [x={threads}, y={JavaSkipList}]
{results/50C2000K.txt};
\addplot [gray,mark=asterisk] table [x={threads}, y={GlobalLockTreap}]
{results/50C2000K.txt};
\addplot [black,mark=x] table [x={threads}, y={DominationLockingTreap}]
{results/50C2000K.txt};
\end{axis}
\end{tikzpicture}
\begin{tikzpicture}
\begin{axis}[mystyle,balanced]
\addplot [blues3,mark=square*] table [x={threads}, y={TimeoutTreap}]
{results/0C2000K.txt}; 
\addplot [orange,mark=diamond*] table [x={threads}, y={DanaAVL}]
{results/0C2000K.txt}; 
\addplot [darkspringgreen,mark=triangle*] table [x={threads}, y={Bronson}]
{results/0C2000K.txt}; 
\addplot [coralred,mark=*] table [x={threads}, y={JavaSkipList}]
{results/0C2000K.txt};
\addplot [gray,mark=asterisk] table [x={threads}, y={GlobalLockTreap}]
{results/0C2000K.txt};
\addplot [black,mark=x] table [x={threads}, y={DominationLockingTreap}]
{results/0C2000K.txt};
\end{axis}
\end{tikzpicture}
\ref{balancedLegened}
\end{center}
\caption{Throughput of balanced data
structures.\label{evaluation:results:balanced}}
\end{figure*}
\section{Related Work}\label{sec:related}
\paragraph{Concurrent Data Structures}
Many sophisticated concurrent data structures (e.g., \cite{ArbelA2014,DrachslerVY2014,NatarajanM2014,BrownER2014,CrainGR2013,BraginskyP2012,
AfekKKMT2012,EllenFRB2010,BronsonCCO2010,HerlihyLLS2007,Michael:1996})
were developed and used in concurrent software systems~\cite{Ohad:OOPSLA11}.
Implementing efficient synchronization for such data structures is considered a challenging and error-prone task~\cite{Ohad:OOPSLA11,Doh:SPAA04,Jin:2012}.
As a result, concurrent data structures are manually implemented by concurrency experts.
This paper shows that (in some cases) an automatic algorithm can produce synchronization which is comparable to synchronization implemented by experts.

\paragraph{Locking Protocols}
Locking protocols are used in databases and shared memory systems to guarantee correctness
of concurrently executing transactions~\cite{Weikum:2001,BHG:Book87}.
Our approach can be seen as a way to extend many existing locking protocols by combining them with an optimistic concurrency control.
In particular, our approach extends the following locking protocols:
two-phase~\cite{Eswaran:1976}, tree locking~\cite{SilberschatzK1980}, DAG locking~\cite{CH:PODS95} and domination locking~\cite{Gueta2011}.
We demonstrate this by showing that extending the  domination locking protocol enables producing efficient concurrency control for
dynamic data structures.


\paragraph{Lock Inference Algorithms}
There has been a lot of work on automatically inferring locks for transactions.
Most of the algorithms in the literature infer locks for following the two-phase
locking protocol~\cite{MZGB:POPL06,Emmi06POPL,gudka2012lock,CCG:PLDI08,HFP:TRANSACT06,CGE:CC08}.
Our approach can potentially be used to optimized the synchronization produced by these algorithms.
For example, for the algorithms that employ a two-phase variant in which all locks are acquired at the beginning of transactions (e.g.,~\cite{gudka2012lock,CCG:PLDI08}),
our approach can be used to defer the locking (e.g., to just before the first write operation) and to eliminate some of the locking operations.


\paragraph{Transactional Memory}
Transactional memory approaches (TMs) dynamically resolve inconsistencies
and deadlocks by rolling back partially completed transactions.
%
Unfortunately, in spite of a lot of effort and many TM implementations (see~\cite{HLR:SLCA2010}), existing TMs
have not been widely adopted due to various concerns~\cite{DuffyTM2010,Cascaval:2008,mckenneyParallel}, including high runtime overhead,
poor performance and limited ability to handle irreversible operations.
In particular, modern concurrent programs (and concurrent data structures) are typically based on hand-crafted synchronization, rather than  on a TM approach~\cite{Ohad:OOPSLA11}.

In a sense, our approach can be seen as a specialized TM approach that can be practically used to handle concurrent data structure.


%\paragraph{Lock Elision for Read-Only Transactions}
\paragraph{Lock Elision}
Our approach is inspired by the idea of \emph{sequential locks}~\cite{mckenneyParallel} and the approach presented in~\cite{Nakaike:2010}.
But  in contrast to the approaches in \cite{mckenneyParallel,Nakaike:2010} which are designed to handle read-only transactions,
our approach handles read-only prefixes of transactions that update the shared memory. 
Moreover, using these approaches for a highly-contended data structure (as in Section~\ref{sec:eval}) will provide limited performance, 
because each update transaction causes all the read-only transactions to abort.

There are some transactional memory techniques to elide locks from arbitrary critical sections (e.g.,~\cite{Rajwar:2002:TLE:635508.605399,Roy:2009:RSS:1519065.1519094,Afek:2014:SHL:2611462.2611482}).
In these techniques a transaction executes the critical section speculatively without acquiring the lock.
When a transaction is aborted, it can acquire the lock and execute the critical section non-speculatively.
In contrast to our approach, these techniques cannot combine speculative and non-speculative execution of the same transaction.







 
\bibliography{myRef}
\bibliographystyle{plain}
\end{document}
