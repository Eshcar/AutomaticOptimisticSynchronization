%%This is a very basic article template.
%%There is just one section and two subsections.

\title{Scalable Automatic Synchronization Combining Optimism with Pessimism}
\documentclass{article}
\usepackage{xspace}
\usepackage{algorithm}
\usepackage[noend]{algpseudocode}
\usepackage{caption}
\usepackage{subcaption}

%-------------Theorem Definitions ---------------%
\newtheorem{theorem}{Theorem}[section]
\newtheorem{lemma}[theorem]{Lemma}
\newtheorem{proposition}[theorem]{Proposition}
\newtheorem{corollary}[theorem]{Corollary}

\newenvironment{proof}[1][Proof]{\begin{trivlist}
\item[\hskip \labelsep {\bfseries #1}]}{\qedsymb\end{trivlist}}
\newenvironment{definition}[1][Definition]{\begin{trivlist}
\item[\hskip \labelsep {\bfseries #1}]}{\end{trivlist}}
\newenvironment{example}[1][Example]{\begin{trivlist}
\item[\hskip \labelsep {\bfseries #1}]}{\end{trivlist}}
\newenvironment{remark}[1][Remark]{\begin{trivlist}
\item[\hskip \labelsep {\bfseries #1}]}{\end{trivlist}}

\newcommand{\qed}{\nobreak \ifvmode \relax \else
      \ifdim\lastskip<1.5em \hskip-\lastskip
      \hskip1.5em plus0em minus0.5em \fi \nobreak
      \vrule height0.75em width0.5em depth0.25em\fi}
\newcommand{\qedsymb}{\hfill{\rule{2mm}{2mm}}}

%--------------------------------------------------%

\newcommand{\code}[1]{\textsf{#1}}
\newcommand{\readV}{\code{read\_version}\xspace} 
\newcommand{\readSet}{\code{read\_set}\xspace}
\newcommand{\writeV}{\code{write\_version}\xspace}

\newcommand{\reqI}{\textbf{LPR1}\xspace}
\newcommand{\reqII}{\textbf{LPR2}\xspace}

\begin{document}
\maketitle

\abstract{We present an \emph{automatic} approach for 
parallelizing sequential data structures in a way that 
is both safe and scalable. While there exist pessimistic 
transformations that make code thread-safe by adding 
(either global or fine-grained) locks, this approach is 
limited in its performance due to synchronization bottlenecks, 
for example, locking the root in a tree data structure. 
In this paper, we improve the performance and scalability 
of such synthesized code by reducing bottlenecks. 
Specifically, we present an automatic approach to eliminate 
many of the locking steps, relying instead on optimistic 
partial traversals of the data structure. 
We realize our approach for tree data structures, 
using the domination locking technique. 
The resulting code scales well, 
significantly outperforms pessimistic approaches, 
and achieves performance close to those of 
custom-tailored concurrent data structures. 
Our work thus shows the promise that automated approaches 
bear for overcoming the difficulty involved in manually 
hand crafting concurrent data structures. 
}
\section{Introduction}

The steady increase in the number of  cores in today's computers is driving software developers to allow more parallelism. 
Indeed, many recent works have developed scalable concurrent data structures~\cite{bronson,dana,citrus,etc}. 
Such efforts are often very successful, achieving performance that scales linearly with the number of threads. 
Nevertheless, each of these project generally focuses on a single data structure 
(for example, a binary search tree~\cite{citrus} or \Idit{give another example}) and manually optimizes its implementation. 
Proving the correctness of such custom-tailored data structures is painstaking 
(for example, the proofs of \cite{x,y,z} are XX,YY, and ZZ pages long \Idit{add info}, respectively). 
We propose an approach to replace this labor-intensive process by automatic means.

One way to automatically convert a sequential data structure into a correct (thread-safe) concurrent one using locks. 
The trivial way to do so is to add a single global lock protecting the entire data structure 
(as in \emph{synchronized methods} in Java$^{TM}$, for example), but this allows no parallelism whatsoever. 
A more sophisticated approach can instrument the code (at compile time) and add fine-grained lock and unlock instructions~\cite{domination,tree-locking,dag,etc}. Such methods are applicable to certain data structure families, for example, Domination Locking~\cite{domination} is applicable to all trees or forests  (including binary search trees, BTrees, Treaps, etc \Idit{more?}), and employs a variant of hand-over-hand locking~\cite{hand-over-hand}, acquiring and releasing locks as it goes down the tree.  Other approaches are applicable to DAGS~\cite{dag-locking}. Unfortunately, to date, solutions of this sort scale poorly. This is due to synchronization bottlenecks such as the root of the tree, which is locked by all operations.

In this paper, we circumvent such synchronization bottlenecks via judicious use of optimism. 
Specifically, we replace many (but not all) locks with speculative execution and later re-validation. 
If re-validation fails, the speculative phase is restarted. 
In striking the balance between optimism and pessimism, we exploit the common nature of data structure operations, 
which typically begin by traversing the data structure to a designated location, and then perform (mostly local) updates at that location. 
Our optimistic execution is limited to the initial read-only part of the code (the data structure traversal)\footnote{Our solution may be seen as a form of software lock elision for read-only operation prefixes.}. 
Unlike most software transactional memory approaches~\cite{stm,tls},  
our synthesized code neither speculatively modifies shared memory contents, nor does it defer writes. 
Hence it never needs to rollback, and saves the overhead for tracking writes in dedicated data structures. 
Further comparison with related work appears in Section~\ref{sec:related}.

Our approach works as follows: Given a sequential data structure implementation, 
it first invokes a given (black-box) mechanism that instruments the code and adds  
fine-grained locks, e.g.,~\cite{domination,tree-locking,dag}. We assume that the locking protocol 
allows early release in the sense that it no longer holds locks on parts of the data structure that it has finished traversing; 
our assumptions are detailed in Section~\ref{sec:model}. We then invoke our algorithm, detailed in Section~\ref{sec:algorithm}, which 
(1) adds version numbers to shared memory objects, 
(2) identifies the read-only prefix of the code, 
(3) replaces locks in the read-only prefix with tracking of the read objects' versions, and 
(4) introduces appropriate re-validation mechanisms.  
Re-validation occurs at the end of the read-only phase, as well as during timeouts and exceptions. 
The latter addresses exceptions and infinite loops that may arise when an operation sees an inconsistent view of the data structure. 
Our code transformation is general, in the sense that it applies to any data structure for which an appropriate locking protocol exists, 
as proven in Section ~\ref{sec:proof}. The transformation is trivial to implement at compile time.

We realize our approach with the Domination Locking scheme~\cite{domination}, 
which is applicable to tree and forest data structures. 
We apply the appropriate code transformations to balanced and unbalanced tree data structure implementations in Java$^{TM}$. 
In Section~\ref{sec:eval} we evaluate the resulting code on a 32-core machine, 
and compare it to fully pessimistic as well as state-of-the-art hand-crafted data structure implementations~\cite{dana,bronson}. 
Our results show that the optimistic approach successfully overcomes synchronization bottlenecks, 
allows the synthesized code to scale linearly with the number of threads, 
and achieve comparable performance to that of custom-tailored solutions.

To conclude, this paper illustrates that automatic synchronization is a promising approach for bringing legacy code to emerging computer architectures. 
While this paper illustrates the method for tree data structures, we believe that the general direction may be more broadly applicable.
Section~\ref{sec:discussion} concludes the paper and touches on some directions for future work.

\section{Preliminaries}
%A \emph{concurrent system} (module) consists of a set of threads, 
\subsection{Shared Memory Model}
A \emph{module} defines a set of \emph{abstract data types} and a 
set of \emph{operations} that may be invoked by
clients of the module.
Each operation is invoked with possible parameters
and returns with a response. 
The invocation is a local step of a thread, 
followed by a execution of a sequence of atomic steps. 
Atomic step is either a computation on local 
variables or a primitive operation on shared variable.
Read and write to shared memory are 
denoted load and store.  
%a load (``\code{x=y.f}'') or 
%a store (``\code{x.f=y}'') operation on shared variables.
The parameters and local variables 
of an operation are private to the
invocation of the operation (thread local). 
There are no static or global variables shared by  
different invocations of the operations. 

A \emph{configuration} is an instantaneous representation 
of the system, including the state of the shared memory
and the local variables. In the \emph{initial configuration} 
all variables hold an initial value. 
An \emph{execution} is an alternating sequence of 
configurations and steps,
$C_0,s_1,\ldots,s_i,C_i,\ldots,$ 
where $C_0$ is the initial configuration,
and each configuration $C_i$ is the result of
executing step $s_i$ on configuration $C_{i-1}$.
 
An execution is \emph{non-interleaved} 
(abbreviated NI-execution) if primitive operation of 
different operations are not interleaved, i.e., 
for every pair of operations invocations $p_i \neq p_j$ 
either all primitive operations of $p_i$ come before any 
primitive operation of $p_j$, or vice versa.

\subsection{Locking Protocols}
Get some text-book definitions of locking protocols. 

In order for our algorithm to work, the initial locking
protocol must have one of the following properties: 

\begin{itemize}
\item The protocols allows early lock release. 
\item The protocols requires that all locks are acquired 
before the first store step.    
\end{itemize}


\subsection{Correctness Conditions}
Given an execution, we say that two primitive operations 
\emph{conflict} if (i) they are executed by two different threads, 
(ii) they access some common global variable or a heap allocated object 
(iii) at least one of the conflicting instruction is a write.


%Let $\pi$ be an execution and $op_i,op_j$ are primitive operations in $\pi$ the
% conflict relation of $\pi$ is: \\ $conf(\pi)=\{(op_i,op_j)\mid op_i$ and $op_j$ are in conflict and $op_i<_{\pi}op_j\}$\\

%Let $\pi$ be an execution the conflict graph $G(\pi)=(V,E)$ is defined by:\\ $V
% = \{$ all operations (threads) of $\pi\}$\\
%$E = \{ (t_i,t_j) \mid t_i\neq t_j$ and $\exists op_i\in t_i, \exists op_j\in
% t_j$ such that $(op_i,op_j)\in conf(\pi) \}$\\

Executions $\pi_1$ and $\pi_2$ are \emph{conflict-equivalent} 
if they include the same primitive operations and they 
agree on the order between conflicting operations. 
An execution is \emph{conflict-serializable} if it is 
conflict-equivalent with a non-interleaved execution.

%DOMINATION LOCKING
%\emph{Domination Locking Protocol} is a conflict serializable
%locking protocol presented in \cite{Gueta2011}. 
%It requires a fixed total order $\leq$ on all heap objects. 
%An execution satisfies the the Domination Locking protocol,
%with respect to $\leq$ if it satisfies the following conditions:

\section{Automatic Transformation}\label{sec:algorithm}

% What we do
Our goal is to automatically transform a given (lock-free) implementation of a sequential data structure to a correct and efficient concurrent one.
We achieve efficiency by carefully introducing a combination of pessimistic and optimistic concurrency control mechanisms into the code. The former are 
added by a given black-box code transformation that ensures some locking protocol. Our algorithm then replaces a subset of the pessimistic
steps with optimistic ones. This section first overviews our general approach to combining optimism and pessimism (Section~\ref{ssec:overview}), and
then delves into the details of our code transformation, illustrated by an example (Section~\ref{ssec:transformation}).

\subsection{Combining Optimism and Pessimism}\label{ssec:overview}

% Optimism
Generally speaking, optimistic concurrency control is a form of lock-free synchronization, which accesses shared variables without locks in the hope that they will not be modified by others before the end of the operation (or more generally, the transaction). To verify the latter, optimistic concurrency control relies on \emph{validation}, which is typically implemented using version numbers. If validation fails, the operation restarts. Optimistic execution of update operations requires either performing roll-back (reverting variables to their old values) upon validation failure, or deferring writes to commit time; both approach induce significant overhead \Idit{citation supporting this?}. 


%The main idea: 
The main idea behind our protocol is judicious use of optimistic synchronization to read 
shared variables without locks as long as no updates occur. Once an operation performs a store step, 
we start using the pessimistic (lock-based) protocol. In other words, we rely on validation at the end of the 
read-only prefix of an operation in order to render redundant all locks that would have been freed
before any change is made.
This scheme is particularly suitable for data structure implementations,
since the common behavior of their operations 
is to first traverse the data structure, and then 
perform (mostly local) modifications at the end of the operation.


Conceptually, our approach divides an operation into three phases: an optimistic \emph{read-only phase},
a pessimistic \emph{update phase} and a \emph{validation phase} that conjoins them. 
The read-only phase traverses the data structure without taking any locks, while maintaining sufficient information to later ensure the correctness of the traversal.
The update phase executes the original locking protocol. 
The validation phase bridges between the optimistic and pessimistic ones.
It first locks the objects required by
the locking protocol and then validates the correctness
of the of the read-only phase, allowing the 
update phase to run as if the an execution of the locking
protocol took place. If the validation fails, the operation 
restarts. In order to avoid livelock, we set a threshold on the number of restarts.
If the threshold is exceeded, the code falls back on pessimistic execution. 

\subsection{Transformation Details}\label{ssec:transformation}
To illustrate our transformation a simple code example is 
provided in Figure \ref{figure:transformation} 
with auxiliary functions in Figures \ref{figure::track} and 
\ref{figure::validate}.

We start our transformation by applying three preparation steps
to the given sequential algorithm.
First, we use the ``black box'' transformation to produce an
algorithm that ensures the locking protocol 
(Figure \ref{figure:transformation:before}). 
Second, we add version numbers to shared memory objects, 
incremented every time the object is locked. 
These version numbers are used 
to validate the correctness of the read-only phase during 
the validation phase as well as at timeouts and exceptions.
Lastly, we identify the read only prefix of each operation. 

Next we explain what the transformation does in each phase (Figure
\ref{figure:transformation:after}). 

\paragraph{Read-only Phase} 
  During this phase the operation maintains a \emph{locked set}
  and a \emph{read set}. 
  The locked set is used to track the lock and unlock steps
  of the locking protocol, thus, we replace each lock step 
  with a step adding the object to the locked set 
  (line \ref{code:lockedSet:add} of auxiliary function 
  track, Figure \ref{figure::track}). 
  and each unlock step with a step removing the object 
  from the locked set (line \ref{code:lockedSet:remove}). 
  (Since the locking protocol might allow re-entrant locks, 
  our locked set allows duplications and the remove 
  operation removes one of the duplications of the object).
  
The read set is used to track all objects accessed by the 
operation in order to later validate that this read set
belongs to a consistent view of shared memory.  
It contains a mapping between references to objects loaded 
to the local version at the time it was read. After 
objects are added to the read set they are checked to be 
unlocked, if the object is locked the operation restarts
from the beginning (lines
\ref{code:track:getVersion}-\ref{code:track:returnFalse} of auxiliary function
track, Figure \ref{figure::track}).

%Object are added to the read set immediately 
%after they are added to the locked set.  

%Incrementing the local version of an object is not 
%done atomically with the lock acquisition, thus, 
%the object is also checked to be unlocked, if it is locked
%the operation restarts from the beginning. 
%This also ensures that the operation never reads old values 
%with a new version. 
 
The read phase does not validate reads during its traversal, 
in order to avoid infinite loops, a timeout is set. 
If the operation reaches the timeout, a read set 
validation takes place (described in the validation phase), 
if it fails the operation restarts from the beginning.
  
  %The use of timeout does not guarantee that the 
  %operation reads a consistent snapshot of the memory, 
  %thus, \ldots 
  %A pseudo code of the transformation can be found in 
  %Figure \ref{figure:readPhaseTransformation}. 

\paragraph{Validation Phase} 
The validation phase is inserted to the code after
the read-only phase (lines
\ref{code:validateLockedSet}-\ref{code:validateGoto2}). 
It locks the objects in the locked set and validates the read set. 

To avoid deadlocks, the locks are acquired using a tryLock
operation, if the tryLock fails, the operation unlocks 
previously acquired locks and restarts from the beginning 
(lines \ref{code:validateLockedSet}-\ref{code:validateGoto1}). 
The calls to tryLock also increment the versions 
of the object,  these objects are also part of the 
operations read set. 
To avoid the operation invalidating itself, 
a successful tryLock operation is followed by incrementing
the version saved in the read set.  

The read set validation is performed as follows, 
each object saved in the read set is checked to be unlocked 
and that the current version matches the version saved in the 
read set (Figure \ref{figure::validate}). If this check fails, the operation releases all its
locks and restarts from the beginning
(lines \ref{code:validateReadSet}-\ref{code:validateGoto2}). 
This guarantees
that the object was not locked from the time it was read until
the time it was validated, since all operations write only to
locked nodes it follows that the object was not changed.  
The read set validation can be viewed as a double collect 
to all objects accessed by the read-only phase. 

%The order between checking the version and checking that the object is unlocked
%is important, as locking and incrementing versions is not atomic. 

%A pseudo code of the \readSet validation can be found in
%Figure \ref{figure:readSetValidation}.
  
\paragraph{Update Phase} 
This phase enforces the locking protocol
  while maintaining the local versions, i.e., the local version of 
  an object is incremented every time it is locked.  
  Once the update phase begins, the operation is guaranteed to to 
  finish without restarts.  


 

\Xomit{  
\subsection{Transformation Details}
We invoke the following steps to a code that implements 
fine grained locking and follows a correct locking
protocol.  

\begin{enumerate}
  \item Add version numbers to shared memory objects, 
  incremented every time the object is locked. These 
  version numbers are used to validate the correctness
  of the optimistic read phase during the validation phase
  as well as timeouts and exceptions.
  
  \item Identify the read only prefix of each data structure 
  operation. In the beginning of the read phase allocate 
  and initialize local \emph{read set} and \emph{timeout}
  objects. The read set object is a storage object used to save 
  references to objects along with version numbers. 
  The timeout object tracks the progress of the operation\ldots 
    
  \item Transform the read only prefix to optimistic traversal 
   by replacing lock steps \ldots 
   Incrementing the version is not atomic with the lock, thus, 
   the object is also checked to be unlocked. 

\end{enumerate}
}
%The use of timeout does not guarantee
%\emph{opacity}~\cite{GuerraouiK2008} or 
%\emph{validity}\cite{LevAriCK2014}. 

%The \readSet validation first checks that the node is unlocked,
%(or locked by the current operation), then it checks that the 
%current version is equal to the version saved in the \readSet. 
 
%ALGORITHM CODE
\renewcommand{\ttdefault}{pcr}
%\algrenewcommand\textkeyword{\texttt}
\algrenewcommand\algorithmicif{\texttt{if}}
\algrenewcommand\algorithmicthen{\texttt{then}}
\algrenewcommand\algorithmicfunction{\textsc{Function}}
\algrenewcommand\algorithmicforall{\texttt{for all}}
\algrenewcommand\algorithmicdo{\texttt{do}}
\algrenewcommand\textproc{\textit}

\begin{figure*}
	\begin{center}
	\begin{subfigure}{.49\textwidth}
		\begin{algorithmic}[1]{}
		{\ttfamily
			\Function{addThird}{Node new} \label{code:begin}
			\Statex ----------------------------
			\State \textbf{Node prev = head} 
			\State prev.lock()
			\State 
			\State \textbf{Node succ = prev.next}
			\State succ.lock()
			\State
			\State prev.unlock()
			\State \textbf{prev = succ}
			\State \textbf{succ = succ.next}
			\State succ.lock()
			\State
			\Statex ----------------------------
			\State
			\State
			\State
			\State
			\State
			\State
			\Statex ----------------------------
			\State \textbf{prev.next = new}
			\State \textbf{new.next = succ}
			\State prev.unlock()
			\State succ.unlock()
			\EndFunction
			}
		\end{algorithmic}
		\caption{ Code synthesized with hand over
		hand locking protocol} \label{figure:transformation:before} 
	\end{subfigure}
	\begin{subfigure}{.49\textwidth}
		\begin{algorithmic}[1]{}
		{\ttfamily
			\Function{addThird}{Node new} \label{code:begin}
			\Statex ----------------------------
			\Comment{\textrm{read-only phase}}
			\State \textbf{Node prev = head}
			\If{!track(prev)}
				\State {goto} \ref{code:begin} \label{code:readGhaseGoto1}
				%\Comment Restart Operation
			\EndIf
			\State \textbf{Node succ = prev.next}
			\If{!track(succ)}
				\State {goto} \ref{code:begin}  \label{code:readGhaseGoto2}
				%\Comment Restart Operation
			\EndIf		
			\State lockedSet.remove(prev) \label{code:lockedSet:remove}
			\State \textbf{prev = succ}
			\State \textbf{succ = succ.next}
			\If{!track(succ)}
				\State {goto} \ref{code:begin} \label{code:readGhaseGoto3}
				%\Comment Restart Operation
			\EndIf
			\Statex ----------------------------
			\Comment{\textrm{validation phase}}
			\If{!lockedSet.tryLockAll()} 	\label{code:validateLockedSet}	
				\State releaseAll()
				\State {goto} \ref{code:begin} \label{code:validateGoto1}
				%\Comment Restart Operation
			\EndIf	
			\If{!readSet.validate()} 		\label{code:validateReadSet}
				\State releaseAll()
				\State {goto} \ref{code:begin} \label{code:validateGoto2}
				%\Comment Restart Operation
			\EndIf	
			\Statex ----------------------------
			\Comment{\textrm{update phase}}
			\State \textbf{prev.next = new}
			\State \textbf{new.next = succ}			
			\State prev.unlock()
			\State succ.unlock()

			\EndFunction
			}
		\end{algorithmic}
		\caption{ Code synthesized with our
		automatic transformation}\label{figure:transformation:after}
	\end{subfigure}
	%\bigskip
	%\hline
	\end{center}
	\caption{Code example. 
	The original sequential code
	is in bold.
			\label{figure:transformation}}
\end{figure*}

\begin{figure}
\begin{algorithmic}[1]{}
		{\ttfamily
		\Function{track}{Node node}
		\State lockedSet.add(node) \label{code:lockedSet:add}
			\State long version = node.getVersion() \label{code:track:getVersion}
			\State readSet.add(node,version)  
			\If{node.isLocked()}
				\State return false \label{code:track:returnFalse}
				%\Comment Restart Operation
			\EndIf
			\State retrun true
		\EndFunction
		}
\end{algorithmic}
\caption{ In read-only phase, locking is replaced by 
tracking locks and read
objects' versions.
\label{figure::track}}
\end{figure}

\begin{figure}
\begin{algorithmic}[1]{}
		{\ttfamily
		\Function{validate}{}()
		\ForAll {Node node in readSet}
			\If{node.isLocked()}
				\State return false
			\EndIf
			\State long version = readSet.getVersion(node)
			\If{node.getVersion() != version}
				\State return false
			\EndIf
		\EndFor
		\State retrun true
		\EndFunction
		}
\end{algorithmic}
\caption{Read set validation.\label{figure::validate}}
\end{figure}


%END CODE 
\section{Algorithm's Correctness} 
We will prove that if the original locking protocol is 
conflict-serializable then our algorithm is conflict-serializable.

Let $\pi$ be an execution of our optimistic automation on a 
sequential algorithm. We will construct an execution $\pi_{LP}$ 
which is an execution following the original locking protocol. 
We will prove that both executions are conflict-equivalent. 
Since any execution of the original locking protocol
is conflict-serializable, then $\pi$ is conflict-serializable. 

Let $p_1,p_2,\ldots,p_n$ be the operations $\in\pi$ ordered by the 
order of execution of the first step of a successful \readSet 
validation. (If some operation does not have such point we omit it).
Let $\pi_{LP} = \pi_{lp1},\pi_{1},\ldots,\pi_{lpi},\pi_{i}$ where 
$\pi_{lpi}$ is a $p_i$-only execution of original locking protocol 
until $p_i$ holds locks only on the local variable locked
in the validation phase of $p_i \in \pi$, and $\pi_i$ is
the interval of $\pi$ starting from the return from the validation of
$pi$ until the first step of the successful \readSet validation of 
$p_{i+1}$ that includes only the operations by $\{p_1,\ldots,p_i\}$.
In other words, we replace the read-phase and validation phase with 
an execution of the original locking protocol, 
taking place at the point just before the \readSet validation starts. 

%TODO connect the requirement on the locking protocol to the construction.
\begin{lemma}
The construction of $\pi_{LP}$ is feasible.  
\end{lemma}
\begin{proof}
Proof by induction on $p_1,p_2,\ldots,p_n$. Base case is immediate. 

Let $\pi' = \pi_{lp1},\pi_{1},\ldots,\pi_{lpk-1},\pi_{k-1}$ be the feasible
construction so far, and let $p_{k}$ be the next operation to be 
added. 

Assume by contradiction that $\pi'\cdot\pi_{lpk}\cdot\pi_{k}$ 
cannot be constructed, thus, some object $v$ that $p_{k}$ locks 
in $\pi_{lpk}$ is already locked 
by $p_j \in \{p_1,p_2,\ldots,p_{k-1}\}$ in
the last configuration of $\pi'$. 
If $p_j$ locked $v$ before $p_k$ read $v$ for the first time, 
then $v$ was locked during the read phase of $p_k$, 
in contradiction to $p_k$ reaching its validation. 
Otherwise, $p_j$ locked $v$ after $p_k$ read $v$. 
If $v$ is still locked during the validation of $p_k$ then 
the validation will fail, contradiction. Alternatively, $v$ 
was unlocked by $p_j$ before $p_k$ validated $v$, 
its version incremented to a version bigger 
than the local version read by $p_k$, 
contradicting the successful validation of $p_k$.  
\end{proof}

\begin{lemma}
$\pi_{LP}$ is conflict-equivalent to $\pi$
\end{lemma}
\begin{proof}
Each operation performs a double collect on all the values it reads. 
The first collect is the read phase and the second is the \readSet 
validation of the validation phase. Since validation was successful, 
both collect are identical, meaning that the values of the \readSet
do not change from the return of the last read of the read phase,
until the first read of the \readSet validation. Therefore, executing 
the original locking of $p_k$ after $\pi' =
\pi_{lp1},\pi_{1},\ldots,\pi_{lpk-1},\pi_{k-1}$ is conflict-equivalent
to the original read phase. The read-write phase remains unchanged, 
maintaining conflict-equivalence to $\pi$.
\end{proof}
\section{Evaluation}\label{sec:eval}

\Xomit{
\paragraph{Implementation Details}
We implemented our automatic transformation in Java$^{TM}$, 
the following section highlight some technical details of 
the implementation.

Version numbers were added as a field in each object, 
incremented inside an acquire function. The
locked set and the read set can be implemented using several Java 
provided data structures. We chose to use arrays as they provide low 
overheads. Timeouts were implemented using a counter, incremented and checked
inside loops (backwards branches) and calls to functions.

Retries can be implemented using the exception mechanism,
that is already required to due to loss of internal consistency. 
However, this mechanism can hinder performance since retries occur 
more often than program exception. Instead we implemented an error mechanism
(inspired by C errno.h), an error object is sent to the operation from 
a wrapper function. If the operation needs to restart it sets the error and
returns. The wrapper checks the error object, if it is set the operation
is called again with a clean error object, otherwise the wrapper returns 
the operation return value. This wrapper is also used to count the retries
and fall back to the full locking protocol.

Operations were delimited with catch blocks, if an exception
is caught a read set validation takes place, 
if the read set validation passes it means that the exception  is inherent
to the original program and is thrown to higher levels, otherwise, the
exception might occurred due to unvalidated reads and the operation is
restarted. 
}

\paragraph{Setup}
In order to apply our approach to a data structure, a
black-box pessimistic locking scheme is required. One such 
approach, presented in \cite{Gueta2011}, automatically 
applies domination locking  to forest-based data structures.  
We follow this approach, and apply a domination locking protocol
followed by our transformation presented in Section~\ref{sec:algorithm}. 
We synthesized concurrent code from two example tree-based 
sequential data structures implemented in Java$^{TM}$: 
a simple unbalanced binary search tree, 
and a treap (randomized binary search tree) \cite{AragonS1989}.
We call the resulting data structures 
automatic binary search tree (\autoTree) and  automatic
treap (\autoTreap), respectively.

We compare the performance of these  automatic implementations
to the following custom tailored approaches: 
\begin{itemize}
\item \danaTree - The locked-based 
				unbalanced tree of Drachsler at al.~\cite{DrachslerVY2014}\footnote{Implementation provided by the authors.}. 
\item \danaAVL - The locked-based relaxed balanced AVL tree of 
				Drachsler et al.~\cite{DrachslerVY2014}\footnote{Implementation available at \\
				\texttt{https://github.com/logicalordering/trees}}.
\item \bronson - The locked based relaxed balanced AVL tree
				of Bronson et al.~\cite{BronsonCCO2010}\footnote{Implementation available at \\
				\texttt{https://github.com/nbronson/snaptree}}.
\item \skiplist - The non-blocking skip-list by Doug 
				Lea included in the 
				the Java$^{TM}$ standard library.
\end{itemize}

We further compare our algorithms to previous automatic approaches, 
namely global locking (\globalTree, \globalTreap) 
and domination locking (\domTree, \domTreap). 

We run our experiments on four Intel Xeon E5-4650 processors, 
each with 8 cores for a total of 32 threads 
(with hyper-threading disabled). 
We used Ubuntu 12.04.4 LTS and Java$^{TM}$ Runtime Environment (build
1.7.0\_51-b13) using the 64-Bit Server VM (build 24.51-b03, mixed mode).


We evaluate performance on a variety of workloads;
each workload is defined by the percentage of read-only
operations (\emph{contains} queries), whereas the remaining operations 
are divided equally between insert and delete.
Our workloads include heavy read-only workloads
(100\% contains operations), medium read-only workloads 
(50\% contains operations) and update only workloads
(0\% contains operations). 

We also consider two sizes of data structures, by using two key ranges
$[0,2\cdot10^4]$ and $[0,2\cdot10^6]$, for each range, the tree is pre-filled until the tree size is 
within 5\% of half the key range.   

We run five seconds trials measuring the total throughput
(number of operations per second) of all threads.
During the trial, each thread continuously executes randomly
chosen operations according to the workload distribution, 
using uniformly random keys from the key range.  
We ran every trial 7 times, and report the average throughput
after eliminating outliers.

\paragraph{Results} Figure \ref{evaluation:results:unbalanced} 
reports the throughput of unbalanced data structures and Figure 
\ref{evaluation:results:balanced} presents
the throughput of the balanced ones. We see that our semi-optimistic
solution is far superior to previous, fully-pessimistic, 
automated approaches. It successfully overcomes the bottlenecks
associated with lock contention, and in many scenarios comes close
to custom-tailored implementations.

The results for the read-only workload show the main overhead
of our automatic approach. By profiling the code, we learned 
that the bulk of this overhead stems from the need to track all read objects,
which is inherent to our automatic transformation. 
This is in contrast with the hand-crafted implementations,
which have no overhead on reads in this scenario, thanks to either 
wait-free reads (in \danaTree and \danaAVL), or optimistic validation (in \bronson). 
%, however, given knowledge on the semantics of
%the data structure a small modification to the automatic code 
%can be applied. 
%For example, we took the \autoTree and removed 
%the \readSet validation for a \getOP operation
%that found the required key. The new operation immediately returns 
%the value without the any validation. 
%Correctness in maintained because if an operation finds the node then
%it is reachable at some configuration during the operation interval, 
%given that the insert operation is correct 
%(never adds a node to detached part of the tree).
%The results of the new optimized code can be found in \ldots 
 
As the ration of updates in the workload increases, our automatic implementation 
closes this gap, and for some operation distributions and 
tree sizes even outperforms some of the hand-crafted algorithms.
These results show that our automatic transformation deals well with update contention. 
This is probably due to the fact that once
an update phase begins, the operation is not delayed due to concurrent 
read-only operations. 

Additionally, the results show that our automatic transformation 
works better on larger data structures. Indeed, in large data structures, 
update operations are more likely to operate on disjoint parts of 
the data, allowing high concurrency. This is especially important 
for the automatic transformation as updates  with overlapping 
locked sets might invalidate each other.  
 


\begin{figure*}
\begin{center}
\begin{tikzpicture}
\begin{axis}[mystyle,unbalanced,
title={\textbf{0\%read-only}},
ylabel = { \textbf{Big Tree ($\sim10^6$ elements)} \\ million ops/sec}]
\addplot [black,mark=square*] table [x={threads}, y={TimeoutBinaryTree}]
{results/0C2000K.txt}; 
\addplot [orange,mark=diamond] table [x={threads}, y={Danaunbalanced}]
{results/0C2000K.txt}; 
\addplot [gray,mark=o] table [x={threads}, y={GlobalLockBinaryTree}]
{results/0C2000K.txt};
\addplot [blues4,mark=x] table [x={threads}, y={DominationLockingBinaryTree}]
{results/0C2000K.txt};
\end{axis}
\end{tikzpicture}
\begin{tikzpicture}
\begin{axis}[mystyle,unbalanced,title={\textbf{50\% read-only}}]
\addplot [black,mark=square*] table [x={threads}, y={TimeoutBinaryTree}]
{results/50C2000K.txt}; 
\addplot [orange,mark=diamond] table [x={threads}, y={Danaunbalanced}]
{results/50C2000K.txt}; 
\addplot [gray,mark=o] table [x={threads}, y={GlobalLockBinaryTree}]
{results/50C2000K.txt};
\addplot [blues4,mark=x] table [x={threads}, y={DominationLockingBinaryTree}]
{results/50C2000K.txt};
\end{axis}
\end{tikzpicture}
\begin{tikzpicture}
\begin{axis}[mystyle,unbalanced,
title= { \textbf{100\% read-only}}]
\addplot [black,mark=square*] table [x={threads}, y={TimeoutBinaryTree}]
{results/100C2000K.txt}; 
\addplot [orange,mark=diamond] table [x={threads}, y={Danaunbalanced}]
{results/100C2000K.txt}; 
\addplot [gray,mark=o] table [x={threads}, y={GlobalLockBinaryTree}]
{results/100C2000K.txt};
\addplot [blues4,mark=x] table [x={threads}, y={DominationLockingBinaryTree}]
{results/100C2000K.txt};
\end{axis}
\end{tikzpicture}


\begin{tikzpicture}
\begin{axis}[mystyle,unbalanced, 
ylabel = { \textbf{Small Tree ($\sim10^4$ elements)} \\ million ops/sec},
xlabel={Threads}]
\addplot [black,mark=square*] table [x={threads}, y={TimeoutBinaryTree}]
{results/0C20K.txt}; 
\addplot [orange,mark=diamond] table [x={threads}, y={Danaunbalanced}]
{results/0C20K.txt}; 
\addplot [gray,mark=o] table [x={threads}, y={GlobalLockBinaryTree}]
{results/0C20K.txt};
\addplot [blues4,mark=x] table [x={threads}, y={DominationLockingBinaryTree}]
{results/0C20K.txt};
\end{axis}
\end{tikzpicture}
\begin{tikzpicture}
\begin{axis}[mystyle,unbalanced, xlabel={Threads}]
\addplot [black,mark=square*] table [x={threads}, y={TimeoutBinaryTree}]
{results/50C20K.txt}; 
\addplot [orange,mark=diamond] table [x={threads}, y={Danaunbalanced}]
{results/50C20K.txt}; 
\addplot [gray,mark=o] table [x={threads}, y={GlobalLockBinaryTree}]
{results/50C20K.txt};
\addplot [blues4,mark=x] table [x={threads}, y={DominationLockingBinaryTree}]
{results/50C20K.txt};
\end{axis}
\end{tikzpicture}
\begin{tikzpicture}
\begin{axis}[mystyle,unbalanced, xlabel={Threads}]
\addplot [black,mark=square*] table [x={threads}, y={TimeoutBinaryTree}]
{results/100C20K.txt}; 
\addplot [orange,mark=diamond*] table [x={threads}, y={Danaunbalanced}]
{results/100C20K.txt}; 
\addplot [gray,mark=o] table [x={threads}, y={GlobalLockBinaryTree}]
{results/100C20K.txt};
\addplot [blues4,mark=x] table [x={threads}, y={DominationLockingBinaryTree}]
{results/100C20K.txt};
\end{axis}
\end{tikzpicture}
\ref{unbalancedLegened}
\end{center}
\caption{Throughput of unbalanced data
structures.
\label{evaluation:results:unbalanced}}
\end{figure*}


\begin{figure*}
\begin{center}

\begin{tikzpicture}
\begin{axis}[mystyle,balanced,
 ylabel = { \textbf{Big Tree ($\sim10^6$ elements)} \\ million ops/sec},
 title={\textbf{0\% read-only}}]
\addplot [black,mark=square*] table [x={threads}, y={TimeoutTreap}]
{results/0C2000K.txt}; 
\addplot [orange,mark=diamond] table [x={threads}, y={DanaAVL}]
{results/0C2000K.txt}; 
\addplot [darkspringgreen,mark=triangle] table [x={threads}, y={Bronson}]
{results/0C2000K.txt}; 
\addplot [coralred,mark=asterisk] table [x={threads}, y={JavaSkipList}]
{results/0C2000K.txt};
\addplot [gray,mark=o] table [x={threads}, y={GlobalLockTreap}]
{results/0C2000K.txt};
\addplot [blues4,mark=x] table [x={threads}, y={DominationLockingTreap}]
{results/0C2000K.txt};
\end{axis}
\end{tikzpicture}
\begin{tikzpicture}
\begin{axis}[mystyle,balanced,title={\textbf{50\% read-only}}]
\addplot [black,mark=square*] table [x={threads}, y={TimeoutTreap}]
{results/50C2000K.txt}; 
\addplot [orange,mark=diamond] table [x={threads}, y={DanaAVL}]
{results/50C2000K.txt}; 
\addplot [darkspringgreen,mark=triangle] table [x={threads}, y={Bronson}]
{results/50C2000K.txt}; 
\addplot [coralred,mark=asterisk] table [x={threads}, y={JavaSkipList}]
{results/50C2000K.txt};
\addplot [gray,mark=o] table [x={threads}, y={GlobalLockTreap}]
{results/50C2000K.txt};
\addplot [blues4,mark=x] table [x={threads}, y={DominationLockingTreap}]
{results/50C2000K.txt};
\end{axis}
\end{tikzpicture}
\begin{tikzpicture}
\begin{axis}[mystyle,balanced, 
				 title={\textbf{100\% read-only}},]
\addplot [black,mark=square*] table [x={threads}, y={TimeoutTreap}]
{results/100C2000K.txt}; 
\addplot [orange,mark=diamond] table [x={threads}, y={DanaAVL}]
{results/100C2000K.txt}; 
\addplot [darkspringgreen,mark=triangle] table [x={threads}, y={Bronson}]
{results/100C2000K.txt}; 
\addplot [coralred,mark=asterisk] table [x={threads}, y={JavaSkipList}]
{results/100C2000K.txt};
\addplot [gray,mark=o] table [x={threads}, y={GlobalLockTreap}]
{results/100C2000K.txt};
\addplot [blues4,mark=x] table [x={threads}, y={DominationLockingTreap}]
{results/100C2000K.txt};
\end{axis}
\end{tikzpicture}

\begin{tikzpicture}
\begin{axis}[mystyle,balanced,
ylabel = { \textbf{Small Tree ($\sim10^4$ elements)} \\ million ops/sec},
	 xlabel={Threads}]
\addplot [black,mark=square*] table [x={threads}, y={TimeoutTreap}]
{results/0C20K.txt}; 
\addplot [orange,mark=diamond] table [x={threads}, y={DanaAVL}]
{results/0C20K.txt}; 
\addplot [darkspringgreen,mark=triangle] table [x={threads}, y={Bronson}]
{results/0C20K.txt}; 
\addplot [coralred,mark=asterisk] table [x={threads}, y={JavaSkipList}]
{results/0C20K.txt};
\addplot [gray,mark=o] table [x={threads}, y={GlobalLockTreap}]
{results/0C20K.txt};
\addplot [blues4,mark=x] table [x={threads}, y={DominationLockingTreap}]
{results/0C20K.txt};
\end{axis}
\end{tikzpicture}
\begin{tikzpicture}
\begin{axis}[mystyle,balanced,  xlabel={Threads}]
\addplot [black,mark=square*] table [x={threads}, y={TimeoutTreap}]
{results/50C20K.txt}; 
\addplot [orange,mark=diamond] table [x={threads}, y={DanaAVL}]
{results/50C20K.txt}; 
\addplot [darkspringgreen,mark=triangle] table [x={threads}, y={Bronson}]
{results/50C20K.txt}; 
\addplot [coralred,mark=asterisk] table [x={threads}, y={JavaSkipList}]
{results/50C20K.txt};
\addplot [gray,mark=o] table [x={threads}, y={GlobalLockTreap}]
{results/50C20K.txt};
\addplot [blues4,mark=x] table [x={threads}, y={DominationLockingTreap}]
{results/50C20K.txt};
\end{axis}
\end{tikzpicture}
\begin{tikzpicture}
\begin{axis}[mystyle,balanced, xlabel={Threads}]
\addplot [black,mark=square*] table [x={threads}, y={TimeoutTreap}]
{results/100C20K.txt}; 
\addplot [orange,mark=diamond] table [x={threads}, y={DanaAVL}]
{results/100C20K.txt}; 
\addplot [darkspringgreen,mark=triangle] table [x={threads}, y={Bronson}]
{results/100C20K.txt}; 
\addplot [coralred,mark=asterisk] table [x={threads}, y={JavaSkipList}]
{results/100C20K.txt};
\addplot [gray,mark=o] table [x={threads}, y={GlobalLockTreap}]
{results/100C20K.txt};
\addplot [blues4,mark=x] table [x={threads}, y={DominationLockingTreap}]
{results/100C20K.txt};
\end{axis}
\end{tikzpicture}
\ref{balancedLegened}





\end{center}
\caption{Throughput of balanced data
structures.
\label{evaluation:results:balanced}}
\end{figure*}
\section{Related Work}\label{sec:related}
\paragraph{Concurrent Data Structures}
Many sophisticated concurrent data structures (e.g., \cite{ArbelA2014,DrachslerVY2014,NatarajanM2014,BrownER2014,CrainGR2013,BraginskyP2012,
AfekKKMT2012,EllenFRB2010,BronsonCCO2010,HerlihyLLS2007,fraser2004practical,Michael:1996})
were developed and used in concurrent software systems~\cite{Ohad:OOPSLA11}.
Implementing efficient synchronization for such data structures is considered a challenging and error-prone task~\cite{Ohad:OOPSLA11,Doh:SPAA04,Jin:2012}.
As a result, concurrent data structures are manually implemented by concurrency experts.
This paper shows that (in some cases) an automatic algorithm can produce synchronization that is comparable to synchronization implemented by experts.

\paragraph{Locking Protocols}
Locking protocols are used in databases and shared memory systems to guarantee correctness
of concurrently executing transactions~\cite{Weikum:2001,BHG:Book87}.
Our approach can be seen as a way to extend many existing locking protocols by combining them with  optimistic concurrency control.
In particular, our approach extends the following locking protocols:
two-phase locking~\cite{Eswaran:1976}, tree locking~\cite{SilberschatzK1980}, DAG locking~\cite{CH:PODS95} and domination locking~\cite{Gueta2011}.
We demonstrate the benefit of such combination by using the  domination locking protocol to produce efficient concurrency control for
dynamic data structures.


\paragraph{Lock Inference Algorithms}
There has been a lot of work on automatically inferring locks for transactions.
Most   algorithms in the literature infer locks that follow the two-phase
locking protocol~\cite{MZGB:POPL06,Emmi06POPL,gudka2012lock,CCG:PLDI08,HFP:TRANSACT06,CGE:CC08}.
Our approach can potentially be used to optimized the synchronization produced by these algorithms.
For example, for  algorithms that employ a two-phase variant in which all locks are acquired at the beginning of a transaction (e.g.,~\cite{gudka2012lock,CCG:PLDI08}),
our approach may be used to defer the locking (e.g., to just before the first write operation) as well as to eliminate some of the locking operations.


\paragraph{Transactional Memory}
Transactional memory approaches (TMs) dynamically resolve inconsistencies
and deadlocks by rolling back partially completed transactions.
%
Unfortunately, in spite of a lot of effort and many TM implementations (see~\cite{HLR:SLCA2010}), existing TMs
have not been widely adopted due to various concerns~\cite{DuffyTM2010,Cascaval:2008,mckenneyParallel}, including high runtime overhead,
poor performance and limited ability to handle irreversible operations.
In particular, modern concurrent programs (and concurrent data structures) are typically based on hand-crafted synchronization, rather than  on a TM approach~\cite{Ohad:OOPSLA11}.

In a sense, our approach can be seen as a specialized TM approach that can be practically used to handle concurrent data structure.


%\paragraph{Lock Elision for Read-Only Transactions}
\paragraph{Lock Elision}
Our approach is inspired by the idea of \emph{sequential locks}~\cite{mckenneyParallel} and the approach presented in~\cite{Nakaike:2010}.
But  in contrast to these approaches,  which are designed to handle read-only transactions,
our approach handles read-only prefixes of transactions (operations) that update the shared memory.
As shown in Section~\ref{sec:eval}, our approach is best suited for update-dominated workloads
Moreover, using these approaches for a highly-contended data structure (as in Section~\ref{sec:eval}) is likely to provide limited performance,
because each update transaction causes many read-only transactions to abort.

There are some transactional memory techniques to elide locks from arbitrary critical sections (e.g.,~\cite{Rajwar:2002:TLE:635508.605399,Roy:2009:RSS:1519065.1519094,Afek:2014:SHL:2611462.2611482}).
In these techniques, a transaction executes the critical section speculatively without acquiring the lock.
When a transaction is aborted, it can acquire the lock and execute the critical section non-speculatively.
In contrast to our approach, these techniques cannot combine speculative and non-speculative execution of the same transaction.







 
\bibliography{myRef}
\bibliographystyle{plain}
\end{document}
