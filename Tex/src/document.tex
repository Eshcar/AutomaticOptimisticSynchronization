%-----------------------------------------------------------------------------
%
%               Template for sigplanconf LaTeX Class
%
% Name:         sigplanconf-template.tex
%
% Purpose:      A template for sigplanconf.cls, which is a LaTeX 2e class
%               file for SIGPLAN conference proceedings.
%
% Guide:        Refer to "Author's Guide to the ACM SIGPLAN Class,"
%               sigplanconf-guide.pdf
%
% Author:       Paul C. Anagnostopoulos
%               Windfall Software
%               978 371-2316
%               paul@windfall.com
%
% Created:      15 February 2005
%
%-----------------------------------------------------------------------------


\documentclass[preprint]{sigplanconf}

% The following \documentclass options may be useful:

% preprint      Remove this option only once the paper is in final form.
% 10pt          To set in 10-point type instead of 9-point.
% 11pt          To set in 11-point type instead of 9-point.
% authoryear    To obtain author/year citation style instead of numeric.

\usepackage{amsmath}


\usepackage{xspace}
\usepackage{algorithm}
\usepackage[noend]{algpseudocode}
\usepackage{caption}
\usepackage{subcaption}
\usepackage{xcolor}
\usepackage{tikz}
\usepackage{pgfplots}


\definecolor{blues1}{RGB}{198, 219, 239}
\definecolor{blues2}{RGB}{158, 202, 225}
\definecolor{blues3}{RGB}{107, 174, 214}
\definecolor{blues4}{RGB}{49, 130, 189}
\definecolor{blues5}{RGB}{8, 81, 156}
\definecolor{antiquefuchsia}{rgb}{0.57, 0.36, 0.51}
\definecolor{asparagus}{rgb}{0.53, 0.66, 0.42}
\definecolor{darkspringgreen}{rgb}{0.09, 0.45, 0.27}
\definecolor{darkslategray}{rgb}{0.18, 0.31, 0.31}
\definecolor{coralred}{rgb}{1.0, 0.25, 0.25}

\pgfplotsset{mystyle/.style={%
        %width=6cm,
        %ylabel={mystyle (kg)},
        %xlabel={Eggs (no.)},
        xmin=1,xmax=32,
        enlargelimits=true,
        %xmajorgrids=false,
		ymajorgrids=true,
        grid=major,
        grid style={dashed, gray!30},
        %symbolic x coords={1,2,4,8,16,32},
        ylabel style={align=center},
        xtick={1,4,8,16,32}, 
        scaled y ticks=base 10:-6,
        ytick scale label code/.code={},
    	yticklabel={\pgfmathprintnumber{\tick}}
        %tick align=outside,
}}

\pgfplotsset{unbalanced/.style={%
	   width=0.35\textwidth,
       legend columns=-1,
	   legend entries={\autoTree, \danaTree ,\lockfreeTree, \stmTree, \domTree, },
	   legend to name=unbalancedLegened,
}}


\pgfplotsset{balanced/.style={%
	   width=0.35\textwidth,
       legend columns=-1,
	   legend entries={\autoTreap, \danaAVL,\bronson, \friendly,  \stmTreap,
	   \domTreap, \globalTreap}, legend to name=balancedLegened,
}}

\pgfplotsset{skiplist/.style={%
	   width=\textwidth,
       ymin=0,ymax=2800000,
       xtick={2,4,8,16,32}, 
       legend columns=-1,
	   legend entries={\autoSkiplist, \kary , \skiplist, \stmSkiplist,
	   \domSkiplist}, legend to name=skiplistLegened,
}}

\pgfplotsset{skiplist1000/.style={%
	   width=\textwidth,
       ymin=0,ymax=150000,
       scaled y ticks=base 10:-3,
       xtick={2,4,8,16,32}, 
       legend columns=-1,
	   legend entries={\autoSkiplist, \kary, \skiplist, \stmSkiplist,
	   \domSkiplist}, legend to name=skiplistLegened1000,
}}

\pgfplotsset{skiplistUpdate/.style={%
	   width=0.35\textwidth,
       ymin=0,ymax=5500000,
       xtick={1,2,4,8,16,32}, 
       legend columns=-1,
	   legend entries={\autoSkiplist, \kary , \skiplist, \stmSkiplist,
	   \domSkiplist}, legend to name=skiplistLegenedUpdate,
}}



\newcommand{\Xomit}[1]{}

%-------------Theorem Definitions ---------------%
\newtheorem{theorem}{Theorem}[section]
\newtheorem{lemma}[theorem]{Lemma}
\newtheorem{claim}[theorem]{Claim}
\newtheorem{observation}[theorem]{Observation}
\newtheorem{proposition}[theorem]{Proposition}
\newtheorem{corollary}[theorem]{Corollary}

\newenvironment{proof}[1][Proof]{\begin{trivlist}
\item[\hskip \labelsep {\bfseries #1}]}{\qedsymb\end{trivlist}}
\newenvironment{definition}[1][Definition]{\begin{trivlist}
\item[\hskip \labelsep {\bfseries #1}]}{\end{trivlist}}
\newenvironment{example}[1][Example]{\begin{trivlist}
\item[\hskip \labelsep {\bfseries #1}]}{\end{trivlist}}
\newenvironment{remark}[1][Remark]{\begin{trivlist}
\item[\hskip \labelsep {\bfseries #1}]}{\end{trivlist}}


\newcommand{\qed}{\nobreak \ifvmode \relax \else
      \ifdim\lastskip<1.5em \hskip-\lastskip
      \hskip1.5em plus0em minus0.5em \fi \nobreak
      \vrule height0.75em width0.5em depth0.25em\fi}
\newcommand{\qedsymb}{\hfill{\rule{2mm}{2mm}}}


%--------------------------------------------------%

\newcommand{\code}[1]{\textsf{#1}}
\newcommand{\readV}{\code{read\_version}\xspace}
\newcommand{\readSet}{\code{read\_set}\xspace}
\newcommand{\writeV}{\code{write\_version}\xspace}

\newcommand{\reqI}{\textbf{LPR1}\xspace}
\newcommand{\reqII}{\textbf{LPR2}\xspace}

%---------Evaluation Macros-----------------------%
\newcommand{\autoTree}{LR-Tree\xspace}
\newcommand{\autoTreap}{LR-Treap\xspace}
\newcommand{\optAutoTree}{Opt-LR-Tree\xspace}
\newcommand{\optAutoTreap}{Opt-LR-Treap\xspace}
\newcommand{\autoSkiplist}{LR-Skiplist\xspace}
\newcommand{\danaTree}{LO-Tree\xspace}
\newcommand{\danaAVL}{LO-AVL\xspace}
\newcommand{\bronson}{Snap-Tree\xspace}
\newcommand{\friendly}{CF-Tree\xspace}
\newcommand{\skiplist}{Java-Skiplist\xspace}
\newcommand{\kary}{k-Tree\xspace}
\newcommand{\lockfreeTree}{LF-Tree\xspace}
\newcommand{\globalTree}{Global-Tree\xspace}
\newcommand{\globalTreap}{Global-Treap\xspace}
\newcommand{\domTree}{Lock-Tree\xspace}
\newcommand{\domTreap}{Lock-Treap\xspace}
\newcommand{\stmTree}{STM-Tree\xspace}
\newcommand{\stmTreap}{STM-Treap\xspace}
\newcommand{\stmSkiplist}{STM-Skiplist\xspace}
\newcommand{\domSkiplist}{Lock-Skiplist\xspace}
\newcommand{\getOP}{\textsc{get}\xspace}


%---------Comments -----------------------%
\newcommand{\Idit}[1]{{\color{red}{[\textbf{Idit:} #1 ]}}}
\newcommand{\guy}[1]{{\color{red}{[\textbf{guy:} #1 ]}}}
\newcommand{\eshcar}[1]{{\textcolor{violet}{\{{\bf eshcar:} \em #1\}}}}
\newcommand{\maya}[1]{{\textcolor{magenta}{\{{\bf maya:} \em #1\}}}}




\begin{document}

\special{papersize=8.5in,11in}
\setlength{\pdfpageheight}{\paperheight}
\setlength{\pdfpagewidth}{\paperwidth}

\conferenceinfo{CONF 'yy}{Month d--d, 20yy, City, ST, Country} 
\copyrightyear{20yy} 
\copyrightdata{978-1-nnnn-nnnn-n/yy/mm} 
\doi{nnnnnnn.nnnnnnn}

% Uncomment one of the following two, if you are not going for the 
% traditional copyright transfer agreement.

%\exclusivelicense                % ACM gets exclusive license to publish, 
                                  % you retain copyright

%\permissiontopublish             % ACM gets nonexclusive license to publish
                                  % (paid open-access papers, 
                                  % short abstracts)

\titlebanner{banner above paper title}        % These are ignored unless
\preprintfooter{short description of paper}   % 'preprint' option specified.

\title{Automatic Lock Removal for Scalable Synchronization}
%\title{Scalable Automatic Synchronization for Concurrent Data Structures}
%\subtitle{Subtitle Text, if any}

\authorinfo{}
           {}
           {}

\maketitle

\begin{abstract}
We present an \emph{automatic} approach for parallelizing sequential 
data structures in a way that is both safe and scalable. 
While there exist pessimistic transformations that make 
code thread-safe by adding (either global or fine-grained) locks, 
this approach is limited in its performance due to synchronization 
bottlenecks, for example, locking the root in a tree data structure. 
In this paper, we improve the performance and scalability of such 
synthesized code by reducing bottlenecks. Specifically, 
we present an automatic approach to eliminate many of the locking steps, 
relying instead on optimistic partial traversals of the data structure. 
We realize our approach for tree data structures using 
the domination locking technique. 
The resulting code scales well, significantly outperforms 
pessimistic approaches, and its performance is comparable 
to that achieved by custom-tailored implementations. 
Our work thus shows the promise that automated approaches 
bear for overcoming the difficulty involved in manually 
hand-crafting concurrent data structures. 

\end{abstract}

\category{CR-number}{subcategory}{third-level}

% general terms are not compulsory anymore, 
% you may leave them out
\terms
Automatic code generation, parallel programming, synchronization

\keywords
Concurrent data structures, lock elision


\section{Introduction} \label{sec:intro}

\subsection{Automatic Lock Removal}
% Parallelizing data strucutres is important for performance
The steady increase in the number of cores in today's computers is driving software developers to allow more and more parallelism.
An important focal point for such efforts is scaling the concurrency of shared data structures, which are often a principal friction point among threads.
%; recent work has illustrated that improved data structure concurrency can lead to
%benefits in overall system performance~\cite{clsm-poster}.
%It is therefore not surprising that many recent works have been dedicated to developing scalable concurrent data
Many recent works have been dedicated to developing scalable concurrent data
structures (e.g.,~\cite{ArbelA2014,DrachslerVY2014,NatarajanM2014,BrownER2014,CrainGR2013,BraginskyP2012,
AfekKKMT2012,EllenFRB2010,BronsonCCO2010,HerlihyLLS2007,fraser2004practical,Michael:1996}),
some of which are widely used in real-world systems~\cite{Ohad:OOPSLA11}.

% They are difficult to build and of resticted use
Each of these projects generally focuses on a single data
structure (for example, a binary search tree~\cite{BronsonCCO2010} or a queue~\cite{Michael:1996}) and manually optimizes its implementation. These data structures are developed by concurrency experts, typically PhDs or PhD candidates.
Proving the correctness of such custom-tailored data structures is painstaking;
for example, the proofs of \cite{BraginskyP2012,EllenFRB2010} are $31$ and $20$ pages long
respectively.
The rationale behind dedicating so much effort to one data structure is that it is
generic and can be used by many applications. Nevertheless,  systems often use data structures in unique ways
that necessitate changing or extending their code (e.g.,~\cite{levelDB,jmonkey,OhadThesis,zyulkyarov2009atomic}), which limits the usability of custom-tailored
implementations. Hence, the return-on-investment for such endeavors may be suboptimal.
Here, we propose an approach to facilitate this labor-intensive process by automatic means,
making scalable synchronization more readily available.

% We give a transformation
Specifically, we present in Section~\ref{sec:algorithm} a source-to-source
code transformation that takes a lock-based concurrent data structure implementation as its input
and generates more scalable code for the same data structure via judicious use of optimism.
%%%Section~\ref{sec:model} details our model and assumptions, and
%%%Section~\ref{sec:algorithm} specifies the transformation.
Our approach combines optimism and pessimism in a new, practical, way.
In striking the balance between the two, we exploit the common access pattern in data structure operations,
(for example, tree insertion or removal), which typically begin by traversing the data structure (to the insertion or removal point), and then perform (mostly) local updates at that location.
Our transformation replaces locking steps in the initial read-only traversal of each operation with
optimistic synchronization, whereas the update phase employs the original lock-based synchronization.
Our work may thus be seen as a form of software lock elision for read-only operation prefixes.

% Best of both worlds
Combining optimism and pessimism allows us to achieve ``the best of both worlds'' -- while the
optimistic traversal increases concurrency and eliminates bottlenecks,
the use of pessimistic updates saves the overhead associated with speculative or deferred shared
memory updates, (as in software transactional memory~\cite{HLR:SLCA2010}).
The partially-optimistic execution is compatible with the original code, which permits us to re-execute operations
pessimistically when too many conflicts occur, avoiding livelocks.
%Furthermore, it allows for code optimizations
%that make the optimistic execution fail in some conflict-free cases (for example, when too many items would have been locked
%by the original code), since we can always fallback upon lock-based execution.

% Properties of our transformation
We show in Section~\ref{sec:proof} that our transformation preserves the external behaviour of the original lock-based code.
%all essential properties of the original code: serializability, linearizability, and deadlock-freedom.
In other words, if the original code is correct (in the sense of serializability, linearizability, and deadlock-freedom), so is the
transformed version. Moreover, the transformation
%does not introduce any new accesses to shared data; in particular, it
refrains from introducing a shared global clock (as used in some transactional memory systems~\cite{DBLP:conf/eurosys/ShalevS06}) or other
shared data structures. Thus, if the original code is \emph{disjoint access parallel}~\cite{Israeli:1994:DIS:197917.198079}, i.e., threads
that access disjoint (abstract) data objects do not contend on (low level) shared memory locations, then this
property holds also for the transformed code.

\subsection{Fully Automatic Parallelization}
% Automatic parallelization
One important use case for our transformation is to apply it in conjunction with automatic lock-based
parallelization mechanisms~\cite{Gueta2011,MZGB:POPL06}.
The latter automatically instrument sequential code (at compile time)
and add fine-grained lock and unlock instructions that ensure its safety in concurrent executions.
%For example, \emph{domination locking}~\cite{Gueta2011} is applicable to trees or forests
%(including binary search trees, b-trees, treaps~\cite{AragonS1989}, and self adjusting heaps~\cite{Sleator:SAH1986:SAH}); it
%employs a variant of hand-over-hand locking~\cite{SilberschatzK1980},
%acquiring and releasing locks as it goes down the tree.
%%Other approaches are applicable to DAGS~\cite{dag-locking}.
Our evaluation shows that, by themselves, solutions of this sort may scale poorly.
%~\cite{Gueta2011}.
This is due to synchronization bottlenecks, e.g., the root of a tree,
which is locked by all operations.
By subsequently applying our transformation, one can optimize
the lock-based code they produce, yielding a \emph{fully automatic approach to
scalable parallelization of sequential code}.

%% Why not read-write locks
%It is worth noting that the aforementioned mechanisms synthesize code that uses conventional symmetric locks,
%which is the type of locks handled by our transformation. We are not aware of any automatic transformation
%inserting read-write locks. We further note that read-write locks
%are more costly than conventional ones, and, moreover, threads using
%read locks contend on the shared memory locations employed by the lock's implementation~\cite{xxx}.
%In contrast, with our transformation, threads executing the optimistic phase do not contend on locks,
%and are completely invisible to other threads.

\subsection{Evaluation}
In Section~\ref{sec:eval} we evaluate our transformation by generating three data structures-- an unbalanced search tree, a treap
(randomized balanced search tree),
and a skip list that supports range queries. We synthesize the first two from sequential implementations using the algorithm of~\cite{Gueta2011}, followed by our transformation.
For the skip list, we manually add fine-grained locks (in a straightforward manner), and then apply our transformation.
All examples are implemented in Java. We evaluate the scalability of the resulting code
in a range of workload scenarios on a $32$-core machine.
In all cases, the lock-based implementations do not scale --
their throughput remains flat as the number of running threads increases. In contrast, the code generated by our transformation
is scalable, and its throughput continues to grow with the number of threads.

We compare our synthesized code to state-of-the-art data structures
that were hand-crafted by experts in the field~\cite{DrachslerVY2014,BronsonCCO2010,ConcurrentSkipList,fraser2004practical}.
%and resulted in publications in leading venues~\cite{DrachslerVY2014,BronsonCCO2010}.
Our results show that the
fully-automatically generated code for the search tree and treap
achieves comparable performance to that of
custom-tailored solutions.

We also consider a data structure that supports range queries, which are required by
many applications (e.g.,~\cite{levelDB,FerroJKRY14}). To this end we implement a skip list -- a data structure that naturally supports range queries.
While range queries implemented using the iterators available in the Java concurrency library's skip list~\cite{ConcurrentSkipList} perform
somewhat better than ones in our synthesized code, it is important to note that these iterators are \emph{not}
linearizable (atomic), and only support so-called weak consistency, whereas range queries in our implementation are linearizable.
A notable previous linearizable implementation is due to Bronson et al.~\cite{BronsonCCO2010},
and it is significantly out-performed by our synthesized code.
%
This illustrates the benefit of the broad applicability
of our automatic approach compared to specific custom-tailored implementations.

Note that the alternative automatic approach we compare with is domination locking~\cite{Gueta2011} which scales poorly.
%\Idit{Say something about transactional memory.}
%The only other automatic parallelization approach we are familiar with is software transactional memory~\cite{xxx},
%which has been shown to perform even worse than a global lock~\cite{}, hence we do not empirically
%compare our solution with it.
Software transactional memory is another approach for automatic parallelization,
yet due to the significant overhead associated with this approach~\cite{Cascaval:2008,DuffyTM2010}, it is often complemented
by custom-tailored data structure implementations whose operations
can be called from within transactions~\cite{Herlihy:2008,Koskinen:2010,NathanBronson11}.
%
Further discussion of related work appears in Section~\ref{sec:related}.

To conclude, this paper demonstrates that automatic synchronization, based on a careful combination of optimistic and
pessimistic concurrency control, is a promising approach for bringing legacy code to emerging computer architectures.
While this paper illustrates the method for tree and skip list data structures, we believe that the general direction may be more broadly applicable, and maybe used with a variety of locking schemes, such as two phase locking.
Section~\ref{sec:discussion} concludes the paper and touches on some directions for future work. 


\section{Model and  Definitions}


\subsection{Shared Memory Data Structures}

We consider an asynchronous shared memory model, where independent threads interact via shared memory objects.
For the sake of our discussion, we do not distinguish among different types of shared memory (e.g., global or heap-allocated). 
In addition, each thread has access to \emph{local} (thread-local) memory.

A \emph{data structure} is an abstract data type exporting a set of \emph{operations}.
A data structure is implemented from a collection of primitive shared \emph{objects} supporting atomic load (read) and store (write) operations.
In Section~\ref{ssec:locking}, we extend the allowed primitive variables to also include locks.  

Each thread executes a sequence of operations, each of which is invoked with certain parameters and returns a response.
An operation's execution consists of a sequence of primitive \emph{steps}, beginning with an \emph{invoke} step, followed by 
atomic accesses to shared objects, and ending with a \emph{return} step. Steps also modify the executing thread's local variables.
A \emph{configuration} is an assignment of values to all shared and local variables. Thus, each step takes the system from one 
configuration to another. Steps are deterministically defined by the data structure's protocol and the current configuration. 
In the \emph{initial configuration}, each variable holds an initial value. 

An \emph{execution} is an alternating sequence of configurations and steps,
$C_0,s_1,\ldots,s_i,C_i,\ldots,$ 
where $C_0$ is an initial configuration,
and each configuration $C_i$ is the result of
executing step $s_i$ on configuration $C_{i-1}$.
An execution is \emph{sequential} if steps of different operations are not interleaved. 
In other words, a sequential execution is a sequence of operation executions.

\subsection{Locking Protocols}
\label{ssec:locking}

A \emph{lock} is a primitive type that supports atomic \emph{lock}, \emph{try\_lock}, and \emph{unlock} operations, 
where try\_lock is a non-blocking attempt to acquire a lock that may fail. 

We assume in this paper a \emph{locking protocol}, which transforms a sequential data-structure to a correct concurrent one by adding locks;
correctness is defined in Section~\ref{sec:linearizable} below.
Examples of such protocols are Tree Locking~\cite{tree-locking}, and Domination Locking~\cite{domination-locking}. 
The locking protocol associates a lock with every primitive shared object used by the data structure, and instruments the sequential code 
by adding lock and unlock operations. Intuitively, such protocols automatically perform some sort of ``hand-over-hand'' locking, acquiring locks as
they traverse a linked-list or tree, and releasing locks on previously traversed nodes. They are typically restricted to tree-like data structures.


We assume that the resulting code (obtained by adding the locks) satisfies the following properties:

\begin{itemize}
\item Every (load or store) access by an operation to a shared object is performed when the executing thread holds the lock on that object.
\item The protocol is deadlock-free, i.e., locking  does not introduce deadlocks.
\item The protocol allows early lock release in the sense that  it never needs to hold a lock on an object that it no longer has a pointer to.
 \end{itemize}

The last condition means that even if the protocol holds a lock on an object it no longer holds a pointer to, it is safe to unlock it at this point (and re-acquire the lock later if it is accessed again ), in the sense that correctness, as defined below, is not breached.  


\subsection{Correctness}
\label{ssec:linearizable} 

The correctness of a data structure is defined in terms of its external behavior, as reflected in values returned by invoked operations. 
This is captured by the notion of a \emph{history} -- the history of an execution $\sigma$ is the subsequence  of $\sigma$ consisting 
of invoke and return steps. The widely-used correctness criteria of linearizability and serializability link the data structure's behavior under concurrency to its allowed behavior in sequential executions. The latter is defined by a \emph{sequential specification}, which is a set of its allowed sequential histories. 

A history $H$ is \emph{linearizable}
\cite{Herlihy:1990:LCC:78969.78972} if there exists $H'$ that can be created by adding zero or more 
return steps to $H$, and there is a sequential permutation $\pi$ of complete($H'$), 
such that: (1) $\pi$ belongs to the sequential specification; and 
(2) every pair of operations that are not interleaved in $\sigma$, appear in the same order in $\sigma$ and in $\pi$. 
A data structure  is \emph{linearizable}, also called \emph{atomic}, if the histories of all of its executions are linearizable.

A history is \emph{serializable} if it satisifes property (1) above. That is, the real-time order of operations is not required.
A data structure  is \emph{serializable} the histories of all of its executions are serializable.

In this paper, we are not concerned with internal consistency (as required e.g., by opacity~\cite{opacity} or the validity notion of~\cite{Kfir}), which 
restricts the configurations an operation might see during its execution. 
This is because our code transformation deals with inconsistencies that may arise when a thread sees an inconsistent view of global variables using 
timouts and exception handlers.   



\section{\ldots}
%The main idea: 
We address the problem of adding an 
optimism to a pessimistic locking protocol. 
Given a sequential implementation and some pessimistic
locking protocol, that performs store steps 
only to locked objects,
we add optimistic synchronization. 
The idea of optimistic synchronization is to load 
shared variables without locks and start using the 
locking protocol only when the operation reaches a store step.
In practice our algorithm shows that there is no need to 
acquire locks if they are freed before any change is made.    
%In order to implement such synchronization scheme, 
%we divide the operation to three \emph{phases}:

This optimistic scheme separates the operation to three
\emph{phases}, an optimistic \emph{read phase},
a pessimistic \emph{read-write phase} and a 
\emph{validation phase} that connect them. 

\subsection{Detailed Algorithm}
Each objects maintains a counter, incremented every time the 
object is locked. This counter is used to validate the correctness
of the optimistic read phase. 

\begin{description}
  \item[Read Phase] Stats at the beginning of the operation and
  ends at any point before the first store operation. 
  During this phase the operation maintains a \readSet, 
  containing references to objects loaded and the local 
  version when it was read. The local versions are incremented 
  during the read-write phase, when the object is locked. 
  Incrementing the version is not atomic with the lock, thus, 
  the object is also checked to be unlocked. 
  The read phase does not validate reads, 
  in order to avoid infinite loops, a timeout is set. 
  If the operation reaches the timeout, 
  a \readSet validation takes place, if it fails 
  the operation restarts from the beginning.
  A pseudo code of the transformation can be found in 
  Figure \ref{figure:readPhaseTransformation}. 
  
  \item[Validation Phase] This phase connects the read phase
  with the read-write phase. It has two requirements: (i) lock 
  local variables of the operation and (ii) ensure that the 
  values read during the read phase are consistent, i.e.
  as if the values were read while executing the locking 
  protocol. To avoid deadlocks, the locks are acquired using 
  a \code{try\_lock} operation, if the \code{try\_lock} fails, 
  the operation restarts from the beginning. Next, the \readSet 
  is validated, if the validation fails the operation restarts from the
  beginning. During the \readSet validation, each reference saved 
  in the \readSet is checked to be unlocked and that the current 
  version matches the version saved in the \readSet. 
  A pseudo code of the \readSet validation can be found in
  Figure \ref{figure:readSetValidation}.
  
  \item[Read-Write Phase] This phase enforces the locking protocol
  while maintaining the local versions, i.e., the local version of 
  an object is incremented every time it is locked.  
  Once the read-write phase begins, the operation is guaranteed to to 
  finish without restarts.  
\end{description}

The use of timeout does not guarantee
\emph{opacity}~\cite{GuerraouiK2008} or 
\emph{validity}\cite{LevAriCK2014}. 

%The \readSet validation first checks that the node is unlocked,
%(or locked by the current operation), then it checks that the 
%current version is equal to the version saved in the \readSet. 
 
%ALGORITHM CODE

\begin{figure}

	\begin{subfigure}{.35\textwidth}
		\begin{algorithmic}[1]{}
			\Function{foo}{\ldots} \label{code:begin}
			\Statex \ldots
			\State x = ptrExp \label{code:readRef}
			\Statex \ldots
			\EndFunction
		\end{algorithmic}
	\end{subfigure}
    \begin{subfigure}{.60\textwidth}
		\begin{algorithmic}[1]{}
			\State temp = ptrExp
			\State version = temp.getVersion()
			\If{y.isLocked()}
				\State \textbf{goto} \ref{code:begin} 
				\Comment Restart Operation
			\EndIf
			\State readSet.add(temp,version) 
			\State x = temp
		\end{algorithmic}
	\end{subfigure}
	\caption{Read phase transformation, the code in line \ref{code:readRef} 
	is replaced with the code on the right. 
			\label{figure:readPhaseTransformation}}
\end{figure}


\begin{figure}
	\begin{algorithmic}[1]{}
	\Function{foo}{\ldots}
			\Statex \ldots
			\Statex \Comment begin \readSet validation
		\ForAll{ (ref,version)$\in$ readSet}
			\If{ref.isLocked() \textbf{and} ref.lockedBy!= self} 
				\State \textbf{goto} \ref{code:begin} 
				\Comment Restart Operation
			\EndIf
			\If{version!= ref.getVersion()}
				\State \textbf{goto} \ref{code:begin} 
				\Comment Restart Operation
			\EndIf
		\EndFor
		\Statex \ldots
		\EndFunction
	\end{algorithmic}
	\caption{\readSet validation \label{figure:readSetValidation}}
\end{figure}
%END CODE 
\newcommand{\op}{\emph{\textsc{op}}}
\newcommand{\opt}{\textsc{opt}}

\section{Analysis}
\label{sec:proof}

We show  three properties of our automatic transformation. First, we prove that the  transformation is correct, i.e., 
every execution of the  synthesized code is equivalent to some execution of the \emph{locking algorithm}, that is, the
sequential code instrumented with the locking protocol.
This implies that if the locking protocol ensures serializability, then so does our protocol.
Second, we show that the equivalent execution preserves the real-time order of the original one, 
which implies that linearizability is also invariant under the transformation.
Finally, we argue that our transformation preserves deadlock-freedom.
In this section, we provide informal correctness arguments. A formal proof is deferred to Appendix~\ref{sec:formal-proof}.

\paragraph{Transformation Correctness}

Let $\pi$ be a finite execution of the transformed algorithm. We will show an equivalent execution of the locking algorithm.
Let \op\ be the set of operations in $\pi$.
First, we project object versions out of $\pi$'s configurations, and remove all accesses (reads and writes) to object versions.
That is, we replace steps that access versions with local steps that modify the operation's local memory only.
(Note that we get an execution with exactly the same invocations, responses, local states, and shared object states, but without 
versions). 
Second, note that each operation $op \in$ \op\ executes at most one successful read-only
phase, namely a complete read-only phase followed by a successful validation phase.
For each such $op$, we remove the prefix of $op$ that precedes the successful read-only phase.
This includes completely removing operations that have no successful read-only phase.
We call the resulting execution $\pi'$.

We next iteratively perturb $\pi'$ to construct an equivalent execution of the locking protocol.
We order the operations in \op\ according to the order in which the first steps of their respective (successful) read set validation phases occur in $\pi'$.
Denote the sequence of these first validations steps $e_1, e_2, \ldots, e_k$, and the respective operations
$op_1, op_2, \ldots, op_k$ in this order. Let $\pi_0 = \pi'$.
In each iteration $i \geq 1$, we construct $\pi_i$ by replacing the execution of $op_i$
in $\pi_{i-1}$ with an execution of $op_i$ that follows the locking protocol. To do this, we first move all steps of $op_i$ that precede 
its first validation step, $e_i$, to occur immediately before $e_i$ (we know that the read steps return the same values, since validation 
succeeds). We then add steps that lock these objects before their read steps (and unlock
them) as dictated by the locking protocol.
%We then add steps that lock all of them immediately before these moved read
% steps. Finally, we add unlock steps immediately before $e_i$ for objects that are in the read set but not in the lock set during $e_i$, 
Finally we remove the tryLock and validation steps by $op_i$. 

Our ability to lock each object in $op_i$'s read set follows from the following observation:
\begin{observation}
Between the first time in which $op_i$ first reads an object $o$ in its read-only phase, and until the first step of $op_i$'s validation phase, no
 thread locks $o$. 
\end{observation}
The observation follows immediately from the fact that $op_i$'s read validation is successful and that every lock step increases the
respective object's version number. 

It is easy to see that $\pi_{i-1}$ and $\pi_i$ are equivalent. By repeating this for all operations in \op, we get an execution $\pi_k$ of the locking 
protocol.

\paragraph{Real-Time Order}
It is easy to see that $\pi_k$ preserves the real-time order of $\pi$, since it does not change the order of invoke or return steps. 

\paragraph{Progress}
The read and validation phases of our instrumented code do not use blocking locks -- the read-phase does not use locks at all, whereas the 
validation phase uses tryLocks. Therefore, both phases are non-blocking. In principle, the optimistic approach may lead to livelocks, but 
our algorithm fall-back on the pessimistic approach following a bounded number of restarts, and hence cannot livelock. We get that any lack
of progress must be due to blocking in the locking algorithm. Since the section of the code executing the locking protocol is unchanged, and
since we ensure that it begins when holding the same locks as in the original protocol, we get that our transformation does not introduce any
source of spurious blocking that is not present in the original locking protocol. 




\section{Evaluation}\label{sec:eval}
In order to apply our approach to a data structure, a
''black-box'' pessimistic locking is required. One such 
approach was presented in \cite{Gueta2011}, automatically 
applying domination locking protocol to forest based data structures.  
We applied domination locking followed by our optimistic 
transformation to two tree based data structures, 
a simple unbalanced binary search tree 
and a treap (randomized binary search tree) \cite{AragonS1989}.

\paragraph{Setup}
We compared the performance of our automatic implementations, 
the automatic binary search tree (\autoTree) and the automatic
treap (\autoTreap), to the following custom tailored approaches: 
\begin{itemize}
\item \danaTree - The locked-based 
				unbalanced tree of Drachsler at al.\cite{DrachslerVY2014}. 
\item \danaAVL - The locked-based relaxed balanced AVL tree of 
				Drachsler et al.\cite{DrachslerVY2014}.
\item \bronson - The locked based relaxed balanced AVL tree
				of Bronson et al.\cite{BronsonCCO2010}.
\item \skiplist - The non-blocking skip-list by Doug 
				Lea included in the 
				the Java standard library.
\end{itemize}

We also compared our algorithms to previous automatic approaches, 
mainly global locking (\globalTree, \globalTreap) 
and domination locking (\domTree, \domTreap). 

%TODO 
We ran our experiments on a \ldots

We evaluated the performance on a variety of workloads, 
each workload is defined by the percentage of read-only
operations (\getOP queries) and the remaining operations 
are divided equally between insert and delete operations.
Our workloads include heavy read-only workloads
(100\%,70\% \getOP operations), medium read-only workload 
(50\% \getOP operations) and update only workload
(0\% \getOP operations). 

We used two key ranges $[0,2\cdot10^5]$ and $[0,2\cdot10^6]$,
for each range, the tree was pre-filled until the tree size was 
within 5\% of half the key range.   

We ran five seconds trials measuring the total throughput
(number of operations per second) of all threads.
During the trial, each thread continuously executed randomly
chosen operations according to the workload distribution 
using uniformly random keys from the key range.  
We ran every trial 7 times, we report the average throughput
while eliminating outliers.

\paragraph{Results} Figure \ref{evaluation:results:unbalanced} 
reports the throughput of unbalanced data structures and Figure 
\ref{evaluation:results:balanced} reports
the throughput of the balanced data structures. 


\begin{figure*}
\begin{center}
\begin{tikzpicture}
\begin{axis}[mystyle,unbalanced]
\addplot [blues3,mark=square*] table [x={threads}, y={TimeoutBinaryTree}]
{results/100C20K.txt}; 
\addplot [orange,mark=diamond*] table [x={threads}, y={Danaunbalanced}]
{results/100C20K.txt}; 
\addplot [gray,mark=asterisk] table [x={threads}, y={GlobalLockBinaryTree}]
{results/100C20K.txt};
\addplot [black,mark=x] table [x={threads}, y={DominationLockingBinaryTree}]
{results/100C20K.txt};
\end{axis}
\end{tikzpicture}
\begin{tikzpicture}
\begin{axis}[mystyle,unbalanced]
\addplot [blues3,mark=square*] table [x={threads}, y={TimeoutBinaryTree}]
{results/100C200K.txt}; 
\addplot [orange,mark=diamond*] table [x={threads}, y={Danaunbalanced}]
{results/100C200K.txt}; 
\addplot [gray,mark=asterisk] table [x={threads}, y={GlobalLockBinaryTree}]
{results/100C200K.txt};
\addplot [black,mark=x] table [x={threads}, y={DominationLockingBinaryTree}]
{results/100C200K.txt};
\end{axis}
\end{tikzpicture}
\begin{tikzpicture}
\begin{axis}[mystyle,unbalanced]
\addplot [blues3,mark=square*] table [x={threads}, y={TimeoutBinaryTree}]
{results/100C2000K.txt}; 
\addplot [orange,mark=diamond*] table [x={threads}, y={Danaunbalanced}]
{results/100C2000K.txt}; 
\addplot [gray,mark=asterisk] table [x={threads}, y={GlobalLockBinaryTree}]
{results/100C2000K.txt};
\addplot [black,mark=x] table [x={threads}, y={DominationLockingBinaryTree}]
{results/100C2000K.txt};
%\addplot  [magenta,mark=pentagon] table [x={threads}, y={rb}]
% {results/100C200K.txt};
\end{axis}
\end{tikzpicture}

\begin{tikzpicture}
\begin{axis}[mystyle,unbalanced]
\addplot [blues3,mark=square*] table [x={threads}, y={TimeoutBinaryTree}]
{results/70C20K.txt}; 
\addplot [orange,mark=diamond*] table [x={threads}, y={Danaunbalanced}]
{results/70C20K.txt}; 
\addplot [gray,mark=asterisk] table [x={threads}, y={GlobalLockBinaryTree}]
{results/70C20K.txt};
\addplot [black,mark=x] table [x={threads}, y={DominationLockingBinaryTree}]
{results/70C20K.txt};
\end{axis}
\end{tikzpicture}
\begin{tikzpicture}
\begin{axis}[mystyle,unbalanced]
\addplot [blues3,mark=square*] table [x={threads}, y={TimeoutBinaryTree}]
{results/70C200K.txt}; 
\addplot [orange,mark=diamond*] table [x={threads}, y={Danaunbalanced}]
{results/70C200K.txt}; 
\addplot [gray,mark=asterisk] table [x={threads}, y={GlobalLockBinaryTree}]
{results/70C200K.txt};
\addplot [black,mark=x] table [x={threads}, y={DominationLockingBinaryTree}]
{results/70C200K.txt};
\end{axis}
\end{tikzpicture}
\begin{tikzpicture}
\begin{axis}[mystyle,unbalanced]
\addplot [blues3,mark=square*] table [x={threads}, y={TimeoutBinaryTree}]
{results/70C2000K.txt}; 
\addplot [orange,mark=diamond*] table [x={threads}, y={Danaunbalanced}]
{results/70C2000K.txt}; 
\addplot [gray,mark=asterisk] table [x={threads}, y={GlobalLockBinaryTree}]
{results/70C2000K.txt};
\addplot [black,mark=x] table [x={threads}, y={DominationLockingBinaryTree}]
{results/70C2000K.txt};
%\addplot  [magenta,mark=pentagon] table [x={threads}, y={rb}]
% {results/100C200K.txt};
\end{axis}
\end{tikzpicture}

\begin{tikzpicture}
\begin{axis}[mystyle,unbalanced]
\addplot [blues3,mark=square*] table [x={threads}, y={TimeoutBinaryTree}]
{results/50C20K.txt}; 
\addplot [orange,mark=diamond*] table [x={threads}, y={Danaunbalanced}]
{results/50C20K.txt}; 
\addplot [gray,mark=asterisk] table [x={threads}, y={GlobalLockBinaryTree}]
{results/50C20K.txt};
\addplot [black,mark=x] table [x={threads}, y={DominationLockingBinaryTree}]
{results/50C20K.txt};
\end{axis}
\end{tikzpicture}
\begin{tikzpicture}
\begin{axis}[mystyle,unbalanced]
\addplot [blues3,mark=square*] table [x={threads}, y={TimeoutBinaryTree}]
{results/50C200K.txt}; 
\addplot [orange,mark=diamond*] table [x={threads}, y={Danaunbalanced}]
{results/50C200K.txt}; 
\addplot [gray,mark=asterisk] table [x={threads}, y={GlobalLockBinaryTree}]
{results/50C200K.txt};
\addplot [black,mark=x] table [x={threads}, y={DominationLockingBinaryTree}]
{results/50C200K.txt};
\end{axis}
\end{tikzpicture}
\begin{tikzpicture}
\begin{axis}[mystyle,unbalanced]
\addplot [blues3,mark=square*] table [x={threads}, y={TimeoutBinaryTree}]
{results/50C2000K.txt}; 
\addplot [orange,mark=diamond*] table [x={threads}, y={Danaunbalanced}]
{results/50C2000K.txt}; 
\addplot [gray,mark=asterisk] table [x={threads}, y={GlobalLockBinaryTree}]
{results/50C2000K.txt};
\addplot [black,mark=x] table [x={threads}, y={DominationLockingBinaryTree}]
{results/50C2000K.txt};
%\addplot  [magenta,mark=pentagon] table [x={threads}, y={rb}]
% {results/100C200K.txt};
\end{axis}
\end{tikzpicture}

\begin{tikzpicture}
\begin{axis}[mystyle,unbalanced]
\addplot [blues3,mark=square*] table [x={threads}, y={TimeoutBinaryTree}]
{results/0C20K.txt}; 
\addplot [orange,mark=diamond*] table [x={threads}, y={Danaunbalanced}]
{results/0C20K.txt}; 
\addplot [gray,mark=asterisk] table [x={threads}, y={GlobalLockBinaryTree}]
{results/0C20K.txt};
\addplot [black,mark=x] table [x={threads}, y={DominationLockingBinaryTree}]
{results/0C20K.txt};
\end{axis}
\end{tikzpicture}
\begin{tikzpicture}
\begin{axis}[mystyle,unbalanced]
\addplot [blues3,mark=square*] table [x={threads}, y={TimeoutBinaryTree}]
{results/0C200K.txt}; 
\addplot [orange,mark=diamond*] table [x={threads}, y={Danaunbalanced}]
{results/0C200K.txt}; 
\addplot [gray,mark=asterisk] table [x={threads}, y={GlobalLockBinaryTree}]
{results/0C200K.txt};
\addplot [black,mark=x] table [x={threads}, y={DominationLockingBinaryTree}]
{results/0C200K.txt};
\end{axis}
\end{tikzpicture}
\begin{tikzpicture}
\begin{axis}[mystyle,unbalanced]
\addplot [blues3,mark=square*] table [x={threads}, y={TimeoutBinaryTree}]
{results/0C2000K.txt}; 
\addplot [orange,mark=diamond*] table [x={threads}, y={Danaunbalanced}]
{results/0C2000K.txt}; 
\addplot [gray,mark=asterisk] table [x={threads}, y={GlobalLockBinaryTree}]
{results/0C2000K.txt};
\addplot [black,mark=x] table [x={threads}, y={DominationLockingBinaryTree}]
{results/0C2000K.txt};
%\addplot  [magenta,mark=pentagon] table [x={threads}, y={rb}]
% {results/100C200K.txt};
\end{axis}
\end{tikzpicture}

\ref{unbalancedLegened}
\end{center}
\caption{Throughput of unbalanced data
structures.\label{evaluation:results:unbalanced}}
\end{figure*}


\begin{figure*}
\begin{center}
\begin{tikzpicture}
\begin{axis}[mystyle,balanced,
				title=100\% read-only operations,
				ylabel = {Key range [$0,2\cdot10^4$]},
				]
\addplot [blues3,mark=square*] table [x={threads}, y={TimeoutTreap}]
{results/100C20K.txt}; 
\addplot [orange,mark=diamond*] table [x={threads}, y={DanaAVL}]
{results/100C20K.txt}; 
\addplot [darkspringgreen,mark=triangle*] table [x={threads}, y={Bronson}]
{results/100C20K.txt}; 
\addplot [coralred,mark=*] table [x={threads}, y={JavaSkipList}]
{results/100C20K.txt};
\addplot [gray,mark=asterisk] table [x={threads}, y={GlobalLockTreap}]
{results/100C20K.txt};
\addplot [black,mark=x] table [x={threads}, y={DominationLockingTreap}]
{results/100C20K.txt};
\end{axis}
\end{tikzpicture}
\begin{tikzpicture}
\begin{axis}[mystyle,balanced,title=50\% read-only operations]
\addplot [blues3,mark=square*] table [x={threads}, y={TimeoutTreap}]
{results/50C20K.txt}; 
\addplot [orange,mark=diamond*] table [x={threads}, y={DanaAVL}]
{results/50C20K.txt}; 
\addplot [darkspringgreen,mark=triangle*] table [x={threads}, y={Bronson}]
{results/50C20K.txt}; 
\addplot [coralred,mark=*] table [x={threads}, y={JavaSkipList}]
{results/50C20K.txt};
\addplot [gray,mark=asterisk] table [x={threads}, y={GlobalLockTreap}]
{results/50C20K.txt};
\addplot [black,mark=x] table [x={threads}, y={DominationLockingTreap}]
{results/50C20K.txt};
\end{axis}
\end{tikzpicture}
\begin{tikzpicture}
\begin{axis}[mystyle,balanced,title=0\% read-only operations]
\addplot [blues3,mark=square*] table [x={threads}, y={TimeoutTreap}]
{results/0C20K.txt}; 
\addplot [orange,mark=diamond*] table [x={threads}, y={DanaAVL}]
{results/0C20K.txt}; 
\addplot [darkspringgreen,mark=triangle*] table [x={threads}, y={Bronson}]
{results/0C20K.txt}; 
\addplot [coralred,mark=*] table [x={threads}, y={JavaSkipList}]
{results/0C20K.txt};
\addplot [gray,mark=asterisk] table [x={threads}, y={GlobalLockTreap}]
{results/0C20K.txt};
\addplot [black,mark=x] table [x={threads}, y={DominationLockingTreap}]
{results/0C20K.txt};
\end{axis}
\end{tikzpicture}


\begin{tikzpicture}
\begin{axis}[mystyle,balanced, 
				ylabel = {Key range [$0,2\cdot10^6$]}, ]
\addplot [blues3,mark=square*] table [x={threads}, y={TimeoutTreap}]
{results/100C2000K.txt}; 
\addplot [orange,mark=diamond*] table [x={threads}, y={DanaAVL}]
{results/100C2000K.txt}; 
\addplot [darkspringgreen,mark=triangle*] table [x={threads}, y={Bronson}]
{results/100C2000K.txt}; 
\addplot [coralred,mark=*] table [x={threads}, y={JavaSkipList}]
{results/100C2000K.txt};
\addplot [gray,mark=asterisk] table [x={threads}, y={GlobalLockTreap}]
{results/100C2000K.txt};
\addplot [black,mark=x] table [x={threads}, y={DominationLockingTreap}]
{results/100C2000K.txt};
\end{axis}
\end{tikzpicture}
\begin{tikzpicture}
\begin{axis}[mystyle,balanced]
\addplot [blues3,mark=square*] table [x={threads}, y={TimeoutTreap}]
{results/50C2000K.txt}; 
\addplot [orange,mark=diamond*] table [x={threads}, y={DanaAVL}]
{results/50C2000K.txt}; 
\addplot [darkspringgreen,mark=triangle*] table [x={threads}, y={Bronson}]
{results/50C2000K.txt}; 
\addplot [coralred,mark=*] table [x={threads}, y={JavaSkipList}]
{results/50C2000K.txt};
\addplot [gray,mark=asterisk] table [x={threads}, y={GlobalLockTreap}]
{results/50C2000K.txt};
\addplot [black,mark=x] table [x={threads}, y={DominationLockingTreap}]
{results/50C2000K.txt};
\end{axis}
\end{tikzpicture}
\begin{tikzpicture}
\begin{axis}[mystyle,balanced]
\addplot [blues3,mark=square*] table [x={threads}, y={TimeoutTreap}]
{results/0C2000K.txt}; 
\addplot [orange,mark=diamond*] table [x={threads}, y={DanaAVL}]
{results/0C2000K.txt}; 
\addplot [darkspringgreen,mark=triangle*] table [x={threads}, y={Bronson}]
{results/0C2000K.txt}; 
\addplot [coralred,mark=*] table [x={threads}, y={JavaSkipList}]
{results/0C2000K.txt};
\addplot [gray,mark=asterisk] table [x={threads}, y={GlobalLockTreap}]
{results/0C2000K.txt};
\addplot [black,mark=x] table [x={threads}, y={DominationLockingTreap}]
{results/0C2000K.txt};
\end{axis}
\end{tikzpicture}
\ref{balancedLegened}
\end{center}
\caption{Throughput of balanced data
structures.\label{evaluation:results:balanced}}
\end{figure*}
\section{Related Work}\label{sec:related}
\paragraph{Concurrent Data Structures}
Many sophisticated concurrent data structures (e.g., \cite{ArbelA2014,DrachslerVY2014,NatarajanM2014,BrownER2014,CrainGR2013,BraginskyP2012,
AfekKKMT2012,EllenFRB2010,BronsonCCO2010,HerlihyLLS2007,Michael:1996})
were developed and used in concurrent software systems~\cite{Ohad:OOPSLA11}.
Implementing efficient synchronization for such data structures is considered a challenging and error-prone task~\cite{Ohad:OOPSLA11,Doh:SPAA04,Jin:2012}.
As a result, concurrent data structures are manually implemented by concurrency experts.
This paper shows that (in some cases) an automatic algorithm can produce synchronization which is comparable to synchronization implemented by experts.

\paragraph{Locking Protocols}
Locking protocols are used in databases and shared memory systems to guarantee correctness
of concurrently executing transactions~\cite{Weikum:2001,BHG:Book87}.
Our approach can be seen as a way to extend many existing locking protocols by combining them with an optimistic concurrency control.
In particular, our approach extends the following locking protocols:
two-phase~\cite{Eswaran:1976}, tree locking~\cite{SilberschatzK1980}, DAG locking~\cite{CH:PODS95} and domination locking~\cite{Gueta2011}.
We demonstrate this by showing that extending the  domination locking protocol enables producing efficient concurrency control for
dynamic data structures.


\paragraph{Lock Inference Algorithms}
There has been a lot of work on automatically inferring locks for transactions.
Most of the algorithms in the literature infer locks for following the two-phase
locking protocol~\cite{MZGB:POPL06,Emmi06POPL,gudka2012lock,CCG:PLDI08,HFP:TRANSACT06,CGE:CC08}.
Our approach can potentially be used to optimized the synchronization produced by these algorithms.
For example, for the algorithms that employ a two-phase variant in which all locks are acquired at the beginning of transactions (e.g.,~\cite{gudka2012lock,CCG:PLDI08}),
our approach can be used to defer the locking (e.g., to just before the first write operation) and to eliminate some of the locking operations.


\paragraph{Transactional Memory}
Transactional memory approaches (TMs) dynamically resolve inconsistencies
and deadlocks by rolling back partially completed transactions.
%
Unfortunately, in spite of a lot of effort and many TM implementations (see~\cite{HLR:SLCA2010}), existing TMs
have not been widely adopted due to various concerns~\cite{DuffyTM2010,Cascaval:2008,mckenneyParallel}, including high runtime overhead,
poor performance and limited ability to handle irreversible operations.
In particular, modern concurrent programs (and concurrent data structures) are typically based on hand-crafted synchronization, rather than  on a TM approach~\cite{Ohad:OOPSLA11}.

In a sense, our approach can be seen as a specialized TM approach that can be practically used to handle concurrent data structure.


%\paragraph{Lock Elision for Read-Only Transactions}
\paragraph{Lock Elision}
Our approach is inspired by the idea of \emph{sequential locks}~\cite{mckenneyParallel} and the approach presented in~\cite{Nakaike:2010}.
But  in contrast to the approaches in \cite{mckenneyParallel,Nakaike:2010} which are designed to handle read-only transactions,
our approach handles read-only prefixes of transactions that update the shared memory. 
Moreover, using these approaches for a highly-contended data structure (as in Section~\ref{sec:eval}) will provide limited performance, 
because each update transaction causes all the read-only transactions to abort.

There are some transactional memory techniques to elide locks from arbitrary critical sections (e.g.,~\cite{Rajwar:2002:TLE:635508.605399,Roy:2009:RSS:1519065.1519094,Afek:2014:SHL:2611462.2611482}).
In these techniques a transaction executes the critical section speculatively without acquiring the lock.
When a transaction is aborted, it can acquire the lock and execute the critical section non-speculatively.
In contrast to our approach, these techniques cannot combine speculative and non-speculative execution of the same transaction.







\section{Discussion}\label{sec:discussion}

%We have made the case that automatic synchronization can be a viable approach for producing scalable concurrent algorithms from %legacy sequential code.
The development of scalable concurrent programs today
heavily relies on custom-tailored implementations, which require painstaking correctness proofs.
In this paper, we have shown a relatively simple  transformation that can facilitate this labor-intensive process, and
thus make scalable synchronization more readily available.
The input for our transformation is a conventional lock-based concurrent program, which may be either constructed manually or
synthesized from sequential code. Our source-to-source transformation then makes judicious use of optimism in order to
eliminate principal concurrency bottlenecks in the given program and improve its scalability. 

We have illustrated our method for a number of data structures -- unbalanced and balanced search trees,  as well as
skip lists supporting range queries. 
In all cases, the transformed code performed significantly better than the original
lock-based one. It also scaled comparably  to hand-crafted
implementations that took considerably more effort to produce.
Moreover, extending the synthesized code with new functionalities
such as range queries was immediate. 
In these examples, we have manually applied our transformation (according to the algorithm presented in Section~\ref{sec:algorithm}). An interesting direction for future work would be to create a tool that automatically applies our transformation at compile time.

Our approach makes use of a common pattern in data structures, where an operation typically begins with a long read-only traversal, followed by a handful of (usually local) modifications.
A promising direction for future work  is to try and
exploit similar patterns in order to parallelize or remove locks in other types of code (not data structures), for example, programs that rely on two-phase locking.
Furthermore, for programs that follow different patterns, other combinations of optimism and pessimism may prove effective.

Finally, there still remains a gap between the performance achievable by manually optimized solutions and what we could achieve automatically. Our algorithm induces inherent overhead for tracking all operations in the read-only phase for later verification.
In specific data structures, these checks might be redundant, but it is difficult  to detect this automatically. We believe that
it may well be possible to enhance  transformations such as ours with computer-assisted optimizations. For example, a programmer may provide hints regarding certain
invariants that are always preserved in the code, in order to eliminate the need for tracking some values for later
validation. Such optimizations have the potential to bridge the remaining performance gap, while requiring far less work
for proving correctness -- instead of proving that the entire construction is correct, the developer would only need to
prove that her program maintains the specific invariants used.

\appendix

\section{Formal Correctness Proof}\label{sec:formal-proof}

We now formalize the correctness arguments made in Section~\ref{sec:proof}. 
First we define our model and the correctness properties of the algorithm for
which we provide the proof.

\paragraph{Model}

We consider an asynchronous shared memory model, where independent threads
interact via shared memory objects. 
Every thread executes a sequence of operations, each of which is invoked with certain parameters and returns a response.
An operation's execution consists of a sequence of primitive \emph{steps}, beginning with an \emph{invoke} step, followed by
atomic accesses to shared objects, and ending with a \emph{return} step. Steps also modify the executing thread's local variables.

A \emph{configuration} is an assignment of values to all shared and local variables. Thus, each step takes the system from one
configuration to another. Steps are deterministically defined by the data structure's protocol and the current configuration.
In the \emph{initial configuration}, each variable holds its initial value.

An \emph{execution} is an alternating sequence of configurations and steps,
$C_0,s_1,C_1, \ldots,s_i,C_i,\ldots,$
where $C_0$ is an initial configuration,
and each configuration $C_i$ is the result of
executing step $s_i$ on configuration $C_{i-1}$.
We only consider finite executions in this paper.
An execution is \emph{sequential} if steps of different operations are not interleaved.
In other words, a sequential execution is a sequence of operation executions.

Two executions are \emph{indistinguishable} to a set of operations if each
operation in the set executes the same steps on shared objects, and
receives the same value from those objects, in both executions. A step $\tau$
by operation $op$ is \emph{invisible} to all other operations 
if the executions with and without $\tau$ are indistinguishable to
$\op\setminus \{op\}$. For example, read steps are invisible.

\paragraph{Correctness}

The correctness of a data structure is defined in terms of its external behavior, as reflected in values returned by invoked operations.
Correctness of a code transformation is proven by showing that the synthesized code's executions are equivalent to ones of the original code,
where two executions are  \emph{equivalent} if every thread invokes the same
operations in the same order  in both executions, and gets the same result for each operation. More formally, we say in this paper that a code transformation is \emph{correct} if every execution of the transformed code
is equivalent to some execution of the original code.

The widely-used correctness criterion of serializability relies on equivalence to sequential executions in order to
link a data structure's behavior under concurrency to its sequentially specified behavior. Since equivalence is transitive,
we get that any code transformation satisfying our correctness notion, when applied to serializable code, yields code that is also serializable.
If the code transformation further ensures the real-time order of operations (i.e., operations that do not overlap appear in the same order in 
executions of the transformed and original code), then linearizability (atomicity) is also invariant under the transformation.
Another important aspect of correctness is preserving the progress conditions of the original code, for example, deadlock-freedom.

In this paper, we are not concerned with internal consistency (as required e.g., by opacity~\cite{GuerraouiK2008} or the validity notion of~\cite{LevAriCK2014}),
which restricts the configurations an operation might see during its execution.
This is because our code transformation uses timeouts and exception handlers to overcome unexpected behavior that may arise when a thread sees an inconsistent view of global variables (similar to~\cite{Nakaike:2010}).


\paragraph{Formal Proof}
We consider a finite execution $\pi$ of the transformed
code, and find an equivalent execution of the original lock-based
code.
Each operation in $\pi$ is an interleaved sequence of read-only phases and validation phases followed by a (single) update phase -- or a prefix of such pattern.
For each operation in $\pi$ we consider its \emph{successful validation}, i.e.,
the last (successful) execution of a validation phase when switching from the read-only
phase to the update phase. Each operation executes at most one successful 
validation. The (complete) read-only phase preceding the successful
validation phase is called \emph{successful read-only
phase} .
Note that each operation executes at most one successful read-only
phase.
Towards proving equivalence to the original code execution, for each operation $op$, we remove the prefix of $op$ that precedes the successful read-only phase.
This includes completely removing operations that have no successful read-only phase.
We call the resulting execution $\hat{\pi}$.
%Finally, we remove all validation phases executed during the successful
%read-only phase that are not the unique successful  validation phase of
%the operation.
These prefixes include read steps as well as tryLock and
unlock steps.
Removing read steps is invisible to other processes. Since we remove all tryLock steps that have failed to acquire locks, 
all remaining tryLock are successful also in $\hat{\pi}$. In addition, since the operation discards all local (private) state when restarting the read-only phase, $\pi$ and $\hat{\pi}$ are indistiguishable to all operations that have completed in $\pi$.
\begin{claim}
\label{claim:pipihat}
$\pi$ and $\hat{\pi}$ are equivalent.
\end{claim}

Denote by $e_1, e_2, \ldots, e_k$ the sequence of the first steps of
the read set validation in the execution of successful validation
phases, by their order in $\hat{\pi}$, where $e_i$ is a step of the operation $op_{i}$ executed by process $p_{i}$.
(Possibly $p_i=p_j$ for $j \neq i$).

For every operation $op_{i}$, consider the partition of $\hat{\pi}$ to
the following intervals $\hat{\pi}=\alpha_i\beta_i\gamma_i$, such that
$\alpha_i$ includes the execution interval of $op_{i}$'s (successful) read-only phase
(denote $op_{i}$'s read set $rs_{i}$); $\beta_i=\beta_{i_1}\beta_{i_2}$, is the
minimal execution interval of $op_{i}$'s successful validation phase;
in $\beta_{i_1}$, $op_{i}$ acquires 
locks on its lock set, denoted $ls_{i}$; 
$e_i$ is the first step of $\beta_{i_2}$, namely the read set validation
interval.

Let \op\ be the set of operations in $\hat{\pi}$.
The next claim follows from the fact that the validation phase of $op_{i}$
in $\beta_i$ is successful, and includes locks and versions
re-validation: 
%\eshcar{need to prove these? or are these clear from the alg description?}

\begin{claim}
\label{claim:locks}
No operation in $\op\setminus\{op_{i}\}$ holds a lock in
$\alpha_i\beta_{i_1}$ that is associated with an object $obj$ in $rs_{i}$ after $op_{i}$'s first
read of $obj$ in $\alpha_i$.
\end{claim}


We next project
object versions out of $\hat{\pi}$'s configurations, and remove all accesses (reads and writes) to object versions.
That is, we replace steps that access versions with local steps that modify the operation's local memory only.
Note that we get an execution with exactly the same invocations, responses, local states, and shared object states, but without 
versions. 
We call the resulting execution $\pi'$.

%Essentially, $\pi'$ is a projection of $\pi$
%excluding versions and all prefixes of the operations preceding their (single)
%successful read-only phase. 
%%, and all failed validation attempts.
%Therefore, all operations that returned a value in $\pi'$ return the same values as in $\pi$:
\begin{claim}
\label{claim:pihatpitag}
$\pi'$ and $\hat{\pi}$ are equivalent.
\end{claim}

Our main lemma constructs the execution of a fully-pessimistic locking code. 
The core idea is to replace the optimistic read-only phase
and validation phase of each operation with a solo execution of the
pessimistic lock-based read phase taking
place at the point where all objects in the lock set are locked.
\begin{lemma}
\label{lemma:pitagtag}
There is an execution of lock-based algorithm that is equivalent to $\pi'$.
\end{lemma}
\begin{proof}
We start with the execution $\pi_0=\pi'$.
For every $i \geq 0$, we show how to perturb $\pi_i$ to
obtain an execution $\pi_{i+1}$. For each $j\geq i+1$, let
  $\beta_{j}^{'}=\beta_{j_1}^{'}\beta_{j_2}^{'}$ be the minimal interval
  containing $op_{j}$'s validation phase in $\pi_{i+1}$, where $\beta_{j_1}^{'}$
  is the minimal interval containing $op_{j}$'s tryLock phase.
  Denote the configuration between $\beta_{j_1}^{'}$ and $\beta_{j_2}^{'}$
  $C_{j}$. In $\pi_{i+1}$ the following conditions are
satisfied:
\begin{enumerate}
  \item \label{cond:lp} The operations $op_{1},\ldots,op_{i}$ follow the
  fully-pessimistic locking algorithm, while the rest of the operations
  $\opt_{i+1}=\op\setminus\{op_{1},\ldots,op_{i}\}$ proceed according to our
  semi-optimistic algorithm.
  \item \label{cond:locks} 
  For $j\geq i+1$, no operation in $\op\setminus\{op_{j}\}$ holds a lock in
  $C_{j}$ that is associated with an object $obj$ in $rs_{j}$.
  \item \label{cond:writes} 
  For $j\geq i+1$, no operation in
  $\op\setminus\{op_{j}\}$ writes to an object $obj$ in $rs_{j}$ after
  $op_{j}$ first read $obj$ before $C_{j}$.
  \item \label{cond:trylocks} 
  For $j\geq i+1$, all try-lock steps by $op_j$ are invisible to
  $\op\setminus\{op_{j}\}$.
  %after $op_{j}$'s last read $obj$ before $\beta_j^{'}$
  \item \label{cond:equiv} $\pi'$ and $\pi_{i+1}$ are equivalent.
\end{enumerate}

For $\opt_{k+1}=\emptyset$, we get an execution where all operations follow the
pessimistic locking algorithm, and by Condition~\ref{cond:equiv} $\pi_{k+1}$ is
equivalent to $\pi'$ and we are done.

The proof is by induction on $i$. For the base case we consider
the execution $\pi'=\pi_0$. Condition~\ref{cond:lp} holds since none of
the operations in this execution follow the full locking algorithm.
Conditions~\ref{cond:locks} and~\ref{cond:writes} hold by
Claim~\ref{claim:locks}, 
and since accesses to objects (other than versions) are similar in $\hat{\pi}$ and
$\pi'$. Condition~\ref{cond:trylocks} holds since by construction, in $\pi'$ every step accessing an object, either
for locking it or for validating it is not locked, finds the object not locked.
.
Condition~\ref{cond:equiv} vacuously holds since $\pi'$ and $\pi_0$ are the
same execution.

For the induction step, assume $\opt_i \neq \emptyset$ and
the execution
$\pi_i=\alpha_i^{'}\beta_{i_1}^{'}\beta_{i_2}^{'}\gamma_i^{'}$ satisfies
the above conditions.
We replace $\pi_i$ with
$\pi_{i+1}=\alpha_i^{''}\beta_{i_1}^{''}\delta_i\beta_{i_2}^{''}\gamma_i^{'}$,
such that $\alpha_i^{''}$, $\beta_{i_1}^{''}$, and $\beta_{i_2}^{''}$ are the
projection of $\alpha_i^{'}$, $\beta_{i_1}^{'}$ and $\beta_{i_2}^{'}$, excluding
the steps by $op_{i}$, while $\delta_i$ is a $p_{i}$-only execution
interval in which $p_{i}$ follows the locking algorithm while
reading $rs_{i}$; after $\delta_{i}$, $p_{i}$ holds the locks on all
objects in $ls_{i}$, and holds no lock on other objects. 
In other words, we replace the optimistic read-only phase and validation phase
of $op_{i}$ with an execution of the original
locking algorithm, taking place at $C_{j}$.
%at the point just before the read set validation starts.

By Condition~\ref{cond:locks} of the induction hypothesis no operation holds 
locks associated with objects in the read set of $op_{i}$ in $C_{i}$, therefore,
$p_{i}$ can acquire the locks on these objects while executing $\delta_{i}$.
By Condition~\ref{cond:writes} of the induction hypothesis no
operation writes to an object $obj$ in $rs_{i}$ after
$op_{i}$ first read $obj$ before $C_{i}$, hence $op_{i}$ reads the same
values in its read set in $\pi_i$ and $\pi_{i+1}$. After $\delta_{i}$,
$op_{i}$ holds the locks on all objects in $ls_i$, hence it can continue with
the execution of the locking algorithm.

In $\alpha_i^{''}\beta_{i_1}^{''}$ we only removed read steps and tryLock steps
by $op_{i}$ that are invisible to all other operations, by
Condition~\ref{cond:trylocks}. Therefore, the executions
$\alpha_i^{'}\beta_{i_1}^{'}$, ending with configuration $C'$, 
and $\alpha_i^{''}\beta_{i_1}^{''}\delta_i$, ending with configuration $C''$, 
are indistinguishable to all operations in $\op\setminus\{op_{i}\}$. 
In addition, in $\beta_{i_2}^{''}$ we only removed invisible read steps.
The values of all shared objects and locks are the same in $C'$ and $C''$,
hence the executions $\alpha_i^{'}\beta_{i_1}^{'}\beta_{i_2}^{'}\gamma_i^{'}$
and $\alpha_i^{''}\beta_{i_1}^{''}\delta_i\beta_{i_2}^{''}\gamma_i^{'}$ are
indistinguishable to all operations in $\op\setminus\{op_{i}\}$. 

The indistinguishability and the induction hypothesis imply that Conditions~\ref{cond:locks},~\ref{cond:writes},~\ref{cond:trylocks} hold.
In addition, this implies that (1)~the projection of the execution $\pi_{i+1}$ on $op_{i}$
follows the full pessimistic locking algorithm satisfying Condition~\ref{cond:lp}, and
(2)~all operations return the same value in $\pi_{i+1}$ as in $\pi'$, which
means Condition~\ref{cond:equiv} holds.

%It is left to show that $\pi_{i+1}$ satisfies
%Conditions~\ref{cond:locks},~\ref{cond:writes},~\ref{cond:trylocks}.
%This is straightforward from the induction hypothesis and the fact that only
%$op_{i}$ changed its excution in the last iteration, and specifically
%removed all its try-lock steps, and since $\delta_i$ precedes $C_{j}$ for all
%$j\geq i+1$ in $\pi_{i+1}$.
\end{proof}

By Lemma~\ref{lemma:pitagtag}, Claim~\ref{claim:pipihat} and Claim~\ref{claim:pihatpitag} we conclude the following
theorem:
\begin{theorem}
Every execution of the transformed code is equivalent to an
execution of the original locking code.
\end{theorem}

%\section{Appendix Title}

%This is the text of the appendix, if you need one.

\acks

Removed for blind review.

%Acknowledgments, if needed.

% We recommend abbrvnat bibliography style.



% The bibliography should be embedded for final submission.

\bibliography{myRef}
\bibliographystyle{abbrvnat}
%\bibliographystyle{plain}

%Maya:  Commented out template bib
%\begin{thebibliography}{}
%\softraggedright
%\bibitem[Smith et~al.(2009)Smith, Jones]{smith02}
%P. Q. Smith, and X. Y. Jones. ...reference text...
%\end{thebibliography}


\end{document}

%                       Revision History
%                       -------- -------
%  Date         Person  Ver.    Change
%  ----         ------  ----    ------

%  2013.06.29   TU      0.1--4  comments on permission/copyright notices

