\section{Automatic Transformation}\label{sec:algorithm}
%The main idea: 
We address the problem of introducing 
optimism to a pessimistic locking protocol, 
performing load and store steps on locked objects only, 
given a sequential algorithm and a ``black box'' 
transformation that ensures this protocol.
The idea of optimistic synchronization is to read 
shared variables without locks and start using the 
locking protocol when the operation reaches a store step.
This scheme is applicable to many data structure implementations
since the common behaviour of data structure operations 
is to first traverse the data structure and 
perform mostly local modifications at the end of the operation.
Our algorithm shows that it is redundant to 
acquire locks if they are freed before any change is made.    

Conceptually, the optimistic scheme separates the operation to three
\emph{phases}, an optimistic \emph{read-only phase},
a pessimistic \emph{update phase} and a 
\emph{validation phase} that conjoins them. 
The optimistic read-only phase traverses the data 
structure without taking any locks while maintaining 
minimal sufficient information to later ensure the 
correctness of the traversal.
The update phase maintains the original locking protocol. 
In order to bridge between the optimistic read phase and the  
pessimistic update phase we use the validation phase. 
The validation phase first locks the objects required by
the locking protocol and then validates the correctness
of the of the read-only phase, allowing the 
update phase to run as if the an execution of the locking
protocol took place.  

  


\subsection{Transformation Details}
We start our transformation by applying three preparation steps
to the given sequential algorithm. 
First, we use the ``black box'' transformation to produce an
algorithm that ensures the locking protocol. Second, we add 
version numbers to shared memory objects, incremented every 
time the object is locked. These version numbers are used 
to validate the correctness of the read-only phase during 
the validation phase as well as at timeouts and exceptions.
Lastly, we identify the read only prefix of each operation. 

Next we explain what the transformation does in each phase. 

\paragraph{Read-only Phase} 
  During this phase the operation maintains a \emph{locked set}
  and a \mph{read set}. 
  The locked set is used to track the lock and unlock steps
  of the locking protocol, thus, we replace each lock step 
  with a a step adding the object to the locked set and each
  unlock step with a step removing the object from the read
  set. (Since the locking protocol might allow re-entrant locks, 
  our locked set allows duplications and the remove operation
  removes one of the duplications of the object).  
  
The read set is used to track all objects accessed by the 
operation in order to later validate that this read set
belongs to a consistent view of shared memory.  
It contains a mapping between references to objects loaded 
to the local version at the time it was read. Object are 
added to the read set immediately after they are added to 
the locked set.  

Incrementing the local version of an object is not 
done atomically with the lock acquisition, thus, 
the object is also checked to be unlocked, if it is locked
the operation restarts from the beginning. 
This also ensures that the operation never reads old values 
with a new version. 
 
The read phase does not validate reads during its traversal, 
in order to avoid infinite loops, a timeout is set. 
If the operation reaches the timeout, a read set 
validation takes place (described in the validation phase), 
if it fails the operation restarts from the beginning.
  
  %The use of timeout does not guarantee that the 
  %operation reads a consistent snapshot of the memory, 
  %thus, \ldots 
  %A pseudo code of the transformation can be found in 
  %Figure \ref{figure:readPhaseTransformation}. 

\paragraph{Validation Phase} 
The validation phase is inserted to the code after
the read-only phase. It locks the objects in the locked set
and validates the read set. 

To avoid deadlocks, the locks are acquired using a try\_lock
operation, if the try\_lock fails, the operation unlocks 
previously acquired locks and restarts from the beginning. 
The calls to try\_lock also increment the versions 
of the object,  these objects are also part of the 
operations read set. 
To avoid the operation invalidating itself, 
a successful try\_lock operation is followed by incrementing
the version saved in the read set.  

The read set validation is performed as follows, 
each object saved in the read set is checked to be unlocked 
and that the current version matches the version saved in the 
read set. If this check fails, the operation releases all its
locks and restarts from the beginning. The order between
checking the version and checking that the object is unlocked
is important, as locking and incrementing versions is not atomic. 

  %A pseudo code of the \readSet validation can be found in
  %Figure \ref{figure:readSetValidation}.
  
\paragraph{Update Phase} 
This phase enforces the locking protocol
  while maintaining the local versions, i.e., the local version of 
  an object is incremented every time it is locked.  
  Once the update phase begins, the operation is guaranteed to to 
  finish without restarts.  


\Xomit{  
\subsection{Transformation Details}
We invoke the following steps to a code that implements 
fine grained locking and follows a correct locking
protocol.  

\begin{enumerate}
  \item Add version numbers to shared memory objects, 
  incremented every time the object is locked. These 
  version numbers are used to validate the correctness
  of the optimistic read phase during the validation phase
  as well as timeouts and exceptions.
  
  \item Identify the read only prefix of each data structure 
  operation. In the beginning of the read phase allocate 
  and initialize local \emph{read set} and \emph{timeout}
  objects. The read set object is a storage object used to save 
  references to objects along with version numbers. 
  The timeout object tracks the progress of the operation\ldots 
    
  \item Transform the read only prefix to optimistic traversal 
   by replacing lock steps \ldots 
   Incrementing the version is not atomic with the lock, thus, 
   the object is also checked to be unlocked. 

\end{enumerate}
}
%The use of timeout does not guarantee
%\emph{opacity}~\cite{GuerraouiK2008} or 
%\emph{validity}\cite{LevAriCK2014}. 

%The \readSet validation first checks that the node is unlocked,
%(or locked by the current operation), then it checks that the 
%current version is equal to the version saved in the \readSet. 
 
%ALGORITHM CODE

\begin{figure*}
	\begin{center}
	\begin{subfigure}{.49\textwidth}
		\begin{algorithmic}[1]{}
			{\ttfamily
			\Function{foo}{Node new} \label{code:begin}
			
			\State prev = head
			\State succ = prev.next
			\State prev = succ
			\State succ = succ.next
			\State new.next = succ
			\State prev.next = new
			
			\EndFunction
			}
		\end{algorithmic}
	%\end{subfigure}
   % \begin{subfigure}{.3\textwidth}
   	\bigskip
		\begin{algorithmic}[1]{}
		{\ttfamily
			\Function{foo}{Node new} \label{code:begin}
			\State Node prev = head
			\State prve.lock()
			\State Node succ = prev.next
			\State succ.lock()
			\State prev.unlock()
			\State prev = succ
			\State succ = succ.next
			\State succ.lock()
			\State new.next = succ
			\State prev.next = new
			\State prev.unlock()
			\State succ.unlock()
			\EndFunction
			}
		\end{algorithmic}
	\end{subfigure}
	\begin{subfigure}{.49\textwidth}
		\begin{algorithmic}[1]{}
		{\ttfamily
			\Function{foo}{Node new} \label{code:begin}
			\State Node prev = head
			\State long version = prev.getVersion()
			\If{prev.isLocked()}
				\State \textbf{goto} \ref{code:begin} 
				%\Comment Restart Operation
			\EndIf
			\State readSet.add(prev,version) 
			\State Node succ = prev.next
			\If{succ.isLocked()}
				\State {\textbf goto} \ref{code:begin} 
				%\Comment Restart Operation
			\EndIf
			\State readSet.add(succ,version) 
			\State prev = succ
			\State succ = succ.next
			\If{succ.isLocked()}
				\State \textbf{goto} \ref{code:begin} 
				%\Comment Restart Operation
			\EndIf
			\State readSet.add(succ,version) 
			%\Statex
			\If{!prev.tryLock()} 		
				\State \textbf{goto} \ref{code:begin} 
				%\Comment Restart Operation
			\EndIf	
			\If{!succ.tryLock()} 		
				\State prev.unlock()
				\State \textbf{goto} \ref{code:begin} 
				%\Comment Restart Operation
			\EndIf	
			\If{!readSet.validate()} 		
				\State prev.unlock()
				\State succ.unlock()
				\State \textbf{goto} \ref{code:begin} 
				%\Comment Restart Operation
			\EndIf	
			%\Statex
			\State new.next = succ
			\State prev.next = new
			\State prev.unlock()
			\State succ.unlock()

			\EndFunction
			}
		\end{algorithmic}
	\end{subfigure}
	%\bigskip
	%\hline
	\end{center}
	\caption{
			\label{figure:transformation}}
\end{figure*}

%END CODE 