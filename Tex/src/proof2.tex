\newcommand{\op}{\emph{\textsc{op}}}
\newcommand{\opt}{\textsc{opt}}

\section{Analysis}
\label{sec:proof}

We show  three properties of our automatic transformation. First, we prove that the  transformation is correct, i.e., 
every execution of the  synthesized code is equivalent to some execution of the \emph{locking algorithm}, that is, the
sequential code instrumented with the locking protocol.
This implies that if the locking protocol ensures serializability, then so does our protocol.
Second, we show that the equivalent execution preserves the real-time order of the original one, 
which implies that linearizability is also invariant under the transformation.
Finally, we argue that our transformation preserves deadlock-freedom.
%%%In this section, we provide informal correctness arguments. A formal proof is deferred to Appendix~\ref{sec:formal-proof}.

\paragraph{Transformation Correctness}

Let $\pi$ be a finite execution of the transformed algorithm. We will show an equivalent execution of the locking algorithm.
Let \op\ be the set of operations in $\pi$.
First, we project object versions out of $\pi$'s configurations, and remove all accesses (reads and writes) to object versions.
That is, we replace steps that access versions with local steps that modify the operation's local memory only.
(Note that we get an execution with exactly the same invocations, responses, local states, and shared object states, but without 
versions). 
Second, note that each operation $op \in$ \op\ executes at most one successful read-only
phase, namely a complete read-only phase followed by a successful validation phase.
For each such $op$, we remove the prefix of $op$ that precedes the successful read-only phase.
This includes completely removing operations that have no successful read-only phase.
We call the resulting execution $\pi'$.

We next iteratively perturb $\pi'$ to construct an equivalent execution of the locking protocol.
We order the operations in \op\ according to the order in which the first steps of their respective (successful) read set validation phases occur in $\pi'$,
and label them $op_1, op_2, \ldots, op_k$ in this order. Let $\pi_0 = \pi'$. In each iteration $i \geq 1$, we construct $\pi_i$ by replacing the execution of $op_i$
in $\pi_{i-1}$ with an execution of $op_i$ that follows the locking protocol. For objects that are in the read set but not in the lock set during the validation phase, this means adding lock and unlock operations exactly where the locking protocol has them. For objects in the lock set, this means removing the tryLock and adding a lock where the locking protocol would lock it; the release of this lock remains unchanged. 
Our ability to lock each object in $op_i$'s read set at 
point where the object was first read follows from the following observation:
\begin{observation}
Between the first time in which $op_i$ first reads an object $o$ in its read-only phase, and until the first step of $op_i$'s validation phase, no
 thread locks $o$. 
\end{observation}
The observation follows immediately from the fact that $op_i$'s read validation is successful and that every lock step increases the
respective object's version number. 

It is easy to see that $\pi_{i-1}$ and $\pi_i$ are equivalent. By repeating this for all operations in \op, we get an execution $\pi_k$ of the locking 
protocol.

\paragraph{Real-Time Order}
It is easy to see that $\pi_k$ preserves the real-time order of $\pi$, since it does not change the order of invoke or return steps. 

\paragraph{Progress}
The read and validation phases of our instrumented code do not use blocking locks -- the read-phase does not use locks at all, whereas the 
validation phase uses tryLocks. Therefore, both phases are non-blocking. In principle, the optimistic approach may lead to livelocks, but 
our algorithm fall-back on the pessimistic approach following a bounded number of restarts, and hence cannot livelock. We get that any lack
of progress must be due to blocking in the locking algorithm. Since the section of the code executing the locking protocol is unchanged, and
since we ensure that it begins when holding the same locks as in the original protocol, we get that our transformation does not introduce any
source of spurious blocking that is not present in the original locking protocol. 


\Xomit{

\subsection{Everything below should move to the appendix}

%
Denote by $e_1, e_2, \ldots, e_i, \ldots$ the sequence of the first steps of the
read set validation in the execution of successful validation phases, by their
order in $\pi$, where $e_i$ is a step of the operation $op_{e_i}$ executed by process $p_{e_i}$.
%Two executions are \emph{equivalent} if every process in their associated
%histories invokes the same operations in the same order and gets the
%same results for each operation. \eshcar{can move this def to the model section}
%

Two executions are \emph{indistinguishable} to a set of operations if each
operation in the set executes the same steps on primitive shared objects, and
receives the same value from those primitives, in both executions.

\eshcar{define invisible step?} 

For every operation $op_{e_i}$, consider the partition of $\pi$ to
the following intervals $\pi=\alpha_i\beta_i\gamma_i$, such that
$\alpha_i$ includes the execution interval of $op_{e_i}$'s read-only phase
(denote $op_{e_i}$'s read set $rs_{e_i}$); $\beta_i=\beta_{i_1}\beta_{i_2}$, is the
minimal execution interval of $op_{e_i}$'s successful validation phase; in
$\beta_{i_1}$, $op_{e_i}$ acquires 
locks on its lock set, denoted $ls_{e_i}$; 
$e_i$ is the first step of $\beta_{i_2}$, namely the read set validation
interval.

The next claim follows from the fact that the validation phase of $op_{e_i}$
in $\beta_i$ is successful, and includes locks and versions
re-validation: \eshcar{need to prove these? or are these clear from the
alg description?}

\begin{claim}
\label{claim:locks}
No operation in $\op\setminus\{op_{e_i}\}$ holds or acquires a lock in
$\alpha_i\beta_{i_1}$ on an object $obj$ in $rs_{e_i}$ after $op_{e_i}$ last
read $obj$ in $\alpha_i$.
\end{claim}

Let $\pi'$ be a projection of $\pi$ excluding all steps accessing versions and
all prefixes of the operations preceding their (single) successful read-only
phase. These prefixes include read steps as well as try-lock and unlock
steps. Removing read steps is invisible to other processes. Since we remove all
try-lock steps that have failed to acquire locks, and some successful try-locks,
all remaining try-lock and lock steps are also successful in $\pi'$. 
Therefore, $\pi'$ is a valid execution such that 
all operations in \op\ return the same values as in $\pi$:
\begin{claim}
\label{claim:pipitag}
$\pi$ and $\pi'$ are equivalent.
\end{claim}
Note that by construction, in $\pi'$ every step accessing a lock object, either
for locking it or for validating it is not locked, finds the object not locked.

Our main lemma constructs the execution of a code with full instrumentation of
the locking protocol. The core idea is to replace the optimistic read-only phase
and validation phase of all operation with a solo execution of the read phase
instrumented with the locking protocol taking place at the point where the try
lock phase is completed.
\begin{lemma}
\label{lemma:pitagtag}
Given a locking protocol, there is an execution $\pi''$ of the code
instrumented with the locking protocol that is equivalent to $\pi'$.
\end{lemma}
\begin{proof}
We start with the execution $\pi_1^{'}=\pi'$.
%$\pi_1^{'}=\alpha_1^{'}\beta_{1}^{'}\gamma_1^{'}$, where $\alpha_1^{'}$
%($\beta_1^{'}$ and $\gamma_1^{'}$) is the projection of $\alpha_1$
%($\beta_1$ and $\gamma_1$, respectively) excluding the steps that were removed
%from $\pi$. 
For every $i \geq 1$, we show how to perturb $\pi_i^{'}$ to
obtain an execution $\pi_{i}^{''}$ in which
\begin{enumerate}
  \item \label{cond:lp} The operations $op_{e_1},\ldots,op_{e_i}$ follow the
  full locking protocol, while the rest of the operations
  $\opt_{i+1}=\op\setminus\{op_{e_1},\ldots,op_{e_i}\}$ do not follow the
  full locking protocol.
  \item \label{cond:locks} For each $j\geq i+1$, Let
  $\beta_{j}^{'}=\beta_{j_1}^{'}\beta_{j_2}^{'}$ be the minimal interval
  containing $op_{e_j}$'s validation phase, where $\beta_{j_1}^{'}$ is the
  minimal interval containing $op_{e_j}$'s try-lock phase in $\pi_{i}^{'}$.
  Denote the configuration between $\beta_{j_1}^{'}$ and $\beta_{j_2}^{'}$
  $C_{e_j}$.
  No operation in $\op\setminus\{op_{e_{j}}\}$ holds a lock in $C_{e_j}$
  on an object $obj$ in $rs_{e_{j}}$, and no operation in
  $\op\setminus\{op_{e_{j}}\}$ writes to an object $obj$ in $rs_{e_{j}}$ after
  $op_{e_j}$ last read $obj$ before $C_{e_j}$.
  %after $op_{e_{j}}$ last read $obj$ before $\beta_j^{'}$
  \item \label{cond:equiv} $\pi^{'}$ and $\pi_{i+1}^{'}$ are equivalent.
\end{enumerate}

For $\opt_{i+1}=\emptyset$, we get an execution where all operations follow the
locking protocol, and by Condition~\ref{cond:equiv} $\pi''=\pi_{i+1}^{'}$ is
equivalent to $\pi^{'}$ and we are done.

The proof is by induction on $i$. For the base case we consider
the execution $\pi^{'}=\pi_1^{'}$. Condition~\ref{cond:lp} holds since none of
the operations in this execution follow the full locking protocol.
Condition~\ref{cond:locks} holds by
Claim~\ref{claim:locks}, since in $\pi'$ we only remove read and lock steps, and
since accesses to objects (other than versions) are similar in $\pi$ and $\pi'$.
Condition~\ref{cond:equiv} vacuously holds since $\pi'$ and $\pi_1^{'}$ are the
same execution.

For the induction step, assume $\opt_i \neq \emptyset$ and
the execution
$\pi_i^{'}=\alpha_i^{'}\beta_{i_1}^{'}\beta_{i_2}^{'}\gamma_i^{'}$ satisfies
the above conditions.
We replace $\pi_i^{'}$ with
$\pi_i^{''}=\alpha_i^{''}\beta_{i_1}^{''}\delta_i\beta_{i_2}^{''}\gamma_i^{'}$,
such that $\alpha_i^{''}$, $\beta_{i_1}^{''}$, and $\beta_{i_2}^{''}$ are the
projection of $\alpha_i^{'}$, $\beta_{i_1}^{'}$ and $\beta_{i_2}^{'}$, excluding
the steps by $op_{e_i}$, while $\delta_i$ is a $p_{e_i}$-only execution
interval in which $p_{e_i}$ follows the locking protocol while
reading $rs_{e_i}$; after $\delta_{i}$, $p_{e_i}$ holds the locks on all
objects in $ls_{e_i}$, and holds no lock on other objects. 
In other words, we replace the optimistic read-only phase and validation phase
of $op_{e_i}$ with an execution of a read phase instrumented with the
locking protocol, taking place at $C_{e_j}$.
%at the point just before the read set validation starts.

By Condition~\ref{cond:locks} of the induction hypothesis no operation holds 
locks on objects in the read set of $op_{e_i}$ in $C_{e_i}$, therefore,
$p_{e_i}$ can acquire the locks on these objects while executing $\delta_{i}$.
In addition, by Condition~\ref{cond:locks} of the induction hypothesis no operation
writes to an object $obj$ in $rs_{e_{i}}$ after
$op_{e_i}$ last read $obj$ before $C_{e_i}$, hence $op_{e_i}$ reads the same
values in its read set in $\pi_i^{'}$ and $\pi_i^{''}$. Since after $\delta_{i}$
$op_{e_i}$ holds the locks on all objects in $ls_{e_i}$, it can continue with
the execution of the locking protocol.

In $\alpha_i^{''}\beta_{i_1}^{''}$ we only removed read steps and try-lock steps
by $op_{e_i}$ that are invisible to all other operations. Therefore, the executions
$\alpha_i^{'}\beta_{i_1}^{'}$, ending with configuration $C'$, and
$\alpha_i^{''}\beta_{i_1}^{''}\delta_i$, ending with configuration $C''$, are
indistinguishable to all operations in $\op\setminus\{op_{e_{i}}\}$. 
In addition, in $\beta_{i_2}^{''}$ we only removed read steps, and since the
values of all shared objects and locks are the same in $C'$ and $C''$, the
executions $\alpha_i^{'}\beta_{i_1}^{'}\beta_{i_2}^{'}\gamma_i^{'}$ and
$\alpha_i^{''}\beta_{i_1}^{''}\delta_i\beta_{i_2}^{''}\gamma_i^{'}$ are
indistinguishable to all operations in $\op\setminus\{op_{e_{i}}\}$. 

This implies that (1)~the projection of the execution $\pi_i^{''}$ on $op_{e_i}$
follows the full locking protocol satisfying Condition~\ref{cond:lp}, and
(2)~all operations return the same value in $\pi_i^{''}$ as in $\pi^{'}$, which
means Condition~\ref{cond:equiv} holds.

It is left to show that $\pi_i^{''}$ satisfies Condition~\ref{cond:locks}. 
This is straightforward from the induction hypothesis and the fact that only
$op_{e_i}$ changed its locking pattern in the last iteration and that
$\delta_i$ precedes $C_{e_j}$ for all $j\geq i+1$ in $\pi_i^{''}$ which we
define to be $\pi_{i+1}^{'}$ for the next iteration.
 
\end{proof}

By Lemma~\ref{lemma:pitagtag} and Claim~\ref{claim:pipitag} we conclude that
\begin{theorem}
Any execution of the optimistic instrumentation is equivalent to an
execution of the locking protocol.
\end{theorem}

}

