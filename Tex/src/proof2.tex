\newcommand{\op}{\emph{\textsc{op}}}
\newcommand{\opt}{\textsc{opt}}

\section{Analysis}
\label{sec:proof}

We show  three properties of our automatic transformation. First, we prove that the  transformation is correct, i.e., 
every execution of the  synthesized code is equivalent to some execution of the \emph{locking algorithm}, that is, the
sequential code instrumented with the locking protocol.
This implies that if the locking protocol ensures serializability, then so does our protocol.
Second, we show that the equivalent execution preserves the real-time order of the original one, 
which implies that linearizability is also invariant under the transformation.
Finally, we argue that our transformation preserves deadlock-freedom.
In this section, we provide informal correctness arguments. A formal proof is deferred to Appendix~\ref{sec:formal-proof}.

\paragraph{Transformation Correctness}

Let $\pi$ be a finite execution of the transformed algorithm. We will show an equivalent execution of the locking algorithm.
Let \op\ be the set of operations in $\pi$.
First, we project object versions out of $\pi$'s configurations, and remove all accesses (reads and writes) to object versions.
That is, we replace steps that access versions with local steps that modify the operation's local memory only.
(Note that we get an execution with exactly the same invocations, responses, local states, and shared object states, but without 
versions). 
Second, note that each operation $op \in$ \op\ executes at most one successful read-only
phase, namely a complete read-only phase followed by a successful validation phase.
For each such $op$, we remove the prefix of $op$ that precedes the successful read-only phase.
This includes completely removing operations that have no successful read-only phase.
We call the resulting execution $\pi'$.

We next iteratively perturb $\pi'$ to construct an equivalent execution of the locking protocol.
We order the operations in \op\ according to the order in which the first steps of their respective (successful) read set validation phases occur in $\pi'$.
Denote the sequence of these first validations steps $e_1, e_2, \ldots, e_k$, and the respective operations
$op_1, op_2, \ldots, op_k$ in this order. Let $\pi_0 = \pi'$.
In each iteration $i \geq 1$, we construct $\pi_i$ by replacing the execution of $op_i$
in $\pi_{i-1}$ with an execution of $op_i$ that follows the locking protocol. To do this, we first move all steps of $op_i$ that precede 
its first validation step, $e_i$, to occur immediately before $e_i$ (we know that the read steps return the same values, since validation 
succeeds). We then add steps that lock these objects before their read steps (and unlock
them) as dictated by the locking protocol.
%We then add steps that lock all of them immediately before these moved read
% steps. Finally, we add unlock steps immediately before $e_i$ for objects that are in the read set but not in the lock set during $e_i$, 
Finally we remove the tryLock and validation steps by $op_i$. 

Our ability to lock each object in $op_i$'s read set follows from the following observation:
\begin{observation}
Between the first time in which $op_i$ first reads an object $o$ in its read-only phase, and until the first step of $op_i$'s validation phase, no
 thread locks $o$. 
\end{observation}
The observation follows immediately from the fact that $op_i$'s read validation is successful and that every lock step increases the
respective object's version number. 

It is easy to see that $\pi_{i-1}$ and $\pi_i$ are equivalent. By repeating this for all operations in \op, we get an execution $\pi_k$ of the locking 
protocol.

\paragraph{Real-Time Order}
It is easy to see that $\pi_k$ preserves the real-time order of $\pi$, since it does not change the order of invoke or return steps. 

\paragraph{Progress}
The read and validation phases of our instrumented code do not use blocking locks -- the read-phase does not use locks at all, whereas the 
validation phase uses tryLocks. Therefore, both phases are non-blocking. In principle, the optimistic approach may lead to livelocks, but 
our algorithm fall-back on the pessimistic approach following a bounded number of restarts, and hence cannot livelock. We get that any lack
of progress must be due to blocking in the locking algorithm. Since the section of the code executing the locking protocol is unchanged, and
since we ensure that it begins when holding the same locks as in the original protocol, we get that our transformation does not introduce any
source of spurious blocking that is not present in the original locking protocol. 



