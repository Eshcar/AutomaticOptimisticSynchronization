\subsection{Global Version Algorithm}
The module maintains a counter denoted \emph{global version}.
In high level, the global version is used to identify 
the order of write operations and freshness of objects.   

In the beginning of the read phase, the operation atomically
reads the global counter and saves the returned value in 
a \readV variable. After each load step, the
operation checks that the version field of the object read
is not larger than the value of \readV. If the checks fails,
the operation restarts from the beginning 
(with a new \readV value). Since the version of the object 
is not changed atomically with the write to the object, 
the object is also checked to be unlocked. 
In this implementation the \readSet contains only 
references to all objects read by the operation.
 
The validation phase have an 
additional  requirement, to acquire a unique \writeV. 
To ensure that versions of objects are not decremented, 
acquiring the \writeV needs to be done atomically with 
the validation of the \readSet. One possibility is holding a
lock on the global version during the validation, 
an optimistic approach is to read the global version 
before validation and incrementing it using a CAS 
operation after the validation. In the \readSet validation 
each object's version is compared with the \readV, 
and is checked to be unlocked. 
%TODO why we check the lock 
%(example of an operation higher in the tree).

The only addition to the read-write phase is writing the 
operation's \writeV to every object that is locked. 



\subsection{Requirements on The Locking Protocol}
For example the \emph{two phase locking (2PL)} protocol 
does not allow early lock release (no locks can be acquired after a 
lock was released), also, it does not require acquiring all locks
before the first store operation. However, a stricter version 
of 2PL locking that requires \reqII can be used to achieve optimism. 
The resulting optimistic protocol would be similar to 
\emph{Transactional Locking 2}\cite{DiceSS2006}.   
