\section{Model and  Definitions}\label{sec:model}


\subsection{Shared Memory Data Structures}

We consider an asynchronous shared memory model, where independent threads interact via shared memory objects.
For the sake of our discussion, we do not distinguish among different types of shared memory (e.g., global or heap-allocated). 
In addition, each thread has access to \emph{local} (thread-local) memory.

A \emph{data structure} is an abstract data type exporting a set of \emph{operations}.
A data structure is implemented from a collection of primitive shared \emph{objects} supporting atomic load (read) and store (write) operations.
In Section~\ref{ssec:locking}, we extend the allowed primitive variables to also include locks.  

Each thread executes a sequence of operations, each of which is invoked with certain parameters and returns a response.
An operation's execution consists of a sequence of primitive \emph{steps}, beginning with an \emph{invoke} step, followed by 
atomic accesses to shared objects, and ending with a \emph{return} step. Steps also modify the executing thread's local variables.
A \emph{configuration} is an assignment of values to all shared and local variables. Thus, each step takes the system from one 
configuration to another. Steps are deterministically defined by the data structure's protocol and the current configuration. 
In the \emph{initial configuration}, each variable holds an initial value. 

An \emph{execution} is an alternating sequence of configurations and steps,
$C_0,s_1,\ldots,s_i,C_i,\ldots,$ 
where $C_0$ is an initial configuration,
and each configuration $C_i$ is the result of
executing step $s_i$ on configuration $C_{i-1}$.
An execution is \emph{sequential} if steps of different operations are not interleaved. 
In other words, a sequential execution is a sequence of operation executions.

\subsection{Locking Protocols}
\label{ssec:locking}

A \emph{lock} is a primitive type that supports atomic \emph{lock}, \emph{try\_lock}, and \emph{unlock} operations, 
where try\_lock is a non-blocking attempt to acquire a lock that may fail. 

We assume in this paper a \emph{locking protocol}, which transforms a sequential data-structure to a correct concurrent one by adding locks;
correctness is defined in Section~\ref{sec:linearizable} below.
Examples of such protocols are Tree Locking~\cite{SilberschatzK1980}, and Domination Locking~\cite{Gueta2011}. 
The locking protocol associates a lock with every primitive shared object used by the data structure, and instruments the sequential code 
by adding lock and unlock operations. Intuitively, such protocols automatically perform some sort of ``hand-over-hand'' locking, acquiring locks as
they traverse a linked-list or tree, and releasing locks on previously traversed nodes. They are typically restricted to tree-like data structures.


We assume that the resulting code (obtained by adding the locks) satisfies the following properties:

\begin{itemize}
\item Every (load or store) access by an operation to a shared object is performed when the executing thread holds the lock on that object.
\item The protocol is deadlock-free, i.e., locking  does not introduce deadlocks.
\item The protocol allows early lock release in the sense that  it never needs to hold a lock on an object that it no longer has a pointer to.
 \end{itemize}

The last condition means that even if the protocol holds a lock on an object it no longer holds a pointer to, it is safe to unlock it at this point (and re-acquire the lock later if it is accessed again ), in the sense that correctness, as defined below, is not breached.  


\subsection{Correctness}
\label{ssec:linearizable} 

The correctness of a data structure is defined in terms of its external behavior, as reflected in values returned by invoked operations. 
This is captured by the notion of a \emph{history} -- the history of an execution $\sigma$ is the subsequence  of $\sigma$ consisting 
of invoke and return steps. The widely-used correctness criteria of linearizability and serializability link the data structure's behavior under concurrency to its allowed behavior in sequential executions. The latter is defined by a \emph{sequential specification}, which is a set of its allowed sequential histories. 

A history $H$ is \emph{linearizable}
\cite{HerlihyW1990} if there exists $H'$ that can be created by adding zero or more 
return steps to $H$, and there is a sequential permutation $\pi$ of complete($H'$), 
such that: (1) $\pi$ belongs to the sequential specification; and 
(2) every pair of operations that are not interleaved in $\sigma$, appear in the same order in $\sigma$ and in $\pi$. 
A data structure  is \emph{linearizable}, also called \emph{atomic}, if the histories of all of its executions are linearizable.

A history is \emph{serializable} if it satisifes property (1) above. That is, the real-time order of operations is not required.
A data structure  is \emph{serializable} the histories of all of its executions are serializable.

In this paper, we are not concerned with internal consistency (as required e.g., by opacity~\cite{GuerraouiK2008} or the validity notion of~\cite{LevAriCK2014}), which 
restricts the configurations an operation might see during its execution. 
This is because our code transformation deals with inconsistencies that may arise when a thread sees an inconsistent view of global variables using 
timouts and exception handlers.   
