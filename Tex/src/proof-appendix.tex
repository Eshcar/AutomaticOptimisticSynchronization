
\section{Formal Correctness Proof}\label{sec:formal-proof}

We now formalize the correctness arguments made in Section~\ref{sec:proof}. 
We consider a finite execution $\pi$ of the transformed algorithm,
and find an equivalent execution of the locking protocol. 
Let \op\ be the set of
operations in $\pi$.
Denote by $e_1, e_2, \ldots, e_k$ the sequence of the first steps of the
read set validation in the execution of successful validation phases, by their
order in $\pi$, where $e_i$ is a step of the operation $op_{i}$ executed by process $p_{i}$.
(Possibly $p_i=p_j$ for $j \neq i$).


For every operation $op_{i}$, consider the partition of $\pi$ to
the following intervals $\pi=\alpha_i\beta_i\gamma_i$, such that
$\alpha_i$ includes the execution interval of $op_{i}$'s read-only phase
(denote $op_{i}$'s read set $rs_{i}$); $\beta_i=\beta_{i_1}\beta_{i_2}$, is the
minimal execution interval of $op_{i}$'s successful validation phase; in
$\beta_{i_1}$, $op_{i}$ acquires 
locks on its lock set, denoted $ls_{i}$; 
$e_i$ is the first step of $\beta_{i_2}$, namely the read set validation
interval.

The next claim follows from the fact that the validation phase of $op_{i}$
in $\beta_i$ is successful, and includes locks and versions
re-validation: 
%\eshcar{need to prove these? or are these clear from the alg description?}

\begin{claim}
\label{claim:locks}
No operation in $\op\setminus\{op_{i}\}$ holds or acquires a lock in
$\alpha_i\beta_{i_1}$ on an object $obj$ in $rs_{i}$ after $op_{i}$'s last
read of $obj$ in $\alpha_i$.
\end{claim}

Two executions are \emph{indistinguishable} to a set of operations if each
operation in the set executes the same steps on primitive shared objects, and
receives the same value from those primitives, in both executions. A step $\tau$
by operation $op$ is \emph{invisible} to all other operations 
if the executions with and without $\tau$ are indistinguishable to
$\op\setminus \{op\}$. For example, read steps are invisible.


As explained in Section~\ref{sec:proof}, we let $\pi'$ be a projection of $\pi$ excluding versions and
all prefixes of the operations preceding their (single) successful read-only
phase. These prefixes include read steps as well as tryLock and unlock
steps. Removing read steps is invisible to other processes. Since we remove all
tryLock steps that have failed to acquire locks, 
all remaining tryLock are successful also in $\pi'$. 
Therefore, in $\pi'$ 
%% Idit: removed below, I don't know what valid is, but it's not an execution of the same protocol without versions
%is a valid execution such that 
all operations in \op\ return the same values as in $\pi$:
\begin{claim}
\label{claim:pipitag}
$\pi$ and $\pi'$ are equivalent.
\end{claim}

Note that by construction, in $\pi'$ every step accessing a lock object, either
for locking it or for validating it is not locked, finds the object not locked.

Our main lemma constructs the execution of a code with full instrumentation of
the locking protocol. The core idea is to replace the optimistic read-only phase
and validation phase of each operation with a solo execution of the read phase
instrumented with the locking protocol taking place at the point where the tryLockAll completes.
\begin{lemma}
\label{lemma:pitagtag}
Given a locking protocol, there is an execution of the code
instrumented with the locking protocol that is equivalent to $\pi'$.
\end{lemma}
\begin{proof}
We start with the execution $\pi_0=\pi'$.
For every $i \geq 0$, we show how to perturb $\pi_i$ to
obtain an execution $\pi_{i+1}$. For each $j\geq i+1$, let
  $\beta_{j}^{'}=\beta_{j_1}^{'}\beta_{j_2}^{'}$ be the minimal interval
  containing $op_{j}$'s validation phase in $\pi_{i+1}$, where $\beta_{j_1}^{'}$
  is the minimal interval containing $op_{j}$'s tryLock phase.
  Denote the configuration between $\beta_{j_1}^{'}$ and $\beta_{j_2}^{'}$
  $C_{j}$. In $\pi_{i+1}$ the following conditions are
satisfied:
\begin{enumerate}
  \item \label{cond:lp} The operations $op_{1},\ldots,op_{i}$ follow the
  full locking protocol, while the rest of the operations
  $\opt_{i+1}=\op\setminus\{op_{1},\ldots,op_{i}\}$ proceed according to our protocol.
  \item \label{cond:locks} 
  For $j\geq i+1$, no operation in $\op\setminus\{op_{j}\}$ holds a lock in
  $C_{j}$ on an object $obj$ in $rs_{j}$.
  \item \label{cond:writes} 
  For $j\geq i+1$, no operation in
  $\op\setminus\{op_{j}\}$ writes to an object $obj$ in $rs_{j}$ after
  $op_{j}$ last read $obj$ before $C_{j}$.
  \item \label{cond:trylocks} 
  For $j\geq i+1$, all try-lock steps by $op_j$ are invisible to
  $\op\setminus\{op_{j}\}$.
  %after $op_{j}$'s last read $obj$ before $\beta_j^{'}$
  \item \label{cond:equiv} $\pi'$ and $\pi_{i+1}$ are equivalent.
\end{enumerate}

For $\opt_{k+1}=\emptyset$, we get an execution where all operations follow the
locking protocol, and by Condition~\ref{cond:equiv} $\pi_{k+1}$ is
equivalent to $\pi'$ and we are done.

The proof is by induction on $i$. For the base case we consider
the execution $\pi'=\pi_0$. Condition~\ref{cond:lp} holds since none of
the operations in this execution follow the full locking protocol.
Conditions~\ref{cond:locks} and~\ref{cond:writes} hold by
Claim~\ref{claim:locks}, and since in $\pi'$ we only remove read and lock steps,
and since accesses to objects (other than versions) are similar in $\pi$ and
$\pi'$. Condition~\ref{cond:trylocks} holds by the observation above.
Condition~\ref{cond:equiv} vacuously holds since $\pi'$ and $\pi_0$ are the
same execution.

For the induction step, assume $\opt_i \neq \emptyset$ and
the execution
$\pi_i=\alpha_i^{'}\beta_{i_1}^{'}\beta_{i_2}^{'}\gamma_i^{'}$ satisfies
the above conditions.
We replace $\pi_i$ with
$\pi_{i+1}=\alpha_i^{''}\beta_{i_1}^{''}\delta_i\beta_{i_2}^{''}\gamma_i^{'}$,
such that $\alpha_i^{''}$, $\beta_{i_1}^{''}$, and $\beta_{i_2}^{''}$ are the
projection of $\alpha_i^{'}$, $\beta_{i_1}^{'}$ and $\beta_{i_2}^{'}$, excluding
the steps by $op_{i}$, while $\delta_i$ is a $p_{i}$-only execution
interval in which $p_{i}$ follows the locking protocol while
reading $rs_{i}$; after $\delta_{i}$, $p_{i}$ holds the locks on all
objects in $ls_{i}$, and holds no lock on other objects. 
In other words, we replace the optimistic read-only phase and validation phase
of $op_{i}$ with an execution of a read phase instrumented with the
locking protocol, taking place at $C_{j}$.
%at the point just before the read set validation starts.

By Condition~\ref{cond:locks} of the induction hypothesis no operation holds 
locks on objects in the read set of $op_{i}$ in $C_{i}$, therefore,
$p_{i}$ can acquire the locks on these objects while executing $\delta_{i}$.
By Condition~\ref{cond:writes} of the induction hypothesis no
operation writes to an object $obj$ in $rs_{i}$ after
$op_{i}$ last read $obj$ before $C_{i}$, hence $op_{i}$ reads the same
values in its read set in $\pi_i$ and $\pi_{i+1}$. After $\delta_{i}$,
$op_{i}$ holds the locks on all objects in $ls_i$, hence it can continue with
the execution of the locking protocol.

In $\alpha_i^{''}\beta_{i_1}^{''}$ we only removed read steps and tryLock steps
by $op_{i}$ that are invisible to all other operations, by
Condition~\ref{cond:trylocks}. Therefore, the executions
$\alpha_i^{'}\beta_{i_1}^{'}$, ending with configuration $C'$, 
and $\alpha_i^{''}\beta_{i_1}^{''}\delta_i$, ending with configuration $C''$, 
are indistinguishable to all operations in $\op\setminus\{op_{i}\}$. 
In addition, in $\beta_{i_2}^{''}$ we only removed invisible read steps.
The values of all shared objects and locks are the same in $C'$ and $C''$,
hence the executions $\alpha_i^{'}\beta_{i_1}^{'}\beta_{i_2}^{'}\gamma_i^{'}$
and $\alpha_i^{''}\beta_{i_1}^{''}\delta_i\beta_{i_2}^{''}\gamma_i^{'}$ are
indistinguishable to all operations in $\op\setminus\{op_{i}\}$. 

This implies that (1)~the projection of the execution $\pi_{i+1}$ on $op_{i}$
follows the full locking protocol satisfying Condition~\ref{cond:lp}, and
(2)~all operations return the same value in $\pi_{i+1}$ as in $\pi'$, which
means Condition~\ref{cond:equiv} holds.

It is left to show that $\pi_{i+1}$ satisfies
Conditions~\ref{cond:locks},~\ref{cond:writes},~\ref{cond:trylocks}.
This is straightforward from the induction hypothesis and the fact that only
$op_{i}$ changed its locking pattern in the last iteration, and specifically
removed all its try-lock steps, and since $\delta_i$ precedes $C_{j}$ for all
$j\geq i+1$ in $\pi_{i+1}$.
 
\end{proof}

By Lemma~\ref{lemma:pitagtag} and Claim~\ref{claim:pipitag} we conclude the following
theorem:
\begin{theorem}
Every execution of the automatic transformation is equivalent to an
execution of the locking protocol.
\end{theorem}
