\section{Related Work}\label{sec:related}
\paragraph{Concurrent Data Structures}
Many sophisticated concurrent data structures (e.g., \cite{ArbelA2014,DrachslerVY2014,NatarajanM2014,BrownER2014,CrainGR2013,BraginskyP2012,
AfekKKMT2012,EllenFRB2010,BronsonCCO2010,HerlihyLLS2007,Michael:1996})
were developed and used in concurrent software systems~\cite{Ohad:OOPSLA11}.
Implementing efficient synchronization for such data structures is considered a challenging and error-prone task~\cite{Ohad:OOPSLA11,Doh:SPAA04,Jin:2012}.
As a result, concurrent data structures are manually implemented by concurrency experts.
This paper shows that (in some cases) an automatic algorithm can produce synchronization which is comparable to synchronization implemented by experts.

\paragraph{Locking Protocols}
Locking protocols are used in databases and shared memory systems to guarantee correctness
of concurrently executing transactions~\cite{Weikum:2001,BHG:Book87}.
Our approach can be seen as a way to extend many existing locking protocols by combining them with an optimistic concurrency control.
In particular, our approach extends the following locking protocols:
two-phase~\cite{Eswaran:1976}, tree locking~\cite{SilberschatzK1980}, DAG locking~\cite{CH:PODS95} and domination locking~\cite{Gueta2011}.
We demonstrate this by showing that extending the  domination locking protocol enables producing efficient concurrency control for
dynamic data structures.


\paragraph{Lock Inference Algorithms}
There has been a lot of work on automatically inferring locks for transactions.
Most of the algorithms in the literature infer locks for following the two-phase
locking protocol~\cite{MZGB:POPL06,Emmi06POPL,gudka2012lock,CCG:PLDI08,HFP:TRANSACT06,CGE:CC08}.
Our approach can potentially be used to optimized the synchronization produced by these algorithms.
For example, for the algorithms that employ a two-phase variant in which all locks are acquired at the beginning of transactions (e.g.,~\cite{gudka2012lock,CCG:PLDI08}),
our approach can be used to defer the locking (e.g., to just before the first write operation) and to eliminate some of the locking operations.


\paragraph{Transactional Memory}
Transactional memory approaches (TMs) dynamically resolve inconsistencies
and deadlocks by rolling back partially completed transactions.
%
Unfortunately, in spite of a lot of effort and many TM implementations (see~\cite{HLR:SLCA2010}), existing TMs
have not been widely adopted due to various concerns~\cite{DuffyTM2010,Cascaval:2008,mckenneyParallel}, including high runtime overhead,
poor performance and limited ability to handle irreversible operations.
In particular, modern concurrent programs (and concurrent data structures) are typically based on hand-crafted synchronization, rather than  on a TM approach~\cite{Ohad:OOPSLA11}.

In a sense, our approach can be seen as a specialized TM approach that can be practically used to handle concurrent data structure.


%\paragraph{Lock Elision for Read-Only Transactions}
\paragraph{Lock Elision}
Our approach is inspired by the idea of \emph{sequential locks}~\cite{mckenneyParallel} and the approach presented in~\cite{Nakaike:2010}.
But  in contrast to the approaches in \cite{mckenneyParallel,Nakaike:2010} which are designed to handle read-only transactions,
our approach handles read-only prefixes of transactions that update the shared memory. 
Moreover, using these approaches for a highly-contended data structure (as in Section~\ref{sec:eval}) will provide limited performance, 
because each update transaction causes all the read-only transactions to abort.

There are some transactional memory techniques to elide locks from arbitrary critical sections (e.g.,~\cite{Rajwar:2002:TLE:635508.605399,Roy:2009:RSS:1519065.1519094,Afek:2014:SHL:2611462.2611482}).
In these techniques a transaction executes the critical section speculatively without acquiring the lock.
When a transaction is aborted, it can acquire the lock and execute the critical section non-speculatively.
In contrast to our approach, these techniques cannot combine speculative and non-speculative execution of the same transaction.






