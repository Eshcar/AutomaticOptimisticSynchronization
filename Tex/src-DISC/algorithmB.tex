\renewcommand{\ttdefault}{pcr}
%\algrenewcommand\textkeyword{\texttt}
\algrenewcommand\algorithmicif{\texttt{if}}
\algrenewcommand\algorithmicthen{\texttt{then}}
\algrenewcommand\algorithmicfunction{\textsc{Function}}
\algrenewcommand\algorithmicforall{\texttt{for all}}
\algrenewcommand\algorithmicdo{\texttt{do}}
\algrenewcommand\textproc{\textit}
\newcommand{\codesize}{\footnotesize}



\section{Transformation}\label{sec:algorithm}

We present an algorithm for a source-to-source transformation, whose
goal is to optimize the code of a given data structure implemented using lock-based concurrency control.
In Section~\ref{ssec:locks}, we detail our assumptions about the given code and the locks it uses.
\Xomit{
%% Idit: omitted to save space
The transformation produces a combination of pessimistic and optimistic synchronization
by replacing part of the locking (in the given code) with optimistic concurrency control.
}
Section~\ref{ssec:overview} overviews our general approach to combining optimism and pessimism,
while
%The synthesized code consists of three phases - an optimistic read-only phase, a validation phase,
%and a pessimistic update phase. 
%The transformation first partitions the given code into a read-only phase and
%an update phase as described in Section~\ref{ssec:extendedTran}, and
%then instruments each of these phases separately and adds the validation phase between the two;
Section~\ref{ssec:transformation} details how the code is instrumented. 
%of each of the three phases is produced.

\subsection{Lock-based Data Structures}\label{ssec:locks}

A \emph{data structure} defines a set of \emph{operations} that may be invoked by
clients of the data structure, potentially concurrently.
%
Operations have parameters and local (private) variables. %, which are private to the invocation, and are thus thread-local.
%
The operations interact via \emph{shared memory variables}, which are also called \emph{shared objects}.
%
Each shared object supports atomic \emph{read} (load) and \emph{write} (store) instructions.
%
%Below, we extend shared objects to also support locks.
More formal definitions appear in Appendix~\ref{sec:formal-proof}. 

In addition, each shared object is associated with a lock, which can be unique to the object or common to several (or even all) objects. 
The object supports atomic \emph{lock} and \emph{unlock} instructions.
Locks are exclusive (i.e., a lock can be held by at most one thread at a time), and blocking.
We assume that in the given code every (read
or write) access by an operation to a shared object is performed when the
executing thread holds the lock associated with that object.


The given code only uses the \emph{lock} and \emph{unlock} instructions, while the transformed code can apply in addition atomic non-blocking
\emph{tryLock}  and \emph{isLockedByAnother} instructions: 
 \emph{tryLock}  returns \emph{false} if the lock is currently held by another thread, otherwise it acquires the lock and returns \emph{true};
 \emph{isLockedByAnother}  returns \emph{true} if and only if the lock is currently held by another thread.


\subsection{Combining Optimism and Pessimism}\label{ssec:overview}

% Optimism
%Generally speaking, 
Optimistic concurrency control is a form of lock-free synchronization, which accesses shared variables without locks in the hope that they will not be modified by others before the end of the operation (or more generally, the transaction). To verify the latter, optimistic concurrency control relies on \emph{validation}, which is typically implemented using \emph{version numbers}. If validation fails, the operation restarts. Optimistic execution of update operations requires either performing roll-back (reverting variables to their old values) upon validation failure, or deferring writes to commit time; both approaches induce significant overhead~\cite{Cascaval:2008}. We therefore refrain from speculative shared memory updates.

%The main idea:
The main idea behind our approach is to judiciously use optimistic synchronization only as long as an operation does not update shared state;
we use a standard approach based on version numbers to allow validation of optimistic reads.
Once an operation
writes to shared memory, we revert to pessimistic (lock-based) synchronization. In
other words, we rely on validation at the end of the read-only prefix of an operation in order to render redundant
locks that would have been acquired and freed before the first update.
This scheme is particularly suitable for data structures,
since the common behavior of their operations
is to first traverse the data structure, and then
perform modifications.

Conceptually, our approach divides an operation into three phases: an optimistic \emph{read-only phase},
a pessimistic \emph{update phase} and a \emph{validation phase} that conjoins them.
The read-only phase traverses the data structure without taking any locks, while maintaining 
in thread-local variables
sufficient information to later ensure the correctness of the traversal.
The read phase is \emph{invisible} to other threads, as it updates no shared variables.
The update phase uses the original pessimistic (lock-based) synchronization, with the addition of updating version numbers.
The validation phase bridges between the optimistic and pessimistic ones.
It first locks the objects for which a lock would have been held at this point by
the original locking code, and then validates the correctness
of the read-only phase. This allows the
update phase to run as if an execution of the original pessimistic synchronization
took place. If the validation fails, the operation
restarts. In order to avoid livelock, we set a threshold on the number of restarts.
If the threshold is exceeded, the code falls back on pessimistic execution.
We show below that
it is safe to do so, since our semi-optimistic code is compatible
with the fully pessimistic one.


\paragraph{Phase Transition}
%\label{ssec:extendedTran}
In many cases, the transition from the read-only phase to the update phase occurs at a statically-defined code location. For example, many data structure operations begin with a read-only traversal to locate the key of interest, and when it is found, proceed to execute code that modifies the data structure. This is the case in all the examples we consider in Section~\ref{sec:eval} below.

More generally, it is possible to switch from the optimistic read-only execution (via the validation phase) to pessimistic execution at any point before the first update. Moreover, the phase transition point can be determined dynamically at run time.

One possible way to dynamically track the execution mode is using a flag \textbf{opt}, initialized to true, indicating the optimistic phase. 
Every shared memory update operation is then instrumented with code that checks \textbf{opt}, and if it is true, executes the validation phase followed by setting \textbf{opt} to false and continuing the execution from the same location. 
%Every lock operation is instrumented with a check of \textbf{opt} to determine whether it is to be executed optimistically or pessimistically. 

\subsection{Transforming the Code Phases}\label{ssec:transformation}

\newcommand{\spOne}{\hspace{-3mm}\ }
\newcommand{\spZero}{\hspace{-3mm}}
\begin{figure*}
\codesize
	\begin{center}
	\begin{subfigure}[b]{.47\textwidth}
		\begin{algorithmic}[1]{}
		{\ttfamily
			\Function{addThird}{List list, Node new} \label{code:begin}
			\Statex -----------------------
				\Comment{\textrm{read-only phase}}
			\State                               \label{code:beginRead}
            \State{\spOne}\textbf{list.lock()}
			\State{\spOne}Node prev = list.head
			\State{\spOne}\textbf{prev.lock()}
            \State{\spOne}\textbf{list.unlock()}
			\State{\spOne}Node succ = prev.next
			\State{\spOne}\textbf{succ.lock()}
			\State{\spOne}\textbf{prev.unlock()}
			\State{\spZero}prev = succ
			\State{\spZero}succ = succ.next
			\State{\spZero}\textbf{succ.lock()}  \label{code:endRead}
			\Statex -----------------------
			\State                               \label{code:beginValidation}
			\State
			\State
			\State
			\State
			\State
			\State
            \State                               \label{code:endValidation}
			\Statex -----------------------
							\Comment{\textrm{update phase}}
			\State{\spZero}prev.next = new       \label{code:beginUpdate}
			\State{\spZero}\textbf{new.lock()}
			\State{\spZero}new.next = succ
            \State
			\State{\spZero}\textbf{prev.unlock()}
            \State
			\State{\spZero}\textbf{new.unlock()}
            \State
			\State{\spZero}\textbf{succ.unlock()}  \label{code:endUpdate}
			\EndFunction
			}
		\end{algorithmic}
		\caption{Code with original locking} \label{figure:transformation:before}
	\end{subfigure}
\hspace{0.03\textwidth}
	\begin{subfigure}[b]{.47\textwidth}
		\begin{algorithmic}[1]{}
		{\ttfamily
			\Function{addThird}{List list, Node new} \label{code:begin}
			\Statex -----------------------
			\Comment{\textrm{read-only phase}}
            \State{\spOne}\textbf{lockedSet.init(), readSet.init()} \label{code:initSets}
            \State{\spOne}\textbf{if !track(list)  then {goto} \ref{code:begin}} \label{code:readGhaseGoto0}
			\State{\spOne}Node prev = list.head
			\State{\spOne}\textbf{if !track(prev)  then {goto} \ref{code:begin}} \label{code:readGhaseGoto1}
            \State{\spOne}\textbf{lockedSet.remove(list)} \label{code:lockedSet:remove1}
			\State{\spOne}Node succ = prev.next
			\State{\spOne}\textbf{if !track(succ) then {goto} \ref{code:begin}}  \label{code:readGhaseGoto2}
			\State{\spOne}\textbf{lockedSet.remove(prev)} \label{code:lockedSet:remove2}
			\State{\spZero}prev = succ
			\State{\spZero}succ = succ.next
			\State{\spZero}\textbf{if !track(succ) then {goto} \ref{code:begin}} \label{code:readGhaseGoto3}
			\Statex -----------------------
			\Comment{\textrm{validation phase}}
			\State{\spZero}\textbf{read fence} \label{code:fence}
			\State{\spZero}\textbf{for all obj in lockedSet do} \label{code:validateLockedSet}	
            \State{\spZero}\ \ \textbf{if !obj.tryLock() then}
            \State{\spZero}\ \ \ \ \ \textbf{unlockAll()}
            \State{\spZero}\ \ \ \ \ \textbf{{goto} \ref{code:begin}} \label{code:validateGoto1}
			\State{\spZero}\textbf{if !validateReadSet() then} 		\label{code:validateReadSet}
				\State{\spZero}\ \ \textbf{unlockAll()}
				\State{\spZero}\ \ \textbf{{goto} \ref{code:begin}} \label{code:validateGoto2}
				%\Comment Restart Operation
			\Statex -----------------------
			\Comment{\textrm{update phase}}
			\State{\spZero}prev.next = new
			\State{\spZero}\textbf{new.lock()}
			\State{\spZero}new.next = succ			
			\State{\spZero}\textbf{prev.incVersion}
			\State{\spZero}\textbf{prev.unlock()}
			\State{\spZero}\textbf{new.incVersion}
			\State{\spZero}\textbf{new.unlock()}
			\State{\spZero}\textbf{succ.incVersion}
			\State{\spZero}\textbf{succ.unlock()}

			\EndFunction
			}
		\end{algorithmic}
		\caption{The code produced by our  transformation}\label{figure:transformation:after}
	\end{subfigure}
	%\bigskip
	%\hline
	\end{center}
\vspace{-4mm}
	\caption{Code transformation example.
	The synchronization code is in bold.
			\label{figure:transformation}}
\end{figure*}

\begin{figure}
\centering
\begin{minipage}{0.49\textwidth}
\centering
\codesize
\begin{algorithmic}[1]{}
		{\ttfamily
		\Function{track}{obj}
		\State{}lockedSet.add(obj) \label{code:lockedSet:add}
			\State long ver = obj.getVersion() \label{code:track:getVersion}
			\State readSet.add($\langle$obj,ver$\rangle$)
%\Statex	\hspace{-30mm} \Comment{\textrm{Eager validation}}
%			\State if {$\exists$<o,v>$\in$readSet such that o=obj and v!=ver} then
			% return false \label{code:track:verifyVersion}			
			\If{$\langle$obj,v$\rangle\in$readSet and v!=ver} \label{code:track:verifyVersion}
				\State return false
			\EndIf
			\If{obj.isLockedByAnother()} \label{code:track:verifyUnlocked}
				\State return false \label{code:track:verifyUnlockedB}
			\EndIf
			\State return true
		\EndFunction
		}
\end{algorithmic}
\caption{In read-only phase, locking is replaced by
tracking locks and read objects' versions.
\label{figure::track}}
\end{minipage}\hfill
\begin{minipage}{0.47\textwidth}
\centering
\codesize
\begin{algorithmic}[1]{}
		{\ttfamily
		\Function{validateReadSet}{}()
		\ForAll {$\langle$obj,ver$\rangle$ in readSet}
			\If{obj.isLockedByAnother()}
			\State return false \Comment{\textrm{validation failed}} \label{code:validate:verifyUnlocked}
			\State \Comment{\textrm{(locked object)}}
			\EndIf
			\If{obj.getVersion() != ver} 
				\State return false \Comment{\textrm{validation failed}} \label{code:validate:verifyVersion}
				\State \Comment{\textrm{(different version)}}
			\EndIf
		\EndFor
		\State retrun true \Comment{\textrm{validation succeed}}
		\EndFunction
		}
\end{algorithmic}
\caption{Read set validation: verify that  objects are unlocked and their versions are unchanged.\label{figure::validate}}
\end{minipage}
\end{figure}

We now describe how we synthesize the code for each of the phases. The regular three-phase flow is described in
Section~\ref{sssec:alg-normal}, and exceptions are described in Section~\ref{sssec:alg-abnormal}.

\subsubsection{Normal Flow}
\label{sssec:alg-normal}

We illustrate the transformation for a simple code snippet that adds a new element as the third node in a linked list. Each node is associated with a  lock.
The original and transformed code are provided in Figure \ref{figure:transformation}. The latter uses
the tracking and validation functions in Figures \ref{figure::track} and
\ref{figure::validate}, resp.
For clarity of exposition, we present a statically instrumented version, without tracking the phases using \textbf{opt}.

%\paragraph{Version Numbers}

Our transformation instruments each lock with an additional field \emph{version}. We assume each object supports \emph{getVersion} and \emph{incVersion} instruction to read and increment the version number of the lock associated with the object. We invoke \emph{incVersion} when holding the lock, and are therefore are not concerned about contention.
%Later we will show that
%If $o$ is not locked, then this field  represents the current version number of $o$.
Note that each lock has its own version, i.e., version numbers of different locks are independent of each other.

\paragraph{Read-only Phase}
% In this phase, we replace all the lock and unlock instructions with local tracking.
In this phase the executing thread is invisible to other thread, i.e., 
 avoids  contention on shared memory both in terms of writing and in terms of locking.
During this phase, our synchronization maintains two thread-local multi-sets: \emph{lockedSet} and \emph{readSet}.
The \emph{lockedSet} tracks the objects that were supposed to be locked by the original synchronization.
%
The \emph{readSet} tracks versions of all objects read by the
operation, in order to allow us to later validate that the operation has observed a consistent view of shared memory.

At the beginning of the read-only phase, we insert code that initializes \emph{lockedSet} and \emph{readSet} to be empty (see  line~\ref{code:initSets} of Figure~\ref{figure:transformation:after}).
Throughout the read-only phase, (i.e., when \textbf{opt} is true with dynamic phase transitions), 
we replace every lock and unlock instruction with the corresponding code in Table~\ref{Ta:readOnlyTransformation}.
A lock instruction on object $o$ is replaced with code that tracks the object and the version of its lock in
 \emph{lockedSet} and \emph{readSet} (see Figure~\ref{figure::track}).
An unlock instruction on object $o$ is replaced with code that removes $o$ from \emph{lockedSet} (see
lines \ref{code:beginRead}-\ref{code:endRead} of Figure~\ref{figure:transformation:after}).

\begin{table}
\codesize
\ttfamily
{\tt
\begin{center}
\begin{tabular}{|l|l|}
\hline
\textbf{Original code} & \textbf{Transformed Code}\\
\hline
\textit{x.lock()}&
\textit{if !track(x) then goto $S$}
\\
\hline
\textit{x.unlock()}&
\textit{lockedSet.remove(x)}
\\
\hline
\end{tabular}
\end{center}
}
\caption{Transformation for read-only phase:
each locking instruction (left column) is replaced with the corresponding code on the right;
 $S$  denotes the beginning of the operation.
}
\label{Ta:readOnlyTransformation}
\end{table}

In Figure~\ref{figure::track} (lines \ref{code:track:verifyVersion}-\ref{code:track:verifyUnlockedB}), we use an eager validation scheme\footnote{Eager validation is not required for correctness.}: 
If the object already exists in \emph{readSet}, we check that the current version of its lock is equal
to the version in \emph{readSet}; and if the versions are different  the operation restarts (line~\ref{code:track:verifyVersion}).
Similarly, it is  checked to be unlocked, and the operation restarts if it is locked (line~\ref{code:track:verifyUnlocked}).


% Optimization
Although it only accesses thread-local data structures, lock tracking induces a certain overhead due to the need to search a lock in the \emph{lockedSet} in order to unlock it. (In our experiments presented below, in large data structures, this overhead slows operations down by up to $40\%$).
First, we observe that 
the \emph{lockedSet} does not need to be tracked in read-only operations, which a compiler can easily detect. 
 We can further avoid this overhead in update operations in certain 
 cases by relying on the structure of the transformed code. For example, if the lock-based code is created from sequential code 
using domination locking~\cite{Gueta2011}, then at any given time in the read phase, it holds locks on a well-defined set of objects -- the ones currently pointed by the operation's local variables. When applying our transformation to code generated by this scheme, we can optimize it to remove lock-tracking, and instead populate the \emph{lockedSet} with the appropriate locks immediately before executing the validation phase.

\paragraph{Validation Phase}
The code of the validation phase is invoked between the read-only phase and the update phase (lines \ref{code:beginValidation}-\ref{code:endValidation} of Figure~\ref{figure:transformation:after}).
It locks the objects that are left in \emph{lockedSet} and validates the objects in \emph{readSet}.
To avoid deadlocks, the locks are acquired using a \emph{tryLock}
instruction.
If any \emph{tryLock} fails, the code unlocks  all
previously acquired locks and restarts from the beginning
(lines \ref{code:validateLockedSet}-\ref{code:validateGoto1}).

The function \emph{validateReadSet} in Figure~\ref{figure::validate} verifies that the objects in the read set have not been updated.
%
The function checks that each object in the read set is not locked by another thread,
and that the current version of the lock associated with the object matches the version saved in the
\emph{readSet}.
This check guarantees that the object was not locked from the time it was read until
the time it was validated.
Since operations write only to
locked objects, it follows that the object was not changed.
This \emph{readSet} validation can be viewed as a double collect~\cite{Afek:1993:ASS:153724.153741}
of all objects accessed by the read-only phase.
%
The operation is restarted if the validation fails (lines \ref{code:validateReadSet}-\ref{code:validateGoto2}).

% Fences
We assume that, following standard practice in lock implementations, 
the function \textit{isLockedByAnother} imposes a \emph{memory fence} (barrier). 
This ensures that the lock and version are read during \textit{track} before the object's value is read optimistically
during the read-only phase. 
To ensure that the second read of the lock and version, during the validation phase, succeeds the 
optimistic read of the object's value, we precede the validation phase with a memory fence as well (line~\ref{code:fence}).  
Note that it suffices to impose a \emph{read fence} (sometimes called acquire or load fence) 
prior to the validation as well as during \textit{isLockedByAnother}, because this part of the code does not include
writes to shared memory.

\paragraph{Update Phase}
In this phase our transformation preserves the original locking while maintaining the versions of the objects, i.e., the version of an object $o$ is incremented every time $o$ is unlocked.
Here, (i.e., in case  \textbf{opt} is false with dynamic phase transitions), 
before each unlock instruction \emph{\ttfamily x.unlock()} we insert the code \emph{\ttfamily x.incVersion()} .
An example is shown in lines \ref{code:beginUpdate}-\ref{code:endUpdate} of Figure~\ref{figure:transformation:after}.


\subsubsection{Exceptions from Regular Flow}
\label{sssec:alg-abnormal}


%\paragraph{Inconsistent Views}
%Other than when reading objects already in the read set, the read phase does not validate past reads during its executions ---
The read phase does not validate past reads during its executions (other than when re-reading the same variable). 
As a result, it may observe an inconsistent state of shared memory, which may lead to infinite loops or spurious exceptions (as explained, e.g., in~\cite{HLR:SLCA2010}).
%
We avoid such infinite loops using a timeout.
If the timeout expires before the read-only phase completes, read set
validation takes place (via the function \emph{validateReadSet}). If the validation fails, the operation is restarted.
This is realized by inserting code that examines the timeout in every loop iteration in the original code.
%
Similarly, we avoid spurious exceptions by catching all exceptions and
performing validation. Here too, if the validation fails, the operation is restarted. Otherwise, the exception is handled as in
the original code.


%\paragraph{Early Phase Branching}
Note that, using our transformation, the shared state at the end of the validation phase
is identical to the state that would have been reached had the code been executed pessimistically from
the outset. Hence, the three-phase version of the code is compatible with the instrumented
pessimistic version. This means that if the optimistic phase is unsuccessful for any reason, we can always
fall back on the pessimistic version. Moreover, we can switch from optimistic to pessimistic synchronization
\emph{at any point} during the read phase.
We use this property in two ways.
%, as we now describe.
First, we avoid livelocks by limiting the number of restarts due to conflicts:
The validation phase tracks the number of restarts in a thread-local variable.
If this number exceeds a certain threshold, we perform the entire operation optimistically.

Second, this property offers the optimistic implementation the liberty of
failing spuriously, even in the absence of conflicts, because it can always fall back on the safe pessimistic version
of the code.
One can take advantage of this liberty, and implement the \emph{readSet} using a constant size array.
%Our implementation takes advantage of this liberty, and uses constant size arrays for the \emph{lockedSet} and \emph{readSet}.
In case the array becomes full,  the optimistic version cannot proceed, but there is no 
need to start the operation anew.
Instead, one can immediately perform the validation phase, which, if successful, switches to a pessimistic modus operandi, after having acquired all the needed locks.
