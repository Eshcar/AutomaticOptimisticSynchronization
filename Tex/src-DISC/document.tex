\documentclass[11pt]{article}
\usepackage{url}                  % format URLs
\usepackage[colorlinks=true,allcolors=blue,breaklinks,draft=false]{hyperref}   % hyperlinks, including DOIs and URLs in bibliography
\usepackage{amsmath}
\usepackage{xspace}
\usepackage{algorithm}
\usepackage[noend]{algpseudocode}
\usepackage{subcaption}
\usepackage{xcolor}
\usepackage{tikz}
\usepackage{pgfplots}


\definecolor{blues1}{RGB}{198, 219, 239}
\definecolor{blues2}{RGB}{158, 202, 225}
\definecolor{blues3}{RGB}{107, 174, 214}
\definecolor{blues4}{RGB}{49, 130, 189}
\definecolor{blues5}{RGB}{8, 81, 156}
\definecolor{antiquefuchsia}{rgb}{0.57, 0.36, 0.51}
\definecolor{asparagus}{rgb}{0.53, 0.66, 0.42}
\definecolor{darkspringgreen}{rgb}{0.09, 0.45, 0.27}
\definecolor{darkslategray}{rgb}{0.18, 0.31, 0.31}
\definecolor{coralred}{rgb}{1.0, 0.25, 0.25}

\pgfplotsset{mystyle/.style={%
        width=0.29\paperwidth,
        %ylabel={mystyle (kg)},
        %xlabel={Eggs (no.)},
        xmin=1,xmax=32,
        enlargelimits=true,
        %xmajorgrids=false,
		ymajorgrids=true,
        grid=major,
        grid style={dashed, gray!30},
        %symbolic x coords={1,2,4,8,16,32},
        ylabel style={align=center},
        xtick={1,4,8,16,32}, 
        scaled y ticks=base 10:-6,
        ytick scale label code/.code={},
    	yticklabel={\pgfmathprintnumber{\tick}}
        %tick align=outside,    
}}

\pgfplotsset{unbalanced/.style={%
	   %width=0.35\textwidth,
       legend columns=-1,
	   legend entries={\autoTree, \danaTree ,\lockfreeTree, \stmTree, \domTree, \optAutoTree, },
	   legend to name=unbalancedLegened,
}}


\pgfplotsset{balanced/.style={%
	   %width=0.35\textwidth,
       legend columns=4,
	   legend entries={\autoTreap, \danaAVL,\bronson, \friendly,  \skiplist, \stmTreap,
	   \domTreap,\optAutoTreap}, legend to name=balancedLegened,
}}

\pgfplotsset{skiplist/.style={%
	   width=\textwidth,
       ymin=0,ymax=2800000,
       xtick={2,4,8,16,32}, 
       legend columns=-1,
	   legend entries={\autoSkiplist, \kary , \skiplist (not atomic), \stmSkiplist,
	   \domSkiplist}, legend to name=skiplistLegened,
}}

\pgfplotsset{skiplist1000/.style={%
	   width=\textwidth,
       ymin=0,ymax=150000,
       scaled y ticks=base 10:-3,
       xtick={2,4,8,16,32}, 
       legend columns=-1,
	   legend entries={\autoSkiplist, \kary, \skiplist (not atomic), \stmSkiplist,
	   \domSkiplist}, legend to name=skiplistLegened1000,
}}

\pgfplotsset{skiplistUpdate/.style={%
	   width=0.35\textwidth,
       ymin=0,ymax=5500000,
       xtick={1,2,4,8,16,32}, 
       legend columns=-1,
	   legend entries={\autoSkiplist, \kary , \skiplist (not atomic), \stmSkiplist,
	   \domSkiplist}, legend to name=skiplistLegenedUpdate,
}}

\newcommand{\Xomit}[1]{}

%-------------Theorem Definitions ---------------%
\newtheorem{theorem}{Theorem}[section]
\newtheorem{lemma}[theorem]{Lemma}
\newtheorem{claim}[theorem]{Claim}
\newtheorem{observation}[theorem]{Observation}
\newtheorem{proposition}[theorem]{Proposition}
\newtheorem{corollary}[theorem]{Corollary}

\newenvironment{proof}[1][Proof]{\begin{trivlist}
\item[\hskip \labelsep {\bfseries #1}]}{\qedsymb\end{trivlist}}
\newenvironment{definition}[1][Definition]{\begin{trivlist}
\item[\hskip \labelsep {\bfseries #1}]}{\end{trivlist}}
\newenvironment{example}[1][Example]{\begin{trivlist}
\item[\hskip \labelsep {\bfseries #1}]}{\end{trivlist}}
\newenvironment{remark}[1][Remark]{\begin{trivlist}
\item[\hskip \labelsep {\bfseries #1}]}{\end{trivlist}}


\newcommand{\qed}{\nobreak \ifvmode \relax \else
      \ifdim\lastskip<1.5em \hskip-\lastskip
      \hskip1.5em plus0em minus0.5em \fi \nobreak
      \vrule height0.75em width0.5em depth0.25em\fi}
\newcommand{\qedsymb}{\hfill{\rule{2mm}{2mm}}}


%--------------------------------------------------%

\newcommand{\code}[1]{\textsf{#1}}
\newcommand{\readV}{\code{read\_version}\xspace}
\newcommand{\readSet}{\code{read\_set}\xspace}
\newcommand{\writeV}{\code{write\_version}\xspace}

\newcommand{\reqI}{\textbf{LPR1}\xspace}
\newcommand{\reqII}{\textbf{LPR2}\xspace}

%---------Evaluation Macros-----------------------%
\newcommand{\autoTree}{LR-Tree\xspace}
\newcommand{\autoTreap}{LR-Treap\xspace}
\newcommand{\autoSkiplist}{LR-Skiplist\xspace}
\newcommand{\danaTree}{LO-Tree\xspace}
\newcommand{\danaAVL}{LO-AVL\xspace}
\newcommand{\bronson}{Snap-Tree\xspace}
\newcommand{\friendly}{CF-Tree\xspace}
\newcommand{\skiplist}{Java-Skiplist\xspace}
\newcommand{\kary}{k-Tree\xspace}
\newcommand{\lockfreeTree}{LF-Tree\xspace}
\newcommand{\globalTree}{Global-Tree\xspace}
\newcommand{\globalTreap}{Global-Treap\xspace}
\newcommand{\domTree}{Lock-Tree\xspace}
\newcommand{\domTreap}{Lock-Treap\xspace}
\newcommand{\stmTree}{STM-Tree\xspace}
\newcommand{\stmTreap}{STM-Treap\xspace}
\newcommand{\stmSkiplist}{STM-Skiplist\xspace}
\newcommand{\domSkiplist}{Lock-Skiplist\xspace}
\newcommand{\getOP}{\textsc{get}\xspace}


%---------Comments -----------------------%
\newcommand{\Idit}[1]{{\color{red}{[\textbf{Idit:} #1 ]}}}
\newcommand{\guy}[1]{{\color{red}{[\textbf{guy:} #1 ]}}}
\newcommand{\eshcar}[1]{{\textcolor{violet}{\{{\bf eshcar:} \em #1\}}}}
\newcommand{\maya}[1]{{\textcolor{magenta}{\{{\bf maya:} \em #1\}}}}


\usepackage{fullpage}

\begin{document}

\begin{titlepage}

\title{Towards Automatic Lock Removal\\ for Scalable Synchronization}
\author{
Maya Arbel\footnotemark[1] \footnotemark[2] \hspace{0.3in} 
Guy Golan Gueta\footnotemark[1]  \hspace{0.3in}
Eshcar Hillel\footnotemark[1]  \hspace{0.3in}
Idit Keidar\footnotemark[1] \footnotemark[2]\\
\\
\large{
\footnotemark[1]Yahoo Labs, Haifa, Israel \hspace{0.3in} 
\footnotemark[2]The Technion, Haifa, Israel
}
}
\date{May 15, 2015}

\maketitle

\begin{abstract}
We present a \emph{code transformation} for concurrent data structures,
which increases their scalability without sacrificing correctness.
Our transformation takes lock-based code, and replaces some of the
locking steps therein with optimistic synchronization, in order to reduce contention. The main idea is to
have each operation perform an optimistic traversal of the data structure
as long as no shared memory locations are updated, and then proceed with
pessimistic code. The transformed code inherits essential
properties of the original one, including linearizability, serializability,
and deadlock freedom.
%It reduces contention, both on locks and on access to shared memory locations.
%When applying our transformation to
%hand-over-hand locking solutions, we obtain significantly superior scalability.

Our work complements existing pessimistic transformations that make
sequential code thread-safe by adding locks.
In essence, we provide a way to optimize such transformations by reducing
synchronization bottlenecks (for example, locking the root of a tree).
The resulting code scales well and significantly outperforms
pessimistic approaches. We further compare our synthesized code to state-of-the-art
data structures implemented by experts.
We find that its performance is comparable %, and sometimes even superior,
to that achieved by the custom-tailored implementations.
Our work thus shows the promise that automated approaches
bear for overcoming the difficulty involved in manually
hand-crafting concurrent data structures.

\end{abstract}

\bigskip\bigskip
\noindent
{\bf
This is a submission to the regular track.\\
It can be considered for the best student paper award:
Maya Arbel is a full-time student.}

\thispagestyle{empty}
\end{titlepage}	




\section{Introduction}

The steady increase in the number of  cores in today's computers is driving software developers to allow more parallelism. 
Indeed, many recent works have developed scalable concurrent data structures~\cite{bronson,dana,citrus,etc}. 
Such efforts are often very successful, achieving performance that scales linearly with the number of threads. 
Nevertheless, each of these project generally focuses on a single data structure 
(for example, a binary search tree~\cite{citrus} or \Idit{give another example}) and manually optimizes its implementation. 
Proving the correctness of such custom-tailored data structures is painstaking 
(for example, the proofs of \cite{x,y,z} are XX,YY, and ZZ pages long \Idit{add info}, respectively). 
We propose an approach to replace this labor-intensive process by automatic means.

One way to automatically convert a sequential data structure into a correct (thread-safe) concurrent one using locks. 
The trivial way to do so is to add a single global lock protecting the entire data structure 
(as in \emph{synchronized methods} in Java$^{TM}$, for example), but this allows no parallelism whatsoever. 
A more sophisticated approach can instrument the code (at compile time) and add fine-grained lock and unlock instructions~\cite{domination,tree-locking,dag,etc}. Such methods are applicable to certain data structure families, for example, Domination Locking~\cite{domination} is applicable to all trees or forests  (including binary search trees, BTrees, Treaps, etc \Idit{more?}), and employs a variant of hand-over-hand locking~\cite{hand-over-hand}, acquiring and releasing locks as it goes down the tree.  Other approaches are applicable to DAGS~\cite{dag-locking}. Unfortunately, to date, solutions of this sort scale poorly. This is due to synchronization bottlenecks such as the root of the tree, which is locked by all operations.

In this paper, we circumvent such synchronization bottlenecks via judicious use of optimism. 
Specifically, we replace many (but not all) locks with speculative execution and later re-validation. 
If re-validation fails, the speculative phase is restarted. 
In striking the balance between optimism and pessimism, we exploit the common nature of data structure operations, 
which typically begin by traversing the data structure to a designated location, and then perform (mostly local) updates at that location. 
Our optimistic execution is limited to the initial read-only part of the code (the data structure traversal)\footnote{Our solution may be seen as a form of software lock elision for read-only operation prefixes.}. 
Unlike most software transactional memory approaches~\cite{stm,tls},  
our synthesized code neither speculatively modifies shared memory contents, nor does it defer writes. 
Hence it never needs to rollback, and saves the overhead for tracking writes in dedicated data structures. 
Further comparison with related work appears in Section~\ref{sec:related}.

Our approach works as follows: Given a sequential data structure implementation, 
it first invokes a given (black-box) mechanism that instruments the code and adds  
fine-grained locks, e.g.,~\cite{domination,tree-locking,dag}. We assume that the locking protocol 
allows early release in the sense that it no longer holds locks on parts of the data structure that it has finished traversing; 
our assumptions are detailed in Section~\ref{sec:model}. We then invoke our algorithm, detailed in Section~\ref{sec:algorithm}, which 
(1) adds version numbers to shared memory objects, 
(2) identifies the read-only prefix of the code, 
(3) replaces locks in the read-only prefix with tracking of the read objects' versions, and 
(4) introduces appropriate re-validation mechanisms.  
Re-validation occurs at the end of the read-only phase, as well as during timeouts and exceptions. 
The latter addresses exceptions and infinite loops that may arise when an operation sees an inconsistent view of the data structure. 
Our code transformation is general, in the sense that it applies to any data structure for which an appropriate locking protocol exists, 
as proven in Section ~\ref{sec:proof}. The transformation is trivial to implement at compile time.

We realize our approach with the Domination Locking scheme~\cite{domination}, 
which is applicable to tree and forest data structures. 
We apply the appropriate code transformations to balanced and unbalanced tree data structure implementations in Java$^{TM}$. 
In Section~\ref{sec:eval} we evaluate the resulting code on a 32-core machine, 
and compare it to fully pessimistic as well as state-of-the-art hand-crafted data structure implementations~\cite{dana,bronson}. 
Our results show that the optimistic approach successfully overcomes synchronization bottlenecks, 
allows the synthesized code to scale linearly with the number of threads, 
and achieve comparable performance to that of custom-tailored solutions.

To conclude, this paper illustrates that automatic synchronization is a promising approach for bringing legacy code to emerging computer architectures. 
While this paper illustrates the method for tree data structures, we believe that the general direction may be more broadly applicable.
Section~\ref{sec:discussion} concludes the paper and touches on some directions for future work.

%\section{Model and Definitions}\label{sec:model}



\paragraph{Shared Memory Data Structures}

A \emph{data structure} defines a set of \emph{operations} that may be invoked by
clients of the data structure, potentially concurrently.
%
Operations have parameters and local variables, which are private to the invocation of the operations.
(Thus, these are thread-local variables.)
%
We assume that the code of each operation is represented by a separate control-flow graph (CFG).

The operations of a data structure interact via a shared memory which is composed of a set of \emph{shared objects}.
%
Each shared object supports atomic \emph{read} (load) and \emph{write} (store) instructions.
%
Below, we extend shared objects to also support locks.

Every thread executes a sequence of operations, each of which is invoked with certain parameters and returns a response.
An operation's execution consists of a sequence of primitive \emph{steps}, beginning with an \emph{invoke} step, followed by
atomic accesses to shared objects, and ending with a \emph{return} step. Steps also modify the executing thread's local variables.

A \emph{configuration} is an assignment of values to all shared objects and local variables. Thus, each step takes the system from one
configuration to another. Steps are deterministically defined by the data structure's code and the current configuration.
We assume that each data structure has a single \emph{initial configuration}.

An \emph{execution} is an alternating sequence of configurations and steps,
$C_0,s_1,C_1, \ldots,s_i,C_i,\ldots,$
where $C_0$ is the initial configuration,
and each configuration $C_i$ is the result of
executing step $s_i$ on configuration $C_{i-1}$.
We only consider finite executions in this paper.
An execution is \emph{sequential} if steps of different operations are not interleaved.
In other words, a sequential execution is a sequence of operation executions.

\paragraph{Locks}
Each shared object serves as a lock for itself.
It supports atomic \emph{lock}, \emph{tryLock}, \emph{unlock} and \emph{isLockedByAnotherThread} instructions.
Locks are exclusive (i.e., a lock can be held by at most one thread at a time).
The execution of a thread
trying to acquire a lock (by a \emph{lock} instruction) which is
held by another thread is blocked until a time when the
lock is available (i.e., is not held by any thread).
The other instructions never block the execution.
The \emph{tryLock} instruction returns \emph{false} if the lock is currently held by another thread, otherwise it acquires the lock and returns \emph{true}.
The \emph{isLockedByAnotherThread} instruction returns \emph{true}, if and only if, the lock is currently held by another thread.

We assume that in the given code every (read
or write) access by an operation to a shared object is performed when the
executing thread holds the lock on that object.
We also assume that the locking in the given code only uses \emph{lock} and \emph{unlock} instructions 
(i.e., the other locking instructions are not used).



\paragraph{Correctness}

The correctness of a data structure is defined in terms of its external behavior, as reflected in values returned by invoked operations.
Correctness of a code transformation is proven by showing that the synthesized code's executions are equivalent to ones of the original code,
where two executions are  \emph{equivalent} if every thread invokes the same
operations in the same order  in both executions, and gets the same result for each operation. More formally, we say in this paper that a code transformation is \emph{correct} if every execution of the transformed code
is equivalent to some execution of the original code.

The widely-used correctness criterion of serializability relies on equivalence to sequential executions in order to
link a data structure's behavior under concurrency to its sequentially specified behavior. Since equivalence is transitive,
we get that any code transformation satisfying our correctness notion, when applied to serializable code, yields code that is also serializable.
If the code transformation further ensures the real-time order of operations (i.e., operations that do not overlap appear in the same order in
executions of the transformed and original code), then linearizability (atomicity) is also invariant under the transformation.
Another important aspect of correctness is preserving the progress conditions of the original code, for example, deadlock-freedom.

In this paper, we are not concerned with internal consistency (as required e.g., by opacity~\cite{GuerraouiK2008} or the validity notion of~\cite{LevAriCK2014}),
which restricts the configurations an operation might see during its execution.
This is because our code transformation uses timeouts and exception handlers to overcome unexpected behavior that may arise when a thread sees an inconsistent view of global variables (similar to~\cite{Nakaike:2010}).




\renewcommand{\ttdefault}{pcr}
%\algrenewcommand\textkeyword{\texttt}
\algrenewcommand\algorithmicif{\texttt{if}}
\algrenewcommand\algorithmicthen{\texttt{then}}
\algrenewcommand\algorithmicfunction{\textsc{Function}}
\algrenewcommand\algorithmicforall{\texttt{for all}}
\algrenewcommand\algorithmicdo{\texttt{do}}
\algrenewcommand\textproc{\textit}



\section{Automatic Transformation}\label{sec:algorithm}

We present an automatic source-to-source transformation, whose 
goal is to optimize the code of a given data structure implemented using lock-based concurrency control.
In Section~\ref{ssec:locks}, we detail our assumptions about the given code and the locks it uses. 

The transformation produces a combination of pessimistic and optimistic synchronization
by replacing part of the locking (in the given code) with optimistic concurrency control.
Section~\ref{ssec:overview} overviews our general approach to combining optimism and pessimism. 

The synthesized code consists of three phases - an optimistic read-only phase, a validation phase,
and a pessimistic update phase. The transformation first partitions the given code into a read-only phase and 
an update phase, then instruments each of these phases separately and adds the validation phase between the two.
Section~\ref{ssec:transformation} describes how the code of each of the three phases is produced, whereas 
Section~\ref{ssec:extendedTran} explains how the code is partitioned into two phases.

\subsection{Lock-based Data Structures}\label{ssec:locks}

A \emph{data structure} defines a set of \emph{operations} that may be invoked by
clients of the data structure, potentially concurrently.
We assume that the code of each operation is represented by a separate \emph{control-flow graph (CFG)}~\cite{xxx}.

Operations have parameters and local variables, which are private to the invocation, and are thus thread-local.
%
The operations interact via \emph{shared memory variables}, which are also called \emph{shared objects}.
%
Each shared object supports atomic \emph{read} (load) and \emph{write} (store) instructions.
%
%Below, we extend shared objects to also support locks.

In addition, each shared object is associated with a unique lock, and supports 
atomic \emph{lock}, \emph{tryLock}, \emph{unlock} and \emph{isLockedByAnotherThread} instructions.
Locks are exclusive (i.e., a lock can be held by at most one thread at a time).
The execution of a thread
trying to acquire a lock (by a \emph{lock} instruction) which is
held by another thread is blocked until a time when the
lock is available (i.e., is not held by any thread).
The other instructions never block.
The \emph{tryLock} instruction returns \emph{false} if the lock is currently held by another thread, otherwise it acquires the lock and returns \emph{true}.
The \emph{isLockedByAnotherThread} instruction returns \emph{true}, if and only if, the lock is currently held by another thread.

We assume that in the given code every (read
or write) access by an operation to a shared object is performed when the
executing thread holds the lock on that object.
We also assume that the locking in the given code only uses the \emph{lock} and \emph{unlock} instructions only.
%(i.e., the other locking instructions are not used).

\subsection{Combining Optimism and Pessimism}\label{ssec:overview}

% Optimism
Generally speaking, optimistic concurrency control is a form of lock-free synchronization, which accesses shared variables without locks in the hope that they will not be modified by others before the end of the operation (or more generally, the transaction). To verify the latter, optimistic concurrency control relies on \emph{validation}, which is typically implemented using version numbers. If validation fails, the operation restarts. Optimistic execution of update operations requires either performing roll-back (reverting variables to their old values) upon validation failure, or deferring writes to commit time; both approaches induce significant overhead~\cite{Cascaval:2008}. We therefore refrain from speculative shared memory updates.



%The main idea:
The main idea behind our approach is judicious use of optimistic synchronization for reading
shared variables without locks, but only as long as the operation does not update shared state. Once an operation
writes to shared memory, we revert to pessimistic (lock-based) synchronization. In
other words, we rely on validation at the end of the read-only prefix of an operation in order to render redundant
locks that would have been acquired and freed before the first update.
This scheme is particularly suitable for data structures,
since the common behavior of their operations
is to first traverse the data structure, and then
perform modifications.

Conceptually, our approach divides an operation into three phases: an optimistic \emph{read-only phase},
a pessimistic \emph{update phase} and a \emph{validation phase} that conjoins them.
The read-only phase traverses the data structure without taking any locks, while maintaining sufficient information to later ensure the correctness of the traversal.
The update phase uses the original pessimistic (lock-based) synchronization.
The validation phase bridges between the optimistic and pessimistic ones.
It first locks the objects for which a lock would have been held at this point by
the original locking code, and then validates the correctness
of the read-only phase. This allows the
update phase to run as if an execution of the original pessimistic synchronization
took place. If the validation fails, the operation
restarts. In order to avoid livelock, we set a threshold on the number of restarts.
If the threshold is exceeded, the code falls back on pessimistic execution.
Note that it is safe to do so, since our semi-optimistic code is compatible
with the fully pessimistic one.


\subsection{Transforming the Code Phases}\label{ssec:transformation}

\newcommand{\spOne}{\hspace{-3mm}\ }
\newcommand{\spZero}{\hspace{-3mm}}
\begin{figure*}
\scriptsize
	\begin{center}
	\begin{subfigure}[t]{.45\textwidth}
		\begin{algorithmic}[1]{}
		{\ttfamily
			\Function{addThird}{List list, Node new} \label{code:begin}
			\Statex ----------------------------
			\State                               \label{code:beginRead}
            \State{\spOne}\textbf{list.lock()}
			\State{\spOne}Node prev = list.head
			\State{\spOne}\textbf{prev.lock()}
            \State{\spOne}\textbf{list.unlock()}
			\State{\spOne}Node succ = prev.next
			\State{\spOne}\textbf{succ.lock()}
			\State{\spOne}\textbf{prev.unlock()}
			\State{\spZero}prev = succ
			\State{\spZero}succ = succ.next
			\State{\spZero}\textbf{succ.lock()}  \label{code:endRead}
			\Statex ----------------------------
			\State                               \label{code:beginValidation}
			\State
			\State
			\State
			\State
			\State
            \State                               \label{code:endValidation}
			\Statex ----------------------------
			\State{\spZero}prev.next = new       \label{code:beginUpdate}
			\State{\spZero}\textbf{new.lock()}
			\State{\spZero}new.next = succ
            \State
			\State{\spZero}\textbf{prev.unlock()}
            \State
			\State{\spZero}\textbf{new.unlock()}
            \State
			\State{\spZero}\textbf{succ.unlock()}  \label{code:endUpdate}
			\EndFunction
			}
		\end{algorithmic}
		\caption{Code with original locking} \label{figure:transformation:before}
	\end{subfigure}
	\begin{subfigure}[t]{.45\textwidth}
		\begin{algorithmic}[1]{}
		{\ttfamily
			\Function{addThird}{List list, Node new} \label{code:begin}
			\Statex ----------------------------
			\Comment{\textrm{read-only phase}}
            \State{\spOne}\textbf{lockedSet.init(), readSet.init()} \label{code:initSets}
            \State{\spOne}\textbf{if !track(list)  then {goto} \ref{code:begin}} \label{code:readGhaseGoto0}
			\State{\spOne}Node prev = list.head
			\State{\spOne}\textbf{if !track(prev)  then {goto} \ref{code:begin}} \label{code:readGhaseGoto1}
            \State{\spOne}\textbf{lockedSet.remove(list)} \label{code:lockedSet:remove1}
			\State{\spOne}Node succ = prev.next
			\State{\spOne}\textbf{if !track(succ) then {goto} \ref{code:begin}}  \label{code:readGhaseGoto2}
			\State{\spOne}\textbf{lockedSet.remove(prev)} \label{code:lockedSet:remove2}
			\State{\spOne}prev = succ
			\State{\spOne}succ = succ.next
			\State{\spZero}\textbf{if !track(succ) then {goto} \ref{code:begin}} \label{code:readGhaseGoto3}
			\Statex ----------------------------
			\Comment{\textrm{validation phase}}
			\State{\spZero}\textbf{for all obj in lockedSet do} \label{code:validateLockedSet}	
            \State{\spZero}\ \ \textbf{if !obj.tryLock() then}
            \State{\spZero}\ \ \ \ \ \textbf{unlockAll()}
            \State{\spZero}\ \ \ \ \ \textbf{{goto} \ref{code:begin}} \label{code:validateGoto1}
			\State{\spZero}\textbf{if !validateReadSet() then} 		\label{code:validateReadSet}
				\State{\spZero}\ \ \textbf{unlockAll()}
				\State{\spZero}\ \ \textbf{{goto} \ref{code:begin}} \label{code:validateGoto2}
				%\Comment Restart Operation
			\Statex ----------------------------
			\Comment{\textrm{update phase}}
			\State{\spZero}prev.next = new
			\State{\spZero}\textbf{new.lock()}
			\State{\spZero}new.next = succ			
			\State{\spZero}\textbf{prev.version++}
			\State{\spZero}\textbf{prev.unlock()}
			\State{\spZero}\textbf{new.version++}
			\State{\spZero}\textbf{new.unlock()}
			\State{\spZero}\textbf{succ.version++}
			\State{\spZero}\textbf{succ.unlock()}

			\EndFunction
			}
		\end{algorithmic}
		\caption{The code produced by our automatic transformation}\label{figure:transformation:after}
	\end{subfigure}
	%\bigskip
	%\hline
	\end{center}
\vspace{-4mm}
	\caption{Code example.
	The synchronization code is in bold.
			\label{figure:transformation}}
\end{figure*}

In this section we assume that the code had already been partitioned into a read-phase and an update phase using 
the transformation described in Section~\ref{ssec:extendedTran}, and 
describe how we synthesize the code for each of the phases.
We illustrate the transformation for a simple code snippet that adds a new element as the third node in a linked list.
The original and transformed code are provided in Figure \ref{figure:transformation}. The latter uses
the tracking and validation functions in Figures \ref{figure::track} and
\ref{figure::validate}, resp.

\paragraph{Code Phases in the CFG}
In this section we assume that 
control-flow-graph (CFG) of each operation has been partitioned (by the transformation in the next section)
into two subgraphs corresponding to two phases -- 
read-only phase $C_r$, and update phase $C_u$. The partitioning satisfies the following:
(1)~every execution of the operation starts in $C_r$ and ends in $C_u$; (2)~there is no edge from $C_u$ to  $C_r$;
and (3) $C_r$ does not contain instructions that write to shared objects.
%
In the example of Figure~\ref{figure:transformation:before}, $C_r$ is the code in lines \ref{code:beginRead}-\ref{code:endRead},
and $C_u$ is the code in lines \ref{code:beginUpdate}-\ref{code:endUpdate}.

\paragraph{Version Numbers}
Our transformation instruments each object $o$ with an additional field \emph{version}.
%Later we will show that 
If $o$ is not locked, then this field  represents the current version number of $o$.
Version numbers are used to validate the correctness of the optimistic execution of the read-only phase.
Note that each object has its own version --- i.e., version numbers of different objects are independent of each other.

\paragraph{Read-only Phase}
In this phase, we replace all the lock and unlock instructions with synchronization that avoids writing to shared memory.
During this phase, our synchronization maintains two thread-local sets: \emph{lockedSet} and \emph{readSet}.
The \emph{lockedSet} is used to track the objects that were supposed to be locked by the original synchronization.
The \emph{readSet} is used to track versions of all objects read by the
operation, in order to allow us to later validate that the operation has observed a consistent view of shared memory.

At the beginning of the read-only phase, we insert code that initializes \emph{lockedSet} and \emph{readSet} to be empty (see  line~\ref{code:initSets} of Figure~\ref{figure:transformation:after}).
We replace every lock and unlock instruction with the corresponding code in Table~\ref{Ta:readOnlyTransformation}.
A lock instruction on object $o$ is replaced with code that tracks the object and its version in 
thread-local variables \emph{lockedSet} and \emph{readSet} (see Figure~\ref{figure::track}).
An unlock instruction on object $o$ is replaced with code that removes $o$ from \emph{lockedSet}.
An example for a transformed code is shown in lines \ref{code:beginRead}-\ref{code:endRead} of Figure~\ref{figure:transformation:after}.

\begin{table}
\scriptsize
\ttfamily
{\tt
\begin{center}
\begin{tabular}{|l|l|}
\hline
\textbf{Original Locking Instruction} & \textbf{Transformed Code}\\
\hline
\textit{x.lock()}&
\textit{if !track(x) then goto $S$}
\\
\hline
\textit{x.unlock()}&
\textit{lockedSet.remove(x)}
\\
\hline
\end{tabular}
\end{center}
}
\caption{Transformation for read-only phase:
each locking instruction (left side) is replaced with the corresponding code on the right;
 $S$  denotes the beginning of the operation.
}
\label{Ta:readOnlyTransformation}
\end{table}

\paragraph{Inconsistent Views}
Other than when reading objects already in the read set, the read phase does not validate past reads during its executions ---
as a result, it may observe an inconsistent state of shared memory.
%
The function \emph{validateReadSet} in Figure~\ref{figure::validate} can be used to validate past reads: it returns \emph{true} if and only if the objects in the read set have not been updated.
%
The function checks that each object in the read set is not locked by another thread,
and that the object's current version matches the version saved in the
read set.
This check guarantees that the object was not locked from the time it was read until
the time it was validated.
Since operations write only to
locked nodes, it follows that the object was not changed.
This read set validation can be viewed as a double collect~\cite{Afek:1993:ASS:153724.153741}
of all objects accessed by the read-only phase.

In order to avoid infinite loops (in the read-only phase) that might occur due to inconsistent reads~\cite{xxx}, a timeout is set.
If the timeout expires before the read-only phase is completed, read set
validation takes place (by invoking the function \emph{validateReadSet}). If the validation fails, the operation is restarted.
This is realized by inserting code that examines the timeout in each backward edge in the CFG of the read-only phase.

Similarly, inconsistent views may lead to spurious exceptions in the read-only phase. These are manifested as branches that 
may lead outside of $C_r$ (end even outside of the operation's code altogether). We instrument such edges in the CFG, and 
perform validation. Here too, if the validation fails, the operation is restarted. Otherwise, the exception is handled as in 
the original code.


\paragraph{Validation Phase}
The code of the validation phase is inserted in each edge from $C_r$ to $C_u$ (i.e., this code is invoked between the read-only phase and the update phase).
This code is shown in lines \ref{code:beginValidation}-\ref{code:endValidation} of Figure~\ref{figure:transformation:after}.
It locks the objects in \emph{lockedSet} and validates the objects in \emph{readSet}.
To avoid deadlocks, the locks are acquired using a tryLock
instruction.
If a tryLock fails, the code unlocks  all
previously acquired locks and restarts from the beginning
(lines \ref{code:validateLockedSet}-\ref{code:validateGoto1}).
Similarity, the operation is restarted if the validation fails (lines \ref{code:validateReadSet}-\ref{code:validateGoto2}).

\paragraph{Update Phase}
In this phase our transformation preserves the original locking while maintaining the versions of the objects, i.e., the version of an object $o$ is incremented every time $o$ is unlocked.
Here, before each unlock instruction \emph{\ttfamily x.unlock()} we insert the code \emph{\ttfamily x.version++} .
An example is shown in lines \ref{code:beginUpdate}-\ref{code:endUpdate} of Figure~\ref{figure:transformation:after}.


\begin{figure}
\scriptsize
\begin{algorithmic}[1]{}
		{\ttfamily
		\Function{track}{obj}
		\State lockedSet.add(obj) \label{code:lockedSet:add}
			\State long ver = obj.version \label{code:track:getVersion}
			\State readSet.add(obj,ver)
			\State if {obj.isLocked()} then return false \label{code:track:returnFalse}
			\State retrun true
		\EndFunction
		}
\end{algorithmic}
\caption{ In read-only phase, locking is replaced by
tracking locks and read
objects' versions.
\label{figure::track}}
\end{figure}




\begin{figure}
\scriptsize
\begin{algorithmic}[1]{}
		{\ttfamily
		\Function{validateReadSet}{}()
		\ForAll {obj in readSet}
			\If{obj.isLockedByAnotherThread()}
			\State return false \Comment{\textrm{validation failed (due to a locked object)}}
			\EndIf
			\State long ver = readSet.getVersion(obj)
			\If{obj.version != ver}
				\State return false \Comment{\textrm{validation failed (due to a different version)}}
			\EndIf
		\EndFor
		\State retrun true \Comment{\textrm{validation succeed}}
		\EndFunction
		}
\end{algorithmic}
\caption{Read set validation.\label{figure::validate}}
\end{figure}

\subsection{Phase Partitioning}\label{ssec:extendedTran}



\begin{figure*}
\scriptsize
	\begin{center}
	\begin{subfigure}[t]{.3\textwidth}
		\begin{algorithmic}[0]{}
		{\ttfamily
			\Function{foo}{X x, int i} \label{codeXXX:aaaaa}
            \Statex --------------------
            \State\hspace{-3mm}{1 :\ x.lock()}
            \State\hspace{-3mm}{2 :\ if i>7  then}
            \State\hspace{-3mm}{3 :\ \ \ x.f2 = i}
            \State\hspace{-3mm}{4 :\ temp = x.f1 + x.f2}
            \State\hspace{-3mm}{5 :\ x.unlock()}
            \State\hspace{-3mm}{6 :\ return temp}
            \Statex
            \State
            \State
            \State
            \State
            \State
            \State
			\EndFunction
			}
		\end{algorithmic}
		\caption{Original code.} \label{figure:autoPartitioning:step1}
	\end{subfigure}
	\begin{subfigure}[t]{.3\textwidth}
		\begin{algorithmic}[0]{}
		{\ttfamily
			\Function{foo}{X x, int i} \label{codeXXX:aaaaa}
            \Statex -------------------- \Comment{\textrm{original code $C$}}
            \State\hspace{-3mm}{1 :\ x.lock()}
            \State\hspace{-3mm}{2 :\ if i>7  then}
            \State\hspace{-3mm}{3 :\ \ \ x.f2 = i}
            \State\hspace{-3mm}{4 :\ temp = x.f1 + x.f2}
            \State\hspace{-3mm}{5 :\ x.unlock()}
            \State\hspace{-3mm}{6 :\ return temp}
            \Statex -------------------- \Comment{\textrm{clone $C_u$}}
            \State\hspace{-3mm}{1':\ x.lock()}
            \State\hspace{-3mm}{2':\ if x.f1>i  then}
            \State\hspace{-3mm}{3':\ \ \ x.f2 = i}
            \State\hspace{-3mm}{4':\ temp = x.f1 + x.f2}
            \State\hspace{-3mm}{5':\ x.unlock()}
            \State\hspace{-3mm}{6':\ return temp}
			\EndFunction
			}
		\end{algorithmic}
		\caption{The code after \emph{CFG cloning}.}
 \label{figure:autoPartitioning:step2}
	\end{subfigure}
	\begin{subfigure}[t]{.35\textwidth}
		\begin{algorithmic}[0]{}
		{\ttfamily
			\Function{foo}{X x, int i} \label{codeXXX:aaaaa}
            \Statex -------------------- \Comment{\textrm{read-only phase $C_r$}}
            \State\hspace{-3mm}{1 :\ x.lock()}
            \State\hspace{-3mm}{2 :\ if i>7  then}
            \State\hspace{-3mm}{3 :\ \ \ \underline{\textbf{goto 3'}}}
            \State\hspace{-3mm}{4 :\ temp = x.f1 + x.f2}
            \State\hspace{-3mm}{5 :\ x.unlock()}
            \State\hspace{-3mm}{6 :\ \underline{\textbf{goto 6'}}}
            \Statex -------------------- \Comment{\textrm{update phase $C_u$}}
            \State\hspace{-3mm}{1' :\ x.lock()}
            \State\hspace{-3mm}{2':\ if x.f1>i  then}
            \State\hspace{-3mm}{3':\ \ \ x.f2 = i}
            \State\hspace{-3mm}{4':\ temp = x.f1 + x.f2}
            \State\hspace{-3mm}{5':\ x.unlock()}
            \State\hspace{-3mm}{6':\ return temp}
			\EndFunction
			}
		\end{algorithmic}
		\caption{The code after \emph{phase separation}.
        %Lines 1-6 are the read-only phase; and lines 1'-6' are the update phase.
} \label{figure:autoPartitioning:step3}
	\end{subfigure}

	\end{center}
\vspace{-4mm}
	\caption{Example for automatic code partitioning to read-only and update phases.}
			\label{figure:autoPartitioning}
\end{figure*}

We now explain our automatic technique to partition code $C$ to read-only and update phases $C_r$ and $C_u$, resp.
By slight abuse of terminology, we use the same notation ($C$, $C_r$, or $C_u$) both for the code section and 
for its corresponding CFG.  
%
We demonstrate our technique for the code in Figure~\ref{figure:autoPartitioning:step1}.
Notice that here, the code location where the shared memory is updated depends on the operation's parameter $i$: if $i>7$ then the code writes to shared memory in line $3$, and otherwise the code does not update shared memory.

\noindent Our technique is realized by the following two steps:
\begin{description}
  \item [CFG cloning:]
Given code $C$ of a data structure operation, we create a clone $C_u$ of $C$, and concatenate $C_u$ to the end of $C$.
For example, Figure~\ref{figure:autoPartitioning:step2} shows the code produced from Figure~\ref{figure:autoPartitioning:step1}.
Each location $l$ in $C$ has a matching location in its clone, denoted by $l'$
(e.g, in Figure~\ref{figure:autoPartitioning:step2}, the matching location of line $5$ is line $5'$).
  \item [Phase separation:]
Next, we modify the original code $C$ to execute only the read-only phase, and branch to $C_u$ when the 
update phase begins. To this end, we replace every shared memory write instruction 
at location $l \in C$ with the instruction \emph{goto l'}.
Likewise, every \emph{return instruction} (or any other instruction that exits from the operation) 
at location $l \in C$ is replaced with the instruction \emph{goto l'}.
The transformed version of $C$ is denoted $C_r$. 
Figure~\ref{figure:autoPartitioning:step3} shows the code produced from Figure~\ref{figure:autoPartitioning:step2}.
\end{description}


For example, in Figure~\ref{figure:autoPartitioning:step3}, lines 1-6 are the read-only phase; and lines 1'-6' are the update phase.

It is easy to see that this new code is equivalent to the original code (in terms of external behavior), and it satisfies the assumption from Section~\ref{ssec:transformation}. Note further that $C_u$ consists of the complete original code, including  
the read-only prefix, which is usually skipped. This property is important in order to allow us to fall back on executing the 
operation pessimistically, as we now explain. 

\paragraph{Pessimistic Fallback}
Note that, using our transformation, the shared state at the end of the validation phase
is identical to the state that would have been reached had the code been executed pessimistically from
the outset. Hence, the three-phase version of the code is compatible with the instrumented
pessimistic version. This means that if the optimistic phase is unsuccessful for any reason, we can always
fall back on the pessimistic version. Moreover, we can switch from optimistic to pessimistic synchronization 
\emph{at any point} during the read phase.
We use this property in two ways, as we now describe.

First, we avoid livelocks by limiting the number of restarts due to conflicts: 
The validation phase tracks the number of restarts in a thread-local variable. 
If this number exceeds a certain threshold, we branch to $S'$, i.e., the location in the clone $C_u$ that matches 
the beginning of the operation $S$. 

Second, this property offers the optimistic implementation the liberty of 
failing spuriously, even in the absence of conflicts, because it can always fall back on the safe pessimistic version
of the code. Our implementation
takes advantage of this liberty, and uses constant size arrays for the \emph{lockedSet} and \emph{readSet}. 
In case either of these arrays becomes full, we cannot proceed with the optimistic version, but also
do not need to start the operation anew. 
Instead, we immediately perform the validation phase, which, if successful, branches to the location in $C_u$
that matches the current code location, after having acquired all the needed locks. 

\newcommand{\op}{\emph{\textsc{op}}}
\newcommand{\opt}{\textsc{opt}}

\section{Correctness Proof}
\label{sec:proof}

We prove that each execution of the optimistic algorithm is equivalent to a
concurrent execution of the sequential code instrumented with the locking
protocol. 
This proves that the optimistic algorithm provides serializability or strict
serializability with respect to the safety guarantees of the locking protocol.

Let $\pi$ be a finite execution of the optimistic algorithm. 
First we remove from the execution all steps that access (read or write to)
versions. We note that accesses to versions are not included in the locking
protocol nor in the sequential code, they are added only as part of the
optimistic instrumentation, therefore removing them does not affect the return
value of the operations in this execution. \eshcar{need to make this argument
stronger?} We then perturb the resulting execution, $\pi'$, in iterations to
construct an execution, $\pi''$, of the locking protocol.

Since a failure of the validation phase triggers a re-try, we assume all
operations \op\ in $\pi$ execute a (single) successful validation phase. We
denote by $e_1, e_2, \ldots, e_i, \ldots$ the sequence of the first steps of the
read set validation in these successful phases, by their order in
$\pi$, where $e_i$ is a step of the operation $op_{e_i}$ executed by process $p_{e_i}$.

Two executions are \emph{equivalent} if every process in their associated
histories invokes the same operations in the same order and gets the
same results for each operation.

Two executions are \emph{indistinguishable} to a set of operations if each
operation in the set executes the same steps on primitive shared objects, and
receives the same value from those primitives in both executions. 

For every operation $op_{e_i}$, consider the partition of $\pi$ to
the following intervals $\pi=\alpha_i\beta_i\gamma_i$, such that
$\alpha_i$ includes the execution interval of $op_{e_i}$'s read phase (denote
$op_{e_i}$'s read set $rs_{e_i}$); $\beta_i=\beta_{i_1}\beta_{i_2}$, is the
execution interval of $op_{e_i}$'s successful validation phase; in
$\beta_{i_1}$, $op_{e_i}$ acquires 
locks on objects pointed by its local variables, denoted $locals_{e_i}$; 
$e_i$ is the first step of $\beta_{i_2}$, namely the read set validation
interval.

The next claim follows from the fact that the validation phase of $op_{e_i}$
in $\beta_i$ is successful, and includes locks and versions
re-validation: \eshcar{need to prove this?}

\begin{claim}
\label{claim:locks}
No operation in $\op\setminus\{op_{e_i}\}$ holds or acquires a lock in
$\alpha_i\beta_{i_1}$ on an object $obj$ in $rs_{e_i}$ after $op_{e_i}$ last
read $obj$ in $\alpha_i$.
\end{claim}


Let $\pi'$ be a projection of $\pi$ excluding all steps accessing versions.
Since the result values returned by all operations in \op\ are the same
in $\pi$ and $\pi'$
\begin{claim}
\label{claim:pipitag}
$\pi$ and $\pi'$ are equivalent.
\end{claim}


Our main lemma construct the execution of the code with full instrumentation of
the locking protocol.
\begin{lemma}
Given a locking protocol, there is an execution $\pi''$ of the code
instrumented with the locking protocol that is equivalent to $\pi'$.
\end{lemma}
\begin{proof}
We start with the execution
$\pi_1^{'}=\alpha_1^{'}\beta_{1}^{'}\gamma_1^{'}$, where $\alpha_1^{'}$
($\beta_1^{'}$ and $\gamma_1^{'}$) is the projection of $\alpha_1$
($\beta_1$ and $\gamma_1$, respectively) excluding the steps accessing the
versions.
For every $i \geq 1$, we show how to perturb $\pi_i^{'}$ to
obtain an execution $\pi_{i}^{''}=\pi_{i+1}^{'}$ in which
\begin{enumerate}
  \item \label{cond:lp} The operations $op_{e_1},\ldots,op_{e_i}$ follow the
  full locking protocol, while the rest of the operations
  $\opt_{i+1}=\op\setminus\{op_{e_1},\ldots,op_{e_i}\}$ do not follow the
  full locking protocol
  \item \label{cond:locks} For each $j\geq i+1$, Let $\beta_j^{'}$ be
the minimal interval containing $op_{e_i}$'s validation phase in
$\pi_{i}^{'}$.
  No operation in $\op\setminus\{op_{e_{j}}\}$ holds a lock in
  $\beta_j^{'}$ on an object $obj$ in $rs_{e_{j}}$ 
  %after $op_{e_{j}}$ last read $obj$ before $\beta_j^{'}$
  \item \label{cond:equiv} $\pi^{'}$ and $\pi_{i+1}^{'}$ are equivalent
\end{enumerate}

For $\opt_{i+1}=\emptyset$, we get an execution where all operations follow the
locking protocol, and by Condition~\ref{cond:equiv} $\pi''=\pi_{i+1}^{'}$ is
equivalent to $\pi^{'}$ and we are done.

The proof is by induction on $i$. For the base case we consider
the execution $\pi^{'}=\pi_1^{'}$. Condition~\ref{cond:lp} holds since none of
the operations in this execution follow the full locking protocol.
Condition~\ref{cond:locks} holds by
Claim~\ref{claim:locks} and since
accesses to objects (other than versions) and locks are similar in $\pi$ and $\pi'$.
Condition~\ref{cond:equiv} vacuously holds since $\pi'$ and $\pi_1^{'}$ are the
same execution.

For the induction step, assume $\opt_i \neq \emptyset$ and
the execution
$\pi_i^{'}=\alpha_i^{'}\beta_i^{'}\gamma_i^{'}$ satisfies the above conditions.
We replace $\pi_i^{'}$ with
$\pi_i^{''}=\alpha_i^{''}\delta_i\beta_i^{''}\gamma_i^{'}$, such that
$\alpha_i^{''}$ and $\beta_i^{''}$ are the projection of $\alpha_i^{'}$ and
$\beta_i^{'}$, respectively, excluding the steps by $op_{e_i}$, while
$\delta_i=\delta_{i_1}\delta_{i_2}$ is a $p_{e_i}$-only execution
interval. In $\delta_{i_1}$, $p_{e_i}$ follows the locking protocol while
reading $rs_{e_i}$; in $\delta_{i_2}$, $p_{e_i}$ unlocks all objects that are
not in $locals_{e_i}$. 
In other words, we replace the optimistic read phase and validation phase of
$op_{e_i}$ with an execution of a read phase instrumented with the
locking protocol, taking place at the point just before the validation phase starts.

By Condition~\ref{cond:locks} of the induction hypothesis no operation holds 
locks on objects in the read set of $op_{e_i}$ at the configuration after
$\alpha_i^{'}$ therefore $p_{e_i}$ can acquire the locks on these
objects while executing $\delta_{i_1}$.
Moreover, by the correctness properties of the locking protocol, it is safe to
release the locks on objects not in $locals_{e_i}$ while executing
$\delta_{i_2}$. The projection of the execution $\pi_i^{''}$ on $op_{e_i}$
follows the full locking protocol and thus Condition~\ref{cond:lp} holds and
$op_{e_i}$ returns the same value in $\pi_i^{''}$ as in $\pi^{'}$.

By Condition~\ref{cond:locks} of the induction hypothesis no operation acquires
locks on objects in $locals_{e_i}$ during $\beta_i^{'}$. Therefore, the executions $\alpha_i^{'}\beta_i^{'}$,
ending with configuration $C'$, and $\alpha_i^{''}\delta_i\beta_i^{''}$, ending
with configuration $C''$, are indistinguishable to all operations in
$\op\setminus\{op_{e_{i}}\}$. Since the values of all shared objects and locks
are the same in $C'$ and $C''$, the
executions $\alpha_i^{'}\beta_i^{'}\gamma_i^{'}$ and
$\alpha_i^{''}\delta_i\beta_i^{''}\gamma_i^{'}$ are indistinguishable to all operations in
$\op\setminus\{op_{e_{i}}\}$. With $op_{e_i}$ having the same return value in
$\pi_i^{''}$ as in $\pi^{'}$ we conclude that Condition~\ref{cond:equiv} holds.

It is left to show that $\pi_i^{''}$ satisfies Condition~\ref{cond:locks}. 
This is straightforward from the induction hypothesis and the fact that only
$op_{e_i}$ changed its locking pattern in the last iteration. \eshcar{this
is were the proof gets incorrect} More specifically, since the additional locks
$op_{e_i}$ acquired are released by the end of $\delta_i$, and by our construction for all $j\geq i+1$, $\beta_j^{'}$ start  






\begin{lemma}
\end{lemma}
\begin{proof}
\end{proof}



\section{Evaluation}
\label{sec:eval}

We evaluate the performance of our approach on 
%two types of data structures. In Section~\ref{sec:readwrite} we consider 
search trees supporting insert, delete, and get operations. 
%whereas Section~\ref{sec:range}focuses on data structures that, in addition, support range queries that retrieve all keys within a given range. 
We compare the throughput of our approach to fully pessimistic solutions applying fine-grain locking, solutions based on software transactional memory, and hand-crafted state-of-the-art data structure
implementations.
%Thesealgorithms also serve as the lock-based reference implementation at the base of our semi-optimistic implementations.We furthercompare our approach to software transactional memory and hand-crafted state-of-the-art data structure implementations supporting the same functionality.

\paragraph{Methodology} We use the micro-benchmark suite \textit{Synchrobench}~\cite{Gramoli2015}, configured as follows. 
%We follow a standard evaluation methodology
%(\cite{DrachslerVY2014,NatarajanM2014,BrownER2014,ArbelA2014}). 
Each experiment
consists of $5$ trials. A trial is a five second run in which each thread continuously executes
randomly chosen operations drawn from the workload distribution, with keys
selected uniformly at random from the range $[0,2\cdot10^6]$.
Each trial is preceded by initiating a new data structure with
%and applying an untimed pre-filling phase, which continues until the size of the data structure is within 5\% of
$10^6$ keys and a warm-up of five seconds.  Our graphs present the average throughput over all trials.
We consider three representative workloads distributions: a
\emph{read-only} workload comprised of $100\%$ lookup operations, a \emph{write-dominated}
workload consisting of insert and delete operations ($50\%$ each), and a
\emph{mixed workload} with $50\%$ lookups, $25\%$ inserts, and $25\%$
deletes.

\paragraph{Platform} All implementations are in Java. We ran the experiments on a dedicated machine with
four Intel Xeon E5-4650 processors, each with $8$ cores, for a total of $32$ threads
(with hyper-threading disabled).
We used Ubuntu 12.04.4 LTS and Java Runtime Environment (build
1.7.0\_51-b13) using the 64-Bit Server VM (build 24.51-b03, mixed mode).

%\subsection{Insert-Delete-Get Operations}
%\label{sec:readwrite}

\paragraph{Implementations}
%We start by benchmarking a search-tree supporting the basic insert,
%delete, and get (lookup) operations. 
%Our experiments evaluate unbalanced as well as balanced trees.
We start from textbook sequential implementations of an unbalanced internal binary
tree and a treap~\cite{AragonS1989}. We next  synthesize 
concurrent lock-based code by (manually) applying the domination locking technique~\cite{Gueta2011} to the sequential
data structures. The resulting algorithms are denoted \domTree and \domTreap.
Then, we manually apply our lock-removal transformation to the reference
implementations by following the algorithm line-by-line (requiring no understading of the base code)
to get our semi-optimistic versions of the code, which we call
\autoTree and \autoTreap, respectively. Note that this solution does not track the  \emph{lockedSet} for read-only operations.
Finally, we apply the optimization described in Section~\ref{sssec:alg-normal}, which eliminates explicit tracking of the \emph{lockedSet} in update operations,
and instead locks 
all objects the thread holds a pointer to in the validation phase; this optimization is applicable since our parallel implementation is synthesized using
domination locking. The resulting algorithms are denoted \optAutoTree and \optAutoTreap.   

For the competition, we parallelize the sequential implementations also using 
Deuce~\cite{Deuce2010}, a Java implementation of  TL2~\cite{DiceSS2006}. The resulting algorithms are denoted \stmTree and \stmTreap. We further compare our implementations to their hand-crafted state-of-the-art counterparts
listed in Table~\ref{table:hand-crafted}.
%\footnote{Unless described otherwise implementations are provided by Synchrobench.}. 


\begin{table}
\begin{tabular}{| l p{2.25in} |l  p{2.25in} |}
\hline 
  {\bf Unbalanced} && {\bf Balanced}  &\\  \hline 
  \textbf{\danaTree} & Locked-based unbalanced tree~\cite{DrachslerVY2014} & \textbf{\danaAVL} & Lock-based relaxed  AVL  tree~\cite{DrachslerVY2014}  \\ 
%				Drachsler et al.
  \textbf{\lockfreeTree} & Lock-free unbalanced tree~\cite{EllenFRB2010}  & \textbf{\bronson} & Lock-based relaxed  AVL tree~\cite{BronsonCCO2010} \\
   && \textbf{\friendly} & Contention-friendly tree~\cite{CrainGR2013}  \\
   && \textbf{Skiplist} & Java lock-free skiplist \\
   \hline 
\end{tabular}
\caption{Hand-crafted state-of-the-art data structures. The code of \danaTree was provided by the authors, all other implementations provided by Synchrobench.\label{table:hand-crafted}}
% ,provided by Synchrobench unless described otherwise

\end{table}



We also measured the performance of global lock-based implementations.
In all workloads, the results were identical or inferior to those
achieved by pessimistic fine-grain locking. We hence
omitted these results to avoid obscuring the presentation.



\paragraph{Results}
Figures~\ref{evaluation:results:unbalanced} and~\ref{evaluation:results:balanced}
show the throughput of unbalanced and balanced data structures, resp. We see that our semi-optimistic
solution, both optimized and unoptimized, 
 is far superior to the fully-pessimistic automated approach; it successfully overcomes the bottlenecks associated with lock contention
in  \domTree and \domTreap. 

Our approach  also outperforms STM by 1.5x to 2.5x.
The additional overhead of STM most likely stems from two reasons: deferring writes to commit time, and 
using a global clock to ensure a consistent view of the read set. The latter is done in order to satisfy opacity~\cite{GuerraouiK2008}, 
which we avoid by ``sandboxing''.
In our experiments, the code \emph{never} incurred a spurious exception or timeout due to
inconsistent reads, and so the sandboxing was not associated with a performance penalty. 

%might be due to the tradeoff between validation overhead and no internal consistency.
%To ensure opacity~\cite{GuerraouiK2008}, each time an object is read in STM, the transaction checks that the version of the lock protecting the object is valid. 
%Our algorithms read the version instead of locking the object which happens typically once during the operation.
%This is where sandboxing pays off -- we allow operations to observe inconsistent views and hence have improved performance.

Our solution comes close to custom-tailored implementations, and the optimized version is even superior to some of them.
The throughput of our read-only operations is up to 1.8x lower than that achieved by the best-in-class. 
By profiling the code, we learned that the bulk of this overhead stems from the need to track all read objects,
which is inherent to our transformation.
This is in contrast with the hand-crafted implementations, which have small overhead on reads that complete without any retries. 
In  workloads that include update operations, our solution is  up to 2.2x slower.
\Idit{Can we say that this only stems from tracking read and locked sets, because we got very few restarts? Can we say what percentage of restarts was?}


\Xomit{
The results for the read-only workload show the main overhead
of our approach. By profiling the code, we learned
that the bulk of this overhead stems from the need to track all read objects,
which is inherent to our transformation.
This is in contrast with the hand-crafted implementations,
which have small overhead on reads that complete without any retries. 

%thanks to either wait-free reads (in \danaAVL, \danaTree, \friendly and \lockfreeTree ), or optimistic validation (in \bronson).

As the ratio of updates in the workload increases, our implementation
closes this gap.
In other words, the transformed code deals well with update contention.
This might be due to the fact that once
an update phase begins, the operation is not delayed due to concurrent
read-only operations.
}

%% Eshcar: discuss small trees? the results are not the same
%We also experimented with smaller trees ($[0,2\cdot10^4]$) to test different contention levels (the results appear in Appendix~\ref{sec:appendix:results}).
%The results show that our transformation works better on larger data structures. 
%Indeed, in large data structures, update operations are more likely to operate on disjoint parts of the data, allowing high concurrency. 
%This is especially important for the automatic transformation as updates with overlapping lock sets might invalidate each other.
%; since the results showed similar trends, they are omitted here.


\begin{figure*}
\begin{center}
\input{plots/unbalanced3}
\end{center}
\caption{Throughput of unbalanced data structures.}
\label{evaluation:results:unbalanced}
\end{figure*}


\begin{figure*}
\begin{center}
\input{plots/balanced3}
\end{center}
\caption{Throughput of balanced data
structures.}
\label{evaluation:results:balanced}
\end{figure*}

\Xomit{
\subsection{Range Queries}
\label{sec:range}

Next we evaluate the performance of our approach when the data
structure supports a more intricate functionality like range queries.

Hand
crafting an implementation of a data structure that supports atomic
(linearizable) range queries is challenging.
The implementations that do support iterating through records may impose an
additional overhead on the regular read and write operations to ensure
progress of range queries.
The results in this section demonstrate that our method
allows generating a correct and efficient code, which is otherwise difficult
to obtain.

We use a skip list, which readily supports range queries by
nature of its linked-structure. The core of the implementation is the key lookup
method; once reaching the key, a key can be added or be removed in place, and
an iteration of subsequent keys can be executed by traversing
the bottom-level linked-list.

The domination locking scheme cannot be efficiently applied to the skip list
structure since it is a DAG rather than a tree. Instead, we
manually devise a pessimistic locking protocol. Our
algorithm, (inspired by the one in~\cite{HerlihyS2008}), applies
hand-over-hand locking at each level, so that at the end of the search, the
operation holds locks on two keys in each level, which define the minimal interval within this level
containing the lookup key (or the first lookup key in the case of a range query). Upon
reaching the bottom level, unnecessary locks are released, as follows: update operations only keep
locks on nodes they intend to modify, whereas
%that will be modified; while holding these locks any
%modification can be executed in isolation from other update operations.
range
queries keep the locks in the level with the minimal interval spanning the
range. Range queries then continue to use hand-over-hand locking to traverse through all keys
within the range. The use of hand-over-hand locking ensures that range queries are atomic
(linearizable), i.e., return a consistent view of the data structure.

This pessimistic lock-based algorithm is denoted \domSkiplist.
As in previous data structures, we  apply the lock-removal transformation to the
reference implementation to get a semi-optimistic algorithm, which we call
\autoSkiplist.

We also applied Deuce to the skip-list sequential implementation. The resulting algorithm is denoted \stmSkiplist.

Our approach is also compared to the aforementioned
state-of-the-art data structures that support range queries. Specifically,
we compare \autoSkiplist to 
\begin{description}
\setlength{\itemsep}{0pt}
\setlength{\parskip}{0pt}
\item[\skiplist] The non-blocking Java skip-list which supports \emph{non}-linarizable range queries through iterators.
\item[\kary] A linearizable, non-blocking $k$-ary search tree
that supports range queries~\cite{BrownA12}\footnote{\url{http://www.cs.toronto.edu/~tabrown/kstrq}}.
%\item[\bronson] provides atomic range queries by traversing a clone of the
%original tree that is lazily generated
\end{description}
To ensure a fair comparison (following~\cite{BrownA12}) the range query operation in all implementations return an array of keys.
For \skiplist this means projecting a subset of the keys, iterating over them, and then copying each of these keys into an array. This does not include a snapshot, so range queries are
not always linearizable. \kary is similar to a b-tree, where the degree of the nodes is at most $k$. In our experiments we set $k$ to 64.

Like many data structure libraries, \friendly and \danaAVL
do not support atomic range queries, and
there is no straightforward way to add them.

\begin{figure*}
	\begin{center}
	\begin{{subfigure}[t]{.35\textwidth}
		\caption{Range queries}
		
\begin{tikzpicture}
\begin{axis}[mystyle,skiplistUpdate,
 ylabel = { million ops/sec},
 title={\textbf{write-dominated}}]
\addplot [black,mark=square*] table [x={threads}, y={LockRemovalSkipList}]
{results/trees-i1000000-u100.log}; 
\addplot [violet,mark=o] table [x={threads}, y={LockFreeKSTRQ}]
{results/trees-i1000000-u100.log}; 
\addplot [teal,mark=pentagon] table [x={threads}, y={LockFreeJavaSkipList}]
{results/trees-i1000000-u100.log};
\addplot [magenta,mark=+] table [x={threads}, y={STMSkipList}]
{results/trees-i1000000-u100.log};
\addplot [blues4,mark=x] table [x={threads}, y={DominationLockingSkipList}]
{results/trees-i1000000-u100.log};
\end{axis}
\end{tikzpicture}
\Xomit{
\begin{tikzpicture}
\begin{axis}[mystyle,skiplistUpdate, 
				 title={\textbf{read-only}},]
\addplot [black,mark=square*] table [x={threads}, y={LockRemovalSkipList}]
{results/trees-i1000000-u0.log}; 
\addplot [violet,mark=o] table [x={threads}, y={LockFreeKSTRQ}]
{results/trees-i1000000-u0.log}; 
\addplot [teal,mark=pentagon] table [x={threads}, y={LockFreeJavaSkipList}]
{results/trees-i1000000-u0.log};
\addplot [magenta,mark=+] table [x={threads}, y={STMSkipList}]
{results/trees-i1000000-u0.log};
\addplot [blues4,mark=x] table [x={threads}, y={DominationLockingSkipList}]
{results/trees-i1000000-u0.log};
\end{axis}
\end{tikzpicture}
}

\ref{skiplistLegenedUpdate}



		\label{evaluation:results:skiplist:scans}
	\end{subfigure}
	\quad\quad
	\begin{subfigure}[t]{.35\textwidth}
		\caption{Insert and delete operations}
		
\begin{tikzpicture}
\begin{axis}[mystyle,skiplistUpdate,
 ylabel = { million ops/sec},
 title={\textbf{write-dominated}}]
\addplot [black,mark=square*] table [x={threads}, y={LockRemovalSkipList}]
{results/trees-i1000000-u100.log}; 
\addplot [violet,mark=o] table [x={threads}, y={LockFreeKSTRQ}]
{results/trees-i1000000-u100.log}; 
\addplot [teal,mark=pentagon] table [x={threads}, y={LockFreeJavaSkipList}]
{results/trees-i1000000-u100.log};
\addplot [magenta,mark=+] table [x={threads}, y={STMSkipList}]
{results/trees-i1000000-u100.log};
\addplot [blues4,mark=x] table [x={threads}, y={DominationLockingSkipList}]
{results/trees-i1000000-u100.log};
\end{axis}
\end{tikzpicture}
\Xomit{
\begin{tikzpicture}
\begin{axis}[mystyle,skiplistUpdate, 
				 title={\textbf{read-only}},]
\addplot [black,mark=square*] table [x={threads}, y={LockRemovalSkipList}]
{results/trees-i1000000-u0.log}; 
\addplot [violet,mark=o] table [x={threads}, y={LockFreeKSTRQ}]
{results/trees-i1000000-u0.log}; 
\addplot [teal,mark=pentagon] table [x={threads}, y={LockFreeJavaSkipList}]
{results/trees-i1000000-u0.log};
\addplot [magenta,mark=+] table [x={threads}, y={STMSkipList}]
{results/trees-i1000000-u0.log};
\addplot [blues4,mark=x] table [x={threads}, y={DominationLockingSkipList}]
{results/trees-i1000000-u0.log};
\end{axis}
\end{tikzpicture}
}

\ref{skiplistLegenedUpdate}



		\label{evaluation:results:skiplist:updates}
	\end{subfigure}
	\ref{skiplistLegened}
	\end{center}
\caption{Small range: all threads execute either small range queries $[10,20]$
or a mix of insert and delete operations.}
\label{evaluation:results:skiplist}
\end{figure*}
}

\begin{figure*}
\begin{center}
\begin{tikzpicture}
\begin{axis}[mystyle,skiplistUpdate,
 %ylabel = { million ops/sec},
  ylabel = { million ops/sec},
 title={\textbf{write-dominated}},
 width=0.28\textwidth,]
\addplot [black,mark=square*] table [x={threads}, y={LockRemovalSkipList}]
{results/trees-i1000000-u100.log}; 
\addplot [violet,mark=o] table [x={threads}, y={LockFreeKSTRQ}]
{results/trees-i1000000-u100.log}; 
\addplot [teal,mark=pentagon] table [x={threads}, y={LockFreeJavaSkipList}]
{results/trees-i1000000-u100.log};
\addplot [magenta,mark=+] table [x={threads}, y={STMSkipList}]
{results/trees-i1000000-u100.log};
\addplot [blues4,mark=x] table [x={threads}, y={DominationLockingSkipList}]
{results/trees-i1000000-u100.log};
\end{axis}
\end{tikzpicture}
\begin{tikzpicture}
\begin{axis}[mystyle,skiplist,
 title={\textbf{range-queries}},
 width=0.28\textwidth,]
 ]
\addplot [black,mark=square*] table [x={threads}, y={LockRemovalSkipList}]
{results/range1.txt}; 
\addplot [violet,mark=o] table [x={threads}, y={LockFreeKSTRQ}]
{results/range1.txt}; 
\addplot [teal,mark=pentagon] table [x={threads}, y={LockFreeJavaSkipList}]
{results/range1.txt};
\addplot [magenta,mark=+] table [x={threads}, y={STMSkipList}]
{results/range1.txt};
\addplot [blues4,mark=x] table [x={threads}, y={DominationLockingSkipList}]
{results/range1.txt};
\end{axis}
\end{tikzpicture}
\begin{tikzpicture}
\begin{axis}[mystyle,skiplist,
% ylabel = { million ops/sec},
 title={\textbf{insert and delete}},
 width=0.28\textwidth,]
 ]
\addplot [black,mark=square*] table [x={threads}, y={LockRemovalSkipList}]
{results/update1.txt}; 
\addplot [violet,mark=o] table [x={threads}, y={LockFreeKSTRQ}]
{results/update1.txt}; 
\addplot [teal,mark=pentagon] table [x={threads}, y={LockFreeJavaSkipList}]
{results/update1.txt};
\addplot [magenta,mark=+] table [x={threads}, y={STMSkipList}]
{results/update1.txt};
\addplot [blues4,mark=x] table [x={threads}, y={DominationLockingSkipList}]
{results/update1.txt};
\end{axis}
\end{tikzpicture}
\begin{tikzpicture}
\begin{axis}[mystyle,skiplistUpdate, 
				 title={\textbf{read-only}},
				 ymax=15000000,
				 width=0.28\textwidth,]
\addplot [black,mark=square*] table [x={threads}, y={LockRemovalSkipList}]
{results/trees-i1000000-u0.log}; 
\addplot [violet,mark=o] table [x={threads}, y={LockFreeKSTRQ}]
{results/trees-i1000000-u0.log}; 
\addplot [teal,mark=pentagon] table [x={threads}, y={LockFreeJavaSkipList}]
{results/trees-i1000000-u0.log};
\addplot [magenta,mark=+] table [x={threads}, y={STMSkipList}]
{results/trees-i1000000-u0.log};
\addplot [blues4,mark=x] table [x={threads}, y={DominationLockingSkipList}]
{results/trees-i1000000-u0.log};
\end{axis}
\end{tikzpicture}

\ref{skiplistLegenedUpdate}

\end{center}
\caption{Small range: (on the right) read-only workload all threads execute small range queries $[10,20]$; (on the left) write-dominated workload all threads execute a mix of insert and delete operations; (in the middle) mixed workload half the threads execute small range queries (middle left)
and half the threads execute insert and delete operations (middle right).}
\label{evaluation:results:skiplist}
\end{figure*}

\begin{figure*}
	\begin{center}
	\begin{subfigure}[t]{.35\textwidth}
		\caption{Range queries}
		
\begin{tikzpicture}
\begin{axis}[mystyle,skiplist,
 ylabel = { million ops/sec},
 %title={\textbf{0\% read-only}}]
 ]
\addplot [black,mark=square*] table [x={threads}, y={LockRemovalSkipList}]
{results/range1.txt}; 
\addplot [violet,mark=o] table [x={threads}, y={LockFreeKSTRQ}]
{results/range1.txt}; 
\addplot [teal,mark=pentagon] table [x={threads}, y={LockFreeJavaSkipList}]
{results/range1.txt};
\addplot [magenta,mark=+] table [x={threads}, y={STMSkipList}]
{results/range1.txt};
\addplot [blues4,mark=x] table [x={threads}, y={DominationLockingSkipList}]
{results/range1.txt};
\end{axis}
\end{tikzpicture}

%\ref{skiplistLegened}



		\label{evaluation:results:range}
	\end{subfigure}
	\quad\quad
	\begin{subfigure}[t]{.35\textwidth}
		\caption{Insert and delete operations}
		
\begin{tikzpicture}
\begin{axis}[mystyle,skiplist,
% ylabel = { million ops/sec},
 %title={\textbf{0\% read-only}}]
 ]
\addplot [black,mark=square*] table [x={threads}, y={TimeoutSkiplist}]
{results/update.txt}; 
\addplot [darkspringgreen,mark=triangle] table [x={threads}, y={Bronson}]
{results/update.txt}; 
\addplot [coralred,mark=asterisk] table [x={threads}, y={JavaSkipList}]
{results/update.txt};
\addplot [blues4,mark=x] table [x={threads}, y={DominationLockingSkiplist}]
{results/update.txt};
\end{axis}
\end{tikzpicture}

%\ref{skiplistLegened}



		\label{evaluation:results:update}
	\end{subfigure}
	\ref{skiplistLegened}
	\end{center}
\caption{Small range: half the threads execute small range queries $[10,20]$
and half the threads execute insert and delete operations.}
\label{evaluation:results:skiplist10}
\end{figure*}


\begin{figure*}
	\begin{center}
	\begin{subfigure}[t]{.35\textwidth}
		\caption{Range queries}
		
\begin{tikzpicture}
\begin{axis}[mystyle,skiplist1000,
 ylabel = { million ops/sec},
 %title={\textbf{0\% read-only}}]
 ]
\addplot [black,mark=square*] table [x={threads}, y={TimeoutSkiplist}]
{results/range1000.txt}; 
\addplot [darkspringgreen,mark=triangle] table [x={threads}, y={Bronson}]
{results/range1000.txt}; 
\addplot [coralred,mark=asterisk] table [x={threads}, y={JavaSkipList}]
{results/range1000.txt};
\addplot [blues4,mark=x] table [x={threads}, y={DominationLockingSkiplist}]
{results/range1000.txt};
\end{axis}
\end{tikzpicture}

%\ref{skiplistLegened}



		\label{evaluation:results:range1000}
	\end{subfigure}
	\quad\quad
	\begin{subfigure}[t]{.35\textwidth}
		\caption{Insert and delete operations}
		
\begin{tikzpicture}
\begin{axis}[mystyle,skiplist,
 ylabel = { million ops/sec},
 %title={\textbf{0\% read-only}}]
 ]
\addplot [black,mark=square*] table [x={threads}, y={LockRemovalSkipList}]
{results/update10001.txt}; 
\addplot [violet,mark=o] table [x={threads}, y={LockFreeKSTRQ}]
{results/update10001.txt}; 
\addplot [teal,mark=pentagon] table [x={threads}, y={LockFreeJavaSkipList}]
{results/update10001.txt};
\addplot [magenta,mark=+] table [x={threads}, y={STMSkipList}]
{results/update10001.txt};
\addplot [blues4,mark=x] table [x={threads}, y={DominationLockingSkipList}]
{results/update10001.txt};
\end{axis}

\end{tikzpicture}

%\ref{skiplistLegened}



		\label{evaluation:results:update1000}
	\end{subfigure}
	\ref{skiplistLegened1000}
	\end{center}
\caption{Half the threads execute large range queries $[1000,2000]$
and half the threads execute insert and delete operations.}
\label{evaluation:results:skiplist1000}
\end{figure*}

\paragraph{Results}
We start the evaluation 
%of data structures supporting range queries 
(Figure~\ref{evaluation:results:skiplist}) with the read-only workload, where all threads execute small range queries between $10$ to $20$ keys. 
%The results reflect the correlation of the implementation to scan-only workload. 
As expected, \kary has the best performance as it is optimal for batch scans. The simplicity of the non-linearizable implementation of \skiplist allows it to perform well. Our semi-optimistic transformed code, outperforms both the STM and the fully-pessimistic fine-grain transformations.

On the other extrem, we evaluated all updates (write-dominated) workloads, where the operations are a mix of insert and delete operations (50\% each). Here, splitting and merging nodes affect the performance of \kary which flattens out at 32 threads. The throughput of \autoSkiplist and \stmSkiplist are comparable, both scale nicely with the number of threads.
Again, we see that \autoSkiplist outperforms
the fully-pessimistic fine-grain one. We believe that,
as in the
domination locking versions of the tree data structures, holding a lock on the head sentinel of the skip list in
\domSkiplist, even for short periods, imposes a major performance penalty,
which is eliminated by our semi-optimistic approach. 
Finally, \autoSkiplist is also superior to the STM implementation
improving throughput by
10x. The improvement in update operations can be attributed to lack of contention on a centralized object like the global version in \stmSkiplist.

Next, we focus on a mixed workload, where half the
threads are dedicated to performing range queries, and the other half perform a
mix of insert and delete operations (50\% each).
This mix allows us to evaluate both the performance of the range queries,
and their impact on concurrent updates and vice versa.
Indeed, the results show that \kary's throughput for range query is comarable to that of \skiplist, and at 32 thread have througput similar to \autoSkiplist.

We also experimented with mixed workloads for large queries with large ranges varying between $1000$ to $2000$ keys (the results appear in Appendix~\ref{sec:appendix:results}). Here the impact of concurrent update operations on range queries is most pronounced in the results of \kary.


The evaluation of range queries focuses on a mixed workload, where half the
threads are dedicated to performing range queries, and the other half perform a
mix of insert and delete operations (50\% each).
This mix allows us to evaluate both the performance of the range queries,
and their impact on concurrent updates and vice versa.
%\eshcar{what about a workload that includes only 100\% range
%queries?}
We experiment with queries with large ranges varying between $1000$ to $2000$ keys
(Figure~\ref{evaluation:results:skiplist1000}) and small ranges
between $10$ to $20$
keys (Figure~\ref{evaluation:results:skiplist10}).

We measure the throughput of range queries and update operations separately.
The overall number of range queries executed per second is reported
in Figures~\ref{evaluation:results:range1000}
and~\ref{evaluation:results:range}, and the overall number of update (insert
and delete) operations executed per second is reported in
Figures~\ref{evaluation:results:update1000} and~\ref{evaluation:results:update}.

Again, we see that our transformed code outperforms
  to the fully-pessimistic fine-grain one. We believe that,
as in the
domination locking versions of the tree data structures, holding a lock on the head sentinel of the skip list in
\domSkiplist, even for short periods, imposes a major performance penalty,
which is eliminated by our  semi-optimistic approach.

It is superior also to the STM implementation
improving throughput by
up to three orders of magnitude -- 500x for large range queries, 2x for small
ranges, and 10x for update operations, regardless of the range size. The improvement in update operations can be attributed to lack of contention on a centralized object like the global version in \stmSkiplist. \eshcar{explain difference in range queries}


The simplicity of the non-linearizable implementation of \skiplist allows it to perform better than \autoSkiplist on range queries.  However, the performance of
\autoSkiplist is almost identical to that of \skiplist --and even better in some cases, when comparing the update operations.
%despite the fact that \skiplist performs inconsistent iterations.

Figure~\ref{evaluation:results:range1000}
shows that, for large ranges, \bronson scales well up to $8$ threads, outperforming
all other implementations, but at $32$ threads its performance deteriorates.
This might be because this implementation is
optimized for full scans or very large range queries running sequentially.
The overhead of initiating a clone of the data structure per range query, which
involves waiting for all
pending update operations to complete, hampers scalability when
a number of queries are executed in parallel. This effect is even more pronounced when the
ranges are small (Figure~\ref{evaluation:results:range}), and the throughput of
\bronson flattens out.

In addition, the lazy cloning required to support
range queries imposes an overhead on  update operations. Copying each node
during downward traversal might be the main impediment preventing these
operations from scaling when the ranges are small (Figure~\ref{evaluation:results:update}) as well as when
they are large (Figure~\ref{evaluation:results:update1000}). In contrast,
the update operations in \autoSkiplist and \skiplist do not take any special
measures for the benefit of concurrent queries (which validate themselves in
\autoSkiplist, and are not atomic in \skiplist), and hence continue to perform well in
their presence.
}

\section{Related Work}\label{sec:related}
\paragraph{Concurrent Data Structures}
Many sophisticated concurrent data structures (e.g., \cite{ArbelA2014,DrachslerVY2014,NatarajanM2014,BrownER2014,CrainGR2013,BraginskyP2012,
AfekKKMT2012,EllenFRB2010,BronsonCCO2010,HerlihyLLS2007,fraser2004practical,Michael:1996})
were developed and used in concurrent software systems~\cite{Ohad:OOPSLA11}.
Implementing efficient synchronization for such data structures is considered a challenging and error-prone task~\cite{Ohad:OOPSLA11,Doh:SPAA04,Jin:2012}.
As a result, concurrent data structures are manually implemented by concurrency experts.
This paper shows that (in some cases) an automatic algorithm can produce synchronization that is comparable to synchronization implemented by experts.

\paragraph{Locking Protocols}
Locking protocols are used in databases and shared memory systems to guarantee correctness
of concurrently executing transactions~\cite{Weikum:2001,BHG:Book87}.
Our approach can be seen as a way to extend many existing locking protocols by combining them with  optimistic concurrency control.
In particular, our approach extends the following locking protocols:
two-phase locking~\cite{Eswaran:1976}, tree locking~\cite{SilberschatzK1980}, DAG locking~\cite{CH:PODS95} and domination locking~\cite{Gueta2011}.
We demonstrate the benefit of such combination by using the  domination locking protocol to produce efficient concurrency control for
dynamic data structures.


\paragraph{Lock Inference Algorithms}
There has been a lot of work on automatically inferring locks for transactions.
Most   algorithms in the literature infer locks that follow the two-phase
locking protocol~\cite{MZGB:POPL06,Emmi06POPL,gudka2012lock,CCG:PLDI08,HFP:TRANSACT06,CGE:CC08}.
Our approach can potentially be used to optimized the synchronization produced by these algorithms.
For example, for  algorithms that employ a two-phase variant in which all locks are acquired at the beginning of a transaction (e.g.,~\cite{gudka2012lock,CCG:PLDI08}),
our approach may be used to defer the locking (e.g., to just before the first write operation) as well as to eliminate some of the locking operations.


\paragraph{Transactional Memory}
Transactional memory approaches (TMs) dynamically resolve inconsistencies
and deadlocks by rolling back partially completed transactions.
%
Unfortunately, in spite of a lot of effort and many TM implementations (see~\cite{HLR:SLCA2010}), existing TMs
have not been widely adopted due to various concerns~\cite{DuffyTM2010,Cascaval:2008,mckenneyParallel}, including high runtime overhead,
poor performance and limited ability to handle irreversible operations.
In particular, modern concurrent programs (and concurrent data structures) are typically based on hand-crafted synchronization, rather than  on a TM approach~\cite{Ohad:OOPSLA11}.

In a sense, our approach can be seen as a specialized TM approach that can be practically used to handle concurrent data structure.


%\paragraph{Lock Elision for Read-Only Transactions}
\paragraph{Lock Elision}
Our approach is inspired by the idea of \emph{sequential locks}~\cite{mckenneyParallel} and the approach presented in~\cite{Nakaike:2010}.
But  in contrast to these approaches,  which are designed to handle read-only transactions,
our approach handles read-only prefixes of transactions (operations) that update the shared memory.
As shown in Section~\ref{sec:eval}, our approach is best suited for update-dominated workloads
Moreover, using these approaches for a highly-contended data structure (as in Section~\ref{sec:eval}) is likely to provide limited performance,
because each update transaction causes many read-only transactions to abort.

There are some transactional memory techniques to elide locks from arbitrary critical sections (e.g.,~\cite{Rajwar:2002:TLE:635508.605399,Roy:2009:RSS:1519065.1519094,Afek:2014:SHL:2611462.2611482}).
In these techniques, a transaction executes the critical section speculatively without acquiring the lock.
When a transaction is aborted, it can acquire the lock and execute the critical section non-speculatively.
In contrast to our approach, these techniques cannot combine speculative and non-speculative execution of the same transaction.







\section{Discussion}\label{sec:discussion}

We have made the case that automatic synchronization can be a viable approach for producing scalable concurrent algorithms from legacy sequential code.
Today, development of such code heavily relies on custom-tailored implementations, which require painstaking correctness proofs. 
 In this paper, we have shown simple automatic transformations that eliminate principal concurrency bottlenecks  to yield code that scales comparably  to hand-crafted  solutions.
While we have illustrated our method only for tree-like data structures, we hope that future work will prove that the approach is more broadly applicable, and can be used to 
synthesize efficient concurrent implementations of additional data structures, for which no such implementations exist yet. 

Our methods make use of a common pattern in data structures, where an operation typically begins with a long read-only traversal, followed by a handful of (usually local) modifications. 
It might be possible to exploit similar patterns in order to parallelize or remove locks in other types of code (not data structures).
Furthermore, for programs that follow different patterns, other combinations of optimism and pessimism may prove effective.

Finally, there still remains a gap between the performance achievable by manually optimized solutions and what we could achieve automatically. Our algorithm induces inherent overhead for tracking all operations in the read-only phase for later verification.
In specific data structures, these checks might be redundant, but it is difficult for a compiler to detect this automatically. We believe that
it may well be possible to enhance automatic transformations such as ours with computer-assisted optimizations. For example, a programmer may provide hints regarding certain 
invariants that are always preserved in the code, in order to eliminate the need for tracking some values for later
validation. Such optimizations have the potential to bridge the remaining performance gap, while requiring far less work 
for proving correctness -- instead of proving that the entire construction is correct, the developer would only need to 
prove that her program maintains the invariants.

%\section{Appendix Title}

%This is the text of the appendix, if you need one.

%\acks

\bibliography{myRef}
\bibliographystyle{abbrv}
%

\appendix

\section{Formal Correctness Proof}\label{sec:formal-proof}

We now formalize the correctness arguments made in Section~\ref{sec:proof}. 
First we define our model and the correctness properties of the algorithm for
which we provide the proof.

\paragraph{Model}

We consider an asynchronous shared memory model, where independent threads
interact via shared memory objects. 
Every thread executes a sequence of operations, each of which is invoked with certain parameters and returns a response.
An operation's execution consists of a sequence of primitive \emph{steps}, beginning with an \emph{invoke} step, followed by
atomic accesses to shared objects, and ending with a \emph{return} step. Steps also modify the executing thread's local variables.

A \emph{configuration} is an assignment of values to all shared and local variables. Thus, each step takes the system from one
configuration to another. Steps are deterministically defined by the data structure's protocol and the current configuration.
In the \emph{initial configuration}, each variable holds its initial value.

An \emph{execution} is an alternating sequence of configurations and steps,
$C_0,s_1,C_1, \ldots,s_i,C_i,\ldots,$
where $C_0$ is an initial configuration,
and each configuration $C_i$ is the result of
executing step $s_i$ on configuration $C_{i-1}$.
We only consider finite executions in this paper.
An execution is \emph{sequential} if steps of different operations are not interleaved.
In other words, a sequential execution is a sequence of operation executions.

Two executions are \emph{indistinguishable} to a set of operations if each
operation in the set executes the same steps on shared objects, and
gets the same value from those objects, in both executions. A step $\tau$
by operation $op$ is \emph{invisible} to all other operations 
if the executions with and without $\tau$ are indistinguishable to
$\op\setminus \{op\}$. For example, read steps are invisible.

\paragraph{Correctness}

The correctness of a data structure is defined in terms of its external behavior, as reflected in values returned by invoked operations.
Correctness of a code transformation is proven by showing that the synthesized code's executions are equivalent to ones of the original code,
where two executions are  \emph{equivalent} if when considering operations that have completed every thread invokes the same
operations in the same order  in both executions, and gets the same result for each operation. More formally, we say in this paper that a code transformation is \emph{correct} if every execution of the transformed code
is equivalent to some execution of the original code.

The widely-used correctness criterion of serializability relies on equivalence to sequential executions in order to
link a data structure's behavior under concurrency to its sequentially specified behavior. Since equivalence is transitive,
we get that any code transformation satisfying our correctness notion, when applied to serializable code, yields code that is also serializable.
If the code transformation further ensures the real-time order of operations (i.e., operations that do not overlap appear in the same order in 
executions of the transformed and original code), then linearizability (atomicity) is also invariant under the transformation.
Another important aspect of correctness is preserving the progress conditions of the original code, for example, deadlock-freedom.

In this paper, we are not concerned with internal consistency (as required e.g., by opacity~\cite{GuerraouiK2008} or the validity notion of~\cite{LevAriCK2014}),
which restricts the configurations an operation might see during its execution.
This is because our code transformation uses timeouts and exception handlers to overcome unexpected behavior that may arise when a thread sees an inconsistent view of global variables (similar to~\cite{Nakaike:2010}).


\paragraph{Formal Proof}
We consider a finite execution $\pi$ of the transformed
code, and find an equivalent execution of the original lock-based
code.
Each operation in $\pi$ is an interleaved sequence of read-only and validation phases followed by a (single) update phase, or a prefix of such pattern.
For each operation in $\pi$ we consider its \emph{successful validation}, i.e.,
the last (successful) execution of a validation phase before switching to the update phase. Each operation executes at most one successful 
validation. The read-only phase preceding the successful
validation phase is called \emph{successful read-only
phase} .
Each operation executes at most one successful read-only
phase.
Towards proving equivalence to the original code execution, for each operation $op$, we remove the prefix of $op$ that precedes the successful read-only phase.
This includes completely removing operations that have no successful read-only phase.
We call the resulting execution $\hat{\pi}$.
%Finally, we remove all validation phases executed during the successful
%read-only phase that are not the unique successful  validation phase of
%the operation.
The removed prefixes include read steps as well as tryLock and
unlock steps.
Removing read steps is invisible to other processes. Since we only remove steps that acquire locks all remaining locking steps in $\hat{\pi}$ have the same affect and get the same response (success or failure) as in $\pi$. Finally, since the operation discards all local (private) state when restarting a read-only phase, $\pi$ and $\hat{\pi}$ are indistiguishable to all operations that have completed in $\pi$.
\begin{claim}
\label{claim:pipihat}
$\pi$ and $\hat{\pi}$ are equivalent.
\end{claim}

Denote by $e_1, e_2, \ldots, e_k$ the sequence of the first steps of
the read set validation in the execution of successful validation
phases, by their order in $\hat{\pi}$, where $e_i$ is a step of the operation $op_{i}$ executed by process $p_{i}$.
(Possibly $p_i=p_j$ for $j \neq i$).

Let \op\ be the set of operations in $\hat{\pi}$.
For every operation $op_{i}$ in \op, consider the partition of $\hat{\pi}$ to
the following intervals $\hat{\pi}=\alpha_i\beta_i\gamma_i$, such that
$\alpha_i$ includes the execution interval of $op_{i}$'s (successful) read-only phase
(denote $op_{i}$'s read set $rs_{i}$); $\beta_i=\beta_{i_1}\beta_{i_2}$, is the
minimal execution interval of $op_{i}$'s successful validation phase;
in $\beta_{i_1}$, $op_{i}$ acquires 
locks on its lock set, denoted $ls_{i}$; 
$e_i$ is the first step of $\beta_{i_2}$, namely the read set validation
interval.

%The next claim follows from the fact that the validation phase of $op_{i}$
%in $\beta_i$ is successful, and includes locks and versions
%re-validation: 
%\eshcar{need to prove these? or are these clear from the alg description?}

\begin{claim}
\label{claim:locks}
No operation in $\op\setminus\{op_{i}\}$ holds a lock in
$\alpha_i\beta_{i_1}$ that is associated with an object $obj$ in $rs_{i}$ after $op_{i}$'s first
read of $obj$ in $\alpha_i$.
\end{claim}
\begin{proof}
Let $lck$ be the lock associated with $obj$. Before reading $obj$ the first time in $\alpha_i$ $op_i$ records the version number of $lck$ (line~\ref{code:track:getVersion} in function \emph{track}) and checks that $lck$ is not held by any other thread (line~\ref{code:track:verifyUnlocked} in function \emph{track}).

We assume the function \emph{isLockedByAnother} imposes a memory fence and that at the beginning of the validation phase there is a read fence. If another thread acquired $lck$ after the first read and did not release it, this is discovered during validation (line~\ref{code:validate:verifyUnlocked} in function \emph{validateReadSet}). If another thread acquired $lck$ after the first read--and therefore after reading the version the first time---and did release the lock, then this thread increased the version number of $lck$ before releasing it. The fencing guarantees that $op_i$ observes the version number has changed (line~\ref{code:validate:verifyVersion} in function \emph{validateReadSet}). Since the validation phase of $op_i$ is successful no operation other than $op_i$ holds $lck$ in
$\alpha_i\beta_{i_1}$.
\end{proof}

We next project
object versions out of $\hat{\pi}$'s configurations, and remove all accesses (reads and writes) to object versions.
That is, we replace steps that access versions with local steps that modify the operation's local memory only.
Note that we get an execution with exactly the same invocations, responses, local states, and shared object states, but without 
versions. 
We call the resulting execution $\pi'$.

%Essentially, $\pi'$ is a projection of $\pi$
%excluding versions and all prefixes of the operations preceding their (single)
%successful read-only phase. 
%%, and all failed validation attempts.
%Therefore, all operations that returned a value in $\pi'$ return the same values as in $\pi$:
\begin{claim}
\label{claim:pihatpitag}
$\pi'$ and $\hat{\pi}$ are equivalent.
\end{claim}

Our main lemma constructs the execution of a fully-pessimistic locking code. 
The core idea is to replace the optimistic read-only phase
and validation phase of each operation with a solo execution of the
pessimistic lock-based read phase taking
place at the point where all objects in the lock set are locked.
\begin{lemma}
\label{lemma:pitagtag}
There is an execution of lock-based algorithm that is equivalent to $\pi'$.
\end{lemma}
\begin{proof}
We start with the execution $\pi_0=\pi'$.
For every $i \geq 0$, we show how to perturb $\pi_i$ to
obtain an execution $\pi_{i+1}$. 
We consider the operations $\op=op_1, \ldots, \op_k$ as defined above by the steps $e_1, e_2, \ldots, e_k$ .
For an operation $op_j$ such that $j\geq i+1$ in $\pi_i$, let
  $\beta_{j}^{'}=\beta_{j_1}^{'}\beta_{j_2}^{'}$ be the minimal interval
  containing $op_{j}$'s validation phase, where $\beta_{j_1}^{'}$
  is the minimal interval containing $op_{j}$'s tryLock phase.
  Denote the configuration between $\beta_{j_1}^{'}$ and $\beta_{j_2}^{'}$
  $C_{j}$. In $\pi_{i}$ the following conditions are
satisfied:
\begin{enumerate}
  \item \label{cond:lp} The operations $op_{1},\ldots,op_{i}$ execute the
  fully-pessimistic locking algorithm, while the rest of the operations
  $\opt_{i}=\op\setminus\{op_{1},\ldots,op_{i-1}\}$ execute our
  semi-optimistic algorithm.
  \item \label{cond:locks} 
  For $j\geq i+1$, no operation in $\op\setminus\{op_{j}\}$ holds a lock in
  $C_{j}$ that is associated with an object $obj$ in $rs_{j}$.
  \item \label{cond:writes} 
  For $j\geq i+1$, no operation in
  $\op\setminus\{op_{j}\}$ writes to an object $obj$ in $rs_{j}$ after
  $op_{j}$ first read $obj$ before $C_{j}$.
  \item \label{cond:trylocks} 
  For $j\geq i+1$, all try-lock steps by $op_j$ are invisible to
  $\op\setminus\{op_{j}\}$.
  %after $op_{j}$'s last read $obj$ before $\beta_j^{'}$
  \item \label{cond:equiv} $\pi'$ and $\pi_{i}$ are equivalent.
\end{enumerate}

For $\opt_{k}=\emptyset$, we get an execution where all operations execute the
pessimistic locking algorithm, and by Condition~\ref{cond:equiv} $\pi_{k+1}$ is
equivalent to $\pi'$ and we are done.

The proof is by induction on $i$. For the base case we consider
the execution $\pi_0$. Condition~\ref{cond:lp} holds since none of
the operations in this execution execute the full locking algorithm.
Conditions~\ref{cond:locks} and~\ref{cond:writes} hold by
Claim~\ref{claim:locks}, 
and since accesses to objects (other than versions) are similar in $\hat{\pi}$ and
$\pi_0$. Condition~\ref{cond:trylocks} holds since by construction, in $\pi_0$ every step accessing an object, either
for locking it or for validating it is not locked, finds the object not locked.
Condition~\ref{cond:equiv} vacuously holds since $\pi'$ and $\pi_0$ are the
same execution.

For the induction step, assume $\opt_i \neq \emptyset$ and
the execution
$\pi_i$ satisfies
the above conditions.
We consider $op_i \in \opt_i$ which partitions $\pi_i$ to $\alpha_i^{'}\beta_{i_1}^{'}\beta_{i_2}^{'}\gamma_i^{'}$. We replace $\pi_i$ with
$\pi_{i+1}=\alpha_i^{''}\beta_{i_1}^{''}\delta_i\beta_{i_2}^{''}\gamma_i^{'}$,
such that $\alpha_i^{''}$, $\beta_{i_1}^{''}$, and $\beta_{i_2}^{''}$ are the
projection of $\alpha_i^{'}$, $\beta_{i_1}^{'}$ and $\beta_{i_2}^{'}$, excluding
the steps by $op_{i}$, while $\delta_i$ is a $p_{i}$-only execution
interval in which $p_{i}$ follows the locking algorithm while
reading $rs_{i}$; after $\delta_{i}$, $p_{i}$ holds the locks on all
objects in $ls_{i}$, and holds no lock on other objects. 
In other words, we replace the optimistic read-only phase and validation phase
of $op_{i}$ with an execution of the original
locking algorithm, taking place at $C_{j}$.
%at the point just before the read set validation starts.

By Condition~\ref{cond:locks} of the induction hypothesis no operation holds 
locks associated with objects in the read set of $op_{i}$ in $C_{i}$, therefore,
$p_{i}$ can acquire the locks on these objects while executing $\delta_{i}$.
By Condition~\ref{cond:writes} of the induction hypothesis no
operation writes to an object $obj$ in $rs_{i}$ after
$op_{i}$ first read $obj$ before $C_{i}$, hence $op_{i}$ reads the same
values in its read set in $\pi_i$ and $\pi_{i+1}$. After $\delta_{i}$,
$op_{i}$ holds the locks on all objects in $ls_i$, hence it can continue with
the execution of the locking algorithm.
This implies that the projection of the execution $\pi_{i+1}$ on $op_{i}$
follows the full pessimistic locking algorithm satisfying Condition~\ref{cond:lp}.

In $\alpha_i^{''}\beta_{i_1}^{''}$ we only removed read steps and tryLock steps
by $op_{i}$ that are invisible to all other operations, by
Condition~\ref{cond:trylocks}. Therefore, the executions
$\alpha_i^{'}\beta_{i_1}^{'}$, ending with configuration $C'$, 
and $\alpha_i^{''}\beta_{i_1}^{''}\delta_i$, ending with configuration $C''$, 
are indistinguishable to all operations in $\op\setminus\{op_{i}\}$. 
In addition, in $\beta_{i_2}^{''}$ we only removed invisible read steps.
The values of all shared objects and locks are the same in $C'$ and $C''$,
hence the executions $\alpha_i^{'}\beta_{i_1}^{'}\beta_{i_2}^{'}\gamma_i^{'}$
and $\alpha_i^{''}\beta_{i_1}^{''}\delta_i\beta_{i_2}^{''}\gamma_i^{'}$ are
indistinguishable to all operations in $\op\setminus\{op_{i}\}$. 

The indistinguishability and the induction hypothesis imply that Conditions~\ref{cond:locks},~\ref{cond:writes},~\ref{cond:trylocks} hold.
In addition, this implies that all completed operations return the same value in $\pi_{i+1}$ and $\pi'$, which
means Condition~\ref{cond:equiv} holds.

%It is left to show that $\pi_{i+1}$ satisfies
%Conditions~\ref{cond:locks},~\ref{cond:writes},~\ref{cond:trylocks}.
%This is straightforward from the induction hypothesis and the fact that only
%$op_{i}$ changed its excution in the last iteration, and specifically
%removed all its try-lock steps, and since $\delta_i$ precedes $C_{j}$ for all
%$j\geq i+1$ in $\pi_{i+1}$.
\end{proof}

By Lemma~\ref{lemma:pitagtag}, Claim~\ref{claim:pipihat} and Claim~\ref{claim:pihatpitag} we conclude the following
theorem:
\begin{theorem}
Every execution of the transformed code is equivalent to an
execution of the original locking code.
\end{theorem}

\section{Additional Results}
\label{sec:appendix:results}

\begin{figure*}
\begin{center}
\begin{tikzpicture}
\begin{axis}[mystyle,unbalanced,
title={\textbf{write-dominated}},
ylabel = { million ops/sec}]
\addplot [black,mark=square*] table [x={threads}, y={LockRemovalTree}]
{results/trees-i1000000-u100.log}; 
\addplot [orange,mark=diamond] table [x={threads}, y={LogicalOrderingTree}]
{results/trees-i10000-u100.log}; 
\addplot [blues5,mark=10-pointed star] table [x={threads}, y={NonBlockingTorontoBSTMap}]
{results/trees-i10000-u100.log}; 
\addplot [magenta,mark=+] table [x={threads}, y={STMBinaryTree}]
{results/trees-i10000-u100.log};
\addplot [blues4,mark=x] table [x={threads}, y={DominationLockingTree}]
{results/trees-i10000-u100.log};
\end{axis}
\end{tikzpicture}
\begin{tikzpicture}
\begin{axis}[mystyle,unbalanced,title={\textbf{mixed workload}}]
\addplot [black,mark=square*] table [x={threads}, y={LockRemovalTree}]
{results/trees-i10000-u50.log}; 
\addplot [orange,mark=diamond] table [x={threads}, y={LogicalOrderingTree}]
{results/trees-i10000-u50.log}; 
\addplot [blues5,mark=10-pointed star] table [x={threads}, y={NonBlockingTorontoBSTMap}]
{results/trees-i10000-u50.log}; 
\addplot [magenta,mark=+] table [x={threads}, y={STMBinaryTree}]
{results/trees-i10000-u50.log};
\addplot [blues4,mark=x] table [x={threads}, y={DominationLockingTree}]
{results/trees-i10000-u50.log};
\end{axis}
\end{tikzpicture}
\begin{tikzpicture}
\begin{axis}[mystyle,unbalanced,
title= { \textbf{read-only}}]
\addplot [black,mark=square*] table [x={threads}, y={LockRemovalTree}]
{results/trees-i10000-u0.log};
\addplot [orange,mark=diamond] table [x={threads}, y={LogicalOrderingTree}]
{results/trees-i10000-u0.log};  
\addplot [blues5,mark=10-pointed star] table [x={threads}, y={NonBlockingTorontoBSTMap}]
{results/trees-i10000-u0.log}; 
\addplot [magenta,mark=+] table [x={threads}, y={STMBinaryTree}]
{results/trees-i10000-u0.log};
\addplot [blues4,mark=x] table [x={threads}, y={DominationLockingTree}]
{results/trees-i10000-u0.log};
\end{axis}
\end{tikzpicture}


\ref{unbalancedLegened}
\end{center}
\caption{Throughput of unbalanced data structures. Small Tree. }
\label{evaluation:results:unbalanced}
\end{figure*}

\begin{figure*}
\begin{center}

\begin{tikzpicture}
\begin{axis}[mystyle,balanced,
 ylabel = { million ops/sec},
 title={\textbf{write-dominated}}]
\addplot [black,mark=square*] table [x={threads}, y={LockRemovalTreap}]
{results/trees-i10000-u100.log}; 
\addplot [orange,mark=diamond] table [x={threads}, y={LogicalOrderingAVL}]
{results/trees-i10000-u100.log}; 
\addplot [darkspringgreen,mark=triangle] table [x={threads}, y={LockBasedStanfordTreeMap}]
{results/trees-i10000-u100.log}; 
\addplot [coralred,mark=asterisk] table [x={threads}, y={LockBasedFriendlyTreeMap}]
{results/trees-i10000-u100.log};
\addplot [magenta,mark=+] table [x={threads}, y={STMTreap}]
{results/trees-i10000-u100.log};
\addplot [blues4,mark=x] table [x={threads}, y={DominationLockingTreap}]
{results/trees-i10000-u100.log};
\end{axis}
\end{tikzpicture}
\begin{tikzpicture}
\begin{axis}[mystyle,balanced,title={\textbf{mixed workload}}]
\addplot [black,mark=square*] table [x={threads}, y={LockRemovalTreap}]
{results/trees-i1000000-u50.log}; 
\addplot [orange,mark=diamond] table [x={threads}, y={LogicalOrderingAVL}]
{results/trees-i10000-u50.log}; 
\addplot [darkspringgreen,mark=triangle] table [x={threads}, y={LockBasedStanfordTreeMap}]
{results/trees-i10000-u50.log}; 
\addplot [coralred,mark=asterisk] table [x={threads}, y={LockBasedFriendlyTreeMap}]
{results/trees-i10000-u50.log};
\addplot [magenta,mark=+] table [x={threads}, y={STMTreap}]
{results/trees-i10000-u50.log};
\addplot [blues4,mark=x] table [x={threads}, y={DominationLockingTreap}]
{results/trees-i10000-u50.log};
\end{axis}
\end{tikzpicture}
\begin{tikzpicture}
\begin{axis}[mystyle,balanced, 
				 title={\textbf{read-only}},]
\addplot [black,mark=square*] table [x={threads}, y={LockRemovalTreap}]
{results/trees-i10000-u0.log}; 
\addplot [orange,mark=diamond] table [x={threads}, y={LogicalOrderingAVL}]
{results/trees-i10000-u0.log}; 
\addplot [darkspringgreen,mark=triangle] table [x={threads}, y={LockBasedStanfordTreeMap}]
{results/trees-i10000-u0.log}; 
\addplot [coralred,mark=asterisk] table [x={threads}, y={LockBasedFriendlyTreeMap}]
{results/trees-i10000-u0.log};
\addplot [magenta,mark=+] table [x={threads}, y={STMTreap}]
{results/trees-i10000-u0.log};
\addplot [blues4,mark=x] table [x={threads}, y={DominationLockingTreap}]
{results/trees-i10000-u0.log};
\end{axis}
\end{tikzpicture}


\ref{balancedLegened}



\end{center}
\caption{Throughput of balanced data
structures. Small Tree.}
\label{evaluation:results:balanced}
\end{figure*}

\begin{figure*}
	\begin{center}
	\begin{subfigure}[t]{.35\textwidth}
		\caption{Range queries}
		
\begin{tikzpicture}
\begin{axis}[mystyle,skiplist1000,
 ylabel = { million ops/sec},
 %title={\textbf{0\% read-only}}]
 ]
\addplot [black,mark=square*] table [x={threads}, y={TimeoutSkiplist}]
{results/range1000.txt}; 
\addplot [darkspringgreen,mark=triangle] table [x={threads}, y={Bronson}]
{results/range1000.txt}; 
\addplot [coralred,mark=asterisk] table [x={threads}, y={JavaSkipList}]
{results/range1000.txt};
\addplot [blues4,mark=x] table [x={threads}, y={DominationLockingSkiplist}]
{results/range1000.txt};
\end{axis}
\end{tikzpicture}

%\ref{skiplistLegened}



		\label{evaluation:results:range1000}
	\end{subfigure}
	\quad\quad
	\begin{subfigure}[t]{.35\textwidth}
		\caption{Insert and delete operations}
		
\begin{tikzpicture}
\begin{axis}[mystyle,skiplist,
 ylabel = { million ops/sec},
 %title={\textbf{0\% read-only}}]
 ]
\addplot [black,mark=square*] table [x={threads}, y={LockRemovalSkipList}]
{results/update10001.txt}; 
\addplot [violet,mark=o] table [x={threads}, y={LockFreeKSTRQ}]
{results/update10001.txt}; 
\addplot [teal,mark=pentagon] table [x={threads}, y={LockFreeJavaSkipList}]
{results/update10001.txt};
\addplot [magenta,mark=+] table [x={threads}, y={STMSkipList}]
{results/update10001.txt};
\addplot [blues4,mark=x] table [x={threads}, y={DominationLockingSkipList}]
{results/update10001.txt};
\end{axis}

\end{tikzpicture}

%\ref{skiplistLegened}



		\label{evaluation:results:update1000}
	\end{subfigure}
	\ref{skiplistLegened1000}
	\end{center}
\caption{Half the threads execute large range queries $[1000,2000]$
and half the threads execute insert and delete operations.}
\label{evaluation:results:skiplist1000}
\end{figure*}


\end{document}
