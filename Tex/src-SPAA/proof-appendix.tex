
\section{Formal Correctness Proof}\label{sec:formal-proof}

We now formalize the correctness arguments made in Section~\ref{sec:proof}. 
First we define our model and the correctness properties of the algorithm for
which we provide the proof.

\paragraph{Model}

We consider an asynchronous shared memory model, where independent threads
interact via shared memory objects. 
Every thread executes a sequence of operations, each of which is invoked with certain parameters and returns a response.
An operation's execution consists of a sequence of primitive \emph{steps}, beginning with an \emph{invoke} step, followed by
atomic accesses to shared objects, and ending with a \emph{return} step. Steps also modify the executing thread's local variables.

A \emph{configuration} is an assignment of values to all shared and local variables. Thus, each step takes the system from one
configuration to another. Steps are deterministically defined by the data structure's protocol and the current configuration.
In the \emph{initial configuration}, each variable holds its initial value.

An \emph{execution} is an alternating sequence of configurations and steps,
$C_0,s_1,C_1, \ldots,s_i,C_i,\ldots,$
where $C_0$ is an initial configuration,
and each configuration $C_i$ is the result of
executing step $s_i$ on configuration $C_{i-1}$.
We only consider finite executions in this paper.
An execution is \emph{sequential} if steps of different operations are not interleaved.
In other words, a sequential execution is a sequence of operation executions.

Two executions are \emph{indistinguishable} to a set of operations if each
operation in the set executes the same steps on primitive shared objects, and
receives the same value from those primitives, in both executions. A step $\tau$
by operation $op$ is \emph{invisible} to all other operations 
if the executions with and without $\tau$ are indistinguishable to
$\op\setminus \{op\}$. For example, read steps are invisible.

\paragraph{Correctness}

The correctness of a data structure is defined in terms of its external behavior, as reflected in values returned by invoked operations.
Correctness of a code transformation is proven by showing that the synthesized code's executions are equivalent to ones of the original code,
where two executions are  \emph{equivalent} if every thread invokes the same
operations in the same order  in both executions, and gets the same result for each operation. More formally, we say in this paper that a code transformation is \emph{correct} if every execution of the transformed code
is equivalent to some execution of the original code.

The widely-used correctness criterion of serializability relies on equivalence to sequential executions in order to
link a data structure's behavior under concurrency to its sequentially specified behavior. Since equivalence is transitive,
we get that any code transformation satisfying our correctness notion, when applied to serializable code, yields code that is also serializable.
If the code transformation further ensures the real-time order of operations (i.e., operations that do not overlap appear in the same order in 
executions of the transformed and original code), then linearizability (atomicity) is also invariant under the transformation.
Another important aspect of correctness is preserving the progress conditions of the original code, for example, deadlock-freedom.

In this paper, we are not concerned with internal consistency (as required e.g., by opacity~\cite{GuerraouiK2008} or the validity notion of~\cite{LevAriCK2014}),
which restricts the configurations an operation might see during its execution.
This is because our code transformation uses timeouts and exception handlers to overcome unexpected behavior that may arise when a thread sees an inconsistent view of global variables (similar to~\cite{Nakaike:2010}).


\paragraph{Formal Proof}
We consider a finite execution $\pi$ of the transformed
algorithm, and find an equivalent execution of the original lock-based
algorithm.
Let \op\ be the set of operations in $\pi$.
For each operation we consider its \emph{successful validation}, i.e.,
the successful execution of a validation phase when switching from the read-only
phase  to the update phase. Each operation executes at most one successful 
validation. Denote by $e_1, e_2, \ldots, e_k$ the sequence of the first steps of
the read set validation in the execution of successful validation
phases, by their order in $\pi$, where $e_i$ is a step of the operation $op_{i}$ executed by process $p_{i}$.
(Possibly $p_i=p_j$ for $j \neq i$).


For every operation $op_{i}$, consider the partition of $\pi$ to
the following intervals $\pi=\alpha_i\beta_i\gamma_i$, such that
$\alpha_i$ includes the execution interval of $op_{i}$'s read-only phase
(denote $op_{i}$'s read set $rs_{i}$); $\beta_i=\beta_{i_1}\beta_{i_2}$, is the
minimal execution interval of $op_{i}$'s successful validation phase;
in $\beta_{i_1}$, $op_{i}$ acquires 
locks on its lock set, denoted $ls_{i}$; 
$e_i$ is the first step of $\beta_{i_2}$, namely the read set validation
interval.

The next claim follows from the fact that the validation phase of $op_{i}$
in $\beta_i$ is successful, and includes locks and versions
re-validation: 
%\eshcar{need to prove these? or are these clear from the alg description?}

\begin{claim}
\label{claim:locks}
No operation in $\op\setminus\{op_{i}\}$ holds or acquires a lock in
$\alpha_i\beta_{i_1}$ on an object $obj$ in $rs_{i}$ after $op_{i}$'s last
read of $obj$ in $\alpha_i$.
\end{claim}


Towards proving equivalence to the original code execution we first project
object versions out of $\pi$'s configurations, and remove all accesses (reads and writes) to object versions.
That is, we replace steps that access versions with local steps that modify the operation's local memory only.
(Note that we get an execution with exactly the same invocations, responses, local states, and shared object states, but without 
versions). 
Second, note that each operation $op \in$ \op\ executes at most one successful read-only
phase, namely a complete read-only phase followed by a successful
 validation phase.
For each such $op$, we remove the prefix of $op$ that precedes the successful read-only phase.
This includes completely removing operations that have no successful read-only phase.
Finally, we remove all validation phases executed during the successful
read-only phase that are not the unique successful  validation phase of
the operation.
We call the resulting execution $\pi'$.

Essentially, $\pi'$ is a projection of $\pi$
excluding versions, all prefixes of the operations preceding their (single)
successful read-only phase, and all failed validation attempts.
These prefixes and validation phases include read steps as well as tryLock and
unlock steps.
Removing read steps is invisible to other processes. Since we remove all tryLock steps that have failed to acquire locks, 
all remaining tryLock are successful also in $\pi'$. 
Therefore, in $\pi'$ 
%% Idit: removed below, I don't know what valid is, but it's not an execution of the same protocol without versions
%is a valid execution such that 
all operations in \op\ return the same values as in $\pi$:
\begin{claim}
\label{claim:pipitag}
$\pi$ and $\pi'$ are equivalent.
\end{claim}

Note that by construction, in $\pi'$ every step accessing a lock object, either
for locking it or for validating it is not locked, finds the object not locked.

Our main lemma constructs the execution of a fully-pessimistic locking code. 
The core idea is to replace the optimistic read-only phase
and validation phase of each operation with a solo execution of the
pessimistic lock-based read phase taking
place at the point where the tryLockAll completes.
\begin{lemma}
\label{lemma:pitagtag}
There is an execution of lock-based algorithm that is equivalent to $\pi'$.
\end{lemma}
\begin{proof}
We start with the execution $\pi_0=\pi'$.
For every $i \geq 0$, we show how to perturb $\pi_i$ to
obtain an execution $\pi_{i+1}$. For each $j\geq i+1$, let
  $\beta_{j}^{'}=\beta_{j_1}^{'}\beta_{j_2}^{'}$ be the minimal interval
  containing $op_{j}$'s validation phase in $\pi_{i+1}$, where $\beta_{j_1}^{'}$
  is the minimal interval containing $op_{j}$'s tryLock phase.
  Denote the configuration between $\beta_{j_1}^{'}$ and $\beta_{j_2}^{'}$
  $C_{j}$. In $\pi_{i+1}$ the following conditions are
satisfied:
\begin{enumerate}
  \item \label{cond:lp} The operations $op_{1},\ldots,op_{i}$ follow the
  fully-pessimistic locking algorithm, while the rest of the operations
  $\opt_{i+1}=\op\setminus\{op_{1},\ldots,op_{i}\}$ proceed according to our
  semi-optimistic algorithm.
  \item \label{cond:locks} 
  For $j\geq i+1$, no operation in $\op\setminus\{op_{j}\}$ holds a lock in
  $C_{j}$ on an object $obj$ in $rs_{j}$.
  \item \label{cond:writes} 
  For $j\geq i+1$, no operation in
  $\op\setminus\{op_{j}\}$ writes to an object $obj$ in $rs_{j}$ after
  $op_{j}$ last read $obj$ before $C_{j}$.
  \item \label{cond:trylocks} 
  For $j\geq i+1$, all try-lock steps by $op_j$ are invisible to
  $\op\setminus\{op_{j}\}$.
  %after $op_{j}$'s last read $obj$ before $\beta_j^{'}$
  \item \label{cond:equiv} $\pi'$ and $\pi_{i+1}$ are equivalent.
\end{enumerate}

For $\opt_{k+1}=\emptyset$, we get an execution where all operations follow the
pessimistic locking algorithm, and by Condition~\ref{cond:equiv} $\pi_{k+1}$ is
equivalent to $\pi'$ and we are done.

The proof is by induction on $i$. For the base case we consider
the execution $\pi'=\pi_0$. Condition~\ref{cond:lp} holds since none of
the operations in this execution follow the full locking algorithm.
Conditions~\ref{cond:locks} and~\ref{cond:writes} hold by
Claim~\ref{claim:locks}, and since in $\pi'$ we only remove read and lock steps,
and since accesses to objects (other than versions) are similar in $\pi$ and
$\pi'$. Condition~\ref{cond:trylocks} holds by the observation above.
Condition~\ref{cond:equiv} vacuously holds since $\pi'$ and $\pi_0$ are the
same execution.

For the induction step, assume $\opt_i \neq \emptyset$ and
the execution
$\pi_i=\alpha_i^{'}\beta_{i_1}^{'}\beta_{i_2}^{'}\gamma_i^{'}$ satisfies
the above conditions.
We replace $\pi_i$ with
$\pi_{i+1}=\alpha_i^{''}\beta_{i_1}^{''}\delta_i\beta_{i_2}^{''}\gamma_i^{'}$,
such that $\alpha_i^{''}$, $\beta_{i_1}^{''}$, and $\beta_{i_2}^{''}$ are the
projection of $\alpha_i^{'}$, $\beta_{i_1}^{'}$ and $\beta_{i_2}^{'}$, excluding
the steps by $op_{i}$, while $\delta_i$ is a $p_{i}$-only execution
interval in which $p_{i}$ follows the locking algorithm while
reading $rs_{i}$; after $\delta_{i}$, $p_{i}$ holds the locks on all
objects in $ls_{i}$, and holds no lock on other objects. 
In other words, we replace the optimistic read-only phase and validation phase
of $op_{i}$ with an execution of a read phase instrumented with the
locking algorithm, taking place at $C_{j}$.
%at the point just before the read set validation starts.

By Condition~\ref{cond:locks} of the induction hypothesis no operation holds 
locks on objects in the read set of $op_{i}$ in $C_{i}$, therefore,
$p_{i}$ can acquire the locks on these objects while executing $\delta_{i}$.
By Condition~\ref{cond:writes} of the induction hypothesis no
operation writes to an object $obj$ in $rs_{i}$ after
$op_{i}$ last read $obj$ before $C_{i}$, hence $op_{i}$ reads the same
values in its read set in $\pi_i$ and $\pi_{i+1}$. After $\delta_{i}$,
$op_{i}$ holds the locks on all objects in $ls_i$, hence it can continue with
the execution of the locking algorithm.

In $\alpha_i^{''}\beta_{i_1}^{''}$ we only removed read steps and tryLock steps
by $op_{i}$ that are invisible to all other operations, by
Condition~\ref{cond:trylocks}. Therefore, the executions
$\alpha_i^{'}\beta_{i_1}^{'}$, ending with configuration $C'$, 
and $\alpha_i^{''}\beta_{i_1}^{''}\delta_i$, ending with configuration $C''$, 
are indistinguishable to all operations in $\op\setminus\{op_{i}\}$. 
In addition, in $\beta_{i_2}^{''}$ we only removed invisible read steps.
The values of all shared objects and locks are the same in $C'$ and $C''$,
hence the executions $\alpha_i^{'}\beta_{i_1}^{'}\beta_{i_2}^{'}\gamma_i^{'}$
and $\alpha_i^{''}\beta_{i_1}^{''}\delta_i\beta_{i_2}^{''}\gamma_i^{'}$ are
indistinguishable to all operations in $\op\setminus\{op_{i}\}$. 

This implies that (1)~the projection of the execution $\pi_{i+1}$ on $op_{i}$
follows the full locking algorithm satisfying Condition~\ref{cond:lp}, and
(2)~all operations return the same value in $\pi_{i+1}$ as in $\pi'$, which
means Condition~\ref{cond:equiv} holds.

It is left to show that $\pi_{i+1}$ satisfies
Conditions~\ref{cond:locks},~\ref{cond:writes},~\ref{cond:trylocks}.
This is straightforward from the induction hypothesis and the fact that only
$op_{i}$ changed its locking pattern in the last iteration, and specifically
removed all its try-lock steps, and since $\delta_i$ precedes $C_{j}$ for all
$j\geq i+1$ in $\pi_{i+1}$.
 
\end{proof}

By Lemma~\ref{lemma:pitagtag} and Claim~\ref{claim:pipitag} we conclude the following
theorem:
\begin{theorem}
Every execution of the automatic transformation is equivalent to an
execution of the locking algorithm.
\end{theorem}
